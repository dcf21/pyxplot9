% types.tex
%
% The documentation in this file is part of Pyxplot
% <http://www.pyxplot.org.uk>
%
% Copyright (C) 2006-2013 Dominic Ford <coders@pyxplot.org.uk>
%               2008-2013 Ross Church
%
% $Id$
%
% Pyxplot is free software; you can redistribute it and/or modify it under the
% terms of the GNU General Public License as published by the Free Software
% Foundation; either version 2 of the License, or (at your option) any later
% version.
%
% You should have received a copy of the GNU General Public License along with
% Pyxplot; if not, write to the Free Software Foundation, Inc., 51 Franklin
% Street, Fifth Floor, Boston, MA  02110-1301, USA

% ----------------------------------------------------------------------------

% LaTeX source for the Pyxplot Users' Guide

\chapter{List of data types}
\label{ch:types_list}

The following is a list of Pyxplot's data types:

\begin{itemize}
\item boolean
\item color
\item date
\item dictionary
\item exception
\item fileHandle
\item function
\item instance
\item list
\item matrix
\item module
\item null
\item number
\item string
\item type
\item vector
\end{itemize}

Each of these data types has a prototype object in the module {\tt types},
which can be called like a function to create a new object of the type. See
Section~\ref{sec:functions_types} for details of the arguments accepted by each
prototype.

All objects in Pyxplot have methods that can be called on them, using the
generic syntax:

\begin{verbatim}
object.methodName(arguments)
\end{verbatim}

\noindent Some methods are common to all objects. For example, all objects have
a method {\tt str()} which produces a string representation of the object, as
used by the {\tt print} command. They also all have a {\tt methods()} method,
which returns a list of the names of all of the methods available for the
object. For example:

\vspace{3mm}
\input{fragments/tex/types_methods.tex}
\vspace{3mm}

\noindent As the above examples show, printing a method object returns brief
documentation about it. The sections below list the methods of each data type.

\section{Methods common to all data types}
\label{sec:common_methods}

\methdef{class()}{returns the class prototype of an object.}
\methdef{contents()}{returns a list of all the methods and internal variables of an object.}
\methdef{data()}{returns a list of all the internal variables (not methods) of an object.}
\methdef{methods()}{returns a list of the methods of an object.}
\methdef{str()}{returns a string representation of an object.}
\methdef{type()}{returns the type of an object.}

\section{The \texttt{boolean} type}
\label{sec:boolean_methods}

The \texttt{boolean} type has no methods other than those common to all types.
Objects of type \texttt{boolean} have no methods other than those common to all types.

\section{The \texttt{color} type}
\label{sec:color_methods}

\methdef{componentsCMYK()}{returns a vector CMYK representation of a color.}
\methdef{componentsHSB()}{returns a vector HSB representation of a color.}
\methdef{componentsRGB()}{returns a vector RGB representation of a color.}
\methdef{toCMYK()}{returns color object containing a CMYK representation of a color.}
\methdef{toHSB()}{returns color object containing an HSB representation of a color.}
\methdef{toRGB()}{returns color object containing an RGB representation of a color.}

\section{The \texttt{date} type}
\label{sec:date_methods}

For more information about manipulating dates in Pyxplot, see
Section~\ref{sec:time_series}.  For more information about manipulating times
and dates in Pyxplot, see Section~\ref{sec:time_series}. Many of the methods
listed below take an optional timezone string as their final argument. This
should be specified in the form {\tt Europe/London}, {\tt America/New\_York} or
{\tt Australia/Perth}, as used by the {\tt tz database}. A complete list of
available timezones can be found here:
\url{http://en.wikipedia.org/wiki/List_of_tz_database_time_zones}.  If
universal time is used, the timezone may be specified as {\tt UTC}.

\methdef{str($<format>,<timezone>$)}{converts a date object to a string with an optional format string supplied as an argument (see the {\tt time.string()} function).}
\methdef{toDayOfMonth($<timezone>$)}{returns the day of the month of a date object in the current calendar.}
\methdef{toDayWeekName($<timezone>$)}{returns the name of the day of the week of a date object.}
\methdef{toDayWeekNum($<timezone>$)}{returns the day of the week (1--7) of a date object.}
\methdef{toHour($<timezone>$)}{returns the integer hour component (0--23) of a date object.}
\methdef{toJD()}{converts a date object to a numerical Julian date.}
\methdef{toMinute($<timezone>$)}{returns the integer minute component (0--59) of a date object.}
\methdef{toMJD()}{converts a date object to a modified Julian date.}
\methdef{toMonthName($<timezone>$)}{returns the name of the month in which a date object falls.}
\methdef{toMonthNum($<timezone>$)}{returns the number (1--12) of the month in which a date object falls.}
\methdef{toSecond($<timezone>$)}{returns the seconds component (0--60) of a date object, including the non-integer component.}
\methdef{toUnix()}{converts a date object to a Unix time.}
\methdef{toYear($<timezone>$)}{returns the year in which a date object falls in the current calendar.}

\section{The \texttt{dictionary} type}
\label{sec:dictionary_methods}

\methdef{delete($s$)}{deletes any element with string key $s$ from the dictionary.}
\methdef{hasKey($x$)}{returns a boolean indicating whether the key $x$ exists in the dictionary.}
\methdef{items()}{returns a list of the [key,value] pairs in a dictionary.}
\methdef{keys()}{returns a list of the keys defined in a dictionary.}
\methdef{len()}{returns the number of entries in a dictionary.}
\methdef{values()}{returns a list of the values in a dictionary.}

\section{The \texttt{exception} type}
\label{sec:exception_methods}

\methdef{raise($x$)}{raises an exception with error message string $x$.}

\section{The \texttt{fileHandle} type}
\label{sec:fileHandle_methods}

\methdef{close()}{closes a file handle.}
\methdef{dump($x$)}{stores a typeable ASCII representation of the object $x$ to a file. Note that this method has no checking for recursive hierarchical data structures.}
\methdef{eof()}{returns a boolean flag to indicate whether the end of a file has been reached.}
\methdef{flush()}{flushes any buffered data which has not yet physically been written to a file.}
\methdef{getPos()}{returns a file handle's current position in a file.}
\methdef{isOpen()}{returns a boolean flag indicating whether a file is open.}
\methdef{read()}{returns the contents of a file as a string.}
\methdef{readline()}{returns a single line of a file as a string.}
\methdef{readlines()}{returns the lines of a file as a list of strings.}
\methdef{setPos($x$)}{sets a file handle's current position in a file.}
\methdef{write($x$)}{writes the string $x$ to a file.}

\section{The \texttt{function} type}

Objects of type \texttt{function} have no methods other than those common to all types.

\section{The \texttt{instance} type}

\methdef{delete($s$)}{deletes any element with string key $s$ from the instance.}
\methdef{hasKey($x$)}{returns a boolean indicating whether the key $x$ exists in the instance.}
\methdef{items()}{returns a list of the [key,value] pairs in a instance.}
\methdef{keys()}{returns a list of the keys defined in a instance.}
\methdef{len()}{returns the number of entries in a instance.}
\methdef{values()}{returns a list of the values in a instance.}

\section{The \texttt{list} type}
\label{sec:list_methods}

\methdef{append($x$)}{appends the object $x$ to a list and returns the new list.}
\methdef{count($x$)}{returns the number of items in a list that equal $x$.}
\methdef{extend($x$)}{appends the members of a list or vector $x$ to the operand and returns the new list.}
\methdef{filter($f$)}{takes a pointer to a function of one argument, $f(a)$. It calls the function for every element of the list, and returns a new list of those elements for which $f(a)$ tests true.}
\methdef{index($x$)}{returns the index of the first element of a list that equals $x$, or $-1$ if no elements match.}
\methdef{insert$n,x$)}{inserts the number $x$ into a list at position $n$, and returns the new list.}
\methdef{len()}{returns the number of elements in a list.}
\methdef{map($f$)}{takes a pointer to a function of one argument, $f(a)$. It calls the function for every element of the list, and returns a list of the results.}
\methdef{max()}{returns the highest-valued item in a list.}
\methdef{min()}{returns the lowest-valued item in a list.}
\methdef{pop()}{returns the last item in a list, and removes it from the list.}
\methdef{reduce($f$)}{takes a pointer to a function of two arguments. It first calls $f(a,b)$ on the first two elements of the list, and then continues through the list calling $f(a,b)$ on the result and the next item in the list. The final result is returned.}
\methdef{reverse()}{reverses the order of the members of a list, and returns the new list.}
\methdef{sort()}{sorts the members of a list into ascending order, and returns the new list.}
\methdef{sortOn($f$)}{sorts the members of a list using the user-supplied function $f(a,b)$ to determine the sort order. $f$ should return $1$, $0$ or $-1$ depending whether $a>b$, $a=b$ or $a<b$.}
\methdef{sortOnElement($n$)}{sorts a list of lists on the $n$th element of each sublist. If $n$ is negative, it counts from the final item of each list, $n=-1$ being the last item.}
\methdef{vector()}{returns the elements in a list as a vector.}

\section{The \texttt{matrix} type}
\label{sec:matrix_methods}

\methdef{det()}{returns the determinant of a square matrix.}
\methdef{diagonal()}{returns a boolean indicating whether a matrix is diagonal.}
\methdef{eigenvalues()}{returns a vector containing the eigenvalues of a square symmetric matrix.}
\methdef{eigenvectors()}{returns a list of the eigenvectors of a square symmetric matrix.}
\methdef{inv()}{returns the inverse of a square matrix.}
\methdef{size()}{returns the dimensions of a matrix.}
\methdef{symmetric()}{returns a boolean indicating whether a matrix is symmetric.}
\methdef{transpose()}{returns the transpose of a matrix.}

\section{The \texttt{module} type}

\methdef{delete($s$)}{deletes any element with string key $s$ from the module.}
\methdef{hasKey($x$)}{returns a boolean indicating whether the key $x$ exists in the module.}
\methdef{items()}{returns a list of the [key,value] pairs in a module.}
\methdef{keys()}{returns a list of the keys defined in a module.}
\methdef{len()}{returns the number of entries in a module.}
\methdef{values()}{returns a list of the values in a module.}

\section{The \texttt{null} type}

Objects of type \texttt{null} have no methods other than those common to all types.

\section{The \texttt{number} type}

Objects of type \texttt{number} have no methods other than those common to all types.

\section{The \texttt{string} type}
\label{sec:string_methods}

\methdef{append($x$)}{appends the string $x$ to the end of a string.}
\methdef{beginsWith($x$)}{returns a boolean indicating whether a string begins with the substring $x$.}
\methdef{endsWith($x$)}{returns a boolean indicating whether a string ends with the substring $x$.}
\methdef{find($x$)}{returns the position of the first occurrence of $x$ in a string, or $-1$ if it is not found.}
\methdef{findAll($x$)}{returns a list of the positions where the substring $x$ occurs in a string.}
\methdef{isalnum()}{returns a boolean indicating whether all of the characters of a string are alphanumeric.}
\methdef{isalpha()}{returns a boolean indicating whether all of the characters of a string are alphabetic.}
\methdef{isdigit()}{returns a boolean indicating whether all of the characters of a string are numeric.}
\methdef{len()}{returns the length of a string.}
\methdef{lower()}{converts a string to lowercase.}
\methdef{lstrip()}{strips whitespace off the beginning of a string.}
\methdef{split()}{returns a list of all the whitespace-separated words in a string.}
\methdef{splitOn(...)}{splits a string whenever it encounters any of the substrings supplied as arguments, and returns a list of the split string segments.}
\methdef{strip()}{strips whitespace off the beginning and end of a string.}
\methdef{rstrip()}{strips whitespace off the end of a string.}
\methdef{upper()}{converts a string to uppercase.}

\section{The \texttt{type} type}

Objects of type \texttt{type} have no methods other than those common to all types.

\section{The \texttt{vector} type}
\label{sec:vector_methods}

\methdef{append($x$)}{appends the number $x$ to a vector and returns the new vector.}
\methdef{extend($x$)}{appends the members of a list or vector $x$ to the operand and returns the new vector.}
\methdef{filter($f$)}{takes a pointer to a function of one argument, $f(a)$. It calls the function for every element of the vector, and returns a new vector of those elements for which $f(a)$ tests true.}
\methdef{insert($n,x$)}{inserts the number $x$ into a vector at position $n$, and returns the new vector.}
\methdef{len()}{returns the number of elements in a vector.}
\methdef{list()}{returns the elements in a vector as a list.}
\methdef{map($f$)}{takes a pointer to a function of one argument, $f(a)$. It calls the function for every element of the vector, and returns a vector of the results.}
\methdef{norm()}{returns the quadrature sum of the elements in a vector.}
\methdef{reduce($f$)}{takes a pointer to a function of two arguments. It first calls $f(a,b)$ on the first two elements of the vector, and then continues through the vector calling $f(a,b)$ on the result and the next item in the vector. The final result is returned.}
\methdef{reverse()}{reverses the order of the elements of a vector, and returns the new vector.}
\methdef{sort()}{sorts the elements of a vector into ascending order, and returns the new vector.}

