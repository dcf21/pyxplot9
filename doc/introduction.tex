% introduction.tex
%
% The documentation in this file is part of PyXPlot
% <http://www.pyxplot.org.uk>
%
% Copyright (C) 2006-2012 Dominic Ford <coders@pyxplot.org.uk>
%               2008-2012 Ross Church
%
% $Id$
%
% PyXPlot is free software; you can redistribute it and/or modify it under the
% terms of the GNU General Public License as published by the Free Software
% Foundation; either version 2 of the License, or (at your option) any later
% version.
%
% You should have received a copy of the GNU General Public License along with
% PyXPlot; if not, write to the Free Software Foundation, Inc., 51 Franklin
% Street, Fifth Floor, Boston, MA  02110-1301, USA

% ----------------------------------------------------------------------------

% LaTeX source for the PyXPlot Users' Guide

\chapter{Introduction}

\label{ch:introduction}

{\sc PyXPlot} is a multi-purpose scientific scripting language, graph plotting
tool, vector graphics suite, and data processing package.  Its interface is
designed so that common tasks -- e.g., plotting labelled graphs of data --
should be accessible via short, simple and intuitive commands. At the same
time, these commands also take many optional settings which allow their output
to be fine-tuned into a wide range of different styles, appropriate for
inclusion in reports, talks or academic journals.

\noindent{\bf A scientific scripting language}

PyXPlot has all the features of a scripting language: string manipulation,
modules and classes. Numbers have physical units, and calculations
automatically return results in appropriate units. \Datafile s can
straightforwardly be converted between physical units. PyXPlot natively handles
dictionaries, lists, dates and file handles. It also supports vector and matrix
algebra, can integrate or differentiate expressions, and numerically solve
systems of equations.

\noindent{\bf A vector graphics suite}

The graphical canvas isn't just for plotting graphs. Circles, polygons and
ellipses can be drawn using simple commands. Colors are a native type of object
for easy customisation. For the mathematically minded, PyXPlot's canvas
interfaces cleanly to its vector math environment, making geometric
construction easy.

\noindent{\bf A data processing package}

PyXPlot can interpolate data, find best fit lines, and compile histograms. It
can Fourier transform data, calculate statistics, and output results to new
\datafile s. Where fine control is needed, custom code can be used to process
every data point in a file.

\section{Compatibility with \gnuplot}

PyXPlot's plotting interface is very similar to that of \gnuplot: the commands
used for plotting simple graphs in the two programs are virtually identical.
Gnuplot users will have a head start with PyXPlot -- simple \gnuplot\ scripts
will often run in PyXPlot with minimal modification -- though the syntax used
for advanced plotting tasks often differs. Although PyXPlot's programming and
mathematical environment is hugely extended over \gnuplot's, we have followed
the latter's preference for short simple command line syntax.

\section{The structure of this manual}

This manual serves both as a tutorial guide to PyXPlot, and also as a reference
manual. Part~I provides a step-by-step tutorial and overview of PyXPlot's
features, including numerous worked examples. Part~II provides a detailed
survey of PyXPlot's plotting and vector graphics commands. Part~III provides an
alphabetical reference to all of PyXPlot's commands, mathematical functions and
plotting options.  Finally, the appendices provide information which is likely
to be of more specialist interest.

\section{An introductory tour}

This section provide a flavour of the wide range of tasks for which PyXPlot can
be used. It is not the place for detailed explanation of the syntax of PyXPlot
commands -- that will come in following sections -- but most of the examples
will work if pasted directly into a PyXPlot command prompt.

\subpart{The mathematical environment}

PyXPlot's mathematical environment comes with many standard functions built-in.
To see a list of them\footnote{See also Chapter~\ref{ch:function_list}.}, type

\begin{verbatim}
print defaults
\end{verbatim}

\noindent or

\begin{verbatim}
show functions
\end{verbatim}

\noindent PyXPlot's built-in functions live in a module called {\tt defaults}
whose members may be listed by {\tt print}ing the module object. Alternatively,
the {\tt show} command is PyXPlot's interactive documentation system.

Taking as an example the built-in function {\tt log10(x)}, samples may be taken
from it as in almost any other programming language:

\vspace{3mm}
\input{fragments/tex/intro_print1.tex}
\vspace{3mm}

\noindent While this returns numerical data, other functions return a variety
of data types. For example, the {\tt primeFactors} function returns a list:

\vspace{3mm}
\input{fragments/tex/intro_print2.tex}
\vspace{3mm}

\noindent the {\tt rgb(r,g,b)} function returns a color object for use in
PyXPlot vector graphics commands:

\vspace{3mm}
\input{fragments/tex/intro_print3.tex}
\vspace{3mm}

\noindent and the {\tt time.fromUnix(t)} function returns a date object from a
Unix time:

\vspace{3mm}
\input{fragments/tex/intro_print4.tex}
\vspace{3mm}

PyXPlot also comes with many built-in physical constants. Many of them live in
the physics module, {\tt phy}, for example, the speed of light:

\vspace{3mm}
\input{fragments/tex/intro_print5.tex}
\vspace{3mm}

\noindent Numbers in PyXPlot have {\bf physical units}, and hence the speed of
light is displayed in km/s. If you would rather know how many miles light
travels in a year, you can change the display unit:

\vspace{3mm}
\input{fragments/tex/intro_print6.tex}
\vspace{3mm}

\noindent It is easy to use PyXPlot as a {\bf desktop calculator} to solve many
problems which would conventionally need careful conversion between physical
units. For example:

\begin{itemize}
\item What is $80^\circ$F in Celsius?

\vspace{3mm}
\input{fragments/tex/intro_tempConvert.tex}
\vspace{3mm}

\item How long does it take for light to travel from the Sun to the
Earth?\footnote{The astronomical unit (AU) is a unit used by astronomers, equal
to the average distance of the Earth from the Sun.}

\vspace{3mm}
\input{fragments/tex/intro_earthDistance.tex}
\vspace{3mm}

\item What wavelength of light corresponds to the ionisation energy of hydrogen
(13.6\,eV)?\footnote{The electron volt (eV) is a unit of energy used by
physicists.}

\vspace{3mm}
\input{fragments/tex/intro_Hionisation.tex}
\vspace{3mm}

\item What is the escape velocity of the Earth?\footnote{The Earth radius and
Earth mass are defined as units in PyXPlot.}

\vspace{3mm}
\input{fragments/tex/intro_escapeVelocity.tex}
\vspace{3mm}
\end{itemize}

\subpart{Graph plotting}

At its simplest, plotting a graph simply involved following the {\tt plot}
command with the name of a function of by plotted, e.g.:

\begin{verbatim}
plot sin(x)
\end{verbatim}

\noindent or by specifying the name of a \datafile:

\begin{verbatim}
plot 'data.dat' using 2:5 select $4>2
\end{verbatim}

\noindent In the example above, the fifth column of a \datafile\ is plotted
against the second, only including those \datapoint s where the fourth column
is larger than two.

In the example below, ranges are specified for the axes, and three Bessel
functions are plotted:

\begin{verbatim}
plot [0:10][-0.5:1] besselJ(0,x), besselJ(1,x), besselJ(2,x)
\end{verbatim}
\begin{center}
\includegraphics[width=8cm]{examples/eps/ex_intro_bessel}
\end{center}

\noindent With a little additional configuration, it is possible to produce
three-dimensional plots like this (see Example~\ref{ex:surface-sinc}):

\begin{center}
\includegraphics[width=8cm]{examples/eps/ex_surface_sinc}
\end{center}

\noindent It is also possible to produce colormaps with custom color scales,
such as this one that uses PyXPlot's in-built functions for converting
wavelengths of light into colors:

\begin{center}
\includegraphics[width=7cm]{examples/eps/ex_spectrum_1}
\end{center}

\noindent or this pair that demonstrate RGB color mixing:

\begin{center}
\includegraphics[width=6cm]{examples/eps/ex_spectrum_colmix1}
\includegraphics[width=6cm]{examples/eps/ex_spectrum_colmix2}
\end{center}

\subpart{Generating data tables}

PyXPlot can also produce tables of data, using a similar syntax to that used
for plotting graphs:

\begin{verbatim}
tabulate tan(x)
\end{verbatim}

\noindent One application of this is filter or re-format the contents of
\datafile s; this example takes only the third and seventh columns out of a
\datafile, and converts the latter from degrees into radians:

\begin{verbatim}
tabulate 'data.dat' using 3:$7*unit(deg)/unit(rad)
\end{verbatim}

\noindent The same effect could be achieved by setting radians as the default
unit of angle:

\begin{verbatim}
set unit of angle radians
tabulate 'data.dat' using 3:$7*unit(deg)
\end{verbatim}

\noindent More sophisticated data processing is also possible; this example
produces a histogram of the values in the fourth column of a datafile, and then
outputs that histogram as a new \datafile:

\begin{verbatim}
histogram h() 'data.dat' using 4
tabulate h(x)
\end{verbatim}

\subpart{Solving equations}

PyXPlot can numerically solve systems of equations; the following example
evaluates $\int_{0\,\mathrm{s}}^{2\,\mathrm{s}} x^2\,\mathrm{d}x$:

\vspace{3mm}
\input{fragments/tex/intro_integration.tex}
\vspace{3mm}

\noindent The {\tt solve} command can be used to solve systems of simultaneous equations, such as this system with two variables:

\vspace{3mm}
\input{fragments/tex/intro_simultaneousEq.tex}
\vspace{3mm}

\noindent The {\tt minimise} and {\tt maximise} commands find the extrema of functions; here they are used to find the minimum of the function $\cos(x)$ closest to $x=0.5$:

\vspace{3mm}
\input{fragments/tex/intro_cosMin.tex}
\vspace{3mm}

All of the examples shown so far have used only real numbers, but PyXPlot can
also perform algebra on complex numbers. By default, evaluation of {\tt
sqrt(-1)} throws an error -- the emergence of complex numbers is often an
indication that a calculation has gone wrong -- but complex arithmetic can be
enabled by typing\indcmd{set numerics complex}

\vspace{3mm}
\input{fragments/tex/intro_complex1.tex}
\vspace{3mm}

\noindent Many of the mathematical functions which are built into PyXPlot can take complex arguments, for example

\vspace{3mm}
\input{fragments/tex/intro_complex2.tex}
\vspace{3mm}

\subpart{Vector graphics}

PyXPlot's graph-plotting canvas can also be used for drawing general vector graphics, using simple commands such as:

\begin{verbatim}
box from -8,-4 to 8,4 with fillcolor green
text "PyXPlot" at 2,3
arrow from 0,0 to -4,2
line from -5,0 to 5,0
\end{verbatim}

\noindent These commands are described in detail in
Chapter~\ref{ch:vector_graphics}. They interface neatly to the vector type in
PyXPlot's mathematical environment, to ease geometric construction. For
example, this example draws a big arrow at angle $\theta$ to the vertical:

\begin{verbatim}
rotate(a) = matrix( [[cos(a),-sin(a)], \
                     [sin(a), cos(a)] ] )
pos = vector(0,5)*unit(cm)
theta = 30*unit(deg)
arrow from 0,0 to rotate(theta)*pos with linewidth 3
\end{verbatim}

\section{License}

This manual and the software which it describes are both copyright \copyright\
Dominic Ford~2006--2012 and Ross Church~2008--2012. They are distributed under
the GNU General Public License (GPL) Version~2, a copy of which is provided in
the {\tt COPYING} file in this distribution.\index{General Public
License}\index{license} Alternatively, it may be downloaded from the web, from
the following location:\\ \url{http://www.gnu.org/copyleft/gpl.html}.

\section{Spelling conventions}

Throughout this manual, US English is used. However, where spelling differs
between US~and UK~English, PyXPlot recognises both variants. For example, where
the word \texttt{color} appears in PyXPlot syntax, it may also be spelt
\texttt{colour}; \texttt{minimize} may also be spelt \texttt{minimise};
\texttt{gray} may also be spelt \texttt{grey}; \texttt{neighbor} may also be
spelt \texttt{neighbour}; etc.

\section{Acknowledgments}

PyXPlot builds on ideas from several pre-existing open-source software
projects.  We admire \gnuplot's simple and intuitive interface, and PyXPlot's
command syntax is intentionally very similar, to the point of backwards
compatibility in many cases. In design PyXPlot's extensions to \gnuplot's
syntax, we have followed the latter's preference for short simple command line
syntax.  In designing, PyXPlot's graphical output engine, we borrowed many
ideas from the PyX\index{PyX} graphics library for Python.

Several individuals deserve credit for their contributions to PyXPlot's
development. Michael Rutter shared his public domain code to converting bitmap
images into PostScript, which we used in the implementation of the {\tt image}
command and the {\tt colormap} plot style. Matthew Smith provided C
implementations of the Airy functions and the Riemann zeta function for general
complex inputs and helped to test PyXPlot's mathematical environment. Zolt\'an
V\"or\"os worked on our development team from 2010 until 2011.

We would also like to thank the users who have got in touch with us by email
since PyXPlot was first released on the web in~2006. Your feedback and
suggestions have been gratefully received.

Final thanks go to our team of alpha testers, without whose work PyXPlot would
doubtless still contain many more bugs.  Especial thanks go to Rachel Holdforth
and Stuart Prescott. Of course, the authors remain solely responsible for any
bugs which remain.

