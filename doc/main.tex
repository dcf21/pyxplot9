% main.tex
%
% The documentation in this file is part of Pyxplot
% <http://www.pyxplot.org.uk>
%
% Copyright (C) 2006-2012 Dominic Ford <coders@pyxplot.org.uk>
%               2008-2012 Ross Church
%
% $Id$
%
% Pyxplot is free software; you can redistribute it and/or modify it under the
% terms of the GNU General Public License as published by the Free Software
% Foundation; either version 2 of the License, or (at your option) any later
% version.
%
% You should have received a copy of the GNU General Public License along with
% Pyxplot; if not, write to the Free Software Foundation, Inc., 51 Franklin
% Street, Fifth Floor, Boston, MA  02110-1301, USA

% ----------------------------------------------------------------------------

% LaTeX source for the Pyxplot Users' Guide

\makeindex
% definitions.tex
%
% The documentation in this file is part of PyXPlot
% <http://www.pyxplot.org.uk>
%
% Copyright (C) 2006-2012 Dominic Ford <coders@pyxplot.org.uk>
%               2008-2012 Ross Church
%
% $Id$
%
% PyXPlot is free software; you can redistribute it and/or modify it under the
% terms of the GNU General Public License as published by the Free Software
% Foundation; either version 2 of the License, or (at your option) any later
% version.
%
% You should have received a copy of the GNU General Public License along with
% PyXPlot; if not, write to the Free Software Foundation, Inc., 51 Franklin
% Street, Fifth Floor, Boston, MA  02110-1301, USA

% ----------------------------------------------------------------------------

% LaTeX source for the PyXPlot Users' Guide

% This file contains a list of macro definitions used in the manual

\def\version{0.9.0}
\def\reldate{Pre-release version}

\newif\ifplastex
\plastexfalse

% Put ticks and crosses next to code examples
\newlength{\dontdowidth}
\setlength{\dontdowidth}{\textwidth}
\addtolength{\dontdowidth}{-2.5cm}
\newenvironment{dontdo}{\vspace{3mm}\noindent\begin{tabular}{p{1cm}p{\dontdowidth}}\noindent{\ifplastex\includegraphics{cross}\else\Large \XSolidBrush\fi }&\noindent\begin{minipage}{\dontdowidth}\tt}{\end{minipage}\end{tabular}\vspace{3mm}}
\newenvironment{dodo}  {\vspace{3mm}\noindent\begin{tabular}{p{1cm}p{\dontdowidth}}\noindent{\ifplastex\includegraphics{tick}\else\Large \CheckmarkBold\fi}&\noindent\begin{minipage}{\dontdowidth}\tt}{\end{minipage}\end{tabular}\vspace{3mm}}

% Place commands in the index in typewriter face
\newcommand\indcmd[1]{\index{#1 command@{\tt #1} command}}
\newcommand\indmod[1]{\index{#1 modifier@{\tt #1} modifier}}
\newcommand\indfun[1]{\index{#1 function@{\tt #1} function}}
\newcommand\indps [1]{\index{#1 plot style@{\tt #1} plot style}\index{plot styles!#1@{\tt #1}}}
\newcommand\indkey[1]{\index{#1 keyword@{\tt #1} keyword}}
\newcommand\indco [1]{\index{coordinate systems!#1@{\tt #1}}}

% As above, but also insert command name in text
\newcommand\indcmdt [1]{{\tt #1} command\indcmd{#1}}
\newcommand\indcmdts[1]{{\tt #1}\indcmd{#1}}
\newcommand\indmodt [1]{{\tt #1}\indmod{#1}}
\newcommand\indfunt [1]{{\tt #1}\indfun{#1}}
\newcommand\indpst  [1]{{\tt #1}\indps{#1}}
\newcommand\indkeyt [1]{{\tt #1}\indkey{#1}}
\newcommand\indcot  [1]{{\tt #1}\indco{#1}}

% Names of software packages where there's controversy over capitalisation
\newcommand\gnuplot{gnuplot}
\newcommand\ghostview{Ghostview}
\newcommand\imagemagick{ImageMagick}

% There's some controversy over whether these should have a space in them
\newcommand\subpart[1]{\vspace{4mm}\noindent{\bf\large #1}\vspace{4mm}}
\newcommand\datafile{datafile}
\newcommand\Datafile{Datafile}
\newcommand\datapoint{datapoint} % "Datum", surely?

\def\sinc{{\rm sinc}}



% Make bold tt work
\usepackage{bold-extra}

% Make box and example float environments
\definecolor{LightGrey}{gray}{0.9}
\usepackage[bf]{caption}
\usepackage{float}
\floatstyle{plain}

% Make box float environment
\newfloat{boxout2}{thp}{lob}
\floatname{boxout2}{Box}

\newcommand{\boxout}[3]{
\begin{boxout2}
\begin{tabular}{|>{\columncolor{LightGrey}}l|}
\hline
\begin{minipage}{\textwidth}
\vspace{2.5mm}
#3
\vspace{2.5mm}
\end{minipage} \\
\hline
\end{tabular}
\caption{#1}
\label{#2}
\end{boxout2}
}

% Make example float environment
\newfloat{exampletag}{thp}{loe}
\floatname{exampletag}{Example}

\newcommand{\example}[3]{
%\afterpage{
\upshape\mdseries\rm
\begin{exampletag}[H]\caption[#2]{#2.}\label{#1}\end{exampletag}
\vspace{-5mm}
\begin{longtable}{|>{\columncolor{LightGrey}}p{\textwidth}|}
\hline \endfoot
\hline \endhead
#3
\end{longtable}%\newpage
%}
}

\newcommand{\nlnp} {\\\hspace{8mm}}
\newcommand{\nlscf}{\vspace{3mm}\\\noindent}
\newcommand{\nlfcf}{\vspace{3mm}\\\hspace{8mm}}

\begin{document}

\begin{titlepage}
\normalsize
\vspace*{0.4cm}
\begin{center}
{\Huge \bf Pyxplot Users' Guide}\\
\end{center}
\vspace*{0.5cm}
\begin{center}
{\LARGE \bf A Scientific Scripting Language, \\ \vspace{2mm} Graph Plotting Suite and \\ \vspace{2mm} Vector Graphics Toolkit. \\}
\end{center}
\vspace*{0.4cm}
\begin{center}
{\Large Version \version \\}
\end{center}
\vspace*{0.0cm}
\begin{center}
\includegraphics[width=10cm]{examples/eps/ex_cover}
\end{center}
\vspace*{0.2cm}
\begin{center}
{\large
Lead Developer: Dominic Ford \\
\vspace{1mm}
Lead Tester: Ross Church \\ 
\vspace{2mm}
Email: \noindent {\tt coders@pyxplot.org.uk} \\
\vspace{4mm}
This manual is also available in HTML, at \\
\url{http://www.pyxplot.org.uk/0.9/doc/html/} \\
}
\end{center}
\vspace*{0.5cm}
\begin{center}
{\Large \reldate \\}
\end{center}
\end{titlepage}

\pagenumbering{roman}

\makeatletter
\renewcommand\l@chapter[2]{%
  \ifnum \c@tocdepth >\m@ne
    \addpenalty{-\@highpenalty}%
    \vskip 1.0em \@plus\p@
    \setlength\@tempdima{2em}%
    \begingroup
      \parindent \z@ \rightskip \@pnumwidth
      \parfillskip -\@pnumwidth
      \leavevmode \bfseries
      \advance\leftskip\@tempdima
      \hskip -\leftskip
      #1\nobreak\hfil \nobreak\hb@xt@\@pnumwidth{\hss #2}\par
      \penalty\@highpenalty
    \endgroup
  \fi}
\renewcommand\l@section{\@dottedtocline{1}{2.0em}{3.0em}}
\renewcommand\l@subsection{\@dottedtocline{2}{5.0em}{4.0em}}
\makeatother
\tableofcontents

\listoffigures
\listof{exampletag}{List of Examples}

\newpage
\pagenumbering{arabic}

\part{Introduction to Pyxplot}
% introduction.tex
%
% The documentation in this file is part of PyXPlot
% <http://www.pyxplot.org.uk>
%
% Copyright (C) 2006-2012 Dominic Ford <coders@pyxplot.org.uk>
%               2008-2012 Ross Church
%
% $Id$
%
% PyXPlot is free software; you can redistribute it and/or modify it under the
% terms of the GNU General Public License as published by the Free Software
% Foundation; either version 2 of the License, or (at your option) any later
% version.
%
% You should have received a copy of the GNU General Public License along with
% PyXPlot; if not, write to the Free Software Foundation, Inc., 51 Franklin
% Street, Fifth Floor, Boston, MA  02110-1301, USA

% ----------------------------------------------------------------------------

% LaTeX source for the PyXPlot Users' Guide

\chapter{Introduction}

\label{ch:introduction}

{\sc PyXPlot} is a multi-purpose scientific scripting language, graph plotting
tool, vector graphics suite, and data processing package.  Its interface is
designed so that common tasks -- e.g., plotting labelled graphs of data --
should be accessible via short, simple and intuitive commands. At the same
time, these commands also take many optional settings which allow their output
to be fine-tuned into a wide range of different styles, appropriate for
inclusion in reports, talks or academic journals.

\noindent{\bf A scientific scripting language}

PyXPlot has all the features of a scripting language: string manipulation,
modules and classes. Numbers have physical units, and calculations
automatically return results in appropriate units. \Datafile s can
straightforwardly be converted between physical units. PyXPlot natively handles
dictionaries, lists, dates and file handles. It also supports vector and matrix
algebra, can integrate or differentiate expressions, and numerically solve
systems of equations.

\noindent{\bf A vector graphics suite}

The graphical canvas isn't just for plotting graphs. Circles, polygons and
ellipses can be drawn using simple commands. Colors are a native type of object
for easy customisation. For the mathematically minded, PyXPlot's canvas
interfaces cleanly to its vector math environment, making geometric
construction easy.

\noindent{\bf A data processing package}

PyXPlot can interpolate data, find best fit lines, and compile histograms. It
can Fourier transform data, calculate statistics, and output results to new
\datafile s. Where fine control is needed, custom code can be used to process
every data point in a file.

\section{Compatibility with \gnuplot}

PyXPlot's plotting interface is very similar to that of \gnuplot: the commands
used for plotting simple graphs in the two programs are virtually identical.
Gnuplot users will have a head start with PyXPlot -- simple \gnuplot\ scripts
will often run in PyXPlot with minimal modification -- though the syntax used
for advanced plotting tasks often differs. Although PyXPlot's programming and
mathematical environment is hugely extended over \gnuplot's, we have followed
the latter's preference for short simple command line syntax.

\section{The structure of this manual}

This manual serves both as a tutorial guide to PyXPlot, and also as a reference
manual. Part~I provides a step-by-step tutorial and overview of PyXPlot's
features, including numerous worked examples. Part~II provides a detailed
survey of PyXPlot's plotting and vector graphics commands. Part~III provides an
alphabetical reference to all of PyXPlot's commands, mathematical functions and
plotting options.  Finally, the appendices provide information which is likely
to be of more specialist interest.

\section{An introductory tour}

This section provide a flavour of the wide range of tasks for which PyXPlot can
be used. It is not the place for detailed explanation of the syntax of PyXPlot
commands -- that will come in following sections -- but most of the examples
will work if pasted directly into a PyXPlot command prompt.

\subpart{The mathematical environment}

PyXPlot's mathematical environment comes with many standard functions built-in.
To see a list of them\footnote{See also Chapter~\ref{ch:function_list}.}, type

\begin{verbatim}
print defaults
\end{verbatim}

\noindent or

\begin{verbatim}
show functions
\end{verbatim}

\noindent PyXPlot's built-in functions live in a module called {\tt defaults}
whose members may be listed by {\tt print}ing the module object. Alternatively,
the {\tt show} command is PyXPlot's interactive documentation system.

Taking as an example the built-in function {\tt log10(x)}, samples may be taken
from it as in almost any other programming language:

\vspace{3mm}
\input{fragments/tex/intro_print1.tex}
\vspace{3mm}

\noindent While this returns numerical data, other functions return a variety
of data types. For example, the {\tt primeFactors} function returns a list:

\vspace{3mm}
\input{fragments/tex/intro_print2.tex}
\vspace{3mm}

\noindent the {\tt rgb(r,g,b)} function returns a color object for use in
PyXPlot vector graphics commands:

\vspace{3mm}
\input{fragments/tex/intro_print3.tex}
\vspace{3mm}

\noindent and the {\tt time.fromUnix(t)} function returns a date object from a
Unix time:

\vspace{3mm}
\input{fragments/tex/intro_print4.tex}
\vspace{3mm}

PyXPlot also comes with many built-in physical constants. Many of them live in
the physics module, {\tt phy}, for example, the speed of light:

\vspace{3mm}
\input{fragments/tex/intro_print5.tex}
\vspace{3mm}

\noindent Numbers in PyXPlot have {\bf physical units}, and hence the speed of
light is displayed in km/s. If you would rather know how many miles light
travels in a year, you can change the display unit, here making use of the fact
that the variable {\tt ans} is always set to the result of the last
calculation:

\vspace{3mm}
\input{fragments/tex/intro_print6.tex}
\vspace{3mm}

\noindent It is easy to use PyXPlot as a {\bf desktop calculator} to solve many
problems which would conventionally need careful conversion between physical
units. For example:

\begin{itemize}
\item What is $80^\circ$F in Celsius?

\vspace{3mm}
\input{fragments/tex/intro_tempConvert.tex}
\vspace{3mm}

\item How long does it take for light to travel from the Sun to the
Earth?\footnote{The astronomical unit (AU) is a unit used by astronomers, equal
to the average distance of the Earth from the Sun.}

\vspace{3mm}
\input{fragments/tex/intro_earthDistance.tex}
\vspace{3mm}

\item What wavelength of light corresponds to the ionisation energy of hydrogen
(13.6\,eV)?\footnote{The electron volt (eV) is a unit of energy used by
physicists.}

\vspace{3mm}
\input{fragments/tex/intro_Hionisation.tex}
\vspace{3mm}

\item What is the escape velocity of the Earth?\footnote{The Earth radius and
Earth mass are defined as units in PyXPlot.}

\vspace{3mm}
\input{fragments/tex/intro_escapeVelocity.tex}
\vspace{3mm}
\end{itemize}

\subpart{Graph plotting}

At its simplest, plotting a graph simply involved following the {\tt plot}
command with the name of a function of by plotted, e.g.:

\begin{verbatim}
plot sin(x)
\end{verbatim}

\noindent or by specifying the name of a \datafile:

\begin{verbatim}
plot 'data.dat' using 2:5 select $4>2
\end{verbatim}

\noindent In the example above, the fifth column of a \datafile\ is plotted
against the second, only including those \datapoint s where the fourth column
is larger than two.

In the example below, ranges are specified for the axes, and three Bessel
functions are plotted:

\begin{verbatim}
plot [0:10][-0.5:1] besselJ(0,x), besselJ(1,x), besselJ(2,x)
\end{verbatim}
\begin{center}
\includegraphics[width=8cm]{examples/eps/ex_intro_bessel}
\end{center}

\noindent With a little additional configuration, it is possible to produce
three-dimensional plots like this (see Example~\ref{ex:surface-sinc}):

\begin{center}
\includegraphics[width=8cm]{examples/eps/ex_surface_sinc}
\end{center}

\noindent It is also possible to produce colormaps with custom color scales,
such as this one that uses PyXPlot's in-built functions for converting
wavelengths of light into colors:

\begin{center}
\includegraphics[width=7cm]{examples/eps/ex_spectrum_1}
\end{center}

\noindent or this pair that demonstrate RGB color mixing:

\begin{center}
\includegraphics[width=6cm]{examples/eps/ex_spectrum_colmix1}
\includegraphics[width=6cm]{examples/eps/ex_spectrum_colmix2}
\end{center}

\subpart{Generating data tables}

PyXPlot can also produce tables of data, using a similar syntax to that used
for plotting graphs:

\begin{verbatim}
tabulate tan(x)
\end{verbatim}

\noindent One application of this is filter or re-format the contents of
\datafile s; this example takes only the third and seventh columns out of a
\datafile, and converts the latter from degrees into radians:

\begin{verbatim}
tabulate 'data.dat' using 3:$7*unit(deg)/unit(rad)
\end{verbatim}

\noindent The same effect could be achieved by setting radians as the default
unit of angle:

\begin{verbatim}
set unit of angle radians
tabulate 'data.dat' using 3:$7*unit(deg)
\end{verbatim}

\noindent More sophisticated data processing is also possible; this example
produces a histogram of the values in the fourth column of a datafile, and then
outputs that histogram as a new \datafile:

\begin{verbatim}
histogram h() 'data.dat' using 4
tabulate h(x)
\end{verbatim}

\subpart{Solving equations}

PyXPlot can numerically solve systems of equations; the following example
evaluates $\int_{0\,\mathrm{s}}^{2\,\mathrm{s}} x^2\,\mathrm{d}x$:

\vspace{3mm}
\input{fragments/tex/intro_integration.tex}
\vspace{3mm}

\noindent The {\tt solve} command can be used to solve systems of simultaneous equations, such as this system with two variables:

\vspace{3mm}
\input{fragments/tex/intro_simultaneousEq.tex}
\vspace{3mm}

\noindent The {\tt minimise} and {\tt maximise} commands find the extrema of functions; here they are used to find the minimum of the function $\cos(x)$ closest to $x=0.5$:

\vspace{3mm}
\input{fragments/tex/intro_cosMin.tex}
\vspace{3mm}

All of the examples shown so far have used only real numbers, but PyXPlot can
also perform algebra on complex numbers. By default, evaluation of {\tt
sqrt(-1)} throws an error -- the emergence of complex numbers is often an
indication that a calculation has gone wrong -- but complex arithmetic can be
enabled by typing\indcmd{set numerics complex}

\vspace{3mm}
\input{fragments/tex/intro_complex1.tex}
\vspace{3mm}

\noindent Many of the mathematical functions which are built into PyXPlot can take complex arguments, for example

\vspace{3mm}
\input{fragments/tex/intro_complex2.tex}
\vspace{3mm}

\subpart{Vector graphics}

PyXPlot's graph-plotting canvas can also be used for drawing general vector graphics, using simple commands such as:

\begin{verbatim}
box from -8,-4 to 8,4 with fillcolor green
text "PyXPlot" at 2,3
arrow from 0,0 to -4,2
line from -5,0 to 5,0
\end{verbatim}

\noindent These commands are described in detail in
Chapter~\ref{ch:vector_graphics}. They interface neatly to the vector type in
PyXPlot's mathematical environment, to ease geometric construction. For
example, this example draws a big arrow at angle $\theta$ to the vertical:

\begin{verbatim}
rotate(a) = matrix( [[cos(a),-sin(a)], \
                     [sin(a), cos(a)] ] )
pos = vector(0,5)*unit(cm)
theta = 30*unit(deg)
arrow from 0,0 to rotate(theta)*pos with linewidth 3
\end{verbatim}

\section{License}

This manual and the software which it describes are both copyright \copyright\
Dominic Ford~2006--2012 and Ross Church~2008--2012. They are distributed under
the GNU General Public License (GPL) Version~2, a copy of which is provided in
the {\tt COPYING} file in this distribution.\index{General Public
License}\index{license} Alternatively, it may be downloaded from the web, from
the following location:\\ \url{http://www.gnu.org/copyleft/gpl.html}.

\section{Spelling conventions}

Throughout this manual, US English is used. However, where spelling differs
between US~and UK~English, PyXPlot recognises both variants. For example, where
the word \texttt{color} appears in PyXPlot syntax, it may also be spelt
\texttt{colour}; \texttt{minimize} may also be spelt \texttt{minimise};
\texttt{gray} may also be spelt \texttt{grey}; \texttt{neighbor} may also be
spelt \texttt{neighbour}; etc.

\section{Acknowledgments}

PyXPlot builds on ideas from several pre-existing open-source software
projects. We like \gnuplot's simple and intuitive interface, and PyXPlot's
command syntax is intentionally very similar, to the point of backwards
compatibility in many cases. Even when designing the entirely new parts of
PyXPlot's syntax, we have followed \gnuplot's preference for short simple
command-line syntax. In designing PyXPlot's graphical output engine, we
borrowed many ideas from the PyX\index{PyX} graphics library for Python.

Several people have contributed code to PyXPlot. Michael Rutter provided us 
with a copy of his public domain code for converting bitmap images into
PostScript, which we used in the implementation of the {\tt image} command and
the {\tt colormap} plot style. Matthew Smith provided C implementations of the
Airy functions and the Riemann zeta function for general complex inputs, and
helped to test PyXPlot's mathematical environment. Zolt\'an V\"or\"os worked on
our development team from 2010 until 2011.  John Walker has published public
domain code implementing RGB rendering of the electromagnetic spectrum, which
we use in the {\tt colors.\-wave\-length()} function.

We would also like to thank all the users who have got in touch with us by
email since PyXPlot was first released on the web in~2006. Your feedback and
suggestions have been gratefully received.

Final thanks go to our team of alpha testers, without whose work PyXPlot would
doubtless still contain many more bugs.  Especial thanks go to Rachel Holdforth
and Stuart Prescott.


% installation.tex
%
% The documentation in this file is part of Pyxplot
% <http://www.pyxplot.org.uk>
%
% Copyright (C) 2006-2012 Dominic Ford <coders@pyxplot.org.uk>
%               2008-2012 Ross Church
%
% $Id$
%
% Pyxplot is free software; you can redistribute it and/or modify it under the
% terms of the GNU General Public License as published by the Free Software
% Foundation; either version 2 of the License, or (at your option) any later
% version.
%
% You should have received a copy of the GNU General Public License along with
% Pyxplot; if not, write to the Free Software Foundation, Inc., 51 Franklin
% Street, Fifth Floor, Boston, MA  02110-1301, USA

% ----------------------------------------------------------------------------

% LaTeX source for the Pyxplot Users' Guide

\chapter{Installation}

\label{ch:installation}

In this chapter we describe how to install Pyxplot on a range of UNIX-like
operating systems.

Pyxplot works on most UNIX-like operating systems. We have tested it under
Linux, Solaris\index{Solaris} and MacOS\index{MacOS}, and believe that it
should work on other similar POSIX systems. We regret that it is not available
for Microsoft Windows, and have no plans for porting it at this time.

\section{Installation within Linux distributions}

By far the easiest way to install Pyxplot under Linux is to use your
distribution's package manager.  If you use a recent release of
Gentoo\index{Gentoo Linux}\index{installation!under Gentoo}\footnote{See
\url{http://gentoo-portage.com/sci-visualization/pyxplot}}, Ubuntu\index{Ubuntu
Linux}\index{installation!under Ubuntu}\footnote{See
\url{http://packages.ubuntu.com/pyxplot}} or Debian\footnote{See
\url{http://packages.debian.org/pyxplot}}, your package manager can install
Pyxplot and all its dependencies for you, though the packaged version may be
several months behind the latest release. Please note that this manual
describes Pyxplot~0.9.x, which is a very substantial upgrade to version~0.8.x.
To install the packaged version of Pyxplot under Debian or Ubuntu, simply type:

\begin{verbatim}
apt-get install pyxplot gv
\end{verbatim}

Users of other distributions, or who want a newer version of Pyxplot, should
use the {\tt .tar.gz} archives available from the Pyxplot website. The process
is described below.

\section{System requirements}

Pyxplot depends on the following software packages, which are not included in
the source tarball:\index{system requirements}

\vspace{0.5cm}
\begin{itemize}
\item fftw (version 2 or, preferably, 3+) \index{fftw}
\item gcc and make\index{gcc}\index{make}
\item Ghostscript \index{Ghostscript}
\item The Gnu Scientific Library (version 1.10+) \index{gsl}
\item ImageMagick \index{ImageMagick}
\item \LaTeX\ (version $2\epsilon$; a full installation is likely to be required in distributions which offer a choice) \index{latex}
\item libpng (version 1.2+) \index{libpng}
\item libxml2 (version 2.6+) \index{libxml}
\item zlib \index{zlib}
\end{itemize}
\vspace{0.5cm}

\noindent It is very strongly recommended that the following software packages
also be installed:

\vspace{0.5cm}
\begin{itemize}
\item cfitsio -- required for Pyxplot to be able to plot \datafile s in FITS format.
\item Ghostview \index{Ghostview} (or {\tt ggv}) -- required for Pyxplot to be able to display plots live on the screen; Pyxplot remains able to generate image files on disk without it. Alternatively, the \indcmdt{set viewer} within Pyxplot allows a different PostScript viewer to be used.
\item gunzip \index{gunzip} -- required for Pyxplot to be able to plot compressed \datafile s in {\tt .gz} format.
\item The Gnu Readline Library (version 5+) \index{readline} -- required for Pyxplot to be able to provide tab completion and command histories in Pyxplot's interactive command-line interface.
\item libkpathsea \index{libkpathsea} -- required to efficiently find the fonts used by \LaTeX.
\item wget \index{wget} -- required for Pyxplot to be able to plot \datafile s directly from the Internet.
\end{itemize}
\vspace{0.5cm}

\noindent In the case of the recommended packages, Pyxplot tests for the
availability of each when it is installed and issues a warning if any are not
found. Installation can proceed, but some of Pyxplot's features will be
disabled. If they are later added to the system, Pyxplot should be reinstalled
to take advantage of their presence.

\subsection{Dependencies in Debian and Ubuntu}

Debian and Ubuntu users can find the above software in the following
packages\footnote{The package names listed here are correct as of Debian
Squeeze and Ubuntu 12.04 (Precise). However, packages occasionally change name
between versions.}: \index{Debian Linux}\index{Ubuntu
Linux}\index{installation!under Debian}\index{installation!under Ubuntu}

\vspace{2mm}
\noindent {\tt fftw3-dev}, {\tt gcc}, {\tt ghostscript}, {\tt gv}, {\tt imagemagick}, {\tt libc6-dev},\newline
\noindent {\tt libcfitsio3-dev}, {\tt libgsl0-dev}, {\tt libkpathsea-dev}, {\tt libpng12-dev},\newline
\noindent {\tt libreadline-dev}, {\tt libxml2-dev}, {\tt make}, {\tt texlive-latex-extra},\newline
\noindent {\tt texlive-latex-recommended}, {\tt texlive-fonts-extra},\newline
\noindent {\tt texlive-fonts-recommended}, {\tt wget}, {\tt zlib1g-dev}.
\vspace{2mm}

\noindent These packages may be installed from a command prompt by typing, all on one line:

\begin{verbatim}
sudo apt-get install fftw3-dev gcc ghostscript gv imagemagick
       libc6-dev libcfitsio3-dev libgsl0-dev libkpathsea-dev
       libpng12-dev libreadline-dev libxml2-dev make
       texlive-latex-extra texlive-latex-recommended
       texlive-fonts-extra texlive-fonts-recommended wget
       zlib1g-dev
\end{verbatim}

\subsection{Dependencies in MacOS}

Users of MacOS~X can find the above software in the following MacPorts packages:
\index{MacOS X}\index{MacPorts}

\vspace{2mm}
\noindent {\tt cfitsio}, {\tt fftw-3}, {\tt ghostscript}, {\tt gsl-devel}, {\tt gv}, {\tt ImageMagick}, {\tt libpng},\newline
\noindent {\tt libxml2}, {\tt readline-5}, {\tt texlive}, {\tt wget}, {\tt zlib}.
\vspace{2mm}

It may then be necessary to run the command
\begin{verbatim}
export C_INCLUDE_PATH=/opt/local/include
\end{verbatim}
before running the {\tt configure} script below.

\section{Installation from source archive}
\index{installation}

First, download the required archive can be downloaded from the front page of
Pyxplot website -- \url{http://www.pyxplot.org.uk}. It is assumed that the
packages listed above have already been installed; if they are not, you will
need to either install them yourself, if you have superuser access to your
machine, or contact your system administrator.\index{installation!user-level}

\begin{itemize}
\item Unpack the distributed .tar.gz:

\begin{verbatim}
tar xvfz pyxplot_0.9.1.tar.gz
cd pyxplot-0.9.1
\end{verbatim}

\item Run the installation script:

\begin{verbatim}
./configure
make
\end{verbatim}

\item Finally, start Pyxplot:

\begin{verbatim}
./bin/pyxplot
\end{verbatim}

\end{itemize}

\subsection{System-wide installation}

Having completed the steps described above, Pyxplot may be installed
system-wide by a superuser with the following additional
step:\index{installation!system-wide}

\begin{verbatim}
sudo make install
\end{verbatim}

By default, the Pyxplot executable installs to {\tt /usr/local/bin/pyxplot}.
If desired, this installation path may be modified in the file {\tt
Makefile.skel}, by changing the variable {\tt USRDIR} in the first line to an
alternative desired installation location.

Pyxplot may now be started by any user of the system, simply by typing:

\begin{verbatim}
pyxplot
\end{verbatim}


% FIRST_STEPS.TEX
%
% The documentation in this file is part of PyXPlot
% <http://www.pyxplot.org.uk>
%
% Copyright (C) 2006-2011 Dominic Ford <coders@pyxplot.org.uk>
%               2008-2011 Ross Church
%
% $Id$
%
% PyXPlot is free software; you can redistribute it and/or modify it under the
% terms of the GNU General Public License as published by the Free Software
% Foundation; either version 2 of the License, or (at your option) any later
% version.
%
% You should have received a copy of the GNU General Public License along with
% PyXPlot; if not, write to the Free Software Foundation, Inc., 51 Franklin
% Street, Fifth Floor, Boston, MA  02110-1301, USA

% ----------------------------------------------------------------------------

% LaTeX source for the PyXPlot Users' Guide

\chapter{First Steps With PyXPlot}
\label{ch:first_steps}

In this chapter, we provide a brief overview of the commands which are used to
produce simple plots in PyXPlot, principally those whose syntax is borrowed
directly from \gnuplot. Users who are already familiar with \gnuplot\ may wish
to skim over this chapter, although there are some subtle differences been
\gnuplot\ and PyXPlot syntax which require care. Particular attention is drawn
to Section~\ref{sec:latex_incompatibility}, which describes the use of \LaTeX\
to render text.

In Chapter~\ref{ch:plotting} we will revisit and extend the material of this
chapter to show how to produce more advanced plots.

\section{Getting Started}

The simplest way to start PyXPlot is to type {\tt pyxplot} at a shell prompt.
This starts an interactive session, and a PyXPlot command-line prompt into
which commands can be typed will appear. PyXPlot can be exited either by typing
\indcmdts{exit} or \indcmdts{quit}, or by pressing CTRL-D. Various switches can
be specified on the shell command line to modify PyXPlot's behaviour; these are
listed in Box~\ref{box:CommandSwitches}.  Of especial interest may be the
switches {\tt -c} and {\tt -m}, which change between the use of
colour-highlighted (default) and non-coloured text.

Typing commands into interactive terminals will probably be a sufficient way to
drive PyXPlot to begin with, but as the number of commands required to set up
plots grows with the complexity of the task, it is likely to become rapidly
more preferable to store these commands in text files called scripts.  Once
such a script has been written, it can be executed automatically by passing the
filename of the command script to PyXPlot on the shell command line, for
example:\index{command-line syntax}

\begin{verbatim}
pyxplot foo.ppl
\end{verbatim}

\boxout{A list of the command line options accepted by PyXPlot.}{box:CommandSwitches}{
From the shell command line, PyXPlot accepts the following switches which
modify its behaviour:\index{command line syntax}
\vspace{0.5cm}

\begin{tabular}{p{3.0cm}p{8.3cm}}
{\tt -h --help} & Display a short help message listing the available command-line switches.\\
{\tt -v --version} & Display the current version number of PyXPlot.\\
{\tt -q --quiet} & Turn off the display of the welcome message on startup. \\
{\tt -V --verbose} & Display the welcome message on startup, as happens by default. \\
{\tt -c --colour} & Use colour highlighting\footnote{This will only function on terminals which support colour output.}, as is the default behaviour, to display output in green, warning messages in amber, and error messages in red.\footnote{The authors apologise to those members of the population who are red/green colour blind, but draw their attention to the following sentence.} These colours can be changed in the {\tt terminal} section of the configuration file; see Section~\ref{sec:configfile_terminal} for more details. \\
{\tt -m --monochrome} & Do not use colour highlighting. \\
\end{tabular}
}

\noindent In this case, PyXPlot would execute the commands in the file {\tt
foo.ppl} and then exit. By convention, we choose to affix the suffix {\tt
.ppl} to the filenames of all PyXPlot command scripts. This is not strictly
necessary, but allows PyXPlot scripts to be easily distinguished from other
text files in a filing system. The filenames of several command scripts may be
passed to PyXPlot on a single command line, indicating that they should be
executed in sequence, as in the example:

\begin{verbatim}
pyxplot foo1.ppl foo2.ppl foo3.ppl
\end{verbatim}

It is also possible to have a single PyXPlot session alternate between running
command scripts autonomously and allowing the user to enter commands
interactively between the running of the scripts. There are two ways of doing
this.  PyXPlot can be passed the magic filename {\tt --} on the command line,
as in the example

\begin{verbatim}
pyxplot foo1.ppl -- foo2.ppl
\end{verbatim}

\noindent where the {\tt --} represents an interactive session which commences
after the execution of {\tt foo1.ppl} and should be terminated by the user in
the usual way, using either the \indcmdts{exit} or \indcmdts{quit} commands.
After the interactive session is finished, PyXPlot will automatically execute
the command script {\tt foo2.ppl}.

From within an interactive session, it is possible to run a command script
using the \indcmdt{load}, as in the example:

\vspace{3mm}
\noindent{\tt pyxplot> {\bf load 'foo.ppl'}}
\vspace{3mm}

\noindent This example would have the same effect as typing the contents of the
file {\tt foo.ppl} into the present interactive terminal.

The \indcmdt{save} may assist in producing PyXPlot command scripts: it stores a
history of the commands which have been typed into the present interactive
session to file.

\boxout{The storage of command histories in PyXPlot.}{box:CommandHistory}{
When PyXPlot is used interactively, its command-line environment is based upon
the GNU Readline Library. This means that the up- and down-arrow keys can be
used to repeat or modify previously executed commands. Each user's command
history is stored in his homespace in a history file called {\tt
.pyxplot\_history}; this file is used by PyXPlot to remember command histories
between sessions. PyXPlot's \indcmdt{save} allows the user to save to a text
file a list of the commands which have been typed into the present session, as
in the following example:\vspace{3mm}

\noindent {\tt save 'output\_filename.ppl' }\vspace{3mm}

The related \indcmdt{history} displays on the terminal a history of all of the
commands which have been typed into this and previous interactive sessions. The
total history can stretch to several hundred lines long, in which case it can
be useful to follow the \indcmdt{history} by an optional number, whereupon it
only displays the last $n$ commands, e.g.:\vspace{3mm}

\noindent {\tt history 20 }
}

\section{First Plots}
\label{sec:first_plots}

The basic workhorse command of PyXPlot is the \indcmdt{plot}, which is used to
produce all plots. The following simple example would plot the trigonometric
function $\sin(x)$:

\begin{verbatim}
plot sin(x)
\end{verbatim}

\begin{center}
\includegraphics[width=8cm]{examples/eps/ex_intro_sine}
\end{center}

\noindent This is one of a large number of standard mathematical functions
which are built into PyXPlot. We will meet more of these in due course, but a
complete list can be found in Appendix~\ref{ch:function_list}.

As well as plotting functions, it is also possible to plot data stored in
files. The following would plot data from a file {\tt data.dat}, taking the
$x$-coordinate of each point from the first column of the \datafile, and the
$y$-coordinate from the second.  The \datafile\ is assumed to be in plain text
format\footnote{If the filename of a \datafile\ ends with a {\tt .gz} suffix,
it is assuming to be gzipped plaintext, and is decoded accordingly. Other
formats of \datafile\ can be opened with the use of input filters; see
Section~\ref{sec:filters}.}, with columns separated by whitespace and/or
commas\footnote{This format is compatible with the Comma Separated Values (CSV)
format produced by many applications, including Microsoft Excel.}\index{csv
files}\index{spreadsheets, importing data from}\index{Microsoft
Excel}\index{gzip}:

\begin{verbatim}
plot 'data.dat'
\end{verbatim}

Several items can be plotted on the same graph by separating them by commas, as
in

\begin{verbatim}
plot 'data.dat', sin(x), cos(x)
\end{verbatim}

\noindent and it is possible to define one's own variables and functions,
and then plot them, as in the example

\begin{verbatim}
a = 0.02
b = -1
c = 5
f(x) = a*(x**3) + b*x + c
plot f(x)
\end{verbatim}

\begin{center}
\includegraphics[width=8cm]{examples/eps/ex_intro_func}
\end{center}

\noindent A complete list of the mathematical operators which can be used to
put together algebraic expressions can be found in
Table~\ref{tab:operators_table}.\index{functions!pre-defined}\index{operators}

\begin{table}
\begin{center}
\begin{tabular}{|>{\columncolor{LightGrey}}l>{\columncolor{LightGrey}}l>{\columncolor{LightGrey}}l|}
\hline
{\bf Symbol} & {\bf Description} & {\bf Operator Associativity} \\
\hline
{\tt **} & Algebraic exponentiation & right \\
\hline
{\tt -} & Unary minus sign & left \\
{\tt not} & Logical not & left \\
\hline
{\tt *} & Algebraic multiplication & left \\
{\tt /} & Algebraic division & left \\
{\tt \%} & Modulo operator & left \\
\hline
{\tt +} & Algebraic sum & left \\
{\tt -} & Algebraic subtraction & left \\
\hline
{\tt <<} & Left binary shift & left \\
{\tt >>} & Right binary shift & left \\
\hline
{\tt <} & Magnitude comparison & right \\
{\tt >} & Magnitude comparison & right \\
{\tt <=} & Magnitude comparison & right \\
{\tt >=} & Magnitude comparison & right \\
\hline
{\tt ==} & Equality comparison & right \\
{\tt !=} & Equality comparison & right \\
{\tt <>} & Alias for {\tt !=} & right \\
\hline
{\tt \&} & Binary and & left \\
\hline
{\tt \^{}} & Binary exclusive or & left \\
\hline
{\tt |} & Binary or & left \\
\hline
{\tt and} & Logical and & left \\
\hline
{\tt or} & Logical or & left \\
\hline
\end{tabular}
\end{center}
\caption{A list of mathematical operators which PyXPlot recognises, in order of
descending precedence. Items separated by horizontal rules are of differing
precedence; those not separated by horizontal rules are of equivalent
precedence. The third column indicates whether strings of operators are
evaluated from left to right (left), or from right to left (right). For
example, the expression {\tt x**y**z} is evaluated as {\tt (x**(y**z))}.}
\label{tab:operators_table}
\end{table}

%\example{ex:introduction}{An example}{
%Here, we have an example.
%}

\section{Comments}

As in any programming language, it is good practice to include comments in
PyXPlot command scripts to explain what each command or block of commands is
doing. Comment lines should begin with a hash character, as in the
example\index{comment lines}\index{command scripts!comment lines}

\begin{verbatim}
# This is a comment
\end{verbatim}

\noindent Comments may also be placed on the same line as commands, as in the
example

\begin{verbatim}
set nokey # I'll have no key on _my_ plot
\end{verbatim}

\noindent In both cases, all of the characters following the hash character are
ignored.

\section{Splitting Long Commands}

Long commands may be split over multiple lines in scripts provided that each
line of the command is terminated with a backslash character, whereupon the
following line will be appended to it, as in this example:

\vspace{3mm}
\noindent{\tt pyxplot> {\bf print~~2~$\backslash$}}\newline
\noindent{\tt .......>~~~~~~~{\bf +3}}\newline
\noindent{\tt 5}
\vspace{3mm}

\noindent Such lines splits will be used extensively in this manual where
command lines are longer than the width of the page.

\section{Printing Text}

PyXPlot's \indcmdt{print} can be used to display strings and the results of
calculations on the terminal, as in the following examples:

\vspace{3mm}
\noindent{\tt pyxplot> {\bf a=2}}\newline
\noindent{\tt pyxplot> {\bf print "Hello World!"}}\newline
\noindent{\tt Hello World!}\newline
\noindent{\tt pyxplot> {\bf print a}}\newline
\noindent{\tt 2}
\vspace{3mm}

Multiple items can be displayed one-after-another on a single line by
separating them with commas. The following example displays the values of the
variable {\tt a} and the function {\tt f(a)} in the middle of a text string:

\vspace{3mm}
\noindent{\tt pyxplot> {\bf f(x) = x**2}}\newline
\noindent{\tt pyxplot> {\bf a=3}}\newline
\noindent{\tt pyxplot> {\bf print "The value of ",a," squared is ",f(a),"."}}\newline
\noindent{\tt The value of 3 squared is 9.}
\vspace{3mm}

A similar effect is often achieved more neatly using the string substitution
operator, {\tt \%}\index{\% operator@{\tt \%} operator}\index{string
operators!substitution}.  The operator is preceded by a format string, in
which the places where numbers and strings are to be substituted are marked by
tokens such as {\tt \%e} and {\tt \%s}. The substitution operator is followed
by a ()-bracketed list of the quantities which are to be substituted into the
format string. This behaviour is similar to that of the Python programming
language's \% operator\footnote{Unlike in Python, the brackets are obligatory;
{\tt '\%d'\%2} is {\it not} valid in PyXPlot, and should be written as {\tt
'\%d'\%(2)}.}, and of the {\tt printf} command in C.  The following examples
demonstrate the use of this operator:

\vspace{3mm}
\noindent{\tt pyxplot> {\bf print "The value of \%d squared is \%d."\%(a,f(a))}}\newline
\noindent{\tt The value of 3 squared is 9.}\newline
\noindent{\tt pyxplot> {\bf print "The \%s of f(\%f) is \%d."\%("value",sqrt(2), $\backslash$}}\newline
\noindent{\tt .......>~~~~~~~~~~~~~~~~~~~~~~~~~~~~~~~~~{\bf f(sqrt(2)) )}}\newline
\noindent{\tt The value of f(1.414214) is 2.}
\vspace{3mm}

The detailed behaviour of the string substitution operator, and a full list of
the substitution tokens which it accepts, are given in
Section~\ref{sec:stringsubop}.

\section{Axis Labels and Titles}
\label{sec:latex_incompatibility}

Labels can be added to the axes of a plot, and a title put at the top.  Labels
should be placed between either single (') or double (") quotes, as in the
following example script:

\begin{verbatim}
set xlabel "Horizontal axis"
set ylabel "Vertical axis"
set title 'A plot with labelled axes'
plot
\end{verbatim}

\begin{center}
\includegraphics[width=8cm]{examples/eps/ex_axislabs}
\end{center}

\noindent These labels and title -- in fact, all text labels which are ever
produced by PyXPlot -- are rendered using the \LaTeX\ typesetting system, and
so any \LaTeX\ commands can be used to produce custom formatting. This allows
great flexibility, but means that care needs to be taken to escape any of
\LaTeX's reserved characters -- i.e. $\backslash$~\&~\%~\#~\{~\}~\$~\_~\^{} or
$\sim$.

Because of the use of quotes to delimit text labels, special care needs to be
taken when apostrophe and quote characters are used. The following command
would raise an error, because the apostrophe would be interpreted as marking
the end of the text label:

\begin{dontdo}
set xlabel 'My plot's X axis'
\end{dontdo}

\noindent The following would achieve the desired effect:

\begin{dodo}
set xlabel "My plot's X axis"
\end{dodo}

To make it possible to render \LaTeX\ strings containing both single and double
quote characters -- for example, the string {\tt J$\backslash$"org's Data},
which puts a German umlaut on the letter o as well as having a possessive
apostrophe -- PyXPlot recognises the backslash character to be an escape
character when followed by either~' or~". A double backslash
($\backslash\backslash$) represents a literal backslash. These are the
\textit{only} cases in which PyXPlot considers $\backslash$ an escape
character. To render the example string above, one would type:\index{escape
characters}\index{backslash character}\index{accented characters}

\begin{dodo}
set xlabel "J$\backslash\backslash\backslash$"org's Data"
\end{dodo}

\noindent In this example, three backslashes are required. The first pair
produce the \LaTeX\ escape character used to produce the umlaut; the second is
a PyXPlot escape character, used so that the~" character is not interpreted as
delimiting the string. \index{escape characters}\index{quote
characters}\index{special characters}

The pre-defined \indfunt{texify()} function may provide some assistance in
generating \LaTeX\ labels: it takes either an algebraic expression, or a string
in quotes, and produces a \LaTeX\ representation of it, as in the following
examples:

\vspace{3mm}
\noindent{\tt pyxplot> {\bf a=50}}\newline
\noindent{\tt pyxplot> {\bf print texify("A \%d\% increase"\%(a))}}\newline
\noindent{\tt A 50$\backslash$\% increase}\newline
\noindent{\tt pyxplot> {\bf print texify(sqrt(x**2+1))}}\newline
\noindent{\tt \$$\backslash$displaystyle $\backslash$sqrt\{x\^{}\{2\}+1\}\$}
\vspace{3mm}

Having set labels and titles, they may be removed thus:

\begin{verbatim}
set xlabel ''
set ylabel ''
set title ''
\end{verbatim}

\noindent These are two other ways of removing the title from a plot:

\begin{verbatim}
set notitle
unset title
\end{verbatim}

The \indcmdt{unset} may be followed by almost any word that can follow the {\tt
set} command, such as {\tt xlabel} or {\tt title}, to return that setting to
its default configuration. The \indcmdt{reset} restores all configurable
parameters to their default states.

\section{Querying the Values of Settings}

As the previous section has demonstrated, the \indcmdt{set} is used in a wide
range of ways to configure the way in which plots appear; we will meet many
more in due course. The corresponding \indcmdt{show} can be used to query the
current values of settings. To query the value of one particular setting, the
\indcmdt{show} should be followed by the name of the setting, as in the
example:

\begin{verbatim}
show title
\end{verbatim}

\noindent Alternatively, several settings may be requested at once, or all
settings beginning with a certain string of characters can be queried, as in
the following two examples:

\begin{verbatim}
show xlabel ylabel key
show g
\end{verbatim}

\noindent The special keyword {\tt settings} may be used to display the values
of all settings which can be set with the {\tt set} command. A list of other
special keywords which the \indcmdt{show} accepts is given in
Table~\ref{tab:show_keywords}.

\begin{table}
\begin{center}
\begin{tabular}{|>{\columncolor{LightGrey}}l>{\columncolor{LightGrey}}p{9cm}|}
\hline
{\bf Query} & {\bf Description} \\ \hline
{\tt all} & Lists all settings.\\
{\tt axes} & Lists all of the currently configured axes.\\
{\tt functions} & Lists all currently defined mathematical functions, both those which are built into PyXPlot and those which the user has defined.\\
{\tt settings} & Lists the current values of all settings which can be set with the {\tt set} command.\\
{\tt units} & Lists all of the physical units which PyXPlot is currently set up to recognise.\\
{\tt userfunctions} & Lists all current user-defined mathematical functions and subroutines.\\
{\tt variables} & Lists the values of all currently-defined variables.\\
\hline
\end{tabular}
\end{center}
\caption{The special keywords which the \indcmdt{show} recognises.}
\label{tab:show_keywords}
\end{table}

Generally, the \indcmdt{show} displays each setting in the form of a
\indcmdt{set} which could be used to set that setting, together with a comment
to briefly explain what effect the setting has. This means that the output can
be pasted directly into another PyXPlot terminal to copy settings from one
session to another. However, it should be noted that some settings, such as
{\tt papersize} are only pastable once the \indcmdt{set numerics typeable} has
been issued, for reasons which will be explained in Section~\ref{sec:pastable}.

When a colour-highlighted interactive session is used, settings which have been
changed are highlighted in yellow, whilst those settings which are unchanged
from PyXPlot's default configuration, or from a user-supplied configuration
file, are shown in green.

\section{Plotting \Datafile s}
\label{sec:plot_datafiles}

In the simple example of the previous section, we plotted the first column of a
\datafile\ against the second. It is possible to plot any arbitrary column of a
\datafile\ against any other; the syntax for doing this is:\indmod{using}

\begin{verbatim}
plot 'data.dat' using 3:5
\end{verbatim}

\noindent This example would plot the contents of the fifth column of the file
{\tt data.dat} on the vertical axis, against the contents of the third column
on the horizontal axis. As mentioned above, columns in \datafile s can be
separated using whitespace and/or commas.  Algebraic expressions may also be
used in place of column numbers, as in the example:

\begin{verbatim}
plot 'data.dat' using (3+$1+$2):(2+$3)
\end{verbatim}

\noindent In such expressions, column numbers are prefixed by dollar signs to
distinguish them from numerical constants. The example above would plot the sum
of the values in the first two columns of the \datafile, plus three, on the
horizontal axis, against two plus the value in the third column on the vertical
axis. The column numbers in such expressions can also be replaced by algebraic
expressions, and so {\tt \$2} can also be written as {\tt \$(2)} or {\tt
\$(1+1)}. In the following example, the \datapoint s are all placed on the
vertical line $x=3$ -- the brackets around the {\tt 3} distinguish it as a
numerical constant rather than a column number -- meanwhile their vertical
positions are drawn from the value of some column $n$ in the \datafile, where
the value of $n$ is itself read from the second column of the \datafile:

\begin{verbatim}
plot 'data.dat' using (3):$($2)
\end{verbatim}

It is also possible to plot data from only selected lines within a \datafile.
When PyXPlot reads a \datafile, it looks for any blank lines in the file. It
divides the \datafile\ up into {\it data blocks}, each being separated from the
next by a single blank line. The first data block is numbered~0, the next~1, and
so on.  \index{datafile format}

When two or more blank lines are found together, the \datafile\ is divided up
into {\it index blocks}. The first index block is numbered~0, the next~1, and
so on. Each index block may be made up of a series of data blocks. To clarify
this, a labelled example \datafile\ is shown in
Figure~\ref{fig:sample_datafile}.

\begin{figure}
\begin{center}
\begin{tabular}{|>{\columncolor{LightGrey}}p{2.2cm}>{\columncolor{LightGrey}}l|}
\hline
{\tt 0.0 \ 0.0} & Start of index 0, data block 0. \\
{\tt 1.0 \ 1.0} & \\
{\tt 2.0 \ 2.0} & \\
{\tt 3.0 \ 3.0} & \\
                & A single blank line marks the start of a new data block. \\
{\tt 0.0 \ 5.0} & Start of index 0, data block 1. \\
{\tt 1.0 \ 4.0} & \\
{\tt 2.0 \ 2.0} & \\
                & A double blank line marks the start of a new index. \\
                & ... \\
{\tt 0.0 \ 1.0} & Start of index 1, data block 0. \\
{\tt 1.0 \ 1.0} & \\
                & A single blank line marks the start of a new data block. \\
{\tt 0.0 \ 5.0} & Start of index 1, data block 1. \\
                & $<$etc$>$ \\
\hline
\end{tabular}
\end{center}
\caption{An example PyXPlot \datafile\ -- the \datafile\ is shown in the two left-hand columns, and commands are shown to the right.}
\label{fig:sample_datafile}
\end{figure}

By default, when a \datafile\ is plotted, all data blocks in all index blocks are
plotted. To plot only the data from one index block, the following syntax may
be used:

\begin{verbatim}
plot 'data.dat' index 1
\end{verbatim}

\noindent To achieve the default behaviour of plotting all index blocks, the
{\tt index} modifier should be followed by a negative number.\indmod{index}

It is also possible to specify which lines and/or data blocks to plot from
within each index. To do so, the \indmodt{every} modifier is used, which takes
up to six values, separated by colons:\label{sec:every}

\begin{verbatim}
plot 'data.dat' every a:b:c:d:e:f
\end{verbatim}

\noindent The values have the following meanings:

\begin{longtable}{p{1.0cm}p{10.5cm}}
$a$ & Plot data only from every $a\,$th line in \datafile. \\
$b$ & Plot only data from every $b\,$th block within each index block. \\
$c$ & Plot only from line $c$ onwards within each block. \\
$d$ & Plot only data from block $d$ onwards within each index block. \\
$e$ & Plot only up to the $e\,$th line within each block. \\
$f$ & Plot only up to the $f\,$th block within each index block. \\
\end{longtable}

\noindent Any or all of these values can be omitted, and so the following would
both be valid statements:

\begin{verbatim}
plot 'data.dat' index 1 every 2:3
plot 'data.dat' index 1 every ::3
\end{verbatim}

\noindent The first would plot only every other \datapoint\ from every third
data block; the second from the third line onwards within each data block.

\index{comment lines!in datafiles}
Comment lines may be included in \datafile s by prefixing them with a hash
character. Such lines are completely ignored by PyXPlot and do not count
towards the one or two blank lines required to separate blocks and index
blocks.  It is usually good practice to include comment lines at the top of
\datafile s to indicate their date and source. In
Section~\ref{sec:special_comments} we will see that PyXPlot can read metadata
from some comment lines which follow particular syntax.

%\example{ex:datafile}{An example}{
%Here, we have an example of plotting a datafile.
%}

\section{Plotting Many \Datafile s at Once}

\index{globbing}\index{wildcards}\index{datafiles!globbing}

The wildcards {\tt *} and {\tt ?} may be used in filenames supplied to the
\indcmdt{plot} to plot many \datafile s at once. The following, for example,
would plot all \datafile s in the current directory with a {\tt .dat} suffix,
using the same plot options:

\begin{verbatim}
plot '*.dat' with linewidth 2
\end{verbatim}

\noindent In the graph's legend, full filenames are displayed, allowing the
\datafile s to be distinguished.

If a blank filename is supplied to the \indcmdt{plot}, the last used \datafile\
is used again, as in the example:

\begin{verbatim}
plot 'data.dat' using 1:2, '' using 2:3
\end{verbatim}

\noindent This can even be used with wildcards, as in the following example:

\begin{verbatim}
plot '*.dat' using 1:2, '' using 2:3
\end{verbatim}

\subsection{Horizontally arranged \Datafile s}
\label{sec:horizontal_datafiles}
\index{horizontal datafiles}\index{datafiles!horizontal}\index{using rows
modifier@{\tt using rows} modifier}\index{using columns modifier@{\tt using
columns} modifier}

PyXPlot also allows rows of data to be plotted against one another. To do so,
the keyword \indkeyt{rows} is placed after the {\tt using} modifier:

\begin{verbatim}
plot 'data.dat' index 1 using rows 1:2
\end{verbatim}

\noindent For completeness, the syntax {\tt using} \indkeyt{columns} is also
accepted, specifying that columns should be plotted against one another as
happens by default:

\begin{verbatim}
plot 'data.dat' index 1 using columns 1:2
\end{verbatim}

When plotting horizontally-arranged \datafile s, the meanings of the {\tt
index} and {\tt every} modifiers are altered slightly. The former continues to
refer to vertically-displaced blocks of data separated by two blank lines.
Blocks, as referenced in the {\tt every} modifier, likewise continue to refer
to vertically-displaced blocks of \datapoint s, separated by single blank
lines. The row numbers passed to the {\tt using} modifier are counted from the
top of the current block.

However, the line numbers specified in the \indmodt{every} modifier -- i.e.\
variables $a$, $c$ and $e$ in the system introduced in the previous section --
now refer to vertical column numbers. For example,

\begin{verbatim}
plot 'data.dat' using rows 1:2 every 2::3::9
\end{verbatim}

\noindent would plot the data in row~2 against that in row~1, using only the
values in every other column, between columns~3 and~9.

\subsection{Choosing which Data to Plot}
\label{sec:select_modifier}

The final modifier which the {\tt plot} command takes to allow the user to
specify which subset(s) of a \datafile\ should be plotted is \indmodt{select}.
This can be used to plot only those \datapoint s in a \datafile\ which specify
some given criterion, as in the following examples:

\begin{verbatim}
plot 'data.dat' select ($8>5)
plot sin(x) select (($1>0) and ($2>0))
\end{verbatim}

\noindent In the second example, two selection criteria are given, combined
with the logical {\tt and} operator. A full list of all of the operators
recognised by PyXPlot, including logical operators, was given in
Table~\ref{tab:operators_table}.

When plotting using \indpst{lines} to connect the \datapoint s (see
Section~\ref{sec:plotting_styles}), the default behaviour is for the lines not
to be broken if a set of \datapoint s are removed by the {\tt select} modifier.
However, this behaviour is sometimes undesirable.  To cause the plotted line to
break when points are removed the \indmodt{discontinuous}\ modifier is supplied
after the {\tt select} modifier, as in the example

\begin{verbatim}
plot sin(x) select ($2>0) discontinuous
\end{verbatim}

\noindent which plots a set of disconnected peaks from the sine function.

%\example{ex:select}{An example}{
%Here, we have an example of the use of the {\tt select} modifier.
%}

\section{The {\tt replot} Command}
\label{sec:replot}

The \indcmdt{replot} may be used to add more datasets to an existing plot, or
to change its axis ranges. For example, having plotted one \datafile\ using the
command

\begin{verbatim}
plot 'datafile1.dat'
\end{verbatim}

\noindent another can be plotted on the same axes using the command

\begin{verbatim}
replot 'datafile2.dat' using 1:3
\end{verbatim}

\noindent or the ranges of the axes on the original plot can be changed using
the command

\begin{verbatim}
replot [0:1][0:1]
\end{verbatim}

\section{Directing Where Output Goes}
\label{sec:directing_output}

By default, when PyXPlot is used interactively, all plots are displayed on the
screen. It is also possible to produce PostScript output, to be read into other
programs or embedded into \LaTeX\ documents, as well as a variety of other
graphical formats. The \indcmdt{set terminal}\footnote{Gnuplot users should
note that the syntax of the {\tt set terminal} command in PyXPlot is somewhat
different from that used by Gnuplot; see Section~\ref{sec:set_terminal}.} is
used to specify the output format that is required, and the \indcmdt{set
output} is used to specify the file to which output should be directed. For
example,

\begin{verbatim}
set terminal postscript
set output 'myplot.eps'
plot 'datafile.dat'
\end{verbatim}

\noindent would output a PostScript plot of data from the file {\tt datafile.dat} to the file
{\tt myplot.eps}.

The \indcmdt{set terminal} can also be used to configure various output options
within each supported file format.  For example, the following commands would
produce black-and-white or colour output respectively:

\begin{verbatim}
set terminal monochrome
set terminal colour
\end{verbatim}

\noindent The former is useful for preparing plots for black-and-white
publications, the latter for preparing plots for colourful presentations.

Both PostScript and Encapsulated PostScript can be produced. The former is
recommended for producing figures to embed into documents, the latter for plots
which are to be printed without further processing. The {\tt postscript}
terminal produces the latter; the {\tt eps} terminal should be used to produce
the former.  Similarly the {\tt pdf} terminal produces files in the Portable
Document Format (PDF)\index{pdf format} read by Adobe Acrobat\index{Adobe
Acrobat}:

\begin{verbatim}
set terminal postscript
set terminal eps
set terminal pdf
\end{verbatim}

It is also possible to produce plots in the gif, png and jpeg graphic formats,
as follows:

\begin{verbatim}
set terminal gif
set terminal png
set terminal jpg
\end{verbatim}

More than one of the above keywords can be combined on a single line, for
example:

\begin{verbatim}
set terminal postscript colour
set terminal gif monochrome
\end{verbatim}

To return to the default state of displaying plots on screen, the {\tt x11}
terminal should be selected:

\begin{verbatim}
set terminal x11
\end{verbatim}

After changing terminals, the \indcmdt{refresh}\footnote{The effect of the {\tt
refresh} command is very similar to that of the {\tt replot} command with no
arguments. The latter simply repeats the last {\tt plot} command. We will see
in Chapter~\ref{ch:vector_graphics} that the {\tt refresh} command is to be
preferred in the current context because it is applicable to vector graphics as
well as to graphs.} is especially useful; it reproduces the last plot to have
been generated in the newly-selected graphical format.  For more details of the
\indcmdt{set terminal}, including how to produce gif and png images with
transparent backgrounds, see Chapter~\ref{ch:image_formats}.

\section{Setting the Size of Output}

The widths of plots may be set by means of two commands -- {\tt set
size}\indcmd{set size} and {\tt set width}\indcmd{set width}. Both are
equivalent, and should be followed by the desired width measured in
centimetres, for example:

\begin{verbatim}
set width 20
\end{verbatim}

The {\tt set size} command can also be used to set the aspect ratio of plots by
following it with the keyword {\tt ratio}\indcmd{set size ratio}. The number
which follows should be the desired ratio of height to width. The following,
for example, would produce plots three times as high as they are wide:

\begin{verbatim}
set size ratio 3.0
\end{verbatim}

\noindent The command {\tt set size noratio} returns to PyXPlot's default
aspect ratio of the golden ratio, i.e.\ $\left((1+\sqrt{5})/2\right)^{-1}$. The
special command {\tt set size square}\indcmd{set size square} sets the aspect
ratio to unity.

\section{Plotting Styles}
\label{sec:plotting_styles}

By default, data from files are plotted with points and functions are plotted
with lines. However, either kinds of data can be plotted in a variety of ways.
To plot a function with points, for example, the following syntax is used:

\begin{verbatim}
plot sin(x) with points
\end{verbatim}

\noindent The number of points displayed (i.e.\ the number of samples of the
function) can be set as follows\indcmd{set samples}:

\begin{verbatim}
set samples 100
\end{verbatim}

\noindent Likewise, \datafile s can be plotted with a line connecting the
\datapoint s:

\begin{verbatim}
plot 'data.dat' with lines
\end{verbatim}

A variety of other styles are available. The \indpst{linespoints} plot style
combines both the \indpst{points} and \indpst{lines} styles, drawing lines
through points. Error bars can also be drawn as follows:\indps{yerrorbars}

\begin{verbatim}
plot 'data.dat' with yerrorbars
\end{verbatim}

\noindent In this case, three columns of data need to be specified: the $x$-
and $y$-coordinates of each \datapoint, plus the size of the vertical error bar
on that \datapoint. By default, the first three columns of the \datafile\ are
used, but as elsewhere (see Section~\ref{sec:plot_datafiles}), the {\tt using}
modifier can be used:

\begin{verbatim}
plot 'data.dat' using 2:3:7 with yerrorbars
\end{verbatim}

Other plot styles supported by PyXPlot are listed in
Section~\ref{sec:list_of_plotstyles}.  More details of the {\tt errorbars} plot
style can be found in Section~\ref{sec:errorbars}. Bar charts will be discussed
in Section~\ref{sec:barcharts}.

\label{sec:pointtype}
The modifiers \indpst{pointtype} and \indpst{linetype}, which can be
abbreviated to {\tt pt} and {\tt lt} respectively, can also be placed after the
{\tt with} modifier. Each should be followed by an integer.  The former
specifies what shape of points should be used to plot the dataset, and the
latter whether a line should be continuous, dotted, dash-dotted, etc.
Different integers correspond to different styles, and are listed in
Chapter~\ref{ch:linetypes_table}.

The default plotting style, used when none is specified to the {\tt plot}
command, can also be changed.  For example:\indcmd{set style data}

\begin{verbatim}
set style data lines
\end{verbatim}

\noindent would change the default style used for plotting data from files to
{\tt lines}. Similarly, the \indcmdt{set style function} changes the default
style used when functions are plotted.

%\example{ex:errorbars}{An example}{
%Here, we have an example of the errorbars plot style.
%}

\section{Setting Axis Ranges}
\label{sec:plot_ranges}

By default, PyXPlot automatically scales axes to some sensible range which
contains all of the plotted data. However, it is possible for the user to
override this and set his own range.\index{axes!setting ranges} This can be
done directly from the plot command, by following the word {\tt plot} with the
syntax {\tt [minimum:maximum]}.\footnote{An alternative valid syntax is to
replace the colon with the word {\tt to}: {\tt [minimum to maximum]}.} The
first specified range applies to the {\tt x}-axis, and the second to the {\tt
y}-axis.\footnote{As will be discussed in Section~\ref{sec:multiple_axes}, if
further ranges are specified, they apply to the {\tt x2}-axis, then the {\tt
y2}-axis, and so forth.} In the following example, the first three cylindrical
Bessel functions are plotted in the range $0<x<10$:

\begin{verbatim}
plot [0:10][-0.5:1] besselJ(0,x), besselJ(1,x), besselJ(2,x)
\end{verbatim}
\begin{center}
\includegraphics[width=8cm]{examples/eps/ex_intro_bessel}
\end{center}

\noindent Any of the values can be omitted, as in the following plot of
three Legendre polynomials:

\begin{verbatim}
set key xcentre
plot [-1:1][:] legendreP(2,x), legendreP(4,x), legendreP(6,x)
\end{verbatim}
\begin{center}
\includegraphics[width=8cm]{examples/eps/ex_intro_legendre}
\end{center}

\noindent Here, we have used the {\tt set key} command to specify that the
plot's legend should be horizontally aligned in the centre of the plot, to
complement the symmetry of the Legendre polynomials. This command will be
described more fully in Section~\ref{sec:legends}.

Alternatively, ranges can be set before the {\tt plot} statement, using the
\indcmdt{set xrange}, as in the examples:

\begin{verbatim}
set xrange [-2:2]
set yrange [a:b]
\end{verbatim}

If an asterisk is supplied in place of either of the limits in this command, then
any limit which had previously been set is switched off, and the axis returns to
its default autoscaling behaviour:

\begin{verbatim}
set xrange [-2:*]
\end{verbatim}

\noindent A similar effect may be obtained using the \indcmdt{set autoscale},
which takes a list of the axes to which it is to apply. Both the upper and
lower limits of these axes are set to scale automatically. If no list is
supplied, then the command is applied to all axes.

\begin{verbatim}
set autoscale x y
set autoscale
\end{verbatim}

The range supplied to the {\tt set xrange} can be followed by the word {\tt
reverse} to indicate that the axis should run from right-to-left, or from
top-to-bottom. In practice, this is of limited use when an explicit range is
specified, as the following two commands are equivalent:

\begin{verbatim}
set xrange [-2:2] reverse
set xrange [2:-2] noreverse
\end{verbatim}

\noindent However, this is useful when axes are set to autoscale:

\begin{verbatim}
set xrange [*:*] reverse
\end{verbatim}

Axes can be set to have logarithmic scales by using the \indcmdt{set logscale},
which also takes a list of axes to which it should apply. Its converse is
\indcmdts{set nologscale}:

\begin{verbatim}
set logscale
set nologscale y x x2
\end{verbatim}

Further discussion of the configuration of axes can be found in
Section~\ref{sec:multiple_axes}.

\example{ex:axislab}{A diagram of the trajectories of projectiles fired with different initial velocities}{
In this example we produce a diagram of the trajectories of projectiles fired
by a cannon at the origin with different initial velocities $v$ and different
angles of inclination $\theta$. According to the equations of motion under
constant acceleration, the distance of such a projectile from the origin after
time $t$ is given by
\begin{eqnarray*}
x(t) & = & vt\cos\,\theta \\
h(t) & = & vt\sin\,\theta + \nicefrac{1}{2}gt^2 \\
\end{eqnarray*}
where $x(t)$ is the horizontal displacement of the projectile and $h(t)$ the
vertical displacement. Eliminating $t$ from these equation gives the expression
\begin{displaymath}
h(x) = x\tan\,\theta - \frac{gx^2}{2v^2\cos^2\,\theta}.
\end{displaymath}
\nlscf
In the script below, we plot this function for five different values of $v$ and
$\theta$.
\nlscf
{\footnotesize
\noindent {\tt g~~~= 9.81~~~~\# Acceleration due to gravity}\newline
\noindent {\tt d2r~= pi/180~~\# Convert degrees to radians}\newline
\noindent {\tt }\newline}\\{\footnotesize
\noindent {\tt \# The mathematical equation of a trajectory}\newline
\noindent {\tt h(x,theta,v) = x*tan(theta*d2r) - 0.5*g*x**2/(v**2*cos(theta*d2r)**2)}\newline
\noindent {\tt }\newline}\\{\footnotesize
\noindent {\tt \# Plot configuration}\newline
\noindent {\tt set xlabel "\$x/{$\backslash$rm m}\$"}\newline
\noindent {\tt set ylabel "\$h/{$\backslash$rm m}\$"}\newline
\noindent {\tt set xrange [0:20]}\newline
\noindent {\tt set yrange [0:]}\newline
\noindent {\tt set key below}\newline
\noindent {\tt set title 'Trajectories of projectiles fired with speed \$v\$ and angle \$$\backslash$theta\$'}\newline
\noindent {\tt plot h(x,30,10) title "\$$\backslash$theta=30\^{}$\backslash$circ;$\backslash$qquad v=10$\backslash$,\{$\backslash$rm m$\backslash$,s\^{}\{-1\}\}\$", $\backslash$}\newline
\noindent {\tt \phantom{xxxxx}h(x,45,10) title "\$$\backslash$theta=45\^{}$\backslash$circ;$\backslash$qquad v=10$\backslash$,\{$\backslash$rm m$\backslash$,s\^{}\{-1\}\}\$", $\backslash$}\newline
\noindent {\tt \phantom{xxxxx}h(x,60,10) title "\$$\backslash$theta=60\^{}$\backslash$circ;$\backslash$qquad v=10$\backslash$,\{$\backslash$rm m$\backslash$,s\^{}\{-1\}\}\$", $\backslash$}\newline
\noindent {\tt \phantom{xxxxx}h(x,30,15) title "\$$\backslash$theta=30\^{}$\backslash$circ;$\backslash$qquad v=15$\backslash$,\{$\backslash$rm m$\backslash$,s\^{}\{-1\}\}\$", $\backslash$}\newline
\noindent {\tt \phantom{xxxxx}h(x,60,15) title "\$$\backslash$theta=60\^{}$\backslash$circ;$\backslash$qquad v=15$\backslash$,\{$\backslash$rm m$\backslash$,s\^{}\{-1\}\}\$"}
}
\nlscf
The resulting plot is shown below:
\nlscf
\begin{center}
\includegraphics{examples/eps/ex_trajectories}
\end{center}
\nlscf
In the next chapter, we will meet various ways in which this script could be
improved. Here, we have used to numerical constants, {\tt g} being the
acceleration due to gravity, and {\tt d2r} being a factor for converting angles
between degrees and radians.  The former is, in fact, already defined within
PyXPlot as a numerical constant, and the latter conversion can be made using
PyXPlot's automatic ability to convert numbers between different physical
units.
}

\section{Interactive Help}

In addition to this {\it Users' Guide}, PyXPlot also has a \indcmdt{help},
which provides a hierarchical source of information. Typing {\tt help} alone
gives a brief introduction to the help system, as well as a list of topics on
which help is available. To display help on any given topic, type {\tt help}
followed by the name of the topic. For example,

\begin{verbatim}
help datafile
\end{verbatim}

\noindent provides information on the format in which PyXPlot expects to read
\datafile s and

\begin{verbatim}
help plot
\end{verbatim}

\noindent provides information on the {\tt plot} command.  Some topics have
sub-topics, which are listed at the end of each page. To view them, add further
words to the end of your help request -- an example might be

\begin{verbatim}
help set title
\end{verbatim}


% calculations.tex
%
% The documentation in this file is part of Pyxplot <http://www.pyxplot.org.uk>
%
% Copyright (C) 2006-2013 Dominic Ford <coders@pyxplot.org.uk>
%               2008-2013 Ross Church
%
% $Id$
%
% Pyxplot is free software; you can redistribute it and/or modify it under the
% terms of the GNU General Public License as published by the Free Software
% Foundation; either version 2 of the License, or (at your option) any later
% version.
%
% You should have received a copy of the GNU General Public License along with
% Pyxplot; if not, write to the Free Software Foundation, Inc., 51 Franklin
% Street, Fifth Floor, Boston, MA  02110-1301, USA

% ----------------------------------------------------------------------------

% LaTeX source for the Pyxplot Users' Guide

\chapter{Performing calculations}

Often calculations need to be performed on data before they are plotted. This
chapter and the next describe the mathematical environment which Pyxplot
provides.

Most of the examples in this chapter act on single numerical values, displaying
the results using the {\tt print} command, demonstrating how Pyxplot may be
used as a desktop calculator. The next chapter will extend this to use of
Pyxplot's mathematical environment to analyse whole datasets and produce plots.

\section{Variables}

Variables can be assigned to hold numerical values using syntax of the form

\begin{verbatim} a = 5.2 * sqrt(64) \end{verbatim}

\noindent which may optionally be written in longhand as\indcmd{let}

\begin{verbatim} let a = 5.2 * sqrt(64) \end{verbatim}

\noindent Variables can subsequently be used by name in mathematical
expressions, for example:

\begin{verbatim} print a / sqrt(64) \end{verbatim}

\noindent Having been defined, variables can later be undefined -- set to have
no value -- using syntax of the form:

\begin{verbatim} a = \end{verbatim}

Variables can also hold non-numeric data, such as strings, colors, dates, lists
and dictionaries. The syntax for defining many of these data structures is
similar to that used by python, for example:

\begin{verbatim}
myList = [8,2,1,7]
myDict = {'john':27 , 'fred':14 , 'lisa':myList}
myDate = time.fromCalendar(2012,7,1,14,30,0)
\end{verbatim}

\noindent More information about Pyxplot's data types can be found in Chapter~\ref{chap:progDataTypes}.

A list of all of the variables which are currently defined can be obtained by
typing {\tt show variables}\indcmd{show variables}. Some constants are
pre-defined by Pyxplot, and so a number of variables are listed even if none
have been set by the user.

\section{Physical constants} \label{sec:constants} \index{physical
constants}\index{constants}

Many mathematical and physical constants are pre-defined in Pyxplot.  A
complete list of these can be found Chapter~\ref{ch:constants}.  Some of these,
for example, {\tt e}, {\tt pi} and {\tt goldenRatio} are standard mathematical
constants which are accessible in the user's default namespace:

\vspace{3mm}
\input{fragments/tex/calc_pi.tex}
\vspace{3mm}

\noindent Others, such as physical constants, are of more specialist interest
and are defined in modules. For example, the speed of light is defined in the
physics module {\tt phy}:

\vspace{3mm}
\input{fragments/tex/calc_c.tex}
\vspace{3mm}

Most of the pre-defined physical constants, such as this one, make use of
Pyxplot's native ability to keep track of the physical units of quantities and
to convert them between different unit systems -- for example, between inches
and centimeters.  This will be explained in more detail in
Section~\ref{sec:units}.

To list all of the functions and variables defined in a module such as {\tt phy}, type

\begin{verbatim}
print phy
\end{verbatim}

\noindent or simply

\begin{verbatim}
phy
\end{verbatim}


\section{Functions} \label{sec:functions}

Many standard mathematical and operating system functions are pre-defined
within Pyxplot's mathematical environment. These range from everyday examples
like trigonometric functions, to very specialised functions; there is even a
function to return the phase of the Moon on any given day. As with the
mathematical constant, common functions are defined in the user's default
namespace, for example

\vspace{3mm}
\input{fragments/tex/calc_exp.tex}
\vspace{3mm}

\noindent whilst others live in modules, for example

\begin{verbatim}
print ast.moonPhase( time.now() )
\end{verbatim}

\noindent which returns the present phase of the Moon in radians, and

\begin{verbatim}
print os.path.filesize("/etc/passwd")
\end{verbatim}

\noindent which returns the size of a file (in units of bytes, of course!).

A complete list of these functioned, sorted by module, can be found in
Chapter~\ref{ch:function_list}.  Another quick way to find out some more
information about a function is the print the function object, for example:

\vspace{3mm}
\input{fragments/tex/calc_log.tex}
\vspace{3mm}

All, of Pyxplot's built-in constants, functions and modules are
contained in the module {\tt defaults}, which can also be printed to view its
contents:

\begin{verbatim}
print defaults
\end{verbatim}

\noindent It is possible to access {\tt pi}, for example, as {\tt defaults.pi},
though in practice this syntax is very rarely needed. All of the objects in the
{\tt defaults} module are always accessible by name (i.e.\ they are always in
any namespace), unless another local or global variable exists with the same
name.

The user can define his own algebraic function definitions using a similar
syntax to that used to declare new variables, as in the examples:

\begin{verbatim}
f()    = pi
g(x)   = x*sin(x)
h(x,y) = x*y
\end{verbatim}

\noindent Function objects are just like any other variables, and can even be
used as arguments to other functions:

\vspace{3mm}
\input{fragments/tex/calc_funcnest.tex}
\vspace{3mm}

User-defined functions can be undefined in the same way as any other variable,
for example by typing:

\begin{verbatim}
f =
\end{verbatim}

Where the logic required to define a particular function is greater than can be
contained in a single algebraic expression, a subroutine should be used (see
Section~\ref{sec:subroutines}); these allow an arbitrary numbers of lines of
Pyxplot code to be executed whenever a function is evaluated.

\subsection{Spliced functions} \index{function splicing} \index{splicing
functions}

The definitions of functions can be declared to be valid only within a certain
domain of argument space, allowing for error checking when models are evaluated
outside their domain of applicability. Furthermore, functions can be given
multiple definitions which are specified to be valid in different parts of
argument space. We term this {\it function splicing}, since multiple algebraic
definitions for a function are spliced together at the boundaries between
their various domains.  The following example would define a function which is
only valid within the range
$-\nicefrac{\pi}{2}<x<\nicefrac{\pi}{2}$:\footnote{The syntax {\tt
[-pi/2:pi/2]} can also be written {\tt [-pi/2 to pi/2]}.}

\begin{verbatim}
truncated_cos(x)[-pi/2:pi/2] = cos(x)
\end{verbatim}

\noindent Attempts to evaluate this function outside of the range in which it
is defined would return an error that the function is not defined at the
requested argument value. Thus, if the above function were to be plotted, no
line would be drawn outside of the range
$-\nicefrac{\pi}{2}<x<\nicefrac{\pi}{2}$. A similar effect could also have been
achieved using the {\tt select} keyword (see
Section~\ref{sec:select_modifier}). Sometimes, however, the desired behaviour
is rather that the function should be zero outside of some region of parameter
space where it has a finite value. This can be achieved as in the following
example:

\begin{verbatim}
f(x) = 0
f(x)[-pi/2:pi/2] = cos(x)
\end{verbatim}

\noindent Plotting this function would yield the following result:

\begin{center}
\includegraphics[width=8cm]{examples/eps/ex_intro_func_splice}
\end{center}

\noindent To produce this function, we have made use of the fact that if there
is an overlap in the domains of validity of multiple definitions of a function,
then later declarations are guaranteed take precedence. The definition that the
function equals zero is valid everywhere, but is overridden in the region
$-\nicefrac{\pi}{2}<x<\nicefrac{\pi}{2}$ by the second function definition.

Where functions have been spliced together, the {\tt show functions} command
will show all of the definitions of the spliced function, together with the
regions of parameter space in which they are used. This is indicated using the
same syntax that is used for defining spliced functions, such that the output can
be stored and pasted into a future Pyxplot session to redefine exactly the same
spliced function.

When a function takes more than one argument, multiple ranges can be specified,
one for each argument. Any of the limits can be left blank if there is no
upper or lower limit upon the value of that particular argument. In the
following example, the function {\tt f(a,b,c)} would only be defined when all
of {\tt a}, {\tt b} and {\tt c} were in the range $-1 \to 1$:

\begin{verbatim}
f(a,b,c)[-1:1][-1:1][-1:1] = a+b+c
\end{verbatim}

Function splicing can be used to define functions which do not have analytic
forms, or which are, by definition, discontinuous, such as top-hat functions or
Heaviside functions. The following example would define $f(x)$ to be a
Heaviside function:

\begin{verbatim}
f(x) = 0
f(x)[0:] = 1
\end{verbatim}

Similar effects may also be achieved using the ternary conditional {\tt ?:}
operator (see Section~\ref{sec:conditional_operator}), for example:

\begin{verbatim}
f(x) = (x>0) ? 1 : 0
\end{verbatim}

\example{ex:funcsplice2}{Modelling a physics problem using a spliced function}{
\noindent{\bf Question}\newline\noindent
A light bead is free to move from side to side between two walls which are
placed at $x=-2l$ and $x=2l$. It is connected to each wall by a light elastic
string of natural length $l$, which applies a force $k\Updelta x$ when extended
by an amount $\Updelta x$, but which applies no force when slack. What is the
total horizontal force on the bead as a function of its horizontal position $x$?
\nlscf
\noindent{\bf Answer}\newline\noindent
This system has three distinct regimes. In the region $-l<x<l$, both strings
are under tension. When $x<-l$, the left-hand string is slack, and only the
right-hand string exerts a force. When $x>l$, the converse is true: only the
left-hand string exerts a force. The case $|x|>2l$ is not possible, as the bead
would have to penetrate the hard walls. It is left as an exercise for the
reader to use Hooke's Law to derive the following expression, but in summary,
the force on the bead can be defined in Pyxplot as:
\nlscf
\input{examples/tex/ex_funcsplice2_1.tex}
\nlscf
\noindent where it is necessary to first define a value for {\tt l} and {\tt
k}. Plotting these functions yields the result:
\nlscf
\begin{center}
\includegraphics[width=\textwidth]{examples/eps/ex_funcsplice2}
\end{center}
\nlscf
Attempting to plot this function with an {\tt x}-axis which extends outside of
the range of values of $x$ for which $F(x)$ is defined, as above, will result
in error messages being returned that the function could not be evaluated at
all argument values. These can be suppressed by typing (see
Section~\ref{sec:num_errs})\vspace{2mm}
\newline\noindent {\tt set numeric errors quiet}
}

\example{ex:funcsplice}{Using a spliced function to calculate the Fibonacci numbers}{
The Fibonacci numbers are defined to be the sequence of numbers in which each
member is the sum of its two immediate predecessors, and the first three
members of the sequence are ${0,1,1}$. Thus, the sequence runs
${0,1,1,2,3,5,8,13,21,34,55,...}$. In this example, we use function splicing to
calculate the Fibonacci sequence in an iterative and highly inefficient way,
hard-coding the first three members of the sequence and then using the
knowledge that all of the subsequent members are the sums of their two
immediate predecessors:

\nlscf
\input{examples/tex/ex_funcsplice_1.tex}
\nlfcf
This method is highly inefficient because each evaluation spawns two further
evaluations of the function, and so the number of operations required to
evaluate {\tt f(x)} scales as $2^x$.  It is inadvisable to evaluate it for
$x\gtrsim25$ unless you're prepared for a long wait.
\nlnp
A much more efficient method of calculating the Fibonacci numbers is to use Binet's formula,
\begin{displaymath}
f(x) = {\psi^x - (1-\psi)^x}{\sqrt{5}},
\end{displaymath}
where $\psi=1+\sqrt{5}/2$ is the golden ratio, which provides an analytic
expression for the sequence.  In the following script, we compare the values
returned by these two implementations. We enable complex arithmetic as Binet's
formula returns complex numbers for non-integer values of $x$.
\nlscf
\input{examples/tex/ex_funcsplice_2.tex}
\nlscf
\begin{center}
\includegraphics[width=\textwidth]{examples/eps/ex_funcsplice}
\end{center}
}

\section{Handling numerical errors}
\label{sec:num_errs}
\index{numerical errors}

By default, an error message is returned whenever calculations return values
which are infinite, as in the case of {\tt 1/0}, or when functions are
evaluated outside the domain of parameter space in which they are defined, as
in the case of {\tt besseli(-1,1)}.  Sometimes this behaviour is desirable: it
flags up to the user that a calculation has gone wrong, and exactly what the
problem is.  At other times, however, these error messages can be undesirable
and may lead you to miss more genuine and serious errors buried in their midst.

For this reason, the issuing of explicit error messages when calculations
return non-finite numeric results can be switched off by typing:
\indcmd{set numeric errors quiet}

\begin{verbatim}
set numeric errors quiet
\end{verbatim}

\noindent Having done this, expressions such as

\begin{verbatim}
x = besseli(-1,1)
\end{verbatim}

\noindent fail silently, and variables which contain non-finite numeric results
are displayed as {\tt NaN}\index{NaN}, which stands for {\it Not a
Number}\index{not a number}.  The issuing of explicit errors may subsequently
be re-enabled by typing: \indcmd{set numeric errors explicit}

\begin{verbatim}
set numeric errors explicit
\end{verbatim}

Having turned off the display of numerical errors, it may be useful to use the
\indcmdt{assert} to throw an error message if a calculation has failed in an
unrecoverable way that the user really ought to know about:

\begin{verbatim}
assert x>0 "Cannot continue with negative x"
\end{verbatim}

The {\tt assert} command should be followed by an algebraic expression which
must be true for execution to continue. If it is false, an error results.
Optionally, an error message can be included, as above, to tell the user what
the problem is.

\section{Working with complex numbers}
\label{sec:complex_numbers}
\index{complex numbers}

In all of the examples given thus far, algebraic expressions have only been
allowed to return real numbers: Pyxplot has not been handling any complex
numbers. Since there are many circumstances in which the data being analysed
may be known for certain to be real, complex arithmetic is disabled in Pyxplot
by default.  Expressions such as {\tt sqrt(-1)} will return either an error or
{\tt NaN}.  The most obvious example of this is the built-in variable {\tt i},
which is set to equal {\tt sqrt(-1)}:

\vspace{3mm}
\noindent\texttt{pyxplot> \textbf{print i}}\newline
\noindent\texttt{nan}
\vspace{3mm}

Complex arithmetic may be enabled by typing
\indcmd{set numeric complex}

\begin{verbatim}
set numeric complex
\end{verbatim}

\noindent and then disabled again by typing
\indcmd{set numeric real}

\begin{verbatim}
set numeric real
\end{verbatim}

\noindent Once complex arithmetic has been enabled, many of Pyxplot's built-in
mathematical functions accept complex input arguments, including the logarithm
function, all of the trigonometric functions, and the exponential function.  A
complete list of functions which accept complex inputs can be found in
Appendix~\ref{ch:function_list}.

Complex number literals can be entered into algebraic expressions in either of
the following two forms:

\begin{verbatim}
print (2 + 3*i       )
print (2 + 3*sqrt(-1))
\end{verbatim}

\noindent The former version depends upon the pre-defined system variable {\tt
i} being defined to equal $\sqrt{-1}$. The user could cause this to stop working,
of course, by re-defining this variable to have a different value.  However, in
this case the variable {\tt i} could straightforwardly be returned to its
default value by typing:

\begin{verbatim}
i=sqrt(-1)
\end{verbatim}

\noindent The user can, of course, define any other variable to equal
$\sqrt{-1}$, thus allowing him to use any other letter, e.g.\ {\tt j}, to
represent the imaginary component of a number.

Several built-in functions are provided for manipulating complex numbers. The
\indfunt{Re(z)} and \indfunt{Im(z)} functions return respectively the real and
imaginary parts of a complex number $z$. The \indfunt{arg(z)} function returns
the complex argument of $z$. And the \indfunt{abs(z)} function returns the
modulus of $z$.  The \indfunt{conjugate(z)} command returns the complex
conjugate of $z$. The following lines of code demonstrate the use of these
functions:

\vspace{3mm}
\input{fragments/tex/calc_complex.tex}
\vspace{3mm}

\section{Working with physical units}
\label{sec:units}
\index{physical units}\index{units}

Pyxplot, and all its mathematical functions, have native support for numbers to
have physical units. This means, for example, that multiplying two lengths
yields an area, and taking passing a selection of lengths to the {\tt max(...)}
function returns the longest of the lengths supplied.

This makes it a powerful desktop tool for converting measurements between
different systems of units -- for example, between imperial and metric units --
or for doing physical calculations.

The special function \indfunt{unit()} is used to specify the physical unit
associated with a quantity. For example, the expression

\begin{verbatim}
print 2*unit(s)
\end{verbatim}

\noindent takes the number~2 and multiplies it by the unit {\tt s}, which is
the SI abbreviation for seconds. Technically, the function {\tt unit(s)}
returns a numeric object equal to one second.

The resulting quantity above, which has dimensions of time, could then, for
example, be divided by the unit {\tt hr} to find the dimensionless number of
hours in two seconds:

\begin{verbatim}
print 2*unit(s)/unit(hr)
\end{verbatim}

Compound units such as miles per hour, which is defined in terms of two other
units, can be used as in

\begin{verbatim}
print 2*unit(miles/hour)
\end{verbatim}

\noindent In many cases, commonly-used units such units have their own explicit
abbreviations, in this case {\tt mph}:

\begin{verbatim}
print 2*unit(mph)
\end{verbatim}

\noindent As these examples demonstrate, the {\tt unit()} function can be
passed a string of units either multiplied together with the {\tt *} operator,
or divided using the {\tt /} operator. Units may be raised to powers with the
{\tt **} operator\footnote{The {\tt \^{}} character may be used as an alias for
the {\tt **} operator, though this notation is arguably confusing, since the
same character is used for the binary exclusive or operator in Pyxplot's normal
arithmetic.}, as in the example:

\vspace{3mm}
\input{fragments/tex/calc_units.tex}
\vspace{3mm}

\noindent As the examples above have demonstrated, units may be referred to by
either their abbreviated or full names, and each of these may be used in either
their singular or plural forms. For example, {\tt s}, {\tt second}, and {\tt
seconds} are all valid and equivalent. A complete list of all of the units
which Pyxplot recognises by default, together with all of their recognised
names, can be found in Appendix~\ref{ch:unit_list}.

SI units may be preceded with SI prefixes\index{units!SI prefixes}, as in
the examples\footnote{As the first of these examples demonstrates, the letter
{\tt u} is used as a Roman-alphabet substitute for the Greek letter $\upmu$.}:

\begin{verbatim}
a = 2*unit(um)
a = 2*unit(micrometers)
\end{verbatim}

When quantities with physical units are substituted into algebraic expressions,
Pyxplot automatically checks that the expression is dimensionally correct
before evaluating it. For example, the following expression is not
dimensionally correct and would return an error because the first term in the
sum has dimensions of velocity, whereas the second term is a length:
\index{units!dimensional analysis}

\begin{dontdo}
a = 2*unit(m)\newline
b = 4*unit(s)\newline
print a/b + a
\end{dontdo}

\noindent Pyxplot continues to throw an error in this case, even when explicit
numerical errors are turned off with the \indcmdt{set numeric errors quiet},
since it is deemed a serious error: the above expression would never be correct
for any values of {\tt a} and {\tt b} given their dimensions.

A large number of units are pre-defined in Pyxplot by default, a complete list
of which can be found in Appendix~\ref{ch:unit_list}.  However, the need may
occasionally arise to define new units. It is not possible to do this from an
interactive Pyxplot terminal, but it is possible to do so from a configuration
script which Pyxplot runs upon start-up. Such configuration scripts will be
discussed in Chapter~\ref{ch:configuration}. New units may either be derived
from existing SI units, alternative measures of existing quantities, or
entirely new base units such as numbers of CPU cycles or man-hours of labour.

\subsection{Treatment of angles in Pyxplot}
\label{sec:angles}
\index{units!angle}\index{angles, handling of}

There are a small number of cases where Pyxplot's handling of units does not
completely follow normal (i.e.\ SI) conventions, which require further
explanation. The most obvious such case is its handling of angles.

By convention, the SI system of units does not have a base unit of angle:
instead, the radian is considered to be a dimensionless unit.  There are some
strong mathematical reasons why this makes sense, since it makes it possible to
write equations such as
\begin{displaymath}
d=\theta r
\end{displaymath}
and
\begin{displaymath}
x = \exp(a+i\theta),
\end{displaymath}
which would otherwise have to be written as, for example,
\begin{displaymath}
d=2\pi\left(\frac{\theta}{2\pi\,\mathrm{rad}}\right) r=\left(\frac{\theta}{\mathrm{rad}}\right) r
\end{displaymath}
in order to be strictly dimensionally correct.

However, it also has some disadvantages since some physical quantities such as
fluxes per steradian are measured per unit angle or per unit solid angle, and
the SI system traditionally\footnote{Radians are sometimes treated in the SI
system as {\it supplementary} or derived units.} offers no way to dimensionally
distinguish these from one another or from quantities with no angular
dependence.  In addition, many of Pyxplot's vector graphics commands take
rotation angles as inputs, and it is useful to express these in units of angle.

In most cases, the user is free to decide whether angles should have units. All
of the following print statements are equivalent:

\begin{verbatim}
print sin(pi)
print sin(180*unit(deg))
print sin(pi *unit(rad))
print sin(0.5*unit(rev))
\end{verbatim}

However, it is useful to be able to define whether inverse trigonometric
functions such as {\tt asin(x)} and {\tt atan(x)} return results with units of
angle, or which are dimensionless. By default, these functions return
dimensionless results, but this may be changed using the commands:

\begin{verbatim}
set unit angle dimensionless
set unit angle nodimensionless
\end{verbatim}

\noindent Note that even when inverse trigonometric functions are set to return
dimensionless outputs, expressions such as {\tt unit(rad)+1} are still
dimensionally incorrect. Functions such as {\tt sin(x)} and {\tt exp(x)}
can always accept inputs which are either dimensionless, or have units of
angle.

\subsection{Converting between different temperature scales}
\index{temperature conversions}\index{units!temperature}

Pyxplot can convert temperatures between different temperature scales, for
example between $^\circ\mathrm{C}$, $^\circ\mathrm{F}$ and K.  However, these
conversions have some subtleties which are unique to temperature conversions.
This means they should be used with some caution.

Consider the following two questions:
\begin{itemize}
\item How many Kelvin corresponds to a temperature of $20^\circ$C?
\item How many Kelvin corresponds to a temperature {\it rise} of $20^\circ$C?
\end{itemize}
The answers to these two questions are 293\,K and 20\,K respectively: although
we are converting from $20^\circ$C in both cases, the corresponding number of
Kelvin depends whether we are talking about an {\it absolute} temperature
or a {\it relative} temperature. A heat capacity of 1\,J/$^\circ$C equals
1\,J/K, even though a temperature of $1^\circ$C does not equal a temperature of
1\,K.

The cause of this problem, and the reason why it rarely affects any physical
units other than temperatures is that there exists such a thing as absolute
temperature.

Take the example of distances. Distances are almost always relative: they
measure distance gaps between points. Occasionally people might choose to
express positions as distance from some particular origin. But if scheme A
involved measuring in meters from New York, and scheme B involved measuring in
feet from Chicago, they wouldn't expect Pyxplot to convert between the two
systems.

The problem of converting between temperature systems is just like this. One
system measures distance in degrees Fahrenheit away from 0$^\circ\mathrm{F}$;
another the distance in degrees Celsius away from
0$^\circ\mathrm{C}$.\footnote{There is one other common example of this
problem. Times are expressed as absolute quantities when dates are written
down. The year {\footnotesize AD}\,1453 implicitly corresponds to a period of
1453 years after the Christian epoch. So, similar problems arise when trying to
convert such a year into the Muslim calendar, which counts from the year
{\footnotesize AD}\,622. Pyxplot can, incidentally, make this conversion, using
date objects, as will be seen in Section~\ref{sec:time_series}.}

As Pyxplot cannot distinguish between absolute and relative temperatures, it
takes a safe approach of performing algebra consistently with any unit of
temperature, never performing automatic conversions between different
temperature scales. A calculation based on temperatures measured in
$^\circ\mathrm{F}$ will produce an answer measured in $^\circ\mathrm{F}$.
However, as converting temperatures between temperature scales is a useful task
which is often wanted, this is allowed, when specifically requested, in the
specific case of dividing one temperature by another unit of temperature to get
a dimensionless number, as in the following example:

\begin{dodo}
print 98*unit(oF) / unit(oC)
\end{dodo}

\noindent Note that the two units of temperature must be placed in separate
{\tt unit(...)} functions. The following is not allowed:

\begin{dontdo}
print 98*unit(oF / oC)
\end{dontdo}

Note that such a conversion always assumes that the temperatures supplied are
{\bf absolute} temperatures. Pyxplot has no facility for converting relative
temperatures between different scales. This must be done manually.

The conversion of derived units of temperature, such as $\mathrm{J}/\mathrm{K}$
or $^\circ\mathrm{C}^2$, to derived units of other temperature scales, such as
$\mathrm{J}/^\circ\mathrm{F}$ or $\mathrm{K}^2$, is not permitted, since in
general these conversions are ill-defined.

The moral of this story is: pick what unit of temperature you want to work in,
convert all of your temperatures to that scale, and then stick to it.

\example{ex:temperature}{Creating a simple temperature conversion scale}{
In this example, we use Pyxplot's automatic conversion of physical units to
create a temperature conversion scale.
\nlscf
\input{examples/tex/ex_tempscale_1.tex}
\nlscf
\begin{center}
\includegraphics{examples/eps/ex_tempscale}
\end{center}
}

\section{Configuring how numbers are displayed}
\label{sec:unitdisp}

\subsection{Display of physical units}

When displaying quantities that have physical units, Pyxplot searches through
its database of known units looking for the most appropriate unit, or
combination of units, to use.  By default, SI units, or combinations of SI
units, are chosen for preference, and SI prefixes such as milli- or kilo- are
applied where appropriate. This behaviour can, however, be extensively
configured.

The most general configuration option allows one of several {\it units
schemes}\index{units!unit schemes} to be selected, each of which comprises a
list of units which are deemed to be members of the particular scheme. For
example, in the CGS unit scheme\index{CGS units}\index{units!CGS}, all lengths
are displayed in centimeters, all masses are displayed in grammes, all energies
are displayed in ergs, and so forth.  In the imperial unit
scheme\index{imperial units}\index{units!imperial}, quantities are displayed in
British imperial units -- inches, pounds, pints, and so forth. In the US unit
scheme, US customary units are used. The current unit scheme can be changed
using the \indcmdt{set unit scheme}:

\vspace{3mm}
\input{fragments/tex/calc_numdisp.tex}
\vspace{3mm}

\noindent A complete list of Pyxplot's unit schemes can be found in
Table~\ref{tab:unit_schemes}.\index{natural units}\index{units!natural}

\begin{table}
\begin{center}
\begin{tabular}{|>{\columncolor{LightGrey}}l>{\columncolor{LightGrey}}p{9cm}|}
\hline
{\bf Name} & {\bf Description} \\
\hline
{\tt ancient} & Ancient units, especially those used in the Authorised Version of the Bible. \\
{\tt CGS} & CGS units. \\
{\tt Imperial} & British imperial units. \\
{\tt Planck} & Planck units, also known as natural units, which make several physical constants equal unity. \\
{\tt SI} & SI units. \\
{\tt US} & US customary units. \\
\hline
\end{tabular}
\end{center}
\caption{A list of Pyxplot's unit schemes.}
\label{tab:unit_schemes}
\end{table}

These units schemes are often sufficient to ensure that most quantities are
displayed in the desired units, but commonly there are a few specific
quantities in any particular piece of work where non-standard units are used.
For example, a study of Jupiter-like planets might express masses in Jupiter
masses, rather than kilograms. A study of the luminosities of stars might
express powers in units of solar luminosities, rather than watts. And a
cosmology paper might express distances in megaparsecs. This level of control is
made available through the \indcmdt{set unit of}. The three examples just given
could be achieved using the following commands:
\begin{verbatim}
set unit of mass Mjupiter
set unit of power solar_luminosity
set unit of length parsec
\end{verbatim}

An astronomer wishing to express masses in Pluto masses would need to first
define the Pluto mass as a user-defined unit, since it is not pre-defined unit
within Pyxplot. In Chapter~\ref{ch:configuration}, we shall see how to define
new units in a configuration script. Having done so, the following syntax would
be allowed:
\begin{verbatim}
set unit of mass Mpluto
\end{verbatim}

The \indcmdt{set unit preferred} offers a slightly more flexible way of
achieving the same result. Whereas the \indcmdt{set unit of} can only operate
on named quantities such as lengths, areas and powers, and cannot act upon
compound units such as {\tt W/Hz}, the \indcmdt{set unit preferred} can act
upon any unit or combination of units:
\begin{verbatim}
set unit preferred parsec
set unit preferred W/Hz
set unit preferred N*m
\end{verbatim}
The latter two examples are particularly useful when working with spectral
densities (powers per unit frequency) or torques (forces multiplied by
distances). Unfortunately, both of these units are dimensionally equal to
energies, and so are displayed by Pyxplot in joules by default. The above
statement overrides such behaviour. Having set a particular unit to be
preferred, this can be unset as in the following example:
\begin{verbatim}
set unit nopreferred parsec
\end{verbatim}

By default, units are displayed in their abbreviated forms, for example {\tt A}
instead of {\tt amperes} and {\tt W} instead of {\tt watts}. Furthermore, SI
prefixes such as milli- and kilo- are applied to SI units where they are
appropriate.\index{SI prefixes}\index{units!SI prefixes} Both of these
behaviours can be turned on or off, in the former case with the commands

\begin{verbatim}
set unit display abbreviated
set unit display full
\end{verbatim}

\noindent and in the latter case using the following pair of commands:

\begin{verbatim}
set unit display prefix
set unit display noprefix
\end{verbatim}

\subsection{Changing the accuracy to which numbers are displayed}

By default, when numbers are displayed, they are printed accurate to eight
significant figures, although fewer figures may actually be displayed if the
final digits are zeros or nines.

This is generally a helpful convention: Pyxplot's internal arithmetic is
generally accurate to around 16 significant figures, and so it is quite
conceivable that a calculation which is supposed to return, say, $1$, may in
fact return 0.999\,999\,999\,999\,999\,9. Likewise, when complex arithmetic is
enabled, routines which are expected to return real numbers may in fact return
results with imaginary parts at the level of one part in $10^{16}$.  By
displaying numbers to only eight significant figures in such cases, the user is
usually shown the `right' answer, instead of a noisy and unattractive one.

However, there may also be cases where more accuracy is desirable, in which
case, the number of significant figures to which output is displayed can be set
using the command\indcmd{set numerics sigfig}

\begin{verbatim}
n = 12
set numerics sigfig n
\end{verbatim}

\noindent where {\tt n} can be any number in the range 1-30. It should be noted
that the number supplied is the {\it minimum} number of significant figures to
which numbers are displayed; on occasion an extra figure may be displayed.

Alternatively, the string substitution operator, described in
Section~\ref{sec:stringsubop} may be used to specify how a number should be
displayed on a one-by-one basis, as in the examples:

\vspace{3mm}
\input{fragments/tex/calc_numsf.tex}

\subsection{Creating pastable text}
\label{sec:pastable}

Pyxplot's default convention of displaying numbers in a format such as

\begin{verbatim}
(2+3i) meters
\end{verbatim}

\noindent is well-suited for creating text which is readable by human users, but
is less well-suited for creating text which can be copied and pasted into
another calculation in another Pyxplot terminal, or for creating text which
could be used in a \latexdcf\ text label on a plot. For this reason, the
\indcmdt{set numerics display} allows the user to choose between three
different ways in which numbers can be displayed:

\vspace{3mm}
\input{fragments/tex/calc_numtype.tex}
\vspace{3mm}

The first case is the default way in which Pyxplot displays numbers. The second
case produces text which forms a valid algebraic expression which could be
pasted into another Pyxplot calculation. The final case produces a string of
\latexdcf\ text which could be used as a label on a plot.

\section{Numerical integration and differentiation}

\index{differentiation}\index{integration} Two special functions,
\indfunt{int\_dx()} and \indfunt{diff\_dx()}, may be used to integrate or
differentiate algebraic expressions numerically.  In each case, the letter {\tt
x} is the dummy variable which is to be used in the integration or
differentiation and may be replaced by any valid variable name of up to
16~characters in length.

The function {\tt int\_dx()} takes three parameters -- firstly the expression
to be integrated, which may optionally be placed in quotes, followed by the
minimum and maximum integration limits. These may have any physical dimensions,
so long as they match, but must both be real numbers. For example, the
following would plot the integral of the function $\sin(x)$:

\begin{verbatim}
plot int_dt('sin(t)',0,x)
\end{verbatim}

The function {\tt diff\_dx()} takes two obligatory parameters plus one further
optional parameter. The first is the expression to be differentiated, which,
as above, may optionally placed in quotes for clarity. This should be followed
by the numerical value $x$ of the dummy variable at the point where the
expression is to be differentiated. This value may have any physical
dimensions, and may be a complex number if complex arithmetic is enabled. The
final, optional, parameter to the {\tt diff\_dx()} function is an approximate
step size, which indicates the range of argument values over which Pyxplot
should take samples to determine the gradient. If no value is supplied, a value
of $10^{-6}x$ is used, replaced by $10^{-6}$ if $x=0$.  The following example
would evaluate the differential of the function $\cos(x)$ with respect to $x$
at $x=1.0$:

\begin{dodo}
print diff\_dx('cos(x)', 1.0)
\end{dodo}

When complex arithmetic is enabled, Pyxplot checks that the function being
differentiated satisfies the Cauchy-Riemann equations, and returns an error if
it does not, to indicate that it is not differentiable.  The following is an
example of a function which is not differentiable, and which throws an error
because the Cauchy-Riemann equations are not satisfied:

\begin{dontdo}
set num comp\newline
print diff\_dx(Re(sin(x)),1)
\end{dontdo}

Advanced users may be interested to know that \indfunt{int\_dx()} function is
implemented using the {\tt gsl\_\-integration\_\-qags()} function of the Gnu
Scientific Library\index{GSL} (GSL), and the \indfunt{diff\_dx()} function is
implemented using the {\tt gsl\_\-deriv\_\-central()} function of the same library.
Any caveats which apply to the use of these routines also apply to Pyxplot's
numerical calculus.

\example{ex:calculus}{Integrating the function $\mathrm{sinc}(x)$}{
The function $\mathrm{sinc}(x)$ cannot be integrated analytically, but it can be shown that
\begin{displaymath}
\int_0^{\pm\infty} \mathrm{sinc}(x)\,\mathrm{d}x = \pm\pi/2 .
\end{displaymath}
In the following script, we use Pyxplot's facilities for numerical integration to produce a plot of
\begin{displaymath}
y=\int_0^{x} \mathrm{sinc}(x)\,\mathrm{d}x .
\end{displaymath}
We reduce the number of samples taken along the abscissa axis to~80, as
evaluation of the numerical integral may be time consuming on older computers.
We use the \indcmdt{set xformat} (see Section~\ref{sec:set_xformat}) to demark
both the {\tt x}- and {\tt y}-axes in fractions of $\pi$:
\nlscf
\input{examples/tex/ex_integration_1.tex}
\nlscf
\centerline{\includegraphics[width=\textwidth]{examples/eps/ex_integration}}
}

\section{Solving systems of equations}

The \indcmdt{solve} can be used to solve systems of one or more simultaneous
equations numerically. It takes as its arguments a comma-separated list of the
equations which are to be solved, and a comma-separated list of the variables
which are to be found. The latter should be prefixed by the word {\tt
via}, to separate it from the list of equations:

\begin{verbatim}
solve <equation 1>, ... via <variable 1>, ...
\end{verbatim}

Note that the time taken by the solver dramatically increases with the number
of variables which are simultaneously found, whereas the accuracy achieved
simultaneously decreases. The following example solves a simple pair of
simultaneous equations of two variables:

\vspace{3mm}
\input{fragments/tex/calc_solve1.tex}
\vspace{3mm}

\noindent No output is returned to the terminal if the numerical solver
succeeds, otherwise an error message is displayed. If any of the fitting
variables are already defined prior to the {\tt solve} command's being called,
their values are used as initial guesses, otherwise an initial guess of unity
for each fitting variable is assumed. Thus, the same \indcmdt{solve} returns
two different values in the following two cases:

\vspace{3mm}
\input{fragments/tex/calc_solve2.tex}
\vspace{3mm}

\noindent In cases where any of the variables being solved for are not
dimensionless, it is essential that an initial guess with appropriate units be
supplied, otherwise the solver will try and fail to solve the system of
equations using dimensionless values:

\begin{dontdo}
x =\newline
y = 5*unit(km)\newline
solve x=y via x
\end{dontdo}

\begin{dodo}
x = unit(m)\newline
y = 5*unit(km)\newline
solve x=y via x
\end{dodo}

The \indcmdt{solve} works by minimising the squares of the residuals of all of the
equations supplied, and so even when no exact solution can be found, the best
compromise is returned. The following example has no solution -- a system of
three equations with two variables is over-constrained -- but Pyxplot
nonetheless finds a compromise solution:

\vspace{3mm}
\input{fragments/tex/calc_solve3.tex}
\vspace{3mm}

When complex arithmetic is enabled, the \indcmdt{solve} allows each of the
variables being fitted to take any value in the complex plane, and thus the
number of dimensions of the fitting problem is effectively doubled -- the real
and imaginary components of each variable are solved for separately -- as in
the following example:

\vspace{3mm}
\input{fragments/tex/calc_solve4.tex}
\vspace{3mm}

\section{Searching for minima and maxima of functions}

The \indcmd{minimize}\indcmd{maximize} {\tt minimize} and {\tt maximize}
commands can be used to find the minima or maxima of algebraic expressions. In
each case, a single algebraic expression should be supplied for optimisation,
together with a comma-separated list of the variables with respect to which it
should be optimised. In the following example, a minimum of the sinusoidal
function $\cos(x)$ is sought:

\vspace{3mm}
\input{fragments/tex/calc_min1.tex}
\vspace{3mm}

\noindent Note that this particular example doesn't work when complex
arithmetic is enabled, since $\cos(x)$ diverges to $-\infty$ at $x=\pi+\infty
i$.

Various caveats apply both to the {\tt minimize} and {\tt maximize} commands,
as well as to the {\tt solve} command.  All of these commands operate by
searching numerically for optimal sets of input parameters to meet the criteria
set by the user. As with all numerical algorithms, there is no guarantee that
the {\it locally} optimum solutions returned are the {\it globally} optimum
solutions. It is always advisable to double-check that the answers returned
agree with common sense.

These commands can often find solutions to equations when these solutions are
either very large or very small, but they usually work best when the solution
they are looking for is roughly of order unity.  Pyxplot does have mechanisms
which attempt to correct cases where the supplied initial guess turns out to be
many orders of magnitude different from the true solution, but it cannot be
guaranteed not to wildly overshoot and produce unexpected results in such
cases.  To reiterate, it is always advisable to double-check that the answers
returned agree with common sense.

\example{ex:eqnsolve}{Finding the maximum of a blackbody curve}{
When a surface is heated to any given temperature $T$, it radiates thermally.
The amount of electromagnetic radiation emitted at any particular frequency,
per unit area of surface, per unit frequency of light, is given by the Planck
Law:
\begin{displaymath}
B_\nu(\nu,T)=\left(\frac{2h^3}{c^2}\right)\frac{\nu^3}{\exp(h\nu/kT)-1}
\end{displaymath}
The visible surface of the Sun has a temperature of approximately
$5800\,\mathrm{K}$ and radiates in such a fashion. In this example, we use the
{\tt solve}, {\tt minimize} and {\tt maximize} commands to locate the frequency
of light at which it emits the most energy per unit frequency interval.  This
task is simplified as Pyxplot has a system-defined mathematical function {\tt
Bv(nu,T)} which evaluates the expression given above.
\nlnp
Below, a plot is shown of the Planck Law for $T=5800\,\mathrm{K}$ to aid in
visualising the solution to this problem:
\nlscf
\begin{center}
\includegraphics[width=\textwidth]{examples/eps/ex_eqnsolve}
\end{center}
\nlnp
To search for the maximum of this function using the \indcmdt{maximize}, we
must provide an initial guess to indicate that the answer sought should have
units of Hz:
\nlscf
\input{fragments/tex/calc_min2.tex}
\nlnp
This maximum could also be sought be searching for turning points in the
function $B_\nu(\nu,T)$, i.e.\ by solving the equation
\begin{displaymath}
\frac{\mathrm{d}B_\nu(\nu,T)}{\mathrm{d}\nu}=0.
\end{displaymath}
This can be done as follows:
\nlscf
\input{fragments/tex/calc_min3.tex}
\nlnp
Finally, this maximum could also be found using Pyxplot's built-in function {\tt Bvmax(T)}:\vspace{2mm}\newline
\input{fragments/tex/calc_min4.tex}
}

\section{Working with time-series data}
\label{sec:time_series}

Time-series data need to be handled carefully. If times and dates are specified
in local time, then conversions may be necessarily between timezones,
especially around the beginning and end of daylight saving time.

Even when this it not an issue, months have different lengths and leap years
have an extra day, which mean it is not straightforward to convert a series of
calendar dates into elapsed times between the \datapoint s.

On a more basic level, even time expressed in hours and minutes are complicated
by being expressed as non-decimal fractions of days.

To simplify the process of working with dates and times, Pyxplot has native
{\tt date} object type, together with pre-defined functions in the {\tt time}
module for creating and manipulating such objects.  A date object represents a
specific moment in time, and can be created from a time and date specified in
any arbitrary timezone. It is then possible to read out the time and date
components of this date object in any other arbitrary timezone.

The functions for creating {\tt date} objects are as follows:

\vspace{2mm}\noindent{\tt time.fromCalendar(year,month,day,hour,min,sec,<timezone>)}\vspace{2mm}

\noindent This function creates a date object from the specified calendar date.
It takes six compulsary numerical inputs: the year, the month number (1--12),
the day of the month (1--31), the hour of day (0--24), the number of minutes
(0--59), and the number of seconds (0--59). To enter dates before
{\footnotesize AD}\,1, a year of~$0$ should be passed to indicate
1\,{\footnotesize BC}, $-1$ should be passed to indicate the year
2\,{\footnotesize BC}, and so forth.

A timezone may optionally be specified as the final argument to the function.
If no timezone is specified, then the default is used, which may be set using
the {\tt set timezone} command. The timezone should be specified as a location string,
of the form {\tt Europe/London}, {\tt America/New\_York} or {\tt
Australia/Perth}, as used by the {\tt tz database}. A complete list of
available timezones can be found here:
\url{http://en.wikipedia.org/wiki/List_of_tz_database_time_zones}.

Daylight saving time will be applied as appropriate for the specified location.
Note that strings such as {\tt GMT}, {\tt EDT} or {\tt CEST} are {\it not}
allowed as timezones; a {\it location} should be specified.

If universal time is used, the timezone may be specified as {\tt UTC}.

\vspace{2mm}\noindent{\tt time.fromUnix(t)}\vspace{2mm}

\noindent This function creates a date object from the specified numerical Unix time -- i.e.\ the number of seconds ellapsed since midnight on 1st January 1970 UTC.

\vspace{2mm}\noindent{\tt time.fromJD(t)}\vspace{2mm}

\noindent This function creates a date object from the specified numerical
Julian date.

\vspace{2mm}\noindent{\tt time.fromMJD(t)}\vspace{2mm}

\noindent This function creates a date object from the
specified numerical modified Julian date.

\vspace{2mm}\noindent{\tt time.now()}\vspace{2mm}

This function takes no arguments, returns a {\tt date} object corresponding to
the current system clock time, as in the following example:

\vspace{3mm}
\noindent\texttt{pyxplot> \textbf{print time.now()}}\newline
\noindent\texttt{Tue 2012 Sep 4 20:57:00 UTC}\newline
\noindent\texttt{pyxplot> \textbf{set timezone "America/Los\_Angeles"}}\newline
\noindent\texttt{pyxplot> \textbf{print time.now()}}\newline
\noindent\texttt{Tue 2012 Sep 4 13:57:41 PDT}\newline
\vspace{3mm}

Note that the date object created by the {\tt time.now()} function is identical
regardless of timezone, but it is {\it displayed} differently depending upon
the current timezone.

The following example creates a date object representing midnight on 1st
January 2000, in universal time, and in Western Australian time:

\vspace{3mm}
\input{fragments/tex/calc_date1.tex}
\vspace{3mm}

Once created, it is possible to add numbers with physical units of time to
dates, as in the following example:

\vspace{3mm}
\input{fragments/tex/calc_date3.tex}
\vspace{3mm}

Standard string representations of calendar dates can be produced with the {\tt
print} command.  It is also possible to use the string substitution operator,
as in {\tt "\%s"\%(date)}, or the {\tt str} method of {\tt date} objects, as in
{\tt date.str()}.

In addition, the {\tt time.string} function can be used to choose a custom
display format for the date, or to specify the timezone in which the date
should be displayed. Its arguments are as follows:

\vspace{2mm}\noindent{\tt time.string(t,<format>,<timezone>)}\vspace{2mm}

\noindent This function returns a string representation of the specified date object $t$. The second argument is optional, and may be used to control the format of the output. If no format string is provided, then the format \newline\noindent{\tt "\%a \%Y \%b \%d \%H:\%M:\%S \%Z"}\newline\noindent is used. In such format strings, the following tokens are substituted for various parts of the date:
\begin{longtable}{|>{\columncolor{LightGrey}}l|>{\columncolor{LightGrey}}l|}
\hline \endfoot
\hline
Token & Value \\
\hline \endhead
{\tt \%\%} & A literal \% sign.\\
{\tt \%a} & Three-letter abbreviated weekday name.\\
{\tt \%A} & Full weekday name.\\
{\tt \%b} & Three-letter abbreviated month name.\\
{\tt \%B} & Full month name.\\
{\tt \%C} & Century number, e.g. 21 for the years 2000-2099.\\
{\tt \%d} & Day of month.\\
{\tt \%H} & Hour of day, in range~00-23.\\
{\tt \%I} & Hour of day, in range~01-12.\\
{\tt \%k} & Hour of day, in range~0-23.\\
{\tt \%l} & Hour of day, in range~1-12.\\
{\tt \%m} & Month number, in range~01-12.\\
{\tt \%M} & Minute, in range~00-59.\\
{\tt \%p} & Either {\tt am} or {\tt pm}.\\
{\tt \%S} & Second, in range~00-59.\\
{\tt \%y} & Last two digits of year number.\\
{\tt \%Y} & Year number.\\
{\tt \%Z} & Timezone name (e.g. UTC, CEST, EDT). \\
\end{longtable}

The third argument is also optional, and specifies the timezone that the time
should be displayed in. As above, this should be specified in the form {\tt
Europe/London}, {\tt America/New\_York} or {\tt Australia/Perth}, as used by
the {\tt tz database}. A complete list of available timezones can be found
here: \url{http://en.wikipedia.org/wiki/List_of_tz_database_time_zones}.  If
universal time is used, the timezone may be specified as {\tt UTC}. If no
timezone is specified, the default is used as set in the {\tt set timezone}
command.

Several functions are provided for converting {\tt date} objects back into
various numerical forms of timekeeping and components of calendar dates, which
are listed below. Where appropriate, an optional timezone may be specified to
obtain a calendar date for a particular location:

\methdef{toDayOfMonth($<timezone>$)}{returns the day of the month of a date object in the current calendar.}
\methdef{toDayWeekName($<timezone>$)}{returns the name of the day of the week of a date object.}
\methdef{toDayWeekNum($<timezone>$)}{returns the day of the week (1--7) of a date object.}
\methdef{toHour($<timezone>$)}{returns the integer hour component (0--23) of a date object.}
\methdef{toJD()}{converts a date object to a numerical Julian date.}
\methdef{toMinute($<timezone>$)}{returns the integer minute component (0--59) of a date object.}
\methdef{toMJD()}{converts a date object to a modified Julian date.}
\methdef{toMonthName($<timezone>$)}{returns the name of the month in which a date object falls.}
\methdef{toMonthNum($<timezone>$)}{returns the number (1--12) of the month in which a date object falls.}
\methdef{toSecond($<timezone>$)}{returns the seconds component (0--60) of a date object, including the non-integer component.}
\methdef{toUnix()}{converts a date object to a Unix time.}
\methdef{toYear($<timezone>$)}{returns the year in which a date object falls in the current calendar.}

For example:

\vspace{3mm}
\input{fragments/tex/calc_date2.tex}
\vspace{3mm}

\subsection{Calendars}

By default, the {\tt time.fromCalendar} function makes a transition from the
old Julian calendar to the new Gregorian calendar at midnight on 14th~September
1752 (Gregorian calendar), when Britain and the British Empire switched
calendars.  Thus, dates between 2nd~September and 14th~September 1752 are not
valid input dates, since they days never occurred in the British calendar.

This behaviour may be changed using the \indcmdt{set calendar}, which offers a
choice of nine different calendars listed in Table~\ref{tab:calendars}. Most of
the these calendars differ only in the date on which the transition is made
between the old (Julian) calendar and the new (Gregorian) calendar.

The exceptions are the Hebrew and Islamic calendars, which have entirely
different systems of months.

\begin{table}
\begin{center}
\begin{tabular}{|>{\columncolor{LightGrey}}l|>{\columncolor{LightGrey}}p{9cm}|}
\hline
{\bf Calendar} & {\bf Description} \\
\hline
British &
Use the Gregorian calendar from 14th~September 1752 (Gregorian), and the Julian calendar prior to 2nd~September 1752 (Julian). \\
French &
Use the Gregorian calendar from 20th~December 1582 (Gregorian), and the Julian
calendar prior to 9th~December 1582 (Julian). \\
Greek &
Use the Gregorian calendar from 1st~March 1923 (Gregorian), and the Julian
calendar prior to 15th~February 1923 (Julian). \\
Gregorian &
Use the Gregorian calendar for all dates. \\
Hebrew &
Use the Hebrew (Jewish) calendar. \\
Islamic &
Use the Islamic (Muslim) calendar. Note that the Islamic calendar is undefined prior to 1st~Muharram {\footnotesize AH}\,1, corresponding to 18th~July {\footnotesize AD}\,622. \\
Julian &
Use the Julian calendar for all dates. \\
Papal &
Use the Gregorian calendar from 15th~October 1582 (Gregorian), and the Julian
calendar prior to 4th~October 1582 (Julian). \\
Russian &
Use the Gregorian calendar from 14th~February 1918 (Gregorian), and the Julian
calendar prior to 31st~January 1918 (Julian). \\
\hline
\end{tabular}
\end{center}
\caption{The calendars supported by the \indcmdt{set calendar}, which can be
used to convert dates between calendar dates and Julian Day numbers.}
\label{tab:calendars}
\end{table}

Optionally, the \indcmdt{set calendar} can be used to set different calendars
to use when converting calendar dates into {\tt date} objects, and when
converting in the opposite direction. This is useful when converting data from
one calendar to another. The syntax used to do this is as follows:
\begin{verbatim}
set calendar in Julian     # only applies to time.fromCalendar()
set calendar out Gregorian # does not apply to time.fromCalendar()
set calendar in Julian out Gregorian   # change both
show calendar              # show calendars currently being used
\end{verbatim}

\example{ex:tolstoy}{Calculating the date of Leo Tolstoy's birth}{
The Russian novelist Leo Tolstoy was born on 28th~August~$1828$ and died on
7th~November~$1910$ in the Russian calendar. What dates do these correspond to
in the Western calendar?
\nlscf
\input{fragments/tex/calc_tolstoy.tex}
}

\subsection{Time intervals}

The time interval between two date objects can be found by subtracting one from
the other.  The following example calculates the time interval between Albert
Einstein's birth and death. The result is returned as a numerical object with
physical dimensions of time:

\vspace{3mm}
\input{fragments/tex/calc_date4.tex}
\vspace{3mm}

The function {\tt time.interval(t1,t2)} has the same effect.  The next example
calculate the time elapsed between the traditional date for the foundation of
Rome by Romulus and Remus in 753\,{\footnotesize BC} and that of the deposition
of the last Emperor of the Western Empire in {\footnotesize AD}\,476:

\vspace{3mm}
\input{fragments/tex/calc_interval.tex}
\vspace{3mm}

The function {\tt time.intervalStr()}\indfun{time.intervalStr({\it
t}$_1$,\-{\it t}$_2$,\-{\it format})} is similar, but returns a textual
representation of the time interval.  It takes an optional third parameter
which specifies the textual format in which the time interval should be
represented. If no format is supplied, then the following verbose format is
used:
\vspace{3mm}\newline
\noindent {\tt "\%Y years \%d days \%h hours \%m minutes and \%s seconds"}
\vspace{3mm}\newline
Table~\ref{tab:time_diff_string_subs} lists the tokens which are substituted
for various parts of the time interval. The following examples demonstrate the
use of the function:

\begin{table}
\begin{center}
\begin{tabular}{|>{\columncolor{LightGrey}}l|>{\columncolor{LightGrey}}l|}
\hline
Token & Substitution value \\
\hline
{\tt \%\%} & A literal \% sign.\\
{\tt \%d} & The number of days elapsed, modulo 365.\\
{\tt \%D} & The number of days elapsed. \\
{\tt \%h} & The number of hours elapsed, modulo 24.\\
{\tt \%H} & The number of hours elapsed.\\
{\tt \%m} & The number of minutes elapsed, modulo 60.\\
{\tt \%M} & The number of minutes elapsed.\\
{\tt \%s} & The number of seconds elapsed, modulo 60.\\
{\tt \%S} & The number of seconds elapsed.\\
{\tt \%Y} & The number of years elapsed.\\
\hline
\end{tabular}
\end{center}
\caption{Tokens which are substituted for various components of the time interval by the {\tt time\_diff\_string} function.}
\label{tab:time_diff_string_subs}
\end{table}

\vspace{3mm}
\input{fragments/tex/calc_interval2.tex}
\vspace{3mm}

\example{ex:timeseries}{A plot of the rate of downloads from an Apache webserver}{
In this example, we use Pyxplot's facilities for handling dates and times to
produce a plot of the rate of downloads from an Apache webserver based upon the
download log which it stores in the file {\tt /var/log/apache2/access.log}.
This file contain a line of the following form for each page or file requested
from the webserver:
\vspace{3mm}\newline\noindent{\tt\footnotesize 127.0.0.1 - - [14/Jun/2012:16:43:18 +0100] "GET / HTTP/1.1" 200 484 "-" "Mozilla/5.0 (X11; Linux x86\_64) AppleWebKit/535.19 (KHTML, like Gecko) Ubuntu/12.04 Chromium/18.0.1025.151 Chrome/18.0.1025.151 Safari/535.19"}\vspace{3mm}\newline
However, Pyxplot's default input filter for {\tt .log} files (see Section~\ref{sec:filters}) manipulates the dates in strings such as these into the form
\vspace{3mm}\newline\noindent{\tt\footnotesize 127.0.0.1 - -~~[~14~~6~~2012~16~43~18~+0100~]~~"GET~~~HTTP 1.1" 200 484 "-" "Mozilla/5.0 (X11; Linux x86\_64) AppleWebKit/535.19 (KHTML, like Gecko) Ubuntu/12.04 Chromium/18.0.1025.151 Chrome/18.0.1025.151 Safari/535.19"}\vspace{3mm}\newline
such that the day, month, year, hour, minute and second components of the date
are contained in the 5th to 10th white-space-separated columns respectively.
In the script below, the {\tt time.fromCalendar()} function and {\tt toUnix()} method are then used to convert
these components into Unix times. The \indcmdt{histogram} (see
Section~\ref{sec:histogram}) is used to sort each of the web accesses recorded
in the Apache log file into hour-sized bins.  Because this may be a
time-consuming process for large log files on busy servers, we use the
\indcmdt{tabulate} (see Section~\ref{sec:tabulate}) to store the data into a
temporary \datafile on disk before deciding how to plot it:
\nlscf
\input{examples/tex/ex_apachelog_1.tex}
\nlscf
Having stored our histogram in the file {\tt apache.dat}, we
now plot the resulting histogram, labelling the horizontal axis with the days
of the week.  The commands used to achieve this will be introduced in
Chapter~\ref{ch:plotting}. The major axis ticks along the horizontal axis are
placed at daily intervals, and minor axis ticks are placed along the axis every
quarter day, i.e.\ every six hours.
\nlscf
\input{examples/tex/ex_apachelog_2.tex}
\nlscf
The plot below shows the graph which results on a moderately busy webserver
which hosts, among many other sites, the Pyxplot website:
\nlscf
\centerline{\includegraphics[width=\textwidth]{examples/eps/ex_apachelog}}
}


% data.tex
%
% The documentation in this file is part of Pyxplot
% <http://www.pyxplot.org.uk>
%
% Copyright (C) 2006-2012 Dominic Ford <coders@pyxplot.org.uk>
%               2008-2012 Ross Church
%
% $Id$
%
% Pyxplot is free software; you can redistribute it and/or modify it under the
% terms of the GNU General Public License as published by the Free Software
% Foundation; either version 2 of the License, or (at your option) any later
% version.
%
% You should have received a copy of the GNU General Public License along with
% Pyxplot; if not, write to the Free Software Foundation, Inc., 51 Franklin
% Street, Fifth Floor, Boston, MA  02110-1301, USA

% ----------------------------------------------------------------------------

% LaTeX source for the Pyxplot Users' Guide

\chapter{Working with data}
\label{ch:numerics}

This chapter returns to Pyxplot's commands for acting on data stored in files.
Chapter~\ref{ch:first_steps} has already introduced the {\tt plot} command,
which draws graphs, but there are also commands for tabulating data to new
\datafile s, for computing histograms, for interpolating data, and for taking
Fourier transforms.

Section~\ref{sec:plot_datafiles} has already introduced the options which can
be used to select data from particular columns or which satisfy particular
criteria: {\tt using}, {\tt index}, {\tt every} and {\tt select}.  These
options are universal to all of Pyxplot's commands which operate on data sets.
In all cases, data sets can be read from files, sampled from functions, or
specified as a colon-separated list of vectors (see
Section~\ref{sec:vectorplot}).

This chapter begins by describing other common features of these commands,
before moving on to describe each command in turn. It leaves the details of the
{\tt plot} command, which was introduced in Chapter~\ref{ch:first_steps}, to be
described in full detail in Chapter~\ref{ch:plotting}.

\section{Input filters}
\label{sec:filters}

By default, Pyxplot expects \datafile s to be in a simple plaintext format
which is described in Section~\ref{sec:plot_datafiles}. However, input filters
provide a mechanism by which \datafile s in arbitrary formats can be read.

An input filter is specified to act on all \datafile s that match some
filename pattern. For example, a filter could be defined to act on all
\datafile s called {\tt *.txt} or {\tt *.dat}. The filter itself takes the form
of a program which is launched by Pyxplot whenever a matching \datafile\ is
read.  The program is passed the filename of the \datafile\ as a command line
argument immediately following any arguments specified in the filter's
definition. It is then expected to return the data contained in the file to
Pyxplot in plaintext format using its {\tt stdout} stream. Any errors which
such a program returns to {\tt stderr} are passed to the user as error
messages.

Pyxplot has five input filters built-in, as the {\tt show filters} command
reveals:

\begin{verbatim}
set filter "*.fits"   "/usr/local/lib/pyxplot/pyxplot_fitshelper"
set filter "*.gz"     "/bin/gunzip -c"
set filter "*.log"    "/usr/local/lib/pyxplot/pyxplot_timehelper"
set filter "ftp://*"  "/usr/bin/wget -O -"
set filter "http://*" "/usr/bin/wget -O -"
\end{verbatim}

The above set of filters allow Pyxplot to read data from gzipped plaintext
\datafile s, from \datafile s available over the web via HTTP\index{HTTP} or
FTP\index{FTP}, and data tables in FITS format\index{FITS format}.  A filter is
also provided for converting textual dates in log files into numbers
representing days, months and years. To add to this list of filters, it is
necessary to write a short program or shell script; the simple filters provided
in Pyxplot's source code for {\tt .log} and {.fits} files may provide a useful
model.

The filter can then be installed using the syntax

\begin{verbatim}
set filter <filenameWildcard> <shellCommand>
\end{verbatim}


\section{Reading data from a pipe}

Pyxplot usually reads data from files, or samples it from functions. However,
it is also possible to read data {\it piped}\index{pipes} into it from other
processes if these are directed to Pyxplot's standard input stream, {\tt
stdin}\index{stdin}.  To do this, the magic filename {\tt -} is used:

\begin{verbatim}
plot '-' with lines
\end{verbatim}

This makes it possible to call Pyxplot from within another program and pass it
data to plot without ever storing the data on disk. Whilst this facility has
great power, it should be used with caution.

It is often very tempting to write programs which both perform calculations and
plot the results immediately, but this can make it difficult to replot graphs
later. A few months after the event, when the need arises to replot the same
data in a different form or in a different style, remembering how to use a
sizable program can prove tricky -- especially if the person struggling to do
so is not you! But a simple \datafile\ is quite straightforward to plot time
and again.

\section{Including data within command scripts}

It is also possible to embed data directly within Pyxplot scripts, which may be
useful when a small number of markers are wanted at particular pre-defined
positions on a graph, or when it is desirable to roll a Pyxplot script and the
data it takes into a single file for easy storage or transmission. To do this,
one uses the magic filename {\tt \--\--} and terminates the data with the string
{\tt END}:

\begin{verbatim}
plot '--' with lines
0 0
1 1
2 0
3 1
END
print "More Pyxplot commands can be placed after END"
\end{verbatim}

\section{Special comment lines in \datafile s}
\label{sec:special_comments}
\index{comment lines!in datafiles}

Whilst most comment lines in \datafile s -- those lines which begin with a hash
character -- are ignored by Pyxplot, lines which begin with any of the
following four strings are parsed:
\begin{verbatim}
# Columns:
# ColumnUnits:

# Rows:
# RowUnits:
\end{verbatim}
The first pair of special comments affect the behaviour of Pyxplot when
plotting {\tt using columns}, while the second pair affect the behaviour of
Pyxplot when plotting {\tt using rows} (see
Section~\ref{sec:horizontal_datafiles}). Within each pair, the first may be
used to tell Pyxplot the names of each of the columns/rows in the \datafile,
while the second may be used to tell Pyxplot the physical units of the values
in each of the columns/rows. These special comments may appear multiple times
throughout a single \datafile\ to indicate changes to the format of the data.

For example, a \datafile\ prefixed with the lines
\begin{verbatim}
# Columns:      Time   Distance
# ColumnUnits:   s      10*m
\end{verbatim}
contains two columns of data, the first containing times measured in seconds
and the second containing distances measured in tens of metres. Note that
because the entries on each of these lines are whitespace-separated, spaces are
not allowed in column names or within units such as {\tt 10*m}. This \datafile\
could be plotted using any of the following forms equivalently
\begin{verbatim}
plot 'data' using Time:Distance
plot 'data' using $Time:$Distance
plot 'data' using 1:2
\end{verbatim}
and the axes of the graph would indicate the units of the data (see
Section~\ref{sec:set_axisunitstyle}).

\section{Tabulating functions and slicing \datafile s}
\label{sec:tabulate}

Pyxplot's \indcmdt{tabulate} is similar to its {\tt plot} command, but instead
of plotting a series of \datapoint s onto a graph, it writes them to a
\datafile. It can be used to produce text files containing samples of
functions, to rearrange/filter the columns in \datafile s, to produce a copy of
a \datafile\ using different physical units, and so on.

The following example would produce a \datafile\ called {\tt gamma.dat}
containing a list of values of the gamma function:

\begin{verbatim}
set output 'gamma.dat'
tabulate [1:5] gamma(x)
\end{verbatim}

One way to tabulate multiple functions into a common file is with the
\indmodt{using} modifier, as in the example

\begin{verbatim}
tabulate [0:2*pi] sin(x):cos(x):tan(x) using 1:2:3:4
\end{verbatim}

\noindent This tabulates the supplied functions horizontally alongside one
another in a series of columns. As many expressions may be supplied to the
\indmodt{using} modifier as columns are wanted.

Alternatively, if a series on functions or \datafile s are listed in a
comma-separated list (as is done in the {\tt plot} command to plot multiple
datasets), the functions are tabulated one after another in a series of index
blocks separated by double linefeeds (see Section~\ref{sec:plot_datafiles}):

\begin{verbatim}
tabulate [0:2*pi] sin(x), cos(x), tan(x)
\end{verbatim}

The \indcmdt{set samples} can be used to control the number of points that are
listed when tabulating functions, in the same way that it controls the number
of data points drawn by the {\tt plot} command:

\begin{verbatim}
set samples 200
\end{verbatim}

\noindent If the abscissa axis is set to be logarithmic then the functions are
evaluated at logarithmically-space points along the axis; otherwise, they are
samples at linearly-spaced points.

The \indmodt{select}, \indmodt{using} and \indmodt{every} modifiers operate in
the same manner in the {\tt tabulate} command as in the {\tt plot} command.
Thus the following example would write out the third, sixth and ninth columns
of the \datafile\ {\tt input.dat}, but only when the arcsine of the value in the
fourth column is positive:

\begin{verbatim}
set output 'filtered.dat'
tabulate 'input.dat' using 3:6:9 select (asin($4)>0)
\end{verbatim}

The numerical display format used for each column of the output file is
automatically chosen to preserve accuracy whilst simultaneously being as easily
human readable as possible.  Thus, columns which contain only integers are
displayed as such, and scientific notation is only used in columns which
contain very large or very small values.

If desired, however, a custom format may be specified using the {\tt with
format} modifier. This can be used both to specify text to appear in between
the columns of data, and to specify the format of the data itself using tokens
such as {\tt \%.5f}, as used by Pyxplot's string substitution operator ({\tt
\%}; see Section~\ref{sec:stringsubop}), and the {\tt sprintf} statement of
many other programming languages.

For example, to tabulate the values of $x^2$ to very many significant figures
with some additional text, one could use:

\begin{verbatim}
tabulate x**2 with format "x = %f ; x**2 = %27.20e"
\end{verbatim}

\noindent This might produce the following output:

\begin{verbatim}
x = 0.000000 ; x**2 =  0.00000000000000000000e+00
x = 0.833333 ; x**2 =  6.94444444444442421371e-01
x = 1.666667 ; x**2 =  2.77777777777778167589e+00
\end{verbatim}

There is flexibility as to how many substitution tokens appear in the format
specification.  If the number of tokens is fewer than the number of columns of
data, then the format is repeated until all the columns have been printed.
Thus, the command

\begin{verbatim}
tabulate x**2 with format "%.3f "
\end{verbatim}

\noindent might produce the output:

\begin{verbatim}
0.000 0.000
0.833 0.694
1.667 2.778
\end{verbatim}

\noindent Note that the space character at the end of the format is important
to ensure that there is a gap between the columns.

If formats are supplied for more columns than are present, then the final
columns are padded with {\tt nan} (not a number).

The data produced by the {\tt tabulate} command can be sorted in order of any
arbitrary metric by supplying an expression after the {\tt sortby} modifier.
The data are sorted in order from the lowest value of this expression to the
highest.


\section{Function fitting}
\label{sec:fit_command}

The \indcmdt{fit} can be used to fit arbitrary functional forms to \datapoint s
read from files. It can be used to produce best-fit lines\index{best fit
lines}\footnote{Another way of producing best-fit lines is to use the {\tt
interpolate} command; more details are given in
Section~\ref{sec:spline_command}} for datasets, or to determine gradients and
other mathematical properties of data by looking at the parameters associated
with the best-fitting functional form.

The following simple example fits a straight line to data in a file called {\tt
data.dat}:

\begin{verbatim}
f(x) = a*x+b
fit f() 'data.dat' index 1 using 2:3 via a,b
\end{verbatim}

\noindent The first line specifies the functional form which is to be used.
The coefficients of this function, {\tt a} and {\tt b}, which are to be
varied during the fitting process, are listed after the keyword \indkeyt{via}
in the {\tt fit} command.  The modifiers \indmodt{index}, \indmodt{every},
\indmodt{select} and \indmodt{using} have the same meanings as in the {\tt plot} command.

When a function of $n$ variables is being fit, at least $n+1$ columns (or rows
-- see Section~\ref{sec:horizontal_datafiles}) of data must be specified after
the {\tt using} modifier. By default, the first $n+1$ columns are used. These
correspond to the values of each of the $n$ arguments to the function, plus
finally the value which the output from the function is aiming to match.

If an additional column is specified, then this is taken to contain the
standard error in the value that the output from the function is aiming to
match, and can be used to weight the \datapoint s which are being used to
constrain the fit.

As an example, below we generate a \datafile\ containing samples of a square
wave using the {\tt tabulate} command and fit the first three terms of a
truncated Fourier series to it:

\begin{verbatim}
set samples 10
set output 'square.dat'
tabulate [-pi:pi] 1-2*heaviside(x)

f(x) = a1*sin(x) + a3*sin(3*x) + a5*sin(5*x)
fit f() 'square.dat' via a1, a3, a5
set xlabel '$x$' ; set ylabel '$y$'
plot 'square.dat' title 'data' with points pointsize 2, \
     f(x) title 'Fitted function' with lines
\end{verbatim}

\begin{center}
\includegraphics[width=8cm]{examples/eps/ex_fitting}
\end{center}

As the \indcmdt{fit} works, it displays statistics including the best fit
values of each of the fitting parameters, the uncertainties in each of them,
and the covariance matrix. These can be useful for analysing the security of
the fit achieved, but calculating the uncertainties in the best fit parameters
and the covariance matrix can be time consuming, especially when many
parameters are being fitted simultaneously. The optional word {\tt
withouterrors} can be included immediately before the filename of the input
\datafile\ to substantially speed up cases where this information is not
required.

By default, the starting values for each of the fitting parameters is
$1.0$. However, if the variables to be used in the fitting process are already
set before the {\tt fit} command is called, these initial values are used
instead. For example, the following would use the initial values
$\{a=100,b=50\}$:
\begin{verbatim}
f(x) = a*x+b
a = 100
b = 50
fit f() 'data.dat' index 1 using 2:3 via a,b
\end{verbatim}

\noindent If any of the fitting coefficients are not dimensionless -- that is,
they have physical units such as meters or seconds -- then an initial value
with the appropriate units {\it must} be specified.

A few points are worth noting:

\begin{itemize}
\item A series of ranges may be specified after the {\tt fit} command, using
the same syntax as in the {\tt plot} command, as described in
Section~\ref{sec:plot_ranges}. If ranges are specified then only \datapoint s
falling within these ranges are used in the fitting process; the ranges refer
to each of the $n$ variables of the fitted function in order:
\begin{verbatim}
fit [0:10] f() 'data.dat' via a
\end{verbatim}

\item As with all numerical fitting procedures, the {\tt fit} command comes
with caveats. It uses a generic fitting algorithm, and may not work well with
poorly behaved or ill-constrained problems. It works best when all of the
values it is attempting to fit are of order unity. For example, in a problem
where $a$ was of order $10^{10}$, the following might fail:
\begin{verbatim}
f(x) = a*x
fit f() 'data.dat' via a
\end{verbatim}
However, better results might be achieved if $a$ were artificially made of
order unity, as in the following script:
\begin{verbatim}
f(x) = 1e10*a*x
fit f() 'data.dat' via a
\end{verbatim}

\item For those interested in the mathematical details, the workings of the
{\tt fit} command are discussed in more detail in Appendix~\ref{ch:fit_maths}.

\end{itemize}

\section{Datafile interpolation}
\label{sec:spline_command}
\index{best fit lines}

The \indcmdt{interpolate} can be used to generate a special function within
Pyxplot's mathematical environment which interpolates a set of \datapoint s
supplied from a \datafile. As with other commands, data can also be supplied
from functions, or from a colon-separated list of vectors (see
Section~\ref{sec:vectorplot}). Either one- or two-dimensional interpolation is
possible. Two-dimensional interpolation is described in the next section.

In the case of one-dimensional interpolation, various different types of
interpolation are supported: linear interpolation, power law interpolation,
polynomial interpolation, cubic spline interpolation and akima spline
interpolation. Stepwise interpolation returns the value of the datapoint
nearest to the requested point in argument space. The use of polynomial
interpolation with large datasets is strongly discouraged, as polynomial fits
tend to show severe oscillations between \datapoint s.

Except in the case of stepwise interpolation, extrapolation is not permitted;
if an attempt is made to evaluate an interpolated function beyond the limits of
the \datapoint s which it interpolates, Pyxplot returns an error or value of
not-a-number.  This behaviour can be configured using the \indcmdt{set numeric
errors quiet} (see Section~\ref{sec:num_errs}).

The \indcmdt{interpolate} has similar syntax to the \indcmdt{fit}:

\begin{verbatim}
interpolate ( akima | linear | loglinear | polynomial |
              spline | stepwise |
              2d [( bmp_r | bmp_g | bmp_b )] )
            [<range specification>] <function name>"()"
            '<filename>'
            [ every <expression> {:<expression} ]
            [ index <value> ]
            [ select <expression> ]
            [ using <expression> {:<expression} ]
\end{verbatim}

A very common application of the \indcmdt{interpolate} is to perform arithmetic
functions such as addition or subtraction on datasets which are not sampled at
the same abscissa values. The following example would plot the difference
between two such datasets:

\begin{verbatim}
interpolate linear f() 'data1.dat'
interpolate linear g() 'data2.dat'
plot [min:max] f(x)-g(x)
\end{verbatim}

\noindent Note that it is advisable to supply a range to the {\tt plot} command
in this example: because the two datasets have been turned into continuous
functions, the {\tt plot} command has to guess a range over which to plot them,
unless one is explicitly supplied.

The \indcmdt{spline} is an alias for {\tt interpolate spline}; the following
two statements are equivalent:

\begin{verbatim}
spline f() 'data1.dat'
interpolate spline f() 'data1.dat'
\end{verbatim}

\example{ex:interpolation}{A demonstration of the {\tt linear}, {\tt spline} and {\tt akima} modes of interpolation}{
In this example, we demonstrate the {\tt linear}, {\tt spline} and {\tt akima}
modes of interpolation using an example \datafile\ with non-smooth data
generated using the {\tt tabulate} command (see Section~\ref{sec:tabulate}):
\nlscf
\noindent{\tt f(x) = 0}\newline
\noindent{\tt f(x)[0:1] = 0.5}\newline
\noindent{\tt f(x)[2:4] = cos((x-3)*pi/2)}\newline
\noindent{\tt set samples 20}\newline
\noindent{\tt tabulate [0:4] f(x)}
\nlscf
Having set three functions to interpolate these non-smooth data in different
ways, we plot them with a vertical offset of~$0.1$ between them for clarity.
The interpolated \datafile is plotted with points three times to show where
each of the interpolation functions is pinned.
\nlscf
\input{examples/tex/ex_interpolation_1.tex}
\nlscf
The resulting plot is shown below:
\nlscf
\centerline{\includegraphics[width=\textwidth]{examples/eps/ex_interpolation}}
}

\subsection{Two-dimensional interpolation}

In the case of two-dimensional interpolation, the type of interpolation to be
used is set using the {\tt interpolate} modifier to the \indcmdt{set samples},
and may be changed at any time after the interpolation function has been
created.  The options available are nearest neighbor interpolation -- which is
the two-dimensional equivalent of stepwise interpolation, inverse square
interpolation -- which returns a weighted average of the supplied \datapoint s,
using the inverse squares of their distances from the requested point in
argument space as weights, and Monaghan Lattanzio interpolation, which uses the
weighting function (Monaghan \& Lattanzio 1985)
\begin{eqnarray*}
w(x) & = 1 - \nicefrac{3}{2}v^2 + \nicefrac{3}{4}v^3 & \,\mathrm{for~}0\leq v\leq 1 \\
     & = \nicefrac{1}{4}(2-v)^3                      & \,\mathrm{for~}1\leq v\leq 2
\end{eqnarray*}
where $v=r/h$ for $h=\sqrt{A/n}$, $A$ is the product
$(x_\mathrm{max}-x_\mathrm{min})(y_\mathrm{max}-y_\mathrm{min})$ and $n$ is the
number of input datapoints. These are selected as follows:

\begin{verbatim}
set samples interpolate nearestNeighbor
set samples interpolate inverseSquare
set samples interpolate monaghanLattanzio
\end{verbatim}

The following example creates a function {\tt quad\-ra\-pole(x,y)} which interpolates a quadrapole:

\begin{verbatim}
set samples interpolate inverseSquare
interpolate 2d quadrapole() '--'
-1 -1  1
-1  1 -1
 1 -1 -1
 1  1  1
END
\end{verbatim}

Finally, data can be imported from graphical images in bitmap ({\tt .bmp})
format to produce a function of two arguments returning a value in the
range~$0\to1$ which represents the data in one of the image's three color
channels. The two arguments are the horizontal and vertical position within the
bitmap image, as measured in pixels. This is done using syntax of the form:

\begin{verbatim}
interpolate 2d bmp_b blue() 'myImg.bmp'
\end{verbatim}

\section{Fourier transforms}

The {\tt fft} and {\tt ifft} commands\indcmd{fft}\indcmd{ifft} take Fourier
transforms and inverse Fourier transforms respectively of data. As with other
commands, data can be supplied from a \datafile, from functions, or from a
colon-separated list of vectors (see Section~\ref{sec:vectorplot}). In each
case, a regular grid of abscissa values must be specified on which to take the
discrete Fourier transform, which can extend over an arbitrary number of
dimensions. The following example demonstrates the syntax of these commands as
applied to a two-dimensional top-hat function:

\begin{verbatim}
step(x,y) = tophat(x,0.2) * tophat(y,0.4)
fft  [  0: 1:0.01][  0: 1:0.01] f() of step()
ifft [-50:49:1   ][-50:49:1   ] g() of f()
\end{verbatim}

\noindent In the {\tt fft} command above, $N_x=100$~equally-spaced samples of
the function {\tt step}$(x,y)$ are taken between limits of $x_\mathrm{min}=0$
and $x_\mathrm{max}=1$ for each of $N_y=100$~equally-spaced values of $y$ on an
identical raster, giving a total of $10^4$ samples. These are converted into a
rectangular grid of $10^4$~samples of the Fourier transform {\tt
f}$(\omega_x,\omega_y)$ at

\begin{eqnarray}
\omega_x = \frac{j}{\Delta x} & \textrm{for} & -\frac{N_x}{2}\leq j <\frac{N_x}{2} \; \left(\textrm{equivalently, for} -\frac{N_x}{2\Delta x}\leq \omega_x <\frac{N_x}{2\Delta x} \right), \nonumber \\
\omega_y = \frac{k}{\Delta y} & \textrm{for} & -\frac{N_y}{2}\leq k <\frac{N_y}{2} \; \left(\textrm{equivalently, for} -\frac{N_y}{2\Delta y}\leq \omega_y <\frac{N_y}{2\Delta y} \right). \nonumber
\end{eqnarray}

\noindent where $\Delta x=x_\mathrm{max}-x_\mathrm{min}$ and $\Delta y$ is
analogously defined. These samples are interpolated stepwise, such that an
evaluation of the function {\tt f}$(\omega_x,\omega_y)$ for general inputs
yields the nearest sample, or zero outside the rectangular grid where samples
are available. In general, even the Fourier transforms of real functions are
complex, and their evaluation when complex arithmetic is not enabled (see
Section~\ref{sec:complex_numbers}) is likely to fail. For this reason, a
warning is issued if complex arithmetic is disabled when a Fourier transform
function is evaluated.

In the example above, we go on to convert this set of samples back into the
function with which we started by instructing the \indcmdt{ifft} to take
$N_x=100$~equally-spaced samples along the $\omega_x$-axis between
$\omega_{x,\mathrm{min}}=-{N_x}/{2\Delta x}$ and
$\omega_{x,\mathrm{max}}=(N_x-1)/{2\Delta x}$, with similar sampling along the
$\omega_y$-axis.

Taking the simpler example of a one-dimensional Fourier transform for clarity,
as might be calculated by the instructions
\begin{verbatim}
step(x) = tophat(x,0.2)
fft  [  0: 1:0.01] f() of step()
\end{verbatim}
the {\tt fft} and {\tt ifft} commands\indcmd{fft}\indcmd{ifft} compute,
respectively, discrete sets of samples $F_m$ and $I_n$ of the functions
$F(\omega_x)$ and $I(x)$, which are given by
\begin{displaymath}
F_j = \sum_{m=0}^{N-1} I_m e^{-2\pi ijm/N} \,\delta x,\;\textrm{for}\; -\frac{N}{2}\leq j <\frac{N}{2} ,
\end{displaymath}
\noindent and
\begin{displaymath}
I_j = \sum_{m=0}^{N-1} F_m e^{ 2\pi ijm/N} \,\delta \omega_x,\;\textrm{for}\; -\frac{N}{2}\leq j <\frac{N}{2} ,
\end{displaymath}
\noindent where:
\begin{tabular}{rcp{9cm}}
$I(x)$        & = & Function being Fourier transformed. \\
$F(\omega_x)$ & = & Fourier transform of $I()$. \\
$N$           & = & The number of values sampled along the abscissa axis. \\
$\delta x$    & = & Spacing of values sampled along the abscissa axis. \\
$\delta \omega_x$ & = & Spacing of abscissa values sampled along the $\omega_x$ axis. \\
$i$           & = & $\sqrt{-1}$. \\
\end{tabular}
\vspace{2mm}

It may be shown in the limit that $\delta x$ becomes small -- i.e.\ when the
number of samples taken becomes very large -- that these sums approximate the
integrals
\begin{equation}
F(\omega_x) = \int I(x) e^{-2\pi ix\omega_x} \,\mathrm{d}x ,
\end{equation}
\noindent and
\begin{equation}
I(x) = \int F(\omega_x) e^{ 2\pi ix\omega_x} \,\mathrm{d}\omega_x .
\end{equation}

Fourier transforms may also be taken of data stored in \datafile s using syntax
of the form
\begin{verbatim}
fft [-10:10:0.1] f() of 'datafile.dat'
\end{verbatim}

\noindent In such cases, the data read from the \datafile\ for an
$N$-dimensional FFT must be arranged in $N+1$ columns\footnote{The {\tt using},
{\tt every}, {\tt index} and {\tt select} modifiers can be used to arrange it
into this form.}, with the first $N$ containing the abscissa values for each of
the $N$ dimensions, and the final column containing the data to be Fourier
transformed. The abscissa values must strictly match those in the raster
specified in the {\tt fft} or {\tt ifft} command, and must be arranged strictly
in row-major order.

\example{ex:fourier}{The Fourier transform of a top-hat function}{
It is straightforward to show that the Fourier transform of a top-hat function
of unit width is the function $\sinc(x^\prime=\pi x)=\sin(x^\prime)/x^\prime$. If
\begin{displaymath}
f(x)=\left\{\begin{array}{l}1\;|x|\leq\nicefrac{1}{2}\\0\;|x|>\nicefrac{1}{2}\end{array}\right. ,
\end{displaymath}
the Fourier transform $F(\omega)$ of $f(x)$ is
\begin{eqnarray*}
F(\omega) & = & \int_0^\infty f(x) \exp \left(-2\pi ix\omega\right) \,\mathrm{d}x
            =   \int_{-\nicefrac{1}{2}}^{\nicefrac{1}{2}} \exp\left(-2\pi ix\omega\right) \,\mathrm{d}x
\\        & = & \frac{1}{2\pi\omega}\left[ \exp\left(\pi i\omega\right) - \exp\left(-\pi i\omega\right) \right]
            = \frac{\sin(\pi\omega)}{\pi\omega}
            = \sinc(\pi\omega).
\end{eqnarray*}
\nlnp
In this example, we demonstrate this numerically by taking the Fourier
transform of such a step function, and comparing the result against the
function {\tt sinc(x)} which is pre-defined within Pyxplot:
\nlscf
\noindent{\tt set numerics complex}\newline
\noindent{\tt step(x) = tophat(x,0.5)}\newline
\noindent{\tt fft [-1:1:0.01] f() of step()}\newline
\noindent{\tt plot [-10:10] Re(f(x)), sinc(pi*x)}
\nlscf
Note that the function {\tt Re(x)} is needed in the {\tt plot} statement here,
since although the Fourier transform of a symmetric function is in theory real,
in practice any numerical Fourier transform will yield a small imaginary
component at the level of the accuracy of the numerical method used. Although
the calculated numerical Fourier transform is defined throughout the range
$-50\leq x<50$, discretised with steps of size $\Updelta x=0.5$, we only plot
the central region in order to show clearly the stepping of the function:
\nlscf
\begin{center}
\includegraphics{examples/eps/ex_fft}
\end{center}
\nlscf
In the following steps, we take the square of the function $\sinc(\pi x)$ just
calculated, and then plot the numerical inverse Fourier transform of the
result:
\nlscf
\noindent{\tt g(x) = f(x)**2}\newline
\noindent{\tt ifft [-50:49.5:0.5] h(x) of g(x)}\newline
\noindent{\tt plot [-2:2] Re(h(x))}
\nlscf
\begin{center}
\includegraphics{examples/eps/ex_fft2}
\end{center}
\nlscf
As can be seen, the result is a triangle function. This is the result which
would be expected from the convolution theorem, which states that when the
Fourier transforms of two functions are multiplied together and then inverse
transformed, the result is the convolution of the two original functions. The
convolution of a top-hat function with itself is, indeed, a triangle function.
}

\subsection{Window functions}
\index{window functions}

A range of commonly-used window functions may automatically be applied to data
as it is read into the {\tt fft} and {\tt ifft} commands; these are listed
together with their algebraic forms in Table~\ref{tab:windowfuncs} and shown in
Figure~\ref{fig:windowfuncs}. In each case, the window functions are given for
sample number $n$, which ranges between $0$ and $N_x$. The window functions may
be invoked using the following syntax:

\begin{verbatim}
fft [...] <out>() of <in>() window <window_name>
\end{verbatim}

\noindent Where multi-dimensional FFTs are performed, window functions are
applied to each dimension in turn.  Other arbitrary window functions may be
implemented by pre-multiplying data before entry to the {\tt fft} and {\tt
ifft} commands.

\newlength{\wfgap}
\setlength{\wfgap}{30pt}

\begin{table}
\begin{center}
\begin{tabular}{|>{\columncolor{LightGrey}}l>{\columncolor{LightGrey}}l|}
\hline
{\bf Window Name} & {\bf Algebraic Definition} \\
\hline
Bartlett     & $\displaystyle w(n) = \left( \frac{2}{N_x-1} \right) \left( \frac{N_x-1}{2} - \left| n - \frac{N_x-1}{2} \right| \right)$ \vphantom{\rule{0pt}{20pt}}\\
BartlettHann & $\displaystyle w(n) = a_0 - a_1\left|\frac{n}{N_x-1}-\frac{1}{2}\right| - a_2\cos\left(\frac{2\pi n}{N_x-1}\right),\;\textrm{for}$ \vphantom{\rule{0pt}{\wfgap}}\\
             & $a_0=0.62,\; a_1=0.48,\; a_2=0.38.$ \vphantom{\rule{0pt}{20pt}}\\
Cosine       & $\displaystyle w(n) = \cos\left(\frac{\pi n}{N_x-1} - \frac{\pi}{2} \right)$ \vphantom{\rule{0pt}{\wfgap}}\\
Gauss        & $\displaystyle w(n) = \exp \left\{ -\frac{1}{2}\left[ \frac{n-(N_x-1)/2}{\sigma(N_x-1)/2} \right]^2 \right\},\;\textrm{for}\;\sigma=0.5$ \vphantom{\rule{0pt}{\wfgap}}\\
Hamming      & $\displaystyle w(n) = 0.54 - 0.46\cos\left(\frac{2\pi n}{N_x-1}\right)$ \vphantom{\rule{0pt}{\wfgap}}\\
Hann         & $\displaystyle w(n) = 0.5 \left[ 1 - \cos\left(\frac{2\pi n}{N_x-1}\right) \right]$ \vphantom{\rule{0pt}{\wfgap}}\\
Lanczos      & $\displaystyle w(n) = \mathrm{sinc}\left( \frac{2n}{N_x-1} - 1 \right)$ \vphantom{\rule{0pt}{\wfgap}}\\
Rectangular  & $\displaystyle w(n) = 1$ \vphantom{\rule{0pt}{\wfgap}}\\
Triangular   & $\displaystyle w(n) = \left( \frac{2}{N_x} \right) \left( \frac{N_x}{2} - \left| n - \frac{N_x-1}{2} \right| \right)$ \vphantom{\rule{0pt}{\wfgap}}\\
\hline
\end{tabular}
\end{center}
\caption{Window functions available in the {\tt fft} and {\tt ifft} commands.}
\label{tab:windowfuncs}
\end{table}

\begin{figure}
\begin{center}
\includegraphics{examples/eps/ex_windowfuncs}
\end{center}
\caption{Window functions available in the {\tt fft} and {\tt ifft} commands.}
\label{fig:windowfuncs}
\end{figure}

\section{Histograms}
\label{sec:histogram}

The \indcmdt{histogram} takes a single column of data and produces a function
that represents the frequency distribution of the supplied data values. The
output function consists of a series of discrete intervals which we term {\it
bins}. Within each interval the output function has a constant value,
determined such that the area under each interval -- i.e.\ the integral of the
function over each interval -- is equal to the number of datapoints found
within that interval.  The following simple example

\begin{verbatim}
histogram f() 'input.dat'
\end{verbatim}

\noindent produces a frequency distribution of the data values found in the
first column of the file {\tt input.dat}, which it stores in the function
$f(x)$. The value of this function at any given point is equal to the number of
items in the bin at that point, divided by the width of the bins used. If the
input datapoints are not dimensionless then the output frequency distribution
adopts appropriate units, thus a histogram of data with units of length has
units of one over length.

The number and arrangement of bins used by the \indcmdt{histogram} can be
controlled by means of various modifiers.  The \indmodt{binwidth} modifier sets
the width of the bins used. The \indmodt{binorigin} modifier controls where
their boundaries lie; the \indcmdt{histogram} selects a system of bins which,
if extended to infinity in both directions, would put a bin boundary at the
value specified in the {\tt binorigin} modifier. Thus, if {\tt binorigin 0.1}
were specified, together with a bin width of~20, bin boundaries might lie
at~$20.1$, $40.1$, $60.1$, and so on. Alternatively global defaults for the bin
width and the bin origin can be specified using the {\tt set binwidth} and {\tt
set binorigin} commands respectively. The example

\begin{verbatim}
histogram h() 'input.dat' binorigin 0.5 binwidth 2
\end{verbatim}

\noindent would bin data into bins between $0.5$ and $2.5$, between $2.5$ and
$4.5$, and so forth.

Alternatively the set of bins to be used can be controlled more precisely using
the \indmodt{bins} modifier, which allows an arbitrary set of bin boundaries to
be specified. The example

\begin{verbatim}
histogram g() 'input.dat' bins (1, 2, 4)
\end{verbatim}

\noindent would bin the data into two bins, $x=1\to 2$ and $x=2\to 4$.

A range can be supplied immediately following the {\tt histogram} command,
using the same syntax as in the {\tt plot} and {\tt fit} commands; if such a
range is supplied, only points that fall within that range will be binned.  In
the same way as in the {\tt plot} command, the \indmodt{index},
\indmodt{every}, \indmodt{using} and \indmodt{select} modifiers can be used to
specify which subsets of a \datafile\ should be used.

Two points about the {\tt histogram} command are worthy of note. First,
although histograms are similar to bar charts, they are not the same.  A bar
chart conventionally has the height of each bar equal to the number of points
that it represents, whereas a histogram is a continuous function in which the
area underneath each interval is equal to the number of points within it.
Thus, to produce a bar chart using the {\tt histogram} command, the end result
should be multiplied by the bin width used.

Second, if the function produced by the {\tt histogram} command is plotted
using the {\tt plot} command, samples are automatically taken not at evenly
spaced intervals along the abscissa axis, but at the centres of each bin. If
the \indpst{boxes} plot style is used, the box boundaries are conveniently
drawn to coincide with the bins into which the data were sorted.

\section{Random data generation}

Pyxplot has functions for generating random numbers from a variety of common
probability distributions. These functions are in the {\tt random} module:

\begin{itemize}
\item {\tt random.random()} -- returns a random real number between 0 and~1.
\indfun{random.random()}
\item {\tt random.binomial($p,n$)} -- returns a random sample from a binomial distribution with $n$ independent trials and a success probability $p$.
\indfun{random.binomial($p,n$)}
\item {\tt random.chisq($\nu$)} -- returns a random sample from a $\chi$-squared distribution with $\nu$ degrees of freedom.
\indfun{random.chisq($\nu$)}
\item {\tt random.gaussian($\sigma$)} -- returns a random sample from a Gaussian (normal) distribution of standard deviation $\sigma$ and centred on zero.
\indfun{random.gaussian($\sigma$)}
\item {\tt random.lognormal($\zeta,\sigma$)} -- returns a random sample from the log normal distribution centred on $\zeta$, and of width $\sigma$.
\indfun{random.lognormal($\zeta,\sigma$)}
\item {\tt random.poisson($n$)} -- returns a random integer from a Poisson distribution with mean $n$.
\indfun{random.poisson($n$)}
\item {\tt random.tdist($\nu$)} -- returns a random sample from a $t$-distribution with $\nu$ degrees of freedom.
\indfun{random.tdist($\nu$)}
\end{itemize}

\noindent These functions all rely on a common underlying random number
generator\footnote{The gsl library's default random number generator, {\tt
gsl\_\-rng\_\-default} is used. As of version~1.15, this maps to {\tt
gsl\_\-rng\_\-mt19937} with a default seed of zero. The various probability
distributions above are sampled using the functions {\tt
gsl\_\-ran\_\-binomial} and similar.}, whose seed may be set using the
\indcmdt{set seed}, which should be followed by any integer. The sequence of
random samples generated is always the same after setting any particular seed.

When Pyxplot starts, the seed is implicitly set to zero. {\bf This means that
Pyxplot always produces the same series of random numbers when restarted.} This
series can be reproduced by typing:
\begin{verbatim}
set seed 0
\end{verbatim}

\noindent For applications where this repeatability is undesirable, the
following command may help, using the system clock as a random seed:
\begin{verbatim}
set seed time.now().toUnix()
\end{verbatim}

\noindent This gives a different sequence of random numbers each second.
However, the user is advised to consider carefully whether this is sufficient
for the particular application being implemented.

\example{ex:random}{Using random numbers to estimate the value of $\pi$}{
Pyxplot's functions for generating random numbers are most commonly used for
adding noise to artificially-generated data. In this example, however, we use
them to implement a rather inefficient algorithm for estimating the value of
the mathematical constant $\pi$.  The algorithm works by spreading
randomly-placed samples in the square $\left\{ -1<x<1;\; -1<y<1\right\}$. The
number of these which lie within the circle of unit radius about the origin are
then counted. Since the square has an area of $4\,\mathrm{unit}^2$ and the
circle an area of $\pi\,\mathrm{unit}^2$, the fraction of the points which
lie within the unit circle equals the ratio of these two areas: $\pi/4$.
\nlnp
The following script performs this calculation using $N=5000$~randomly placed
samples. Firstly, the positions of the random samples are generated using the
{\tt random()} function, and written to a file called {\tt random.dat} using the
{\tt tabulate} command. Then, the {\tt foreach datum} command -- which will be
introduced in Section~\ref{sec:foreach_datum} -- is used to loop over these,
counting how many lie within the unit circle.
\nlscf
\input{examples/tex/ex_pi_estimation_1.tex}
\nlfcf
On the author's machine, this script returns a value of $3.1352$ when executed
using the random samples which are returned immediately after starting Pyxplot.
This method of estimating $\pi$ is well modelled as a Poisson process, and the
uncertainty in this result can be estimated from the Poisson distribution to be
$1/\sqrt{N}$. In this case, the uncertainty is $0.01$, in close agreement with
the deviation of the returned value of $3.1352$ from more accurate measures of
$\pi$.
\nlnp
With a little modification, this script can be adapted to produce a diagram of
the datapoints used in its calculation. Below is a modified version of the
second half of the script, which loops over the \datapoint s stored in the
\datafile\ {\tt random.dat}. It uses Pyxplot's vector graphics commands, which
will be introduced in Chapter~\ref{ch:vector_graphics}, to produce such a
diagram:

\nlscf
\input{examples/tex/ex_pi_estimation_2.tex}
\nlscf
The graphical output from this script is shown below. The number of datapoints
has been reduced to {\tt Nsamples}$=500$ for clarity:
\nlscf
\begin{center}
\includegraphics{examples/eps/ex_pi_estimation}
\end{center}
}


% programming.tex
%
% The documentation in this file is part of Pyxplot
% <http://www.pyxplot.org.uk>
%
% Copyright (C) 2006-2012 Dominic Ford <coders@pyxplot.org.uk>
%               2008-2012 Ross Church
%
% $Id$
%
% Pyxplot is free software; you can redistribute it and/or modify it under the
% terms of the GNU General Public License as published by the Free Software
% Foundation; either version 2 of the License, or (at your option) any later
% version.
%
% You should have received a copy of the GNU General Public License along with
% Pyxplot; if not, write to the Free Software Foundation, Inc., 51 Franklin
% Street, Fifth Floor, Boston, MA  02110-1301, USA

% ----------------------------------------------------------------------------

% LaTeX source for the Pyxplot Users' Guide

\chapter{Programming: Pyxplot's data types}
\label{chap:progDataTypes}

This chapter describes Pyxplot's built-in object types, which include lists,
dictionaries, vectors, matrices and file handles. A comprehensive list of all
of Pyxplot's object types can be found in Chapter~\ref{ch:types_list}, which
also lists the methods available in each object type.

All objects in Pyxplot, including numbers, have {\it methods}, which act on or
return information about the object.  Some methods are common to all objects:
for example, the {\tt str()} method which returns a string representation of
the object (as used by the {\tt print} command). All objects have a method
called {\tt methods()}, which returns a list of the names of all of methods of
that object:

\begin{verbatim}
print pi.str()
print "My son, it's a wisp of fog.".methods()
\end{verbatim}

\noindent Methods are like functions in that printing them returns brief
documentation about them:

\vspace{3mm}
\input{fragments/tex/prog_methods.tex}
\vspace{3mm}

\section{Instantiating objects}

A list of all of Pyxplot's built-in object types can be found in the {\tt
types} module, which contains the {\tt object prototypes} for each type:

\vspace{3mm}
\input{fragments/tex/prog_types.tex}
\vspace{3mm}

\noindent These object prototypes can be called like functions to produce an
instance of each data type. Each prototype can take various different kinds of
argument; for example, the {\tt number} prototype can take a {\tt number}, {\tt
boolean} or a {\tt string} from which to create a number:

\vspace{3mm}
\input{fragments/tex/prog_types2.tex}
\vspace{3mm}

\noindent Full documentation of the types of inputs supported by each prototype
are listed in Section~\ref{sec:functions_types}.

In many cases there are much more succinct ways of creating objects of each type. For example, lists can be creating by enclosing a comma-separated list of elements in square brackets:

\vspace{3mm}
\input{fragments/tex/prog_types3.tex}
\vspace{3mm}

\noindent Dictionaries can be can be creating by enclosing key--value pairs in curly brackets:

\vspace{3mm}
\input{fragments/tex/prog_types4.tex}
\vspace{3mm}


\section{Strings}
\label{sec:stringvars}
\index{variables!string}

Strings can be enclosed either in single ({\tt '}) or double ({\tt "}) quotes.
Strings may also be enclosed by three quote characters in a row: either {\tt
'''} or {\tt """}. Special care needs to be taken when using apostrophes or
quotes in single-quote delimited strings, as these characters may be
misinterpreted as string delimiters, as in the example:

\begin{dontdo}
'Robert's data'
\end{dontdo}

\noindent This easiest way to avoid such problems is to use three quotes:

\begin{dodo}
'''Robert's data'''
\end{dodo}

Special characters such as tabs and newlines can be inserted into strings using
escape codes such as {\tt $\backslash$t} and {\tt $\backslash$n}; see
Table~\ref{tab:escape_sequences2} for a list of these. The following string is
split over three lines:

\vspace{3mm}
\input{fragments/tex/fs_print4.tex}
\vspace{3mm}

Sometimes these escape codes can be rather annoying, especially when entering
\LaTeX control codes, which all begin with backslash characters. Rather than
having to escape every backslash, it is generally easier to prefix the string
with the character {\tt r}, which turns off all escape codes:

\vspace{3mm}
\input{fragments/tex/fs_print5.tex}
\vspace{3mm}

\begin{table}
\begin{center}
\begin{tabular}{|>{\columncolor{LightGrey}}l>{\columncolor{LightGrey}}l|}
\hline
{\bf Escape sequence} & {\bf Description} \\
\hline
{\tt $\backslash$?} & Question mark \\
{\tt $\backslash$'} & Apostrophe \\
{\tt $\backslash$"} & Double quote \\
{\tt $\backslash\backslash$} & Literal backslash \\
{\tt $\backslash$a} & Bell character \\
{\tt $\backslash$b} & Backspace \\
{\tt $\backslash$f} & Formfeed \\
{\tt $\backslash$n} & Newline \\
{\tt $\backslash$r} & Carriage return \\
{\tt $\backslash$t} & Horizontal tab \\
{\tt $\backslash$v} & Vertical tab \\
\hline
\end{tabular}
\end{center}
\caption{A complete list of Pyxplot's string escape sequences. These are a subset of those available in C.}
\label{tab:escape_sequences2}
\end{table}

Once defined, a string variable can be used anywhere in Pyxplot where a quoted
string could have been used, for example in the {\tt set title} command:

\begin{verbatim}
plotname = "Insert title here"
set title plotname
\end{verbatim}

Strings can be concatenated together using the {\tt +} operator:

\vspace{3mm}
\input{fragments/tex/fs_print6.tex}
\vspace{3mm}

\subsection{The string substitution operator}
\label{sec:stringsubop}

Most string manipulations are performed using the string substitution operator,
{\tt \%}\index{\% operator@{\tt \%} operator}. This operator should be preceded
by a format string, such as {\tt x=\%f}, in which tokens such as {\tt \%f} mark
places where numbers and strings should be substituted. The substitution
operator is followed by a bracketed list of the quantities which should be
substituted in place of these tokens in the format string. This behaviour is
similar to that of the Python programming language's \% operator\footnote{As in
Python, the brackets are optional when only one item is being substituted. For
example, {\tt '\%d'\%2} is equivalent to {\tt '\%d'\%(2)}.} and of the {\tt
printf} statement in C.  For example, to concatenate the two strings contained
in the variables {\tt a} and {\tt b} into a single string variable {\tt c}, one
would issue the command:\index{string operators!concatenation}

\begin{verbatim}
c = "%s%s"%(a,b)
\end{verbatim}

One application of this operator might be to label plots with the title of the
\datafile\ being plotted, as in the following example:
\begin{verbatim}
filename="data_file.dat"
title=r"A plot of the data in {\tt %s}."%(filename)
set title title
plot filename
\end{verbatim}

The syntax of the substitution tokens placed in the format string is similar to
that used by many other languages and is as follows. All substitution tokens
begin with a {\tt \%} character, after which there may be placed, in order:

\begin{enumerate}
\item An optional minus sign, to specify that the substituted item should be left-justified.
\item An optional integer specifying the minimum character width of the substituted item, or a {\tt *} (see below).
\item An optional decimal point/period ({\tt .}) separator.
\item An optional integer, or a {\tt *} (see below), specifying either (a) the maximum number of characters to be printed from a string, or (b) the number of decimal places of a floating-point number to be displayed, or (c) the minimum number of digits of an integer to be displayed, padded to the left with zeros.
\item A conversion character.
\end{enumerate}

\noindent The conversion character is a single character which specifies what
kind of substitution should take place. Its possible values are listed in
Table~\ref{tab:conversion_chars}. Note that where numerical quantities with
physical units are provided, the physical units are not displayed unless the
{\tt \%s} token is used. Although it is not an error to pass a quantity with
physical units to, for example, the {\tt \%f} substitution token, it is good
practice to divide the quantity by a suitable unit first to make it
dimensionless, to be certain of the unit in which it will be displayed.

\begin{table}
\begin{center}
\begin{tabular}{|>{\columncolor{LightGrey}}l>{\columncolor{LightGrey}}p{9cm}|}
\hline
{\bf Character} & {\bf Substitutes} \\
\hline
{\tt d}, {\tt i}   & An integer value. \\
{\tt e}, {\tt E}   & A floating-point value in scientific notation using either the character {\tt e} or {\tt E} to indicate exponentiation. \\
{\tt f}            & A floating-point value without the use of scientific notation. \\
{\tt g}, {\tt G}   & A floating-point value, either using scientific notation, if the exponent is greater than the precision or less than $-4$, otherwise without the use of scientific notation. \\
{\tt o}            & An integer value in octal (base~8). \\
{\tt s}, {\tt S}, {\tt c} & A string, if a string is provided, or a numerical quantity, with units, if such is provided. \\
{\tt x}, {\tt X}   & An integer value in hexadecimal (base~16). \\
{\tt \%}           & A literal {\tt \%} sign. \\
\hline
\end{tabular}
\end{center}
\caption{The conversion characters recognised by the string substitution operator, {\tt \%}.}
\label{tab:conversion_chars}
\end{table}

Where the character {\tt *} is specified for either the character width or the
precision of the substitution token, an integer is read from the list of items
to be substituted, as happens in C's {\tt printf} command:

\vspace{3mm}
\input{fragments/tex/prog_stringsub.tex}
\vspace{3mm}

\subsection{Converting strings to numbers}

Strings which contain numerical data can be converted to numbers by passing
them to the object {\tt types.number()}, as in the example:

\vspace{3mm}
\input{fragments/tex/prog_stringnum.tex}
\vspace{3mm}

\noindent It is an error to try to convert a string to a number if it does not contain a correctly-formatted number:

\begin{dontdo}
types.number("this is not a number")
\end{dontdo}

\subsection{Slicing strings}

Segments of strings can be cut out by using square brackets to slice the string:

\vspace{3mm}
\input{fragments/tex/prog_stringslice.tex}
\vspace{3mm}

\noindent If a single number is placed in the square brackets, and single
character is taken out of the string. If two colon-separated numbers are
specified, {\tt [x:y]} then the substring from character position {\tt x} up to
but not including {\tt y} is returned. If either {\tt x} or {\tt y} are
omitted, then the start or end of the string is used respectively. If either
number of negative, then it counts from the end of the string, $-1$ being the
last character in the string.

\subsection{String methods}

Strings have many methods for performing simple string manipulations. Here we
list their names using the {\tt foreach} command, which will be introduced in
the next chapter:

\vspace{3mm}
\input{fragments/tex/prog_stringmethodlist.tex}
\vspace{3mm}

\noindent Full documentation of them can be found in
Section~\ref{sec:string_methods}. As in {\tt python}, the {\tt strip()} method
removes whitespace characters from the beginning and end of strings, and the
{\tt split()} method splits a string up into whitespace-separated words. The
{\tt splitOn(x)} method splits a string on all occurrences of the sub-string
{\tt x}.  The following simple examples demonstrate the use of some of them:

\vspace{3mm}
\input{fragments/tex/prog_stringmethods.tex}
\vspace{3mm}

\vspace{3mm}
\input{fragments/tex/prog_stringmethods2.tex}
\vspace{3mm}

\subsection{Regular expressions}

String variables can be modified using the search-and-replace string
operator\index{string operators!search and replace}\footnote{Programmers with
experience of {\tt perl} will recognise this syntax.}, =$\sim$\index{=$\sim$
operator}, which takes a regular expression with a syntax similar to that
expected by the shell command {\tt sed}\index{sed shell command@{\tt sed} shell
command} and applies it to the relevant string variable.\footnote{Regular
expression syntax is a massive subject, and is beyond the scope of this manual.
The official GNU documentation for the {\tt sed} command is heavy reading, but
there are many more accessible tutorials on the web.}\index{regular
expressions} In the following example, the first instance of the letter {\tt s} in
the string variable {\tt twister} is replaced with the letters {\tt th}:

\vspace{3mm}
\input{fragments/tex/prog_re1.tex}
\vspace{3mm}

Note that only the {\tt s} (substitute) command of {\tt sed} is implemented in
Pyxplot. Any character can be used in place of the {\tt /} characters in the
above example, for example:

\begin{verbatim}
twister =~ s's'th'
\end{verbatim}

\noindent Flags can be passed, as in {\tt sed} or {\tt perl}, to modify the
precise behaviour of the regular expression. In the following example the {\tt
g} flag is used to perform a global search-and-replace of all instances of the
letter {\tt s} with the letters {\tt th}:

\vspace{3mm}
\input{fragments/tex/prog_re2.tex}
\vspace{3mm}

\noindent Table~\ref{tab:re_flags} lists all of the regular expression flags
recognised by the =$\sim$ operator.

\begin{table}
{\footnotesize
\begin{tabular}{|>{\columncolor{LightGrey}}p{5mm}>{\columncolor{LightGrey}}p{10.5cm}|}
\hline
{\tt g} & Replace {\it all} matches of the pattern; by default, only the first match is replaced. \\
{\tt i} & Perform case-insensitive matching, such that expressions like {\tt [A-Z]} will match lowercase letters, too. \\
{\tt l} & Make {\tt $\backslash$w}, {\tt $\backslash$W}, {\tt $\backslash$b}, {\tt $\backslash$B}, {\tt $\backslash$s} and {\tt $\backslash$S} dependent on the current locale. \\
{\tt m} & When specified, the pattern character {\tt \^{}} matches the beginning of the string and the beginning of each line immediately following each newline. The pattern character {\tt \$} matches at the end of the string and the end of each line immediately preceding each newline. By default, {\tt \^{}} matches only the beginning of the string, and {\tt \$} only the end of the string and immediately before the newline, if present, at the end of the string. \\
{\tt s} & Make the {\tt .} special character match any character at all, including a newline; without this flag, {\tt .} will match anything except a newline. \\
{\tt u} & Make {\tt $\backslash$w}, {\tt $\backslash$W}, {\tt $\backslash$b}, {\tt $\backslash$B}, {\tt $\backslash$s} and {\tt $\backslash$S} dependent on the Unicode character properties database. \\
{\tt x} & This flag allows the user to write regular expressions that look nicer. Whitespace within the pattern is ignored, except when in a character class or preceded by an un-escaped backslash. When a line contains a {\tt \#}, neither in a character class nor preceded by an un-escaped backslash, all characters from the left-most such {\tt \#} through to the end of the line are ignored. \\
\hline
\end{tabular}}
\caption{A list of the flags accepted by the =$\sim$ operator. Most are rarely used, but the {\tt g} flag is very useful.}
\label{tab:re_flags}
\end{table}

\section{Lists}

List objects hold ordered sequences of other Pyxplot objects, which may
include lists and dictionaries to create hierarchical data structures. They are
created by enclosing a comma-separated list of objects by square brackets:

\begin{verbatim}
a = [10,colors.green,"bottles"]
\end{verbatim}

Once created, more items can be added to a list using its {\tt append(item)}
and {\tt insert(n,item)} methods, where the latter inserts an item at position
$n$:

\vspace{3mm}
\input{fragments/tex/prog_listappend.tex}
\vspace{3mm}

\noindent
A complete list of the methods available on lists (itself a list of strings)
can be found by calling the method {\tt [].methods()}; they are also listed in
Section~\ref{sec:list_methods}. As with string methods, documentation of list
methods is returned if the method object is printed:

\vspace{3mm}
\input{fragments/tex/prog_listmethods.tex}
\vspace{3mm}

Most methods that operate on lists, for example, append, extend and sort
operations, return the list as their output. Unless this is stored in a
variable, Pyxplot prints this return value to the terminal. In some cases this
is useful: in the example above, it allowed us to see how the list was changing
when we called its {\tt append()} and {\tt insert()} methods. Often, however,
this terminal spam is unwanted. The \indcmdt{call} allows methods to be called
without printing their output, which is discarded:

\vspace{3mm}
\input{fragments/tex/prog_listappend2.tex}
\vspace{3mm}

\subsection{Using lists as stacks}

The following example demonstrates the use of a list as a stack; note that the
last item added to the stack is the first one to be popped:

\vspace{3mm}
\input{fragments/tex/prog_liststack.tex}
\vspace{3mm}

\subsection{Using lists as buffers}

The following example demonstrates the use of a list as a buffer in which the
first item added to the stack is the first one to be popped:

\vspace{3mm}
\input{fragments/tex/prog_listbuffer.tex}
\vspace{3mm}

\noindent In the final line, we make use of the fact that a list tests true if
it contains any items, or false if it is empty.

\subsection{Sorting lists}

Methods are provided for sorting data in lists. The simplest of these is the
{\tt sort()} method, which sorts the members of a list into order of ascending
value.\footnote{Non-numeric items are assigned arbitrary but consistent values
for the purposes of sorting.  Booleans are always lower-valued than numbers,
but numbers are lower-valued than lists. Longer lists are always higher valued
than shorter lists; larger dictionaries are always higher-valued than smaller
dictionaries.} The {\tt reverse()} method can be used to invert the order of
the list afterwards if descending order is wanted.

\vspace{3mm}
\input{fragments/tex/prog_listsort.tex}
\vspace{3mm}

\subsubsection{Custom sorting}

Often, however, a custom ordering is wanted. The {\tt sortOn(f)} method takes a
function of two arguments as its input. The function {\tt f(a,b)} should return
$-1$ if {\tt a} is to be placed before {\tt b} in the sorted list, $1$ if {\tt
a} is to be placed after {\tt b} in the sorted list, and zero if the two
elements have equal ranking.

\noindent The {\tt cmp(a,b)} function is often useful in making comparison
functions for use with the {\tt sortOn(f)} method: it returns either $-1$, $0$
or $1$ depending on Pyxplot's default way of comparing two objects. In the
example below, we pass it the magnitude of {\tt a} and {\tt b} to sort a list
in order of magnitude.

\vspace{3mm}
\input{fragments/tex/prog_listsort2.tex}
\vspace{3mm}

In this example, the {\tt range(start,end,step)} function is used to generate a
raster of values between $-8$ and $8$. It outputs a vector, which is converted
into a list using the vector's {\tt list()} method. More information about
vectors is in Section~\ref{sec:vectors}.

The subroutine command, which is often used to implement more complicated
sorting functions, will be covered in Section~\ref{sec:subroutines}. For
example, the function used above could have been written:

\begin{verbatim}
subroutine absCmp(a,b)
 {
  return cmp(abs(a),abs(b))
 }
\end{verbatim}

\subsubsection{Sorting lists of lists}

The {\tt sortOnElement(n)} method can be used to sort a list of lists on the
$n$th sub-element of each sublist.

\vspace{3mm}
\input{fragments/tex/prog_listsort3.tex}
\vspace{3mm}

\subsection{Iterating over lists}

The \indcmdt{foreach} can be used to iterate over the members of a list; it
will be covered in more detail in Section~\ref{sec:foreach}. The following
example iterates over the words in a sentence:

\vspace{3mm}
\input{fragments/tex/prog_listiter.tex}
\vspace{3mm}

\subsection{Calling functions with lists of arguments}

The \indfunt{call($f,a$)} function can be used to call a function with an
arbitrary list of arguments:

\vspace{3mm}
\input{fragments/tex/prog_listcall.tex}
\vspace{3mm}

\subsection{List mapping and filtering}
\label{sec:listfilter}

The methods {\tt filter(f)}, {\tt map(f)} and {\tt reduce(f)} can be used to
perform actions on all of the members of a list in turn. {\tt filter(f)} takes
a function of one argument as its argument, and returns a new list of all of
the members {\tt x} of the original list for which {\tt f(x)} tests true. For
example:

\vspace{3mm}
\input{fragments/tex/prog_listfilter.tex}
\vspace{3mm}

The method {\tt map(f)} also takes a function of one argument as its argument,
and returns a list of the results {\tt f(x)} for each of the members {\tt x} of
the original list. In other words, if {\tt f} were {\tt sin}, and the original
list contained values of {\tt x}, the result would be a list of values of {\tt
sin(x)}. This example converts a list of numbers into Roman numerals:

\vspace{3mm}
\input{fragments/tex/prog_listmap.tex}
\vspace{3mm}

The method {\tt reduce(f)} takes a pointer to a function of two arguments. It
first calls $f(a,b)$ on the first two elements of the list, and then continues
through the list calling $f(a,b)$ on the result and the next item in the list.
The final result is returned:

\vspace{3mm}
\input{fragments/tex/prog_listreduce.tex}
\vspace{3mm}


\subsection{Vectors versus lists}

Vectors are similar to lists, except that all of their elements must be real
numbers, and that all of the elements of any given vector must share common
physical dimensions.  Vectors are stored much more efficiently in memory than
lists, since information about the types and physical units of each of the
elements need not be stored. In addition they support a wide range of vector
and matrix arithmetic operations.

Data from lists can also be plotted onto graphs, but the list must first be
converted into a vector. See~\ref{sec:vectors} for more information.

\section{Dictionaries}

Dictionaries, also known as associative arrays or content-addressable memories
in other programming languages, store collections of objects, each of which has
a unique name (or key). Objects are addressed by name, rather than by number:

\vspace{3mm}
\input{fragments/tex/prog_dictinit.tex}
\vspace{3mm}

As the first line of this example shows, dictionaries can be created by
enclosing a list of key--value pairs in curly brackets. As in python, a colon
separates each key from its corresponding value, while the list of key--value
pairs are comma-separated. That is, the general syntax is:
\begin{verbatim}
{ key1:value1 , key2:value2 , ... }
\end{verbatim}

It is also possible to generate an empty dictionary, as {\tt \{\}}. Items can
later be referenced or assigned by name, where the name is placed in square
brackets after the name of the dictionary. Items can be deleted with the
dictionary's {\tt delete(key)} method.

It is not an error to assign an item to a name which is already defined in the
dictionary; the new assignment overwrites the old object with that name. It is,
however, an error to attempt to access a key which is not defined in the
dictionary. The method {\tt hasKey(key)} may be used to test whether a key is
defined before attempting to access it.

Unlike in python, keys {\bf must} be strings.

\section{Vectors and matrices}
\label{sec:vectors}

Vectors are similar to lists, except that all of their elements must be real
numbers, and that all of the elements of any given vector must share common
physical dimensions.  Vectors are stored much more efficiently in memory than
lists, since information about the types and physical units of each of the
elements need not be stored. In addition they support a wide range of vector
and matrix arithmetic operations.

For example, applying the addition {\tt +} operator to two lists concatenates
the lists together, meanwhile the same operator applied to two vectors performs
vector addition:

\vspace{3mm}
\input{fragments/tex/prog_vectoradd.tex}
\vspace{3mm}

In fact, whilst vectors do support the same {\tt append} and {\tt extend}
methods as lists, to add either a single new element, or a list of new
elements, to the end of the vector, these are very time consuming methods to
run. It is much more efficient to create a vector of the desired length, and
then to populate it with elements:

\vspace{3mm}
\input{fragments/tex/prog_vectorfill.tex}
\vspace{3mm}

As the above example demonstrates, the {\tt vector()} prototype can take not
only a list or vector from which to make a vector object copy, but
alternatively a single integer, which creates a zeroed vector of the specified
length.

Similarly, the {\tt matrix()} prototype can create a matrix from a list of
lists, a list of vectors, a series of list arguments, a series of vector
arguments, or two integers. In the final case, a zero matrix with the specified
number of rows and columns is returned. If a matrix is specified as a series of
vectors, these are taken to be the columns of the matrix; but if the matrix is
specified as a series of lists, these are taken to be the rows of the matrix:

\vspace{3mm}
\input{fragments/tex/prog_matrixinit.tex}
\vspace{3mm}

Like vectors, matrices can have physical units, and adding two matrices together performs element-wise addition:

\vspace{3mm}
\input{fragments/tex/prog_matrixadd.tex}
\vspace{3mm}

\subsection{Dot and cross products}

The dot product of two vectors can be found simply by multiplying the two vectors together:

\vspace{3mm}
\input{fragments/tex/prog_vectoradd.tex}
\vspace{3mm}

\noindent
The cross product of two vectors, which is only defined for pairs of three-element vectors, can be found by passing the two vectors to the {\tt cross(a,b)} function:

\vspace{3mm}
\input{fragments/tex/prog_vectorcross.tex}
\vspace{3mm}


\subsection{Matrix algebra}

Matrices can be multiplied by one another and by vectors to perform matrix
arithmetic. This not only allows matrix equations to be solved, but also allows
transformation matrices to be applied to vector positions on the vector
graphics canvas. All of Pyxplot's vector graphics commands, which will be
described in detail in Chapter~\ref{ch:vector_graphics} can accept positions on
the canvas as either comma-separated numerical components, or as vector
objects. The following example demonstrates the use of a rotation matrix:

\begin{verbatim}
rotate(a) = matrix( [[cos(a),-sin(a)], \
                     [sin(a), cos(a)] ] )
pos = vector(0,5)*unit(cm)
theta = 30*unit(deg)
arrow from 0,0 to rotate(theta)*pos with linewidth 3
\end{verbatim}

In addition to matrix multiplication, other arithmetic operations are available
via the methods of matrix objects. Their methods {\tt diagonal()} and {\tt
symmetric()} return {\tt true} or {\tt false} as appropriate. Their {\tt size()}
method returns the vector size of the matrix (rows, columns). Their {\tt det()}
method returns the determinate of the matrix and their {\tt transpose()} method
returns the matrix transpose.

Among more complex operations, {\tt inv()} returns the inverse of a matrix,
{\tt eigen\-values()} returns a vector of the matrix's eigen\-values, and {\tt
eigen\-vectors()} returns a list of the matrix's corresponding eigen\-vectors.

\subsection{Plotting data from vectors}

Vectors can be used to pass calculated data to the \indcmdt{plot} for plotting.
Instead of supplying the name of a \datafile, or a function to be plotted, a
series of colon-separated vector objects should be passed to the
\indcmdt{plot}. Each of the vectors should be the same length; the $n$th
elements of each of the vectors are put together to form the columns of data
for the $n$th \datapoint.

The following example draws 100 random points on a graph:

\begin{verbatim}
N=100
a=vector(N) ; b=vector(N)
for i=0 to 99 { a[i]=random.random() ; }
for i=0 to 99 { b[i]=random.random() ; }
plot [0:1][0:1] a:b
\end{verbatim}

Vectors support the same {\tt filter()}, {\tt map()} and {\tt reduce()} methods
as lists (see Section~\ref{sec:listfilter}), and these can prove especially
useful for preparing data for plotting. The following example selects fifty
random points along the $x$-axis, and uses them to plot $\sin(x)$:

\begin{verbatim}
N=50
a=vector(N)
for i=0 to 99 { a[i]=random.random() ; }
b=a.map(sin)
plot [0:1][0:1] a:b
\end{verbatim}

\section{Colors}

Most of Pyxplot's graph plotting and vector graphics commands have settings for
specifying colors. A selection of widely-used colors may be specified by name,
for example {\tt red} and {\tt blue}. However, greater freedom in choice of
color is available by passing these commands objects of type {\tt color}.

Several functions are available for making color objects:

\begin{itemize}
\item {\tt gray(x)} returns a shade of gray. The argument $x$ should be in the range 0--1. If $x=0$, black is returned; if $x=1$, white is returned.
\item {\tt rgb(r,g,b)} returns a color with the specified RGB components, which should be in the range 0--1.
\item {\tt cmyk(c,m,y,k)} returns a color with the specified CMYK components, which should be in the range 0--1.
\item {\tt hsb(h,s,b)} returns a color with the specified coordinates in hue--saturation--brightness color space, which should be in the range 0--1.
\end{itemize}

\noindent In addition, color objects corresponding to all of Pyxplot's built-in
named colors can be found in the {\tt colors} module.

\begin{verbatim}
a = colors.red
b = rgb(0,0.5,0)
box from 0,0 to 3,3 with color a fillcolor b lw 5
\end{verbatim}

Once a color object has been made, various operations are supported.
Multiplying or dividing a color by a number changes the brightness of the
color. When two colors are compared, brighter colors are greater than darker
colors. When two colors are added together, they are additively mixed in RGB
space, so that adding red and green together produces yellow. When one color is
subtracted from another, the opposite happens, so that yellow minus green is
red.

\noindent The methods available on {\tt color} objects are listed in
Section~\ref{sec:color_methods}.

\subsection{Color representations of the electromagnetic spectrum}

Two functions, in the {\tt colors} module, provide color objects which
approximate the color of particular wavelengths of light, or of electromagnetic
spectra.

\funcdef{colors.wavelength($\lambda$,$norm$)}{returns a color representation of monochromatic light at wavelength $\lambda$, normalised to brightness $norm$. A value of $norm=1$ is recommended for plotting the complete span of the electromagnetic spectrum without colors clipping to white.}
\funcdef{colors.spectrum($spec,norm$)}{returns a color representation of the spectrum $spec$, normalised to brightness $norm$. $spec$ should be a function object that takes a single input (wavelength) with units of length, and may return an output with arbitrary units.}

\noindent For an example of the use of these functions, see
Section~\ref{sec:colormaps}.

\section{Dates}

Pyxplot has a {\tt date} object type which simplifies the process of working
with dates and times.  Pyxplot provides a range of pre-defined functions, in
the {\tt time} module, for creating and manipulating {\tt date} objects. The
functions for creating {\tt date} objects are as follows:

\funcdef{time.fromCalendar($year,month,day,hour,min,sec$)}{creates a date
object from the specified calendar date. It takes six inputs: the year, the
month number (1--12), the day of the month (1--31), the hour of day (0--24),
the number of minutes (0--59), and the number of seconds (0--59). To enter
dates before {\footnotesize AD}\,1, a year of~$0$ should be passed to indicate
1\,{\footnotesize BC}, $-1$ should be passed to indicate the year
2\,{\footnotesize BC}, and so forth. The \texttt{set calendar} command is used
to change the current calendar.}
\funcdef{time.fromJD($t$)}{creates a date object from the specified numerical Julian date.}
\funcdef{time.fromMJD($t$)}{creates a date object from the specified numerical modified Julian date.}
\funcdef{time.fromUnix($t$)}{creates a date object from the specified numerical Unix time.}
\funcdef{time.now()}{creates a date object representing the present time.}

The following example creates a date object representing midnight on 1st January 2000:

\vspace{3mm}
\input{fragments/tex/calc_date1.tex}
\vspace{3mm}

Once created, it is possible to add numbers with physical units of time to
dates, as in the following example:

\vspace{3mm}
\input{fragments/tex/calc_date3.tex}
\vspace{3mm}

\noindent In addition, if one date is subtracted from
another date, the time interval between the two dates is returned as a number
with physical dimensions of time:

\vspace{3mm}
\input{fragments/tex/calc_interval.tex}
\vspace{3mm}

Standard string representations of calendar dates can be produced with the {\tt
print} command.  It is also possible to use the string substitution operator,
as in {\tt "\%s"\%(date)}, or the {\tt str} method of {\tt date} objects, as in
{\tt date.str()}.  In addition, the {\tt time.string} function can be used to
choose a custom display format for the date; for more information, see
Section~\ref{sec:time_series}.

Several functions are provided for converting {\tt date} objects back into various numerical forms of timekeeping and components of calendar dates:

\methdef{toDayOfMonth()}{returns the day of the month of a date object in the current calendar.}
\methdef{toDayWeekName()}{returns the name of the day of the week of a date object.}
\methdef{toDayWeekNum()}{returns the day of the week (1--7) of a date object.}
\methdef{toHour()}{returns the integer hour component (0--23) of a date object.}
\methdef{toJD()}{converts a date object to a numerical Julian date.}
\methdef{toMinute()}{returns the integer minute component (0--59) of a date object.}
\methdef{toMJD()}{converts a date object to a modified Julian date.}
\methdef{toMonthName()}{returns the name of the month in which a date object falls.}
\methdef{toMonthNum()}{returns the number (1--12) of the month in which a date object falls.}
\methdef{toSecond()}{returns the seconds component (0--60) of a date object, including the non-integer component.}
\methdef{toUnix()}{converts a date object to a Unix time.}
\methdef{toYear()}{returns the year in which a date object falls in the current calendar.}

For example:

\vspace{3mm}
\input{fragments/tex/calc_date2.tex}
\vspace{3mm}

More information on the manipulation of {\tt date} objects can be found in Section~\ref{sec:time_series}.

\section{Modules and classes}

Modules provide a convenient way to group functions and variables together,
just as Pyxplot's built-in functions are grouped into modules such as {\tt os}
and {\tt random}.  New modules can be created by calling the {\tt module}
object, which is a synonym for {\tt types.module}. Once created, a module is
like a dictionary, except that its elements can be accessed both as {\tt
module[item]} and more commonly as {\tt module.item}:

\vspace{3mm}
\input{fragments/tex/prog_mod1.tex}
\vspace{3mm}

Modules can also serve as class prototypes. If a module is called like a
function, the return value is an {\it instance} of the module:

\vspace{3mm}
\input{fragments/tex/prog_mod2.tex}
\vspace{3mm}

The module {\it instance} inherits all of the functions and variables of its
parent object, but may also contain its own additional functions and variables,
some of which may supersede those in the parent object if they have the same
name. When functions or subroutines of a module instance are called, the
special variable {\tt self} is defined to equal the module instance object.
This allows the function to store private data in the module instance, or to
call other methods on the instance.

\vspace{3mm}
\input{fragments/tex/prog_mod3.tex}
\vspace{3mm}

As this example demonstrates, it is also possible to hierarchically instantiate
modules: {\tt tiddles} is an instance of {\tt cat}, which is itself an instance
of {\tt animal}.

\section{File handles}

File handles provide a means of reading data directly from text files in
Pyxplot scripts, or of writing data to files. Files are opened using the {\tt
open()} function:

\vspace{2mm}
\funcdef{open($x$[,$y$])}{opens the file $x$ with string access mode $y$, and returns a file handle object.}
\vspace{4mm}

\noindent The most commonly used access modes are {\tt "r"}, to open a file read-only, {\tt "w"}, to open a file for writing, erasing any pre-existing file of the same filename, and {\tt "a"}, to append data to the end of a file.

Alternatively, if what is wanted is a temporary scratch space, the {\tt os.tmpfile()} function should be used:

\vspace{2mm}
\funcdef{os.tmpfile()}{returns a file handle for a temporary file.}
\vspace{4mm}

\noindent The resulting file handle is open for both reading and writing.

The following methods are defined for file handles:

\methdef{close()}{closes a file handle.}
\methdef{dump($x$)}{stores a typeable ASCII representation of the object $x$ to a file. Note that this method has no checking for recursive hierarchical data structures.}
\methdef{eof()}{returns a boolean flag to indicate whether the end of a file has been reached.}
\methdef{flush()}{flushes any buffered data which has not yet physically been written to a file.}
\methdef{getPos()}{returns a file handle's current position in a file.}
\methdef{isOpen()}{returns a boolean flag indicating whether a file is open.}
\methdef{read()}{returns the contents of a file as a string.}
\methdef{readline()}{returns a single line of a file as a string.}
\methdef{readlines()}{returns the lines of a file as a list of strings.}
\methdef{setPos($x$)}{sets a file handle's current position in a file.}
\methdef{write($x$)}{writes the string $x$ to a file.}

\subsection{Storing data structures in text files}

The {\tt dump(x)} method of file handles is provided as a means of writing a
typeable ASCII representation of the object {\tt x} to file, for later recovery
using the {\tt load} command. It is similar to the {\tt pickle()} function in
Python.

There is no limit to the depth to which it will traverse hierarchically nested
data structures, and will produce output of infinite length if there is
recursive nesting.

Note that it is not able to store representations of function definitions or
file handles, which are stored as null objects; class instances lose their
relationship with their parents and are stored as free-standing modules.

% flowctrl.tex
%
% The documentation in this file is part of Pyxplot
% <http://www.pyxplot.org.uk>
%
% Copyright (C) 2006-2012 Dominic Ford <coders@pyxplot.org.uk>
%               2008-2012 Ross Church
%
% $Id$
%
% Pyxplot is free software; you can redistribute it and/or modify it under the
% terms of the GNU General Public License as published by the Free Software
% Foundation; either version 2 of the License, or (at your option) any later
% version.
%
% You should have received a copy of the GNU General Public License along with
% Pyxplot; if not, write to the Free Software Foundation, Inc., 51 Franklin
% Street, Fifth Floor, Boston, MA  02110-1301, USA

% ----------------------------------------------------------------------------

% LaTeX source for the Pyxplot Users' Guide

\chapter{Programming: flow control}

This chapter describes Pyxplot's facilities for automating
repetitive tasks by using loops. At the end, we turn to Pyxplot's
interaction with the shell and filing system in which it operates, introducing
a simple framework for automatically re-executing Pyxplot scripts whenever they
change, allowing plots to be automatically regenerated whenever the scripts
used to produce them are modified.

\section{Conditionals}

The {\tt if} statement\indcmd{if} can be used to conditionally execute a series
of commands only when a certain criterion is satisfied. In its simplest form,
its syntax is

\begin{verbatim}
if <expression> { .... }
\end{verbatim}

\noindent where the expression can take the form of, for example, {\tt x<0} or
{\tt y==1}. Note that the operator {\tt ==} is used to test the equality of two
algebraic expressions; the operator {\tt =} is only used to assign values to
variables and functions. A full list of the operators available can be found in
Table~\ref{tab:operators_table}. As in many other programming languages,
algebraic expressions are deemed to be true if they evaluate to any non-zero
value, and false if they exactly equal zero. Thus, the following two examples
are entirely legal syntax, and the first {\tt print} statement will execute,
but the second will not:

\begin{verbatim}
if 2*3 {
  print "2*3 is True"
 }

if 2-2 {
  print "2-2 is False"
 }
\end{verbatim}

\noindent The variables {\tt true} and {\tt false} are predefined constants, making the following syntax legal:

\begin{verbatim}
if false {
  print "Never gets here"
 }
\end{verbatim}

As in C, the block of commands which are to be conditionally executed is
enclosed in braces (i.e.\ {\tt \{~\}}).  The closing brace must be on a line by
itself at the end of the block, or separated from the last command in the block
by a semi-colon.

\begin{verbatim}
if (x==0)
 {
  print "x is zero"
 }

if (x==0) { print "x is zero" ; }
\end{verbatim}

After such an {\tt if} clause, it is possible to string together further
conditions in {\tt else if} clauses, perhaps with a final {\tt else} clause, as
in the example:

\begin{verbatim}
if (x==0)
 {
  print "x is zero"
 } else if (x>0) {
  print "x is positive"
 } else {
  print "x is negative"
 }
\end{verbatim}

Here, as previously, the first script block is executed if the first
conditional, {\tt x==0}, is true. If this script block is not executed, the
second conditional, {\tt x>0}, is then tested. If this is true, then the second
script block is executed.  The final script block, following the {\tt else}, is
executed if none of the preceding conditionals have been true. Any number of
{\tt else if} statements can be chained one after another, and a final {\tt
else} statement can always be optionally supplied. The {\tt else} and {\tt else
if} statements must always be placed on the same line as the closing brace of
the preceding script block.

The precise way in which a string of {\tt else if} statements are arranged in a
Pyxplot script is a matter of taste: the following is a more compact but
equivalent version of the example given above:

\begin{verbatim}
if      (x==0) { print "x is zero"     ; } \
else if (x> 0) { print "x is positive" ; } \
else           { print "x is negative" ; }
\end{verbatim}

\section{For loops}
\indcmd{for}

For loops may be used to execute a series of commands multiple times. Pyxplot
allows {\tt for} loops to follow either the syntax of the BASIC programming
language, or the C syntax:

\begin{verbatim}
for <variable> = <start> to <end> [step <step>]
                          [loopname <loopname>]
  <code>

for (<initialise>; <criterion>; <step>)
  <code>
\end{verbatim}

\noindent Here, {\tt <code>} may be substituted by any block on Pyxplot command
enclosed in braces \{\}. The closing brace must be on a new line after the last
command of the block.

The first form is similar to how the \indcmdt{for} works in BASIC.  The first
time that the script block is executed, the variable named at the start of the
{\tt for} statement has the value given for {\tt start}.  Upon each iteration
of the loop, this is incremented by amount {\tt step}. The loop finishes when
the value exceeds {\tt end}. If {\tt step} is negative, then {\tt end} is
required to be less than or equal to {\tt start}. A step size of zero is
considered to be an error.  The iterator variable can have any physical
dimensions, so long as {\tt start}, {\tt end} and {\tt step} all have the same
dimensions, but the iterator variable must always be a real number. If no step
size is given then a step size of unity is assumed.  As an example, the
following script would print the numbers 0, 2, 4, 6 and 8:

\begin{verbatim}
for x = 0 to 10 step 2
 {
  print x
 }
\end{verbatim}

In the C form of the \indcmdt{for}, three expressions are provided, separated
by semicolons. These are evaluated (a) when the loop initialises, (b) as a
boolean test of whether the loop should continue iterating, and (c) on each
loop to increment/decrement variables as required. For example:

\begin{verbatim}
for (i=1,j=1; i<=256; i*=2,j++) { print "%3d %3d"%(j,i); }
\end{verbatim}

The syntax

\begin{verbatim}
for (a; b; c) { ... ; }
\end{verbatim}

is equivalent to

\begin{verbatim}
a; while (b) { ... ; c ; }
\end{verbatim}

The optional {\tt loopname} which can be specified in the {\tt for} statement
is used in conjunction with the {\tt break} and {\tt continue} statements which
will be introduced in Section~\ref{sec:breakcontinue}.

\section{Foreach loops}
\indcmd{foreach}
\index{wildcards}

Foreach loops may be used to run a script block once for each item in a list.
The list may either take the form of a list object, or a filename wildcard, as
in the following examples:

\begin{verbatim}
foreach x in [-1,pi,10]
 {
  print x
 }

foreach x in "*.dat"
 {
  print x
 }
\end{verbatim}

The first of these loops would iterate three times, with the variable {\tt x}
holding the values $-1$, $\pi$ and $10$ in turn. The second of these loops
would search for any \datafile s in the user's current directory with filenames
ending in {\tt .dat} and iterate for each of them. As previously, the wildcard
character {\tt *} matches any string of characters, and the character {\tt ?}
matches any single character. Thus, {\tt foo?.dat} would match {\tt foo1.dat}
and {fooX.dat}, but not {\tt foo.dat} or {\tt foo10.dat}. The effect of the
{\tt print} statement in this particular example would be rather similar to
typing:

\begin{verbatim}
!ls *.dat
\end{verbatim}

An error is returned if there are no files in the present directory which match
the supplied wildcard. The following example would produce plots of all of the
\datafile s in the current directory with filenames {\tt foo\_*.dat} or {\tt
bar\_*.dat} as {\tt eps} files with matching filenames:

\begin{verbatim}
set terminal eps
foreach x in "foo_*.dat" "bar_*.dat"
 {
  outfilename =  x
  outfilename =~ s/dat/eps/
  set output outfilename
  plot x using 1:2
 }
\end{verbatim}

\section{Foreach datum loops}
\label{sec:foreach_datum}

Foreach datum loops are similar to foreach loops in that they run a script
block once for each item in a list.  In this case, however, the list in
question is the list of \datapoint s in a \datafile, samples of a function, or
values in a vector. The syntax of the \indcmdt{foreach datum} is similar to
that of the commands met in the previous chapter for acting upon \datafile s:
the standard modifiers {\tt every}, {\tt index}, {\tt select} and {\tt using}
can be used to select which columns of the \datafile, and which subset of the
datapoints, should be used:

\begin{verbatim}
foreach datum i,j,name in "data.dat" using 1:2:"%s"%($3)
  <code>

foreach datum x,y,z in sin(x):cos(x)
  <code>

foreach datum a,b in vector_a:vector_b
  <code>
\end{verbatim}

The \indcmdt{foreach datum} is followed by a comma-separated list of the
variable(s) which are to be read from the input data on each iteration of the
loop. The {\tt using} modifier specifies the columns or rows of data which are
to be used to set the values of each variable. In the first example above, the
third variable, {\tt name}, is set as a string, indicating that it will be set
to equal whatever string of text is found in the third column of the \datafile.

\example{ex:meansd}{Calculating the mean and standard deviation of data}{
The following Pyxplot script calculates the mean and standard deviation of a
set of \datapoint s using the \indcmdt{foreach datum}:
\nlscf
\noindent{\tt N\_data = 0}\newline
\noindent{\tt sum\_x  = 0}\newline
\noindent{\tt sum\_x2 = 0}\newline
\\
\noindent{\tt foreach datum x in '--'}\newline
\noindent{\tt \phantom{x}\{}\newline
\noindent{\tt \phantom{xx}N\_data ++}\newline
\noindent{\tt \phantom{xx}sum\_x  += x}\newline
\noindent{\tt \phantom{xx}sum\_x2 += x**2}\newline
\noindent{\tt \phantom{x}\}}\newline
\noindent{\tt 1.3}\newline
\noindent{\tt 1.2}\newline
\noindent{\tt 1.5}\newline
\noindent{\tt 1.1}\newline
\noindent{\tt 1.3}\newline
\noindent{\tt END}\newline
\\
\noindent{\tt mean = sum\_x / N\_data}\newline
\noindent{\tt SD   = sqrt(sum\_x2 / N\_data - mean**2)}\newline
\\
\noindent{\tt print "Mean = \%s"\%mean}\newline
\noindent{\tt print "SD   = \%s"\%SD}
\nlscf
\noindent For the data supplied, a mean of $1.28$ and a standard deviation of
$0.133$ are returned.
}

\section{While and do loops}

The \indcmdt{while} may be used to continue running a script block until some
stopping criterion is met. Two types of while loop are supported:

\begin{verbatim}
while <criterion> [ loopname <name> ]
 {
  ....
 }

do [ loopname <name> ]
 {
  ....
 } while <criterion>
\end{verbatim}
\indcmd{do}

In the former case, the enclosed script block is executed repeatedly, and the
algebraic expression supplied to the \indcmdt{while} is tested immediately
before each repetition. If it tests false, then the loop finishes.  The latter
case is very similar, except that the supplied algebraic expression is tested
immediately {\it after} each repetition. Thus, the former example may never
actually execute the supplied script block if the looping criterion tests false
upon the first iteration, but the latter example is always guaranteed to run
its script block at least once.

The following example would continue looping indefinitely until stopped by the
user, since the value {\tt 1} is considered to be true:

\begin{verbatim}
while (1)
 {
  print "Hello, world!"
 }
\end{verbatim}

\section{The {\tt break} and {\tt continue} statements}
\label{sec:breakcontinue}
\indcmd{break}
\indcmd{continue}

The {\tt break} and {\tt continue} statements may be placed within loop
structures to interrupt their iteration. The {\tt break} statement terminates
execution of the smallest loop currently being executed, and Pyxplot resumes
execution at the next statement after the closing brace which marks the end of
that loop structure. The {\tt continue} statement terminates execution of the
{\it current iteration} of the smallest loop currently being executed, and
execution proceeds with the next iteration of that loop, as demonstrated by the
following pair of examples:

\vspace{3mm}
\input{fragments/tex/flow_break.tex}
\vspace{3mm}

Note that if several loops are nested, the {\tt break} and {\tt continue}
statements only act upon the innermost loop. If either statement is encountered
outside of a loop structure, an error results. Optionally, the {\tt for}, {\tt
foreach}, {\tt do} and {\tt while} commands may be supplied with a name for the
loop, prefixed by the word {\tt loopname}, as in the examples:

\begin{verbatim}
for i=0 to 4 loopname iloop
...
foreach i in "*.dat" loopname DatafileLoop
...
\end{verbatim}

\noindent When loops are given such names, the {\tt break} and {\tt continue}
statements may be followed by the name of the loop to be broken out of,
allowing the user to act upon loops other than the innermost one.

\section{The conditional operator}
\label{sec:conditional_operator}

The conditional operator provides a compact means of inserting conditional
expressions.  Following the syntax of C, it takes three arguments and is
written as {\tt a ? b : c}. The first argument, {\tt a} is a truth criterion to
be tested. If the criterion is true, then the operator returns its second
argument, {\tt b} as its output. Otherwise, the function's third argument {\tt
c} is returned.

\vspace{2mm}
\input{fragments/tex/flow_conditional.tex}

\section{Subroutines}
\indcmd{subroutine}
\label{sec:subroutines}

Subroutines are similar to mathematical functions (see
Section~\ref{sec:functions}), and once defined, can be used anywhere in
algebraic expressions, just as functions can be.  However, instead of being
defined by a single algebraic expression, whenever a subroutine is evaluated, a
block of Pyxplot commands of arbitrary length is executed. This gives much
greater flexibility for implementing complex algorithms. Subroutines are
defined using the following syntax:
\begin{verbatim}
subroutine <name>(<variable1>,...)
 {
  ...
  return <value>
 }
\end{verbatim}
Where {\tt name} is the name of the subroutine, {\tt variable1} is an argument
taken by the subroutine, and the value passed to the {\tt return} statement is
the value returned to the caller. Once the {\tt return} statement is reached,
execution of the subroutine is terminated. The following two examples would
produce entirely equivalent results:
\begin{verbatim}
f(x,y) = x*sin(y)

subroutine f(x,y)
 {
  return x*sin(y)
 }
\end{verbatim}
In either case, the function/subroutine could be evaluated by typing:
\begin{verbatim}
print f(1,pi/2)
\end{verbatim}
If a subroutine ends without any value being returned using the {\tt return}
statement, then a value of zero is returned.

Subroutines may serve one of two purposes. In many cases they are used to
implement complicated mathematical functions for which no simple algebraic
expression may be given. Secondly, they may be used to repetitively execute a
set of commands whenever they are required. In the latter case, the subroutine
may not have a return value, but may merely be used as a mechanism for
encapsulating a block of commands.  In this case, the \indcmdt{call} may be
used to execute a subroutine, discarding any return value which it may produce,
as in the example:

\vspace{3mm}
\input{fragments/tex/flow_subroutine.tex}
\vspace{3mm}

\example{ex:newton}{An image of a Newton fractal}{
Newton fractals are formed by iterating the equation
\begin{displaymath}
z_{n+1} = z_n - \frac{f(z_n)}{f^\prime(z_n)},
\end{displaymath}
subject to the starting condition that $z_0=c$, where $c$ is any complex number
$c$ and $f(z)$ is any mathematical function. This series is the Newton-Raphson
method for numerically finding solutions to the equation $f(z)=0$, and with
time usually converges towards one such solution for well-behaved functions.
The complex number $c$ represents the initial guess at the position of the
solution being sought. The Newton fractal is formed by asking which solution
the iteration converges upon, as a function of the position of the initial
guess $c$ in the complex plane. In the case of the cubic polynomial
$f(z)=z^3-1$, which has three solutions, a map might be generated with points
colored red, green or blue to represent convergence towards the three roots.
\nlnp
If $c$ is close to one of the roots, then convergence towards that particular
root is guaranteed, but further afield the map develops a fractal structure. In
this example, we define a Pyxplot subroutine to produce such a map as a
function of $c=x+iy$, and then plot the resulting map using the {\tt colormap}
plot style (see Section~\ref{sec:colormaps}).  To make the fractal prettier --
it contains, after all, only three colors as strictly defined -- we vary the
brightness of each point depending upon how many iterations are required before
the series ventures within a distance of $|z_n-r_i|<10^{-2}$ of any of the
roots $r_i$.

\nlscf
\input{examples/tex/ex_newton_1.tex}
\nlscf
\begin{center}
\includegraphics[width=8cm]{examples/eps/ex_newton}
\end{center}
}

\example{ex:pendulum}{The dynamics of the simple pendulum}{
The equation of motion for a pendulum bob may be derived from the rotational
analogue to Newton's Second Law, $G=I\ddot\theta$ where $G$ is torque, $I$ is
moment of inertia and $\theta$ is the displacement of the pendulum bob from the
vertical. For a pendulum of length $l$, with a bob of mass $m$, this
equation becomes $-mgl\sin\theta=ml^2\ddot\theta$. In the small-angle
approximation, such that $\sin\theta\approx\theta$, it reduces to
the equation for simple harmonic motion, with the solution
\\ \noindent
\begin{equation}
\theta_\mathrm{approx}=\omega\sin\left(\sqrt{\frac{g}{l}}t\right).
\label{eq:pendulum_approx}
\end{equation}
\nlnp
A more exact solution requires integration of the second-order differential
equation of motion including the $\sin\theta$ term. This integral cannot be
done analytically, but the solution can be written in the form
\\ \noindent
\begin{equation}
\theta_\mathrm{exact}(t) = 2\sin^{-1}\left[ k\,\mathrm{sn}\left(\sqrt{\frac{g}{l}}t,k\right)\right].
\label{eq:pendulum_exact}
\end{equation}
\\ \noindent
where $\mathrm{sn}(u,m)$ is a Jacobi elliptic function and
$k=\sin\left(\omega/2\right)$.  The Jacobi elliptic function cannot be
analytically computed, but can be numerically approximated using the {\tt
jacobi\_\-sn(u,m)} function in Pyxplot.
\nlnp
Below, we produce a plot of Equations~(\ref{eq:pendulum_approx}) and
(\ref{eq:pendulum_exact}).  The horizontal axis is demarked in units of the
dimensionless period of the pendulum to eliminate $g$ and $l$, and a swing
amplitude of $\pm30^\circ$ is assumed:
\nlscf
\input{examples/tex/ex_pendulum_1.tex}
\nlscf
\centerline{\includegraphics[width=9cm]{examples/eps/ex_pendulum}}
\nlnp
As is apparent, at this amplitude, the exact solution begins to deviate
noticeably from the small-angle solution within 2--3 swings of the pendulum. We
now seek to quantify more precisely how long the two solutions take to diverge
by defining a subroutine to compute how long $T$ it takes before the two
solutions to deviate by some amount $\psi$. We then plot these times as a
function of amplitude $\omega$ for three deviation thresholds. Because this
subroutine takes a significant amount of time to run, we only compute~40
samples for each value of $\psi$:
\nlscf
\input{examples/tex/ex_pendulum_2.tex}
\nlscf
\centerline{\includegraphics[width=9cm]{examples/eps/ex_pendulum2}}
}

\section{Macros}
\index{macros}
\index{@ operator@{\tt @} operator}

The $@$ operator can be used for literal substitution of the content of a
string variable into the command line.  The name of the string variable follows
the $@$ sign, and its content is expanded to the command line, as in this
example
\begin{verbatim}
mac = "with lines lw 2 lt 1"
plot sin(x) @mac
\end{verbatim}
which is equivalent to 
\begin{verbatim}
plot sin(x) with lines lw 2 lt 1
\end{verbatim}

The macro, being a string, can contain any characters, but as with other
variable names, the name of the macro can contain only alphanumeric characters
and the underscore sign. This also means that any operator, with the exception
of the {\tt and} and {\tt or} operators, can signify the end of the macro name,
without the need for a trailing white space. Therefore, in the example
\begin{verbatim}
foo = "50"
print @foo*3
\end{verbatim}
the end result is 150; {\tt 50*3} is passed to the command line interpreter.

\section{The \indcmdt{exec}}

The \indcmdt{exec} can be used to execute Pyxplot commands contained within
string variables. For example:

\begin{verbatim}
terminal="eps"
exec "set terminal %s"%(terminal)
\end{verbatim}

\noindent It can also be used to write obfuscated Pyxplot scripts, and its use
should be minimized wherever possible.

\section{Assertions}
\index{assertions}

The \indcmdt{assert} can be used to assert that a logical expression, such as
{\tt x>0}, is true. An error is reported if the expression is false, and
optionally a string can be supplied to provide a more informative error message
to the user:

\begin{verbatim}
assert x>0
assert y<0 "y must be less than zero."
\end{verbatim}

The \indcmdt{assert} can also be used to test the version number of Pyxplot. It
is possible to test either that the version is newer than or equal to a
specific version, using the {\tt $>$=} operator, or that it is older than a
specific version, using the {\tt $<$} operator, as demonstrated in the
following examples:

\begin{verbatim}
assert version >= 0.8.2
assert version <  0.8  "This script is designed for Pyxplot 0.7"
\end{verbatim}


\section{Raising exceptions}

Pyxplot's {\tt raise(e,s)} function is used to raise exceptions when error
conditions are met.  Its first argument {\tt e} specifies the type of
exception, and should be an object of type {\tt exception}. The second argument
should be an error message string.  Pyxplot has a range of default exception
types, which can be found as {\tt exception} objects in the module {\tt
exceptions}. Alternatively, the object {\tt types.exception} may be called with
a single string argument to make a new exception type. For example:

\begin{verbatim}
raise(exceptions.syntax , "Input could not be parsed")

a=types.exception("user error")
raise(a, "The user made a mistake")
\end{verbatim}

Alternatively, {\tt exception} objects have a method {\tt raise(s)} which can be called as follows:

\begin{verbatim}
a=types.exception("user error")
a.raise("The user made a mistake")
\end{verbatim}


\section{Shell commands}

Shell commands\index{shell commands!executing} may be executed directly from
within Pyxplot by prefixing them with an \indcmdts{!} character. The
remainder of the line is sent directly to the shell, for example:

\begin{verbatim}
!ls -l
\end{verbatim}

\noindent Semi-colons cannot be used to place further Pyxplot commands after a
shell command on the same line.

\begin{dontdo}
!ls -l ; set key top left
\end{dontdo}

It is also possible to substitute the output of a shell command into a Pyxplot
command. To do this, the shell command should be enclosed in back-quotes (`),
as in the following example:\index{backquote character}\index{shell
commands!substituting}

\begin{verbatim}
a=`ls -l *.ppl | wc -l`
print "The current directory contains %d Pyxplot scripts."%(a)
\end{verbatim}

It should be noted that back-quotes can only be used outside quotes. For
example,

\begin{dontdo}
set xlabel '`ls`'
\end{dontdo}

\noindent will not work. The best way to do this would be:

\begin{dodo}
set xlabel `echo "'" ; ls ; echo "'"`
\end{dodo}

Note that it is not possible to change the current working directory by sending
the {\tt cd} command to a shell, as this command would only change the working
directory of the shell in which the single command is executed:

\begin{dontdo}
!cd ..
\end{dontdo}

\noindent Pyxplot has its own \indcmdt{cd} for this purpose, as well as its own
\indcmdt{pwd}:

\begin{dodo}
cd ..
\end{dodo}

\section{Script watching: pyxplot\_watch}

Pyxplot includes a simple tool for watching command script files and executing
them whenever they are modified. This may be useful when developing a command
script, if one wants to make small modifications to it and see the results in a
semi-live fashion. This tool is invoked by calling the {\tt
pyxplot\_watch}\index{pyxplot\_watch}\index{watching scripts} command from a
shell prompt. The command-line syntax of {\tt pyxplot\_watch} is similar to
that of Pyxplot itself, for example:

\begin{verbatim}
pyxplot_watch script.ppl
\end{verbatim}

\noindent would set {\tt pyxplot\_watch} to watch the command script file
{\tt script.ppl}. One difference, however, is that if multiple script files are
specified on the command line, they are watched and executed independently,
\textit{not} sequentially, as Pyxplot itself would do. Wildcard characters can
also be used to set {\tt pyxplot\_watch} to watch multiple
files.\footnote{Note that {\tt pyxplot\_watch *.script} and
{\tt pyxplot\_watch $\backslash$*.script} will behave differently in most
UNIX shells.  In the first case, the wildcard is expanded by your shell, and a
list of files passed to {\tt pyxplot\_watch}. Any files matching the
wildcard, created after running {\tt pyxplot\_watch}, will not be picked up.
In the latter case, the wildcard is expanded by {\tt pyxplot\_watch} itself,
which {\it will} pick up any newly created files.}

This is especially useful when combined with \ghostview's\index{Ghostview}
watch facility. For example, suppose that a script {\tt foo.ppl} produces
PostScript output {\tt foo.ps}. The following two commands could be used to
give a live view of the result of executing this script:

\begin{verbatim}
gv --watch foo.ps &
pyxplot_watch foo.ppl
\end{verbatim}


\part{Plotting and vector graphics}
% plotting.tex
%
% The documentation in this file is part of Pyxplot
% <http://www.pyxplot.org.uk>
%
% Copyright (C) 2006-2012 Dominic Ford <coders@pyxplot.org.uk>
%               2008-2012 Ross Church
%
% $Id$
%
% Pyxplot is free software; you can redistribute it and/or modify it under the
% terms of the GNU General Public License as published by the Free Software
% Foundation; either version 2 of the License, or (at your option) any later
% version.
%
% You should have received a copy of the GNU General Public License along with
% Pyxplot; if not, write to the Free Software Foundation, Inc., 51 Franklin
% Street, Fifth Floor, Boston, MA  02110-1301, USA

% ----------------------------------------------------------------------------

% LaTeX source for the Pyxplot Users' Guide

\chapter{Plotting: a complete guide}
\label{ch:plotting}

This part of the manual provides a complete description of Pyxplot's commands
for producing graphs and vector graphics. This chapter extends the overview of
the \indcmdt{plot} in Chapter~\ref{ch:first_steps}, providing a systematic
description of how the appearance of plots can be configured.  Subsequent
chapters describe how to produce graphical output in a range of image formats
(Chapter~\ref{ch:image_formats}), how to produce galleries of multiple plots
side-by-side, and how to produce more sophisticated vector graphics
(Chapter~\ref{ch:vector_graphics}).

\section{The {\tt with} modifier}
\label{sec:with_modifier}

In Chapter~\ref{ch:first_steps} an overview of the syntax of the \indcmdt{plot}
was provided, including the {\tt every}, {\tt index}, {\tt select} and {\tt
using} modifiers, which can be used to control {\it which} data should be
plotted. The {\tt with} modifier controls {\it how} data should be plotted. For
example, the statement
\begin{verbatim}
plot "data.dat" index 1 using 4:5 with lines
\end{verbatim}
specifies that data should be plotted with lines connecting each \datapoint\ to
its neighbors. We term the keyword {\tt lines} a {\it plot style}. The {\tt
with} modifier can also be followed by a variety of settings which fine-tune
aspects of how data are displayed.  For example, the statement
\begin{verbatim}
plot "data.dat" with lines linewidth 2.0
\end{verbatim}
would connect \datapoint s with a line of twice the default width.

The next section will provide a complete list of all of Pyxplot's plot styles
-- i.e.\ the words which may be used in place of {\tt lines}. First we list all
of the modifiers such as {\tt line\-width} which may be used to alter the exact
appearance of these plot styles. These are as follows:

\begin{itemize}
\item \indmodt{color} -- used to select the color in which the dataset is to be plotted. It should be followed either by a number, to select a color from the present palette (see Section~\ref{sec:palette}); by a recognised color name, a complete list of which can be found in Section~\ref{sec:color_names}; or by a color object, such as may be created by the functions {\tt gray(g)}, {\tt rgb(r,g,b)}, {\tt cmyk(c,m,y,k)} or {\tt hsb(h,s,b)}. This modifier may also be spelt {\tt colour}.\index{colors!setting for datasets}
\item \indmodt{fillcolor} -- used to select the color in which the dataset is filled. The color may be specified using any of the styles listed for {\tt color}. May also be spelt {\tt fillcolour}.
\item \indmodt{linetype} -- used to numerically select the type of line -- for example, solid, dotted, dashed, etc.\ -- which should be used by line-based plot styles. A complete list of Pyxplot's numbered line types can be found in Chapter~\ref{ch:linetypes_table}. May be abbreviated {\tt lt}.
\item \indmodt{linewidth} -- used to select the width of line which should be used by line-based plot styles, where unity represents the default width. May be abbreviated {\tt lw}.
\item \indmodt{pointlinewidth} -- used to select the width of line which should be used to stroke points in point-based plot styles, where unity represents the default width. May be abbreviated {\tt plw}.
\item \indmodt{pointsize} -- used to select the size of drawn points, where unity represents the default size. May be abbreviated {\tt ps}.
\item \indmodt{pointtype} -- used to numerically select the type of point -- for example, crosses, circles, etc.\ -- used by point-based plot styles. A complete list of Pyxplot's numbered point types can be found in Chapter~\ref{ch:linetypes_table}. May be abbreviated {\tt pt}.
\end{itemize}

Any number of these modifiers may be placed sequentially after the keyword {\tt
with}, as in the following examples:

\begin{verbatim}
plot 'datafile' using 1:2 with points pointsize 2
plot 'datafile' using 1:2 with lines color red linewidth 2
plot 'datafile' using 1:2 with lp col 1 lw 2 ps 3
\end{verbatim}

\noindent Where modifiers take numerical values, expressions of the form {\tt
\$2+1}, similar to those supplied to the {\tt using} modifier, may be used to
read numbers from the supplied data set. In this case, each datapoint will be
displayed in a different style or in a different color (in the example given,
depending on the values in the second column of the supplied data).

The following example would plot a \datafile\ with {\tt points}, drawing the
position of each point from the first two columns of the supplied \datafile\
and the size of each point from the third column:
\begin{verbatim}
plot 'datafile' using 1:2 with points pointsize $3
\end{verbatim}

Not all of these modifiers are applicable to all of Pyxplot's plot styles. For
example, the {\tt line\-width} modifier has no effect on plot styles which do
not draw lines between datapoints. Where modifiers are applied to plot styles
for which they have no defined effect, the modifier has no effect, but no error
results.  Table~\ref{tab:style_modifiers} lists which modifiers act on which
plot styles.

\begin{table}
\centerline{\includegraphics[width=\textwidth]{examples/eps/ex_plotstyletab}}
\caption{A list of the plot styles affected by each style modifiers.}
\label{tab:style_modifiers}
\end{table}

\subsection{The palette}
\label{sec:palette}
\index{palette}\index{colors!setting the palette}

Wherever Pyxplot takes a color as an input to a command, the user has three
options for how it may be specified.  A selection of widely-used colors may be
specified by name, for example {\tt red} and {\tt blue}. A complete list of
such colors may be found in Section~\ref{sec:color_names}. Alternatively, an
object of type {\tt color} may be provided, such as {\tt rgb(0,1,0)}, {\tt
hsb(0.5,0.5,0.5)}, {\tt gray(0.2)}, {\tt colors.green + colors.red}, or {\tt
colors.yellow - colors.green}.

The third option is to specify a numbered color from Pyxplot's {\it palette}.
By default, this contains a series of visually distinctive colors which are,
insofar as possible, also distinctive to users with most common forms of color
blindness:

\centerline{\includegraphics[width=\textwidth]{examples/eps/ex_palettelist}}

The first color is number~1, the second number~2, and so forth. As
well as being accessible by number, these colors also form the default series
which Pyxplot chooses for successive datasets when their colors are not
individually specified.

The current palette may be queried using the \indcmdt{show palette}, and
changed using the \indcmdt{set palette}, which takes a comma-separated list of
colors, as in the example:

\begin{verbatim}
set palette brickRed, limeGreen, cadetBlue
\end{verbatim}

\noindent The palette is treated as a cyclic list, and so in the above example,
color number~4 would map to {\tt brickRed}, as would color numbers~1 and~8. The
default palette which Pyxplot uses on startup may be changed by setting up a
configuration file, as described in Chapter~\ref{ch:configuration}.

If a non-integer color is requested from the palette, for example color~1.5,
then a color is returned which is half-way in between colors~1 and~2 in RGB
space; in this case, brown. This can be used to produce custom color gradients,
as the following example demonstrates (the {\tt colormap} plot style will be
described in Section~\ref{sec:colormaps}):

\vspace{2mm}
\input{examples/tex/ex_colgradient_1.tex}
\vspace{2mm}

\centerline{\includegraphics[width=10cm]{examples/eps/ex_colgradient}}


\subsection{Default settings}

In addition to setting these parameters on a per-dataset basis, the {\tt
linewidth}, {\tt pointlinewidth} and {\tt pointsize} settings can also have
their default values changed for all datasets as in the following examples:
\begin{verbatim}
set linewidth 1
set pointlinewidth 2
set pointsize 3
plot "datafile"
\end{verbatim}
In each case, the normal default values of these settings are~1. The default
values of the {\tt color}, {\tt linetype} and {\tt pointtype} settings depend
whether the current graphic output device is set to produce color or
monochrome output (see Chapter~\ref{sec:set_terminal}).

In the case of color output, the colors of each of the comma-separated datasets
plotted on a graph are drawn sequentially from the currently-selected palette,
and all lines are drawn as solid lines ({\tt line\-type~1}). The symbols used
to draw each dataset are drawn sequentially from Pyxplot's available point
types. In the case of monochrome output, all datasets are plotted in black and
both the line types and point types used to draw each dataset are drawn
sequentially from Pyxplot's available options.

The following simple example demonstrates this:
\begin{verbatim}
set terminal color
plot [][6:0] 1 with lp, 2 with lp, 3 w lp, 4 w lp, 5 w lp
set terminal monochrome
replot
\end{verbatim}
\centerline{\includegraphics[width=\textwidth]{examples/eps/ex_col_vs_mono}}

\section{Pyxplot's plot styles}
\label{sec:list_of_plotstyles}

This section provides a complete list of Pyxplot's {\it plot styles}, arranged
into groups for clarity.  Table~\ref{tab:plot_style_columns} summarises the
columns of data expected by each plot style when used on two- and
three-dimensional plots. The following sections describe each of these plot
styles in turn.

\begin{table}
\begin{tabular}{|rll|}
\hline
{\bf Style} & {\bf Columns (2D plots)} & {\bf Columns (3D plots)} \\
\hline
{\tt arrows\_head} & $(x_1,y_1,x_2,y_2)$ & $(x_1,y_1,z_1,x_2,y_2,z_2)$ \\
{\tt arrows\_nohead} & $(x_1,y_1,x_2,y_2)$ & $(x_1,y_1,z_1,x_2,y_2,z_2)$ \\
{\tt arrows\_twohead} & $(x_1,y_1,x_2,y_2)$ & $(x_1,y_1,z_1,x_2,y_2,z_2)$ \\
{\tt boxes} & $(x,y)$ & $(x,y)$ \\
{\tt colormap} & $(x,y,c_1,\ldots)$ & $(x,y,c_1,\ldots)$ \\
{\tt contourmap} & $(x,y,c_1,\ldots)$ & $(x,y,c_1,\ldots)$ \\
{\tt dots} & $(x,y)$ & $(x,y,z)$ \\
{\tt FilledRegion} & $(x,y)$ & $(x,y)$ \\
{\tt fsteps} & $(x,y)$ & $(x,y)$ \\
{\tt histeps} & $(x,y)$ & $(x,y)$ \\
{\tt impulses} & $(x,y)$ & $(x,y,z)$ \\
{\tt lines} & $(x,y)$ & $(x,y,z)$ \\
{\tt LinesPoints} & $(x,y)$ & $(x,y,z)$ \\
{\tt LowerLimits} & $(x,y)$ & $(x,y,z)$ \\
{\tt points} & $(x,y)$ & $(x,y,z)$ \\
{\tt stars} & $(x,y)$ & $(x,y,z)$ \\
{\tt steps} & $(x,y)$ & $(x,y)$ \\
{\tt surface} & $(x,y,z)$ & $(x,y,z)$ \\
{\tt UpperLimits} & $(x,y)$ & $(x,y,z)$ \\
{\tt wboxes} & $(x,y,w)$ & $(x,y,w)$ \\
{\tt XErrorBars} & $(x,y,\sigma_x)$ & $(x,y,z,\sigma_x)$ \\
{\tt XErrorRange} & $(x,y,x_\mathrm{min},x_\mathrm{max})$ & $(x,y,z,x_\mathrm{min},x_\mathrm{max})$ \\
{\tt XYErrorBars} & $(x,y,\sigma_x,\sigma_y)$ & $(x,y,z,\sigma_x,\sigma_y)$ \\
{\tt XYErrorRange} & $(x,y,x_\mathrm{min},x_\mathrm{max},y_\mathrm{min},y_\mathrm{max})$ & $(x,y,z,x_\mathrm{min},x_\mathrm{max},y_\mathrm{min},y_\mathrm{max})$ \\
{\tt XYZErrorBars} & $(x,y,z,\sigma_x,\sigma_y,\sigma_z)$ & $(x,y,z,\sigma_x,\sigma_y,\sigma_z)$ \\
{\tt XYZErrorRange} & $(x,y,z,x_\mathrm{min},x_\mathrm{max},y_\mathrm{min},$ -- & $(x,y,z,x_\mathrm{min},x_\mathrm{max},y_\mathrm{min},$ -- \\
                    & -- $y_\mathrm{max},z_\mathrm{min},z_\mathrm{max})$ & -- $y_\mathrm{max},z_\mathrm{min},z_\mathrm{max})$ \\
{\tt XZErrorBars} & $(x,y,z,\sigma_x,\sigma_z)$ & $(x,y,z,\sigma_x,\sigma_z)$ \\
{\tt XZErrorRange} & $(x,y,z,x_\mathrm{min},x_\mathrm{max},z_\mathrm{min},z_\mathrm{max})$ & $(x,y,z,x_\mathrm{min},x_\mathrm{max},z_\mathrm{min},z_\mathrm{max})$ \\
{\tt YErrorBars} & $(x,y,\sigma_y)$ & $(x,y,z,\sigma_y)$ \\
{\tt YErrorRange} & $(x,y,y_\mathrm{min},y_\mathrm{max})$ & $(x,y,z,y_\mathrm{min},y_\mathrm{max})$ \\
{\tt YErrorShaded} & $(x,y_\mathrm{min},y_\mathrm{max})$ & $(x,y_\mathrm{min},y_\mathrm{max})$ \\
{\tt YZErrorBars} & $(x,y,z,\sigma_y,\sigma_z)$ & $(x,y,z,\sigma_y,\sigma_z)$ \\
{\tt YZErrorRange} & $(x,y,z,y_\mathrm{min},y_\mathrm{max},z_\mathrm{min},z_\mathrm{max})$ & $(x,y,z,y_\mathrm{min},y_\mathrm{max},z_\mathrm{min},z_\mathrm{max})$ \\
{\tt ZErrorBars} & $(x,y,z,\sigma_z)$ & $(x,y,z,\sigma_z)$ \\
{\tt ZErrorRange} & $(x,y,z,z_\mathrm{min},z_\mathrm{max})$ & $(x,y,z,z_\mathrm{min},z_\mathrm{max})$ \\
\hline
\end{tabular}
\caption{A summary of the columns of data expected by each of Pyxplot's plot styles when used on two- and three-dimensional plots.}
\label{tab:plot_style_columns}
\end{table}

\subsection{Lines and points}

The following is a list of Pyxplot's simplest plot styles, all of which take
two columns of input data on 2D plots (three columns on 3D plots), which
represent the $x$-, $y$- (and $z$-)coordinates of the positions of each point:
\begin{itemize}
\item \indpst{dots} -- places a small dot at each datum.
\item \indpst{lines} -- connects adjacent \datapoint s with straight lines.
\item \indpst{linespoints} -- a combination of both lines and points.
\item \indpst{lowerlimits} -- places a lower-limit sign (\includegraphics{examples/eps/ex_lowerlimit}) at each datum.\index{lower-limit datapoints}
\item \indpst{points} -- places a marker symbol at each datum.
\item \indpst{stars} -- similar to {\tt points}, but uses a different set of marker symbols, based on the stars drawn in Johann Bayer's highly ornate star atlas {\it Uranometria} of 1603.
\item \indpst{upperlimits} -- places an upper-limit sign (\includegraphics{examples/eps/ex_upperlimit}) at each datum.\index{upper-limit datapoints}
\end{itemize}

\example{ex:hrdiagram}{A Hertzsprung-Russell diagram}{
Hertzsprung-Russell (HR) diagrams are scatter-plots of the luminosities of
stars plotted against their colors, on which most normal stars lie
along a tight line called the main sequence, whilst unusual classes of stars --
giants and dwarfs -- can be readily identified on account of their not lying
along this main sequence. The principal difficulty in constructing accurate HR
diagrams is that the luminosities $L$ of stars can only be calculated from
their observed brightnesses $F$, using the relation $L=Fd^2$ if their distances
$d$ are known. In this example, we construct an HR diagram using observations
made by the European Space Agency's {\it Hipparcos} spacecraft, which
accurately measured the distances of over a million stars between 1989 and
1993.
\nlnp
The Hipparcos catalogue can be downloaded for free from
\url{ftp://cdsarc.u-strasbg.fr/pub/cats/I/239/hip_main.dat.gz}; a description
of the catalogue can be found at
\url{http://cdsarc.u-strasbg.fr/viz-bin/Cat?I/239}. In summary, though the data
is arranged in a non-standard format which Pyxplot cannot read without a
special input filter, the following Python script generates a text file with
four columns containing the magnitudes $m$, $B-V$ colors and parallaxes $p$ of
the stars, together with the uncertainties in the parallaxes. From these
values, the absolute magnitudes $M$ of the stars -- a measure of their
luminosities -- can be calculated using the expression
$M=m+5\log_{10}\left(10^{2}p\right)$, where $p$ is measured in
milli-arcseconds:
\nlscf
\noindent{\tt for line in open("hip\_main.dat"):}\newline
\noindent{\tt \phantom{x}try:}\newline
\noindent{\tt \phantom{xx}Vmag  = float(line[41:46])}\newline
\noindent{\tt \phantom{xx}BVcol = float(line[245:251])}\newline
\noindent{\tt \phantom{xx}parr  = float(line[79:86])}\newline
\noindent{\tt \phantom{xx}parre = float(line[119:125])}\newline
\noindent{\tt \phantom{xx}print "\%s,\%s,\%s,\%s"\%(Vmag, BVcol, parr, parre)}\newline
\noindent{\tt \phantom{x}except ValueError: pass}
\nlscf
The resultant four columns of data can then be plotted in the {\tt dots} style
using the following Pyxplot script. Because the catalogue is very large, and
many of the parallax datapoints have large errorbars producing large
uncertainties in their vertical positions on the plot, we use the {\tt select}
statement to pick out those datapoints with parallax signal-to-noise ratios of
better than~20.
\nlscf
\input{examples/tex/ex_hrdiagram_1.tex}
\nlscf
\centerline{\includegraphics[width=10cm]{examples/eps/ex_hrdiagram}}
}

\subsection{Error bars}
\index{errorbars}\label{sec:errorbars}

The following pair of plot styles allow datapoints to be plotted with errorbars
indicating the uncertainties in either their vertical or horizontal positions:
\begin{itemize}
\item \indpst{yerrorbars}
\item \indpst{xerrorbars}
\end{itemize}
Both of these take three columns of input data on 2D plots (or four on
3D plots). The first two (or three) of these represent the $x$-, $y$- (and
$z$-) coordinates of the central position of each errorbar, while the last
represents the uncertainty in either the $x$- and $y$-coordinate. The
plot style \indpst{errorbars} is an alias for \indpst{yerrorbars}.
Additionally, the following plot style allows datapoints to be plotted with
both horizontal and vertical errorbars:
\begin{itemize}
\item \indpst{xyerrorbars}
\end{itemize}
This plot style takes four (or five) columns of data as input, the first two
(or three) of which represent the $x$-, $y$- (and $z$-) coordinates of the
central position of each errorbar. The last but one column gives the
uncertainty in the $x$-coordinate, and the last column gives the uncertainty in
the $y$-coordinate.

Each of the plot styles listed above has a corresponding partner which takes
minimum and maximum limits for each errorbar, equivalent to writing
$5^{+2}_{-3}$, in place of a single symmetric uncertainty:
\begin{itemize}
\item \indpst{xerrorrange} -- takes four (or five) columns of data.
\item \indpst{yerrorrange} -- takes four (or five) columns of data.
\item \indpst{xyerrorrange} -- takes six (or seven) columns of data.
\end{itemize}
The plot style \indpst{errorrange} is an alias of \indpst{yerrorrange}.

Corresponding plot styles also exist to plot data with errorbars along the
$z$-axes of three-dimensional plots: {\tt zerrorbars}, {\tt zerrorrange}, {\tt
xzerrorbars}, {\tt xzerrorrange}, {\tt yzerrorbars}, {\tt yzerrorrange}, {\tt
xyzerrorbars}, {\tt xyzerrorrange}. Though it does not make sense to use these
on two-dimensional plots, it is not an error to do so; they expect the same
number of columns of input data on both two- and three-dimensional plots.

\subsection{Shaded regions}

The following plot styles allow regions of graphs to be shaded with color:

\begin{itemize}
\item \indpst{yerrorshaded}
\item \indpst{shadedregion}
\end{itemize}

Both fill specified regions of graphs with the selected
{\tt fillcolor} and draw a line around the boundary of the region with the
selected {\tt color}, {\tt linetype} and {\tt linewidth}.

They differ in the format in which they expect the input data to be arranged.
The \indpst{yerrorshaded} plot style is similar to the \indpst{yerrorrange}
plot style: it expects three columns of data, specifying the $x$-coordinate and
the minimum and maximum extremes of the vertical errorbar on each \datapoint.
The region contained between the upper and lower limits of these error bars is
filled with color.  Note that the \datapoint s must be sorted in order of
either increasing or decreasing $x$-coordinate for sensible behaviour.

The \indpst{shadedregion} plot style takes only two columns of input data,
specifying the $x$- and $y$-coordinates of a series of \datapoint s which are
to be joined in a join-the-dots fashion. At the end of each dataset, the drawn
path is closed and filled.

The use of these plot styles on three-dimensional graphs may produce unexpected
results.

\subsection{Barcharts and histograms}
\label{sec:barcharts}
\index{bar charts}

The following plot styles allow barcharts to be produced:
\begin{itemize}
\item \indpst{boxes}
\item \indpst{impulses}
\item \indpst{wboxes}
\end{itemize}
These styles differ in where the horizontal interfaces between the
bars are placed along the abscissa axis and how wide the bars are.  In the
\indpst{boxes} plot style, the interfaces between the bars are at the midpoints
between the specified \datapoint s by default (see, for example,
Figure~\ref{fig:ex_barchart2}a).  Alternatively, the widths of the bars may be
set using the {\tt set boxwidth} command. In this case, all of the bars will be
centered on their specified $x$-coordinates, and have total widths equalling
that specified in the \indcmdt{set boxwidth}. Consequently, there may be gaps
between them, or they may overlap, as seen in Figure~\ref{fig:ex_barchart2}(b).

\begin{figure}
\begin{center}
\includegraphics[width=\textwidth]{examples/eps/ex_barchart2}
\end{center}
\caption[A gallery of the various bar chart styles which Pyxplot can produce]
{A gallery of the various bar chart styles which Pyxplot can produce.
See the text for more details.}
\label{fig:ex_barchart2}
\end{figure}

Having set a fixed box width, the default behaviour of scaling box widths
automatically may be restored either with the {\tt unset boxwidth} command,
or by setting the boxwidth to a negative width.

In the \indpst{wboxes} plot style, the width of each bar is specified manually
as an additional column of the input \datafile.  This plot style expects three
columns of data to be provided: the $x$- and $y$-coordinates of each bar in the
first two, and the width of the bars in the third.
Figure~\ref{fig:ex_barchart2}(c) shows an example of this plot style in use.

Finally, in the \indpst{impulses} plot style, the bars all have zero width; see
Figure~\ref{fig:ex_barchart1}(c) for an example.

In all of these plot styles, the bars originate from the line $y=0$ by default,
as is normal for a histogram. However, should it be desired for the bars to
start from a different vertical line, this may be achieved by using the
\indcmdt{set boxfrom}, for example:

\begin{verbatim}
set boxfrom 5
\end{verbatim}

\noindent In this case, all of the bars would now originate from the line
$y=5$. Figure~\ref{fig:ex_barchart1}(b) shows the kind of effect that is
achieved; for comparison, Figure~\ref{fig:ex_barchart1}(a) shows the same bar
chart with the boxes starting from their default position of $y=0$.

\begin{figure}
\begin{center}
\includegraphics[width=\textwidth]{examples/eps/ex_barchart1}
\end{center}
\caption[A second gallery of the various bar chart styles which Pyxplot can
produce]
{A second gallery of the various bar chart styles which Pyxplot can
produce. See the text for more details.  The script and data file used to
produce this image are available on the Pyxplot website at
\protect\url{http://www.pyxplot.org.uk/examples/Manual/03barchart1/}.}
\label{fig:ex_barchart1}
\end{figure}

The bars may be filled using the {\tt with} \indmodt{fillcolor} modifier,
followed by the name of a color:

\begin{verbatim}
plot 'data.dat' with boxes fillcolor blue
plot 'data.dat' with boxes fc 4
\end{verbatim}

\noindent Figures~\ref{fig:ex_barchart2}(b) and (d) demonstrate the use of
filled bars.

The {\tt boxes} and {\tt wboxes} plot styles expect identically-formatted data
when used on two- and three-dimensional plots; in the latter case, all bars are
drawn in the plane $z=0$. The {\tt impulses} plot style takes an additional
column of data on three-dimensional plots, specifying the $z$-coordinate at
which each impulse should be drawn.

\subsubsection{Stacked bar charts}

If multiple \datapoint s are supplied to the \indpst{boxes} or \indpst{wboxes}
plot styles at a common $x$-coordinate, then the bars are stacked one above
another into a stacked barchart. Consider the following \datafile:

\begin{verbatim}
1 1
2 2
2 3
3 4
\end{verbatim}

\noindent The second bar at $x=2$ would be placed on top of the first, spanning
the range $2<y<5$, and having the same width as the first. If plot colors are
being automatically selected from the palette, then a different palette color
is used to plot the upper bar.

\subsection{Steps}

The following plot styles allow data to be plotted with a series of horizontal
steps associated with each supplied \datapoint:
\begin{itemize}
\item \indpst{steps}
\item \indpst{fsteps}
\item \indpst{histeps}
\end{itemize}
Each of these styles takes two columns of data, containing the $x$- and
$y$-coordinates of each \datapoint.  They expect identically-formatted data
regardless of whether they are used on two- and three-dimensional plots; in the
latter case, the steps are drawn in the plane $z=0$.

An example of their appearance  is shown in Figures~\ref{fig:ex_barchart1}(d),
(e) and (f); for clarity, the positions of each of the supplied \datapoint s
are marked by red crosses.

These plot styles differ in their placement of the edges of each of the
horizontal steps.  The \indpst{steps} plot style places the right-most edge of
each step on the \datapoint\ it represents.  The \indpst{fsteps} plot style
places the left-most edge of each step on the \datapoint\ it represents.  The
\indpst{histeps} plot style centers each step on the \datapoint\ it represents.

\subsection{Arrows}

The following plot styles allow arrows or lines to be drawn on graphs with
positions dictated by a series of \datapoint s:
\begin{itemize}
\item \indpst{arrows\_head}
\item \indpst{arrows\_nohead}
\item \indpst{arrows\_twohead}
\end{itemize}
The plot style of \indpst{arrows} is an alias for \indpst{arrows\_head}.  Each
of these plot styles take four columns of data on two-dimensional plots --
$x_1$, $y_1$, $x_2$ and $y_2$ -- or six columns of data on three-dimensional
plots with additional $z$-coordinates. Each \datapoint\ results in an arrow
being drawn from the point $(x_1,y_1,z_1)$ to the point $(x_2,y_2,z_2)$. The
three plot styles differ in the kinds of arrows that they draw:
\indpst{arrows\_head} draws an arrow head on each arrow at the point
$(x_2,y_2,z_2)$; \indpst{arrows\_nohead} draws simple lines without arrow heads
on either end; \indpst{arrows\_twohead} draws arrow heads on both ends of each
arrow.

\example{ex:vortex}{A diagram of fluid flow around a vortex}{
In this example we produce a velocity map of fluid circulating in a vortex. For
simplicity, we assume that the fluid in the core of the vortex, at radii $r<1$,
is undergoing solid body rotation with velocity $v\propto r$, and that the
fluid outside this core is behaving as a free vortex with velocity $v\propto
1/r$. First of all, we use a simple python script to generate a \datafile\ with
the four columns:
\nlscf
\noindent{\tt from math import *}\newline
\noindent{\tt for i in range(-19,20,2):}\newline
\noindent{\tt \phantom{x}for j in range(-19,20,2):}\newline
\noindent{\tt \phantom{xx}x = float(i)/2}\newline
\noindent{\tt \phantom{xx}y = float(j)/2}\newline
\noindent{\tt \phantom{xx}r = sqrt(x**2 + y**2) / 4}\newline
\noindent{\tt \phantom{xx}theta = atan2(y,x)}\newline
\noindent{\tt \phantom{xx}if (r $<$ 1.0): v = 1.3*r}\newline
\noindent{\tt \phantom{xx}else        : v = 1.3/r}\newline
\noindent{\tt \phantom{xx}vy = v *  cos(theta)}\newline
\noindent{\tt \phantom{xx}vx = v * -sin(theta)}\newline
\noindent{\tt \phantom{xx}print "\%7.3f \%7.3f \%7.3f \%7.3f"\%(x,y,vx,vy)}
\nlscf
This data can then be plotted using the following Pyxplot script:
\nlscf
\input{examples/tex/ex_vortex_1.tex}
\nlscf
\centerline{\includegraphics[width=10cm]{examples/eps/ex_vortex}}
}

\subsection{Color maps, contour maps and surface plots}

The following plot styles differ from those above in that, regardless of
whether a three-dimensional plot is being produced, they read in datapoints
with $x$, $y$ and $z$ coordinates in three columns. The first two are useful
for producing two-dimensional representations of $(x,y,z)$ surfaces using
colors or contours to show the magnitude of $z$, while the third is useful for
producing three-dimensional graphs of such surfaces:
\begin{itemize}
\item colormap
\item contourmap
\item surface
\end{itemize}
They are discussed in detail in Sections~\ref{sec:colormaps},
\ref{sec:contourmaps} and \ref{sec:surfaces} respectively.

\section{Labelling datapoints}

The {\tt label} modifier to the {\tt plot} command may be used to add text
labels next to datapoints, as in the following examples:
\begin{verbatim}
set samples 8
plot [2:5] x**2 label "$x=%.2f$"%($1) with points

plot 'datafile' using 1:2 label "%s"%($3)
\end{verbatim}

\noindent Note that if a particular column of a \datafile\ contains strings
which are to be used as labels, as in the second example above, syntax such as
{\tt "\%s"\%(\$3)} must be used to explicitly read the data as strings rather
than as numerical quantities.  As Pyxplot treats any whitespace as separating
columns of data, such labels cannot contain spaces, though \latexdcf's {\tt
$\sim$} character (a non-breaking space) can be used to achieve a space.

Data points can be labelled when plotted in any of the following plot styles:
{\tt arrows} (and similar styles), {\tt dots}, {\tt errorbars} (and similar
styles), {\tt errorrange} (and similar styles), {\tt impulses}, {\tt
linespoints}, {\tt lowerlimits}, {\tt points}, {\tt stars} and {\tt
upperlimits}. It is not possible to label datapoints plotted in other styles.
Labels are rendered in the same color as the datapoints with which they are
associated.

\section{The {\tt style} keyword}

At times, the string of style keywords placed after the {\tt with} modifier in
{\tt plot} commands can grow rather unwieldy in its length. For clarity,
frequently used plot styles can be stored as numbered plot {\it styles}.  The
syntax for setting a numbered plot style is:

\begin{verbatim}
set style 2 points pointtype 3
\end{verbatim}

\noindent where the {\tt 2} is the identification number of the style. In a
subsequent {\tt plot} statement, this style can be recalled as follows:

\begin{verbatim}
plot sin(x) with style 2
\end{verbatim}

\section{Plotting functions in exotic styles}

The use of plot styles which take more than two columns of input data to plot
functions requires more than one function to be supplied.  When functions are
plotted with syntax such as

\begin{verbatim}
plot sin(x) with lines
\end{verbatim}

\noindent two columns of data are generated: the first contains values of $x$
-- plotted against the horizontal axis -- and the second contains values of
$\sin(x)$ -- plotted against the vertical axis. Syntax such as

\begin{verbatim}
plot f(x):g(x) with yerrorbars
\end{verbatim}

\noindent generates three columns of data. As before, the first contains values
of $x$. The second and third contain samples from the colon-separated functions
$f(x)$ and $g(x)$. Specifically, in this example, $g(x)$ provides the
uncertainty in the value of $f(x)$.  The {\tt using} modifier may also be used
in combination with such syntax, as in

\begin{verbatim}
plot f(x):g(x) using 2:3
\end{verbatim}

\noindent though this example is not sensible. $g(x)$ would be plotted on the
{\tt y}-axis, against $f(x)$ on the {\tt x}-axis. However, this is unlikely to be
sensible because the range of values of $x$ substituting into these expressions
would correspond to the range of the plot's horizontal axis. The result might
be particularly unexpected if the above were attempted with an autoscaling
horizontal axis -- Pyxplot would find itself autoscaling the {\tt x}-axis range
to the spread of values of $f(x)$, but find that this itself changed depending
on the range of the {\tt x}-axis. In this case, the user should have used the
{\tt parametric} plot option described in the next section.

\section{Plotting parametric functions}
\label{sec:parametric_plotting}

Parametric functions are functions expressed in forms such as
\begin{eqnarray*}
x & = & r \sin(t)  \\
y & = & r \cos(t) ,
\end{eqnarray*}
where separate expressions are supplied for the ordinate and abscissa values as
a function of some free parameter $t$. The above example is a parametric
representation of a circle of radius $r$. Before Pyxplot can usefully plot
parametric functions, it is generally necessary to stipulate the range of
values of $t$ over which the function should be sampled. This may be done using
the \indcmdt{set trange}, as in the example
\begin{verbatim}
set trange [unit(0*rad):unit(2*pi*rad)]
\end{verbatim}
or in the {\tt plot} command itself. By default, values in the range $0\leq
t\leq1$ are used. Note that the \indcmdt{set trange} differs from other
commands for setting axis ranges in that auto-scaling is not an allowed
behaviour; an explicit range {\it must} be specified for $t$.

Having set an appropriate range for $t$, parametric functions may be plotted by
placing the keyword {\tt parametric} before the list of functions to be
plotted, as in the following simple example which plots a circle:
\begin{verbatim}
set trange [unit(0*rev):unit(1*rev)]
plot parametric sin(t):cos(t)
\end{verbatim}
Optionally, a range for $t$ can be specified on a plot-by-plot basis
immediately after the keyword {\tt parametric}, and thus the effect above could
also be achieved using:
\begin{verbatim}
plot parametric [unit(0*rev):unit(1*rev)] sin(t):cos(t)
\end{verbatim}
The only difference between parametric function plotting and ordinary function
plotting -- other than the change of dummy variable from {\tt x} to {\tt t} --
is that one fewer column of data is generated. Thus, whilst
\begin{verbatim}
plot f(x)
\end{verbatim}
generates two columns of data, with values of $x$ in the first column,
\begin{verbatim}
plot parametric f(t)
\end{verbatim}
generates only one column of data.

\example{ex:spirograph}{Spirograph patterns}{
Spirograph patterns are produced when a pen is tethered to the end of a rod
which rotates at some angular speed $\omega_1$ about the end of another rod,
which is itself rotating at some angular speed $\omega_2$ about a fixed central
point. Spirographs are commonly implemented mechanically as wheels within
wheels -- epicycles within deferents, mathematically speaking -- but in this
example we implement them using the parametric functions
\begin{eqnarray*}
x & = & r_1 \sin(t) + r_2 \sin(t r_1 / r_2) \\
y & = & r_1 \cos(t) + r_2 \cos(t r_1 / r_2) \\
\end{eqnarray*}
which are simply the sum of two circular motions with angular velocities
inversely proportional to their radii. The complexity of the resulting
spirograph pattern depends on how rapidly the rods return to their starting
configuration; if the two chosen angular speeds for the rods have a large
lowest common multiple, then a highly complicated pattern will result. In the
example below, we pick a ratio of $8:15$:
\nlscf
\input{examples/tex/ex_spirograph_1.tex}
\nlscf
\centerline{\includegraphics[width=8cm]{examples/eps/ex_spirograph}}
\nlscf
Other ratios of {\tt r1}:{\tt r2} such as $7:19$ and $5:19$ also produce
intricate patterns.
}

\subsection{Two-dimensional parametric surfaces}

Pyxplot can also plot datasets which can be parameterised in terms of two free
parameters $u$ and $v$. This is most often useful in conjunction with the {\tt
surface} plot style, allowing any $(u,v)$-surface to be plotted (see
Section~\ref{sec:surfaces} for details of the {\tt surface} plot style).
However, it also works in conjunction with any other plot style, allowing, for
example, $(u,v)$-grids of points to be constructed.

As in the case of parametric lines above, the range of values that each free
parameter should take must be specified. This can be done using the \indcmd{set
urange}{\tt set urange} and \indcmd{set vrange}{\tt set vrange} commands. These
commands also act to switch Pyxplot between one- and two-dimensional parametric
function evaluation; whilst the {\tt set trange} command indicates that the
next parametric function should be evaluated along a single raster of values of
$t$, the {\tt set urange} and {\tt set vrange} commands indicate that a grid of
$(u,v)$ values should be used. By default, the range of values used for $u$ and
$v$ is $0\to 1$.

The number of samples to be taken can be specified using a command of the
form\indcmd{set sample grid}
\begin{verbatim}
set sample grid 20x50
\end{verbatim}
which specifies that~20 different values of $u$ and~50 different values of $v$
should be used, yielding a total of~1000 datapoints. The following example uses
the {\tt lines} plot style to generate a sequence of cross-sections through a
two-dimensional Gaussian surface:

\vspace{2mm}
\input{examples/tex/ex_datagrid_1.tex}
\vspace{2mm}

\centerline{\includegraphics[width=8cm]{examples/eps/ex_datagrid}}

The ranges of values to use for $u$ and $v$ may alternatively be specified on a dataset-by-dataset
basis within the plot command, as in the example
\begin{verbatim}
plot parametric [0:1][0:1] u:v , \
     parametric [0:1] sin(t):cos(t)
\end{verbatim}

\example{ex:torus}{A three-dimensional view of a torus}{
In this example we plot a torus, which can be parametrised in terms of two
free parameters $u$ and $v$ as
\begin{eqnarray*}
x & = & (R + r\cos(v))\cos(u)  \\
y & = & (R + r\cos(v))\sin(u)  \\
z & = & r\sin(v)  ,
\end{eqnarray*}
where $u$ and $v$ both run in the range $[0:2\pi]$, $R$ is the distance of the
tube's center from the center of the torus, and $r$ is the radius of the tube.
\nlscf
\input{examples/tex/ex_torus_1.tex}
\nlscf
\centerline{\includegraphics[width=8cm]{examples/eps/ex_torus}}	
\nlscf
}

\example{ex:trefoil}{A three-dimensional view of a trefoil knot}{
In this example we plot a trefoil knot, which is the simplest non-trivial knot
in topology.  Simply put, this means that it is not possible to untie the knot
without cutting it.  The knot follows the line
\begin{eqnarray*}
x & = & (2 + \cos(3t))\cos(2t)  \\
y & = & (2 + \cos(3t))\sin(2t)  \\
z & = & \sin(3t)  ,
\end{eqnarray*}
but in this example we construct a tube around this line using the following
parameterisation:
\begin{eqnarray*}
x & = & \cos(2u)\cos(v) + r\cos(2u)(1.5+\sin(3u)/2)   \\
y & = & \sin(2u)\cos(v) + r\sin(2u)(1.5+\sin(3u)/2) \\
z & = & \sin(v)+R\cos(3u) ,
\end{eqnarray*}
where $u$ and $v$ run in the ranges $[0:2\pi]$ and $[-\pi:\pi]$ respectively,
and $r$ and $R$ determine the size and thickness of the knot as in an analogous
fashion to the previous example.
\nlscf
\input{examples/tex/ex_trefoil_1.tex}
\nlscf
\centerline{\includegraphics[width=8cm]{examples/eps/ex_trefoil}}
\nlscf
}

\section{Graph legends}
\index{keys}\index{legends}
\label{sec:legends}

By default, plots are displayed with legends in their top-right corners. The
textual description of each dataset is auto-generated from the command used
to plot it. Alternatively, the user may specify his own description for each
dataset by following the {\tt plot} command with the \indmodt{title} modifier,
as in the following examples:

\begin{verbatim}
plot sin(x) title 'A sine wave'
plot cos(x) title ''
\end{verbatim}

In the latter case a blank title is specified, which indicates to Pyxplot that
no entry should be made for the dataset in the legend. This allows for legends
which contain only a subset of the datasets on a plot. Alternatively, the
production of the legend can be completely turned off for all datasets using
the command \indcmdts{set nokey}. Having issued this command, the production of
keys can be resumed using the \indcmdt{set key}.

The \indcmdt{set key} can also be used to dictate how legends should be
positioned on graphs, using a syntax along the lines of:

\begin{verbatim}
set key top right
\end{verbatim}

The following recognised positional keywords are self-explanatory:
\indkeyt{top}, \indkeyt{bottom}, \indkeyt{left}, \indkeyt{right},
\indkeyt{xcenter} and \indkeyt{ycenter}. Any single instance of the
\indcmdt{set key} can be followed by one horizontal alignment keyword and one
vertical alignment keyword; these keywords also affect the justification of the
legend -- for example, the keyword \indkeyt{left} aligns the legend with its
left edge against the left edge of the plot.

Alternatively, the position of the legend can be indicated using one of the
keywords \indkeyt{outside}, \indkeyt{below} or \indkeyt{above}. These cannot be
combined with the horizontal and vertical alignment keywords above, and are
used to indicate that the legend should be placed, respectively, outside the
plot on its right side, centered beneath the plot, and centered above the plot.

Two comma-separated positional offset coordinates may be specified following
any of the named positions listed above to fine-tune the position of the legend
-- the first value is assumed to be a horizontal offset and the second a
vertical offset. Either may have units of length, or, if they are
dimensionless, are assumed to be measured in centimeters, as the following
examples demonstrate:

\begin{verbatim}
set key bottom left 0.0 -2
set key top xcenter 2*unit(mm),2*unit(mm)
\end{verbatim}

By default, entries in the legend are automatically sorted into an appropriate
number of columns. The number of columns to be used, can, instead, be
stipulated using the \indcmdt{set keycolumns}. This should be followed by
either the integer number of desired columns, or by the keyword {\tt auto} to
indicate that the default behaviour of automatic formatting should be resumed:

\begin{verbatim}
set keycolumns 2
set keycolumns auto
\end{verbatim}

\section{Configuring axes}

\subsection{Adding additional axes}
\label{sec:multiple_axes}

By default, plots have only one horizontal {\tt x}-axis and one vertical {\tt
y}-axis.  Additional axes may be added parallel to these and are referred to
as, for example, the {\tt x2} axis, the {\tt x3} axis, and so forth up to a
maximum of {\tt x127}.  In keeping with this nomenclature, the first axis in
each direction can be referred to interchangeably as, for example, {\tt x} or
{\tt x1}, or as {\tt y} or {\tt y1}.  Further axes are automatically generated
when statements such as the following are issued:

\begin{verbatim}
set x2label 'A second horizontal axis'
\end{verbatim}

\noindent Such axes may alternatively be created explicitly using the
\indcmdt{set axis}, as in the example

\begin{verbatim}
set axis x3
\end{verbatim}

\noindent or removed explicitly using the \indcmdt{unset axis}, as in the
example

\begin{verbatim}
unset axis x3
\end{verbatim}

\noindent In either case, multiple axes can be created or removed in a single
statement, as in the examples

\begin{verbatim}
unset axis x3x5x6 y2
set axis x2y2
\end{verbatim}

\noindent The first axes {\tt x1} and {\tt y1} -- and {\tt z1} on
three-dimensional plots -- are unique in that they cannot be removed; all plots
must have at least one axis in each perpendicular direction.  Thus, the command
{\tt unset axis x1} does not remove this first axis, but merely returns it to
its default configuration.  It should be noted that if the following two
commands are typed in succession, the second may not entirely negate the first:

\begin{verbatim}
set x3label 'foo'
unset x3label
\end{verbatim}

\noindent If an {\tt x3}-axis did not previously exist, then the first will
have implicitly created one. This would need to be removed with the {\tt unset
axis x3} command if it was not desired.

\subsection{Selecting which axes to plot against}

The axes against which data are plotted can be selected by passing the {\tt
axes} modifier to the {\tt plot} command. By default, data is plotted against
the first horizontal axis and the first vertical axis. In the following {\tt
plot} command the second horizontal axis and the third vertical axis would be
used:
\begin{verbatim}
plot f(x) axes x2y3
\end{verbatim}
It is also possible to plot data using a vertical axis as the abscissa axis
using syntax such as:
\begin{verbatim}
plot f(x) axes y3x2
\end{verbatim}
Similar syntax is used when plotting three-dimensional graphs, except that
three perpendicular axes should be specified.

\subsection{Plotting quantities with physical units}
\label{sec:set_axisunitstyle}

When data with non-dimensionless physical units are plotted against an axis,
for example using any of the statements
\begin{verbatim}
plot [0:10] x*unit(m)
plot [0:10] x using 1:$2*unit(m)
plot [0*unit(m):1*unit(m)] x**2

set unit angle nodimensionless ; plot [0:1] asin(x)
\end{verbatim}
the axis is set to share the particular physical dimensions of that unit, and
thereafter no data with any other physical dimensions may be plotted against
that axis. When the axis comes to be drawn, Pyxplot makes a decision about
which physical unit should be used to label the axis. For example, in the
default SI system and with no preferred unit of length set, axes with units of
length might be displayed in millimeters, meters or kilometers depending on
their scales.

The chosen unit is indicated in one of three styles in the axis label, selected
using the \indcmdt{set axisunitstyle}:
\begin{verbatim}
set axisunitstyle ratio
set axisunitstyle bracketed
set axisunitstyle squarebracketed
\end{verbatim}
The effect of these three options, respectively, is shown below for an axis
with units of momentum. In each case, the axis label was set simply using
\begin{verbatim}
set xlabel "Momentum"
\end{verbatim}
and the subsequent text was appended automatically by Pyxplot:

\vspace{3mm}
\centerline{\includegraphics[width=10cm]{examples/eps/ex_axisunits}}
\vspace{3mm}

When the \indcmdt{set xformat} is used (see Section~\ref{sec:set_xformat}), no
indication of the units associated with axes are appended to axis labels, as
the \indcmdt{set xformat} can be used to hard-code this information. The user
must include this information in the axis label manually if it is needed.

\subsection{Specifying the positioning of axes}

By default, the {\tt x1}-axis is placed along the bottom of graphs and the {\tt
y1}-axis is placed up the left-hand side of graphs. On three-dimensional plots,
the {\tt z1}-axis is placed at the front of the graph. The second set of axes
are placed opposite the first: the {\tt x2}-, {\tt y2}- and {\tt z2}-axes are
placed respectively along the top, right and back sides of graphs.
Higher-numbered axes are placed alongside the {\tt x1}-, {\tt y1}- and {\tt
z1}-axes.

However, the position of any axis can be explicitly set using syntax of the
form:
\begin{verbatim}
set axis x top
set axis y right
set axis z back
\end{verbatim}
Horizontal axes can be set to appear either at the {\tt top} or {\tt bottom};
vertical axes can be set to appear either at the {\tt left} or {\tt right}; and
$z$-axes can be set to appear either at the {\tt front} or {\tt back}.

\subsection{Configuring the appearance of axes}

The \indcmdt{set axis} also accepts the following keywords alongside the
positional keywords listed above, which specify how the axis should appear:
\begin{itemize}
\item {\tt arrow} -- Specifies that an arrowhead should be drawn on the right/top end of the axis. [{\bf Not default}].
\item {\tt atzero} -- Specifies that rather than being placed along an edge of the plot, the axis should mark the lines where the perpendicular axes {\tt x1}, {\tt y1} and/or {\tt z1} are zero. [{\bf Not default}].
\item {\tt automirrored} -- Specifies that an automatic decision should be made between the behaviour of {\tt mirrored} and {\tt nomirrored}. If there are no axes on the opposite side of the graph, a mirror axis is produced. If there are already axes on the opposite side of the graph, no mirror axis is produced. [{\bf Default}].
\item {\tt fullmirrored} -- Similar to {\tt mirrored}. Specifies that this axis should have a corresponding twin placed on the opposite side of the graph with mirroring ticks and labelling. [{\bf Not default}; see {\tt automirrored}].
\item {\tt invisible} -- Specifies that the axis should not be drawn; data can still be plotted against it, but the axis is unseen. See Example~\ref{ex:australia} for a plot where all of the axes are invisible.
\item {\tt linked} -- Specifies that the axis should be linked to another axis; see Section~\ref{sec:linked_axes}.
\item {\tt mirrored} -- Specifies that this axis should have a corresponding twin placed on the opposite side of the graph with mirroring ticks but with no labels on the ticks. [{\bf Not default}; see {\tt automirrored}].
\item {\tt noarrow} -- Specifies that no arrowheads should be drawn on the ends of the axis. [{\bf Default}].
\item {\tt nomirrored} -- Specifies that this axis should not have any corresponding twins. [{\bf Not default}; see {\tt automirrored}].
\item {\tt notatzero} -- Opposite of {\tt atzero}; the axis should be placed along an edge of the plot. [{\bf Default}].
\item {\tt notlinked} -- Specifies that the axis should no longer be linked to any other; see Section~\ref{sec:linked_axes}. [{\bf Default}].
\item {\tt reversearrow} -- Specifies that an arrowhead should be drawn on the left/bottom end of the axis. [{\bf Not default}].
\item {\tt twowayarrow} -- Specifies that arrowheads should be drawn on both ends of the axis. [{\bf Not default}].
\item {\tt visible} -- Specifies that the axis should be displayed; opposite of {\tt invisible}. [{\bf Default}].
\end{itemize}

The following simple examples demonstrate the use of some of these configuration options:
\begin{verbatim}
set axis x atzero twoway
set axis y atzero twoway
plot [-2:8][-10:10]
\end{verbatim}

\centerline{\includegraphics[width=8cm]{examples/eps/ex_axisatzero}}

\begin{verbatim}
set axis x atzero arrow
set axis y atzero twoway
plot [0:10][-10:10]
\end{verbatim}

\centerline{\includegraphics[width=8cm]{examples/eps/ex_axisatzero2}}

\begin{verbatim}
set axis x notatzero arrow nomirror
set axis y notatzero arrow nomirror
plot [0:10][0:20]
\end{verbatim}

\centerline{\includegraphics[width=8cm]{examples/eps/ex_axisatzero3}}

\subsection{Setting the color of axes}

The colors of axes\index{axes!color}\index{colors!axes} can be controlled
via the \indcmdts{set axescolor}.  The following example would set axes to be
drawn in blue:

\begin{verbatim}
set axescolor blue
\end{verbatim}

\noindent Any of the color names listed in Section~\ref{sec:color_names} can be
used, as can any object of type {\tt color}, e.g.\ {\tt rgb(0.2,0.1,0.8)}.

\subsection{Specifying where ticks should appear along axes}

By default, Pyxplot places a series of tick marks at significant points along
each axis, with the most significant points being labelled.  Labelled tick
marks are termed {\it major} ticks, and unlabelled tick marks are termed {\it
minor} ticks.  The position and appearance of the major ticks along the {\tt
x}-axis can be configured using the \indcmdt{set xtics}, which has the
following syntax:

\begin{verbatim}
set xtics
    [ ( axis | border | inward | outward | both ) ]
    [ ( autofreq
          | [<minimum>,] <increment> [, <maximum>]
          | \( { '<label>' <position> } \)
         ] )
\end{verbatim}

The corresponding {\tt set mxtics} command, which has the same syntax as above,
configures the appearance of the minor ticks along the {\tt x}-axis. Analogous
commands such as {\tt set ytics} and {\tt set mx2tics} configure the major and
minor ticks along other axes.

The keywords \indkeyt{inward}, \indkeyt{outward} and \indkeyt{both} are used to
configure how the ticks appear -- whether they point inward, towards the plot,
as is default, or outwards towards the axis labels, or in both directions.  The
keyword \indkeyt{axis} is an alias for \indkeyt{inward}, and \indkeyt{border}
an alias for \indkeyt{outward}.

The remaining options are used to configure where along the axis ticks are
placed. If a series of comma-separated values {\tt <minimum>, <increment>,
<maximum>} are specified, then ticks are placed at evenly spaced intervals
between the specified limits. The {\tt <minimum>} and {\tt <maximum>} values
are optional; if only one value is specified then it is taken to be the step
size between ticks. If two values are specified, then the first is taken to be
{\tt <minimum>}. In the case of logarithmic axes, {\tt <increment>} is applied
as a multiplicative step size, and should be dimensionless. For example:
\begin{verbatim}
set xtics 0,1,10 # Ticks at 0,1,2,...,10
set log x
set xtics 2,2  # Ticks at 2,4,8,16 ...
\end{verbatim}

Alternatively, if a bracketed list of quoted tick labels and tick positions are
provided, then ticks can be placed on an axis manually and each given its own
textual label. The quoted tick labels may be omitted, in which case they are
automatically generated:
\begin{verbatim}
set xtics ("a" 1, "b" 2, "c" 3)
set xtics (1,2,3)
\end{verbatim}
The keyword \indkeyt{autofreq} overrides any manual selection of ticks which
may have been placed on an axis and resumes the automatic placement of ticks
along it. The \indcmdt{show xtics}, together with its companions such as {\tt
show x2tics} and {\tt show ytics}, may be used to query the current ticking
options. The \indcmdt{set noxtics} may be used to stipulate that no ticks
should appear along a particular axis:

\begin{verbatim}
set noxtics
show xtics
\end{verbatim}

\example{ex:axistics}{A plot of the function $\exp(x)\sin(1/x)$}{
In this example we produce a plot illustrating some of the zeroes of the
function $\exp(x)\sin(1/x)$.  We set the {\tt x}-axis to have tick marks at
$x=0.05$, $0.1$, $0.2$ and $0.4$.  The {\tt x2}-axis has custom labelled ticks
at $x=1/\pi, 2/\pi$, etc., pointing outwards from the plot.  The left-hand {\tt
y}-axis has tick marks placed automatically whereas the {\tt y2}-axis has no
tics at all.
\nlscf
\input{examples/tex/ex_axistics_1.tex}
\nlscf
\centerline{\includegraphics[width=9cm]{examples/eps/ex_axistics}}
}

\subsection{Configuring how tick marks are labelled}
\label{sec:set_xformat}

By default, the major tick marks along axes are labelled with representations
of the values represented at each point, each accurate to the number of
significant figures specified using the \indcmdt{set numerics sigfig}. These
labels may appear as decimals, such as $3.142$, in scientific notion, as in
$3\times10^8$, or, on logarithmic axes where a base has been specified for the
logarithms, using syntax such as\footnote{Note that the {\tt x} axis must be
referred to as {\tt x1} here to distinguish this statement from {\tt set log
x2}.}
\begin{verbatim}
set log x1 2
\end{verbatim}
in a format such as $1.5\times2^8$.

The \indcmdt{set xformat} -- together with its companions such as {\tt set
yformat}\footnote{There is no {\tt set mxformat} command since minor axis ticks
are never labelled unless labels are explicitly provided for them using the
syntax {\tt set mxtics (...)}.} -- is used to manually specify an explicit
format for the axis labels to take, as demonstrated by the following pair of
examples:
\begin{verbatim}
set xformat "%.2f"%(x)
set yformat "%s$^\prime$"%(y/unit(feet))
\end{verbatim}
The first example specifies that ordinate values should be displayed to two
decimal places along the {\tt x}-axis; the second specifies that distances should
be displayed in feet along the {\tt y}-axis. Note that the dummy variable used to
represent the ordinate value is {\tt x}, {\tt y} or {\tt z} depending on the
direction of the axis, but that the dummy variable used in the {\tt set
x2format} command is still {\tt x}. The following pair of examples both have
the equivalent effect of returning the {\tt x2}-axis to its default system of
tick labels:
\begin{verbatim}
set x2format auto
set x2format "%s"%(x)
\end{verbatim}

The following example specifies that ordinate values should be displayed as
multiples of $\pi$:
\begin{verbatim}
set xformat "%s$\pi$"%(x/pi)
plot [-pi:2*pi] sin(x)
\end{verbatim}

\noindent\centerline{\includegraphics[width=8cm]{examples/eps/ex_axistics2}}

Note that where possible, Pyxplot intelligently changes the positions along
axes where it places the ticks to reflect significant points in the chosen
labelling system.  The extent to which this is possible depends on the format
string supplied. It is generally easier when continuous-varying numerical
values are substituted into strings, rather than discretely-varying values or
strings. Thus, rather than

\begin{dontdo}
set xformat "\%d"\%(floor(x))
\end{dontdo}

\noindent the following is preferred

\begin{dodo}
set xformat "\%d"\%(x)
\end{dodo}

\noindent and rather than

\begin{dontdo}
set xformat "\%s"\%date.str()
\end{dontdo}

\noindent the following is preferred

\begin{dodo}
set xformat "\%d/\%02d/\%d"\%(date.toDayOfMonth(), $\backslash$\newline
date.toMonthNum(), date.toYear())
\end{dodo}

\subsubsection{Changing the slant of axis labels}

The \indcmdt{set xformat} and its companions may also be followed by keywords
which control the angle at which tick labels are drawn. By default, all tick
labels are written horizontally, a behaviour which may be reproduced by issuing
the command:
\begin{verbatim}
set xformat auto horizontal
\end{verbatim}
Alternatively, tick labels may be set to be written vertically, by issuing the command
\begin{verbatim}
set xformat auto vertical
\end{verbatim}
or to be written at any clockwise rotation angle from the horizontal using commands of the form
\begin{verbatim}
set xformat auto rotate 10
\end{verbatim}

Axis labels may also be made to appear at arbitrary rotations using commands such as
\begin{verbatim}
set unit angle nodimensionless
set xlabel "I'm upside down" rotate unit(0.5*revolution)
\end{verbatim}

\subsubsection{Removing axis tick labels}

Axes may be set to have no textual labels associated with the ticks along them
using the command:
\begin{verbatim}
set xformat ""
\end{verbatim}
This is particularly useful when compiling galleries of plots using linked axes
(see the next section) and the multiplot environment (see
Chapter~\ref{ch:vector_graphics}).

\subsection{Linked axes}
\label{sec:linked_axes}

Often it may be desired that multiple axes on a graph share a common range, or
be related to one another by some algebraic expression. For example, a plot
with wavelength $\lambda$ of light as one axis may usefully also have parallel
axes showing frequency of light $\nu=c/\lambda$ or photon energy
$E=hc/\lambda$. The following example sets the {\tt x2} axis to share a common
range with the {\tt x} axis:
\begin{verbatim}
set axis x2 linked x
\end{verbatim}
An algebraic relationship between two axes may be set by stating the algebraic
relationship after the keyword {\tt using}, as in the following example which
implement the formulae shown above for the frequency and energy of photons of
light as a function of their wavelength:
\begin{verbatim}
set xrange [200*unit(nm):unit(800*nm)]
set axis x2 linked x1 using phy.c/x
set axis x3 linked x2 using phy.h*x
\end{verbatim}
As in the {\tt set xformat} command, a dummy variable of {\tt x}, {\tt y} or
{\tt z} is used in the linkage expression depending on the direction of the
axis being linked to, but a dummy variable of {\tt x} is still used when
linking to, for example, the {\tt x2} axis.

As these examples demonstrate, the functions used to link axes need not be
linear. In fact, axes with any arbitrary mapping between position and value can
be produced by linked in a non-linear fashion to another linear axis, which, if
desired, can then be hidden using the {\tt set axis invisible} command.
Multi-valued mappings are also permitted. Any data plotted against the
following {\tt x2}-axis for a suitable range of {\tt x}-axis
\begin{verbatim}
set axis x2 linked x1 using x**2
\end{verbatim}
would appear twice, symmetrically on either side of $x=0$.

Insofar as is possible, linked axes autoscale intelligently when no range is
set.  Thus, if the {\tt x2}-axis is linked to the {\tt x}-axis, and no range to
set for the {\tt x}-axis, the {\tt x}-axis will autoscale to include all of the
data plotted against both itself and the {\tt x2}-axis. Similarly, if the {\tt
x2}-axis is linked to the {\tt x}-axis by means of some algebraic expression,
the {\tt x}-axis will attempt to autoscale to include the data plotted against
the {\tt x2}-axis, though in some cases -- especially with non-monotonic
linking functions -- this may prove too difficult. Where Pyxplot detects that
it has failed, a warning is issued recommending that a hard range be set for --
in this example -- the {\tt x}-axis.

\example{ex:multiaxes}{A plot of many blackbodies demonstrating the use of linked axes}{
In this example we produce a plot of blackbody spectra for five different
temperatures $T$, using the Planck formula
\begin{displaymath}
B_\nu(\nu,T)=\left(\frac{2h^3}{c^2}\right)\frac{\nu^3}{\exp(h\nu/kT)-1}
\end{displaymath}
which is evaluated in Pyxplot by the system-defined mathematical function {\tt
Bv(nu,T)}. We use the axis linkage commands listed as an example in the text of
Section~\ref{sec:linked_axes} to produce three parallel horizontal axes showing
wavelength of light, frequency of light and photon energy.
\nlscf
\input{examples/tex/ex_multiaxes_1.tex}
\nlscf
\centerline{\includegraphics[width=10cm]{examples/eps/ex_multiaxes}}
}

\example{ex:cmbrtemp}{A plot of the temperature of the CMBR as a function of redshift demonstrating non-linear axis linkage}{
In this example we produce a plot of the temperature of the cosmic microwave
background radiation (CMBR) as a function of time $t$ since the Big Bang, on
the {\tt x}-axis, and equivalently as a function of redshift $z$, on the {\tt
x2}-axis.  The specialist cosmology function
\indfunt{ast\_\-Lcdm\_\-z($t$,\-$H_0$,\-$\Omega_\mathrm{M}$,\-$\Omega_\Uplambda$)}
is used to make the highly non-linear conversion between time $t$ and redshift
$z$, adopting some standard values for the cosmological parameters $H_0$,
$\Omega_\mathrm{M}$ and $\Omega_\Uplambda$. Because the temperature of the CMBR
is most easily expressed as a function of redshift as $T=2.73\,\mathrm{K}/(1+z)$,
we plot this function against axis {\tt x2}.
\nlscf
\input{examples/tex/ex_cmbrtemp_1.tex}
\nlscf
\centerline{\includegraphics[width=8cm]{examples/eps/ex_cmbrtemp}}
}

\section{Gridlines}

Gridlines may be placed on a plot and subsequently removed via the statements:

\begin{verbatim}
set grid
set nogrid
\end{verbatim}

\noindent respectively. The following commands are also valid:

\begin{verbatim}
unset grid
unset nogrid
\end{verbatim}

\noindent By default, gridlines are drawn from the major and minor ticks of the
default horizontal and vertical axes (which are the first axes in each
direction unless set otherwise in the configuration file; see
Chapter~\ref{ch:configuration}).  However, the axes which should be used may be
specified after the \indcmdt{set grid}\index{grid}:

\begin{verbatim}
set grid x2y2
set grid x x2y2
\end{verbatim}

\noindent The top example would connect the gridlines to the ticks of the {\tt
x2}- and {\tt y2}-axes, whilst the lower would draw gridlines from both the
{\tt x}- and the {\tt x2}-axes.

If one of the specified axes does not exist, then no gridlines will be drawn in
that direction.  Gridlines can subsequently be removed selectively from some
axes via:

\begin{verbatim}
set nogrid x2x3
\end{verbatim}

\label{sec:set_colors}
The colors of gridlines\index{grid!color}\index{colors!grid} can be
controlled via the \indcmdts{set gridmajcolor} and \indcmdts{set
gridmincolor} commands, which control the gridlines emanating from major and
minor axis ticks respectively. The following example would set the minor grid
lines on a graph to be drawn in blue:

\begin{verbatim}
set gridmajcolor gray70
set gridmincolor blue
\end{verbatim}

\noindent Any of the color names listed in Section~\ref{sec:color_names} can
be used, as can any object of type {\tt color}.

\section{Clipping behaviour}

The treatment of datapoints close to the edges of plots may be specified using
the \indcmdt{set clip}, which provides two options. Either datapoints close to
the axes can be clipped and not allowed to overrun the axes -- specified by
{\tt set clip} -- or such datapoints may be allowed to extend over the lines of
the axes -- specified by {\tt set noclip} and the default behaviour.

\section{Labelling graphs}

The {\tt set arrow}\indcmd{set arrow} and \indcmdt{set label}s allow arrows and
text labels to be added to graphs to label significant points or to add simple
vector graphics to them.

\subsection{Arrows}

\label{sec:set_arrow}\index{arrows} The \indcmdt{set arrow} may be used to
draw arrows on top of graphs; its syntax is illustrated by the following simple
example:

\begin{verbatim}
set arrow 1 from 0,0 to 1,1
\end{verbatim}

\noindent Optionally, a third coordinate may be specified. On 2D plots, this is
ignored. If no third coordinate is supplied then a value of $z=0$ is
substituted when the arrow is plotted on 3D graphs. The number {\tt 1}
immediately following \indcmdts{set arrow} specifies an identification number
for the arrow, allowing it to be subsequently removed via the command

\begin{verbatim}
unset arrow 1
\end{verbatim}

\noindent or equivalently, via\indcmd{set noarrow}

\begin{verbatim}
set noarrow 1
\end{verbatim}

\noindent or to be replaced with a different arrow by issuing a new command of
the form {\tt set arrow 1~...}.  The {\tt set arrow} command may be followed by
the keyword {\tt with} to specify the style of the arrow. The keywords
\indkeyt{nohead}, \indkeyt{head} and \indkeyt{twohead}, placed after the
keyword {\tt with}, can be used to generate arrows with no arrow heads, normal
arrow heads, or with two arrow heads.  \indkeyt{twoway} is an alias for
\indkeyt{twohead}, as in the following example:

\begin{verbatim}
set arrow 1 from 0,0 to 1,1 with twoway
\end{verbatim}

\noindent Line types, line widths and colors can also be specified after the
keyword {\tt with}, as in the example:

\begin{verbatim}
set arrow 1 from 0,0 to 1,1 with nohead \
linetype 1 c blue
\end{verbatim}

The coordinates for the start and end points of the arrow can be specified in a
range of coordinate systems. The coordinate system to be used should be
specified immediately before the coordinate value. The default system,
\indcot{first} measures the graph using the {\tt x}- and {\tt y}-axes. The
\indcot{second} system uses the {\tt x2}- and {\tt y2}-axes. \indcot{axis<n>}
specifies that the position is to be measured along the $n\,$th horizontal or
vertical axis -- for example, {\tt axis3}.\indcmd{set arrow} This allows the
graph to be measured with reference to any arbitrary axis on plots which make
use of large numbers of parallel axes (see Section~\ref{sec:multiple_axes}).
The \indcot{page} and \indcot{graph} systems both measure in centimeters from
the origin of the graph. In the following example, we use these specifiers, and
specify coordinates using variables rather than doing so explicitly:

\begin{verbatim}
x0 = 0.0
y0 = 0.0
x1 = 1.0
y1 = 1.0
set arrow 1 from first  x0, first  y0 \
            to   screen x1, screen y1 \
            with nohead
\end{verbatim}

\subsection{Text labels}

Text labels may be placed on plots using the \indcmdt{set label}. As with all
textual labels in Pyxplot, these are rendered in \latexdcf:

\begin{verbatim}
set label 1 'Hello World' at 0,0
\end{verbatim}

As in the previous section, the number {\tt 1} is a reference number, which
allows the label to be removed by either of the following two commands:

\begin{verbatim}
set nolabel 1
unset label 1
\end{verbatim}

\noindent The positional coordinates for the text label, placed after the {\tt
at} keyword, can be specified in any of the coordinate systems described for
arrows above. As above, either two or three coordinates may be supplied. A
rotation angle may optionally be specified after the keyword \indkeyt{rotate},
to rotate text counter-clockwise by a given angle, measured in degrees. For
example, the following would produce upward-running text:

\begin{verbatim}
set label 1 'Hello World' at axis3 3.0, axis4 2.7 rotate 90
\end{verbatim}

A color can also be specified, if desired, using the {\tt with color}
modifier.  For example, the following would produce a green label at the origin:

\begin{verbatim}
set label 2 'This label is green' at 0, 0 with color green
\end{verbatim}

\index{fontsize}\index{text!size} The size of the text can be set using the {\tt with fontsize} modifier:

\begin{verbatim}
set label 3 'A Big Blue Label' at 0,0 with col blue fontsize 4
\end{verbatim}

\noindent
Alternatively, it may be set globally using the \indcmdt{set fontsize}. This
applies not only to the {\tt set label} command, but also to plot titles, axis
labels, keys, etc. The value supplied should be a multiplicative factor greater
than zero; a value of~{\tt 2} would cause text to be rendered at twice its
normal size, and a value of~{\tt 0.5} would cause text to be rendered at half
its normal size.

\index{text!color}\index{colors!text} The \indcmdt{set textcolor} can be
used to globally set the color of all text output, and applies to all of the
text that the {\tt set fontsize} command does. It is especially useful when
producing plots to be embedded in presentation slideshows, where bright text on
a dark background may be desired. It should be followed either by an integer,
to set a color from the present palette, or by a color name. A list of the
recognised color names can be found in Section~\ref{sec:color_names}.  For
example:

\begin{verbatim}
set textcolor 2
set textcolor blue
\end{verbatim}

\index{text!alignment}\index{alignment!text}By default, each label's specified
position corresponds to its bottom left corner. This alignment may be changed
with the \indcmdts{set texthalign} and \indcmdts{set textvalign} commands. The
former takes the options \indkeyt{left}, \indkeyt{center} or \indkeyt{right},
and the latter takes the options \indkeyt{bottom}, \indkeyt{center} or
\indkeyt{top}, for example:

\begin{verbatim}
set texthalign right
set textvalign top
\end{verbatim}

\example{ex:hlines}{A diagram of the atomic lines of hydrogen}{
The wavelengths of the spectral lines of atomic hydrogen are given by the Rydberg formula,
\begin{displaymath}
\frac{1}{\lambda} = R_\mathrm{H}\left(\frac{1}{n^2}-\frac{1}{m^2}\right),
\end{displaymath}
where $\lambda$ is wavelength, $R_\mathrm{H}$ is the Rydberg constant,
predefined in Pyxplot as the variable {\tt phy\_Ry}, and {\tt n} and {\tt m}
are positive non-zero integers such that {\tt m>n}. The first few series are
called the Lyman series ({\tt n}$=1$), the Balmer series ({\tt n}$=2$), the
Paschen series ({\tt n}$=3$) and the Brackett series ({\tt n}$=4$). Within each
series, the lines are given Greek letter designations -- $\alpha$ for {\tt
m}$=${\tt n}$+1$, $\beta$ for {\tt m}$=${\tt n}$+2$, and so forth.
\nlnp
In the following example, we produce a diagram of the lines in the first four
series, drawing the first~20 lines within each. At the bottom of the diagram,
we overlay indications of the wavelengths of ten color filters commonly used
by astronomers (data taken from Binney \& Merrifield, {\it Galactic Astronomy},
Princeton, 1998).
\nlscf
\input{examples/tex/ex_hlines_1.tex}
\nlscf
\begin{center}
\includegraphics[width=11cm]{examples/eps/ex_hlines}
\end{center}
}

\newpage
\example{ex:australia}{A map of Australia}{
In this example, we use Pyxplot to plot a map of Australia, using a coastal
outline obtained from \protect\url{http://www.maproom.psu.edu/dcw/}. We use the
{\tt set label} command to label the states and major cities. The files {\tt
ex\_\-map\_\-1.dat.gz} and {\tt ex\_\-map\_\-2.dat} can be found in the Pyxplot
installation tarball in the directory {\tt doc/\-examples/}.
\nlscf
\input{examples/tex/ex_map_1.tex}
\nlscf
\begin{center}
\includegraphics[width=\textwidth]{examples/eps/ex_map}
\end{center}
}

\section{Color maps}
\label{sec:colormaps}

Color maps provide a graphical means of producing two-dimensional
representations of $(x,y,z)$ surfaces, or equivalently of producing maps of the
values $z(x,y)$ of functions of two variables. Each point in the $(x,y)$ plane
is assigned a color which indicates the value $z$ associated with that point.
In this section, we refer to the third coordinate as $c_1$ rather than
$z$, to distinguish it from the third axes of three-dimensional
plots\footnote{When color maps are plotted on three-dimensional graphs, they
appear in a flat plane on one of the back faces of the plot selected using the
{\tt axes} modifier to the {\tt plot} command, and the $c_1$-axis associated
with each are entirely independent of the plot's $z$-axis.}.

In the following simple example, a color map of the complex argument of the
Riemann zeta function $\zeta(z)$ is produced, taking the $(x,y)$ plane to be an
Argand plane, with $x$ being the real axis, and $y$ being the imaginary axis.
Each point in the plane has an associated value of $c_1$.

\vspace{2mm}
\input{examples/tex/ex_zeta_arg_1.tex}
\vspace{2mm}

\centerline{\includegraphics[width=8cm]{examples/eps/ex_zeta_arg}}

The \indcmdt{set c1range} sets the range of values of $c_1$ to be assigned
colors between black and white. By default, the lowest and highest values of
$c_1$ found in the color map is assigned to black and white.

The \indcmdt{set c1format} controls the format of the axis labels placed along
the color scale bar on the right-hand side of the plot. In this case, they are
marked as multiples of $\pi$.

The \indcmdt{set samples grid} sets the dimensions of the grid of samples -- or
pixels -- used to render the color map. If either value is replaced with an
asterisk ({\tt *}) then the current number of samples set in the {\tt set
samples} command is substituted.

If a \datafile\ is supplied to the {\tt colormap} plot style, then the
datapoints need not lie on the specified regular grid, but are first re-sampled
onto this grid using the interpolation method specified using the \indcmdt{set
samples interpolate} (see Section~\ref{sec:spline_command}). Three methods are
available. {\tt nearest\-Neigh\-bor} uses the value of $c_1$
associated with the datapoint closest to each grid point, producing color maps
which look like Voronoi diagrams. {\tt inverse\-Square} interpolation returns a
weighted average of the supplied \datapoint s, using the inverse squares of
their distances from each grid point as weights. {\tt monag\-han\-Lattan\-zio}
interpolation uses the weighting function of Monaghan \& Lattanzio (1985) which
is described further in Section~\ref{sec:spline_command}).

In the following example, a color map of a quadrupole is produced using four
input datapoints:

\vspace{2mm}
\input{examples/tex/ex_quadrupole_1.tex}
\vspace{2mm}

\centerline{\includegraphics[width=8cm]{examples/eps/ex_quadrupole}}

\subsection{Custom color mappings}
\label{sec:colmap_custom}

The default mapping used between values of $c_1$ and color is a grayscale
mapping. This is scaled such that the smallest value of $c_1$ in the map
corresponds to black, and largest value corresponds to white.

Alternatively, the user can supply any algebraic expressions for converting
values of $c_1$ into colors.  Moreover, these custom color mappings need not be
one-parameter mappings depending only on a single variable $c_1$, but can
depend on up to four quantities $c_1$, $c_2$, $c_3$ and $c_4$. This makes it
possible, for example, to represent both the amplitude and complex phase of a
quantity in a single color map.

Pyxplot's \indpst{colormap} plot style takes between three and seven columns of
data, which may be supplied either from one or more function(s), or from a
\datafile. If data is read from a \datafile, then the first two columns of
output data are assumed to contain the respective positions of each datapoint
along the $x$-axis and the $y$-axis. The next column contains the value $c_1$,
and may be followed by up to three further optional values $c_2$, $c_3$ and
$c_4$.

In the case where one or more function(s) are supplied, they are assumed to be
functions of both $x$ and $y$, and are sampled at a grid of points in the
$(x,y)$ plane; the first supplied function returns the value $c_1$, and may be
followed by up to three further optional functions for $c_2$, $c_3$ and $c_4$..

The color mapping is set using the \indcmdt{set colormap}, which takes an
algebraic expression which should be a function of the variables {\tt c1}, {\tt
c2}, {\tt c3} and {\tt c4}. This should evaluate either to a color object or a
number (in which case a color is drawn from the current palette).

\begin{verbatim}
set colormap <expr> [ mask <expr> ]
\end{verbatim}

\noindent If the optional mask expression is supplied, then any areas in a
color map where this expression evaluates to false (e.g.\ zero) are made
transparent. The following color mapping, which is the default, produces a
grayscale color mapping of the third column of data supplied to the
\indpst{colormap} plot style; further columns of data, if supplied, are not
used:

\begin{verbatim}
set c1range [*:*] renormalise noreverse
set colormap gray(c1)
\end{verbatim}

The \indcmdt{set c<n>range} command specifies how the values of $c_n$ are
processed before being used in the expressions supplied to the \indcmdt{set
colormap}. It has the following syntax:

\begin{verbatim}
set c<n>range [ <range> ]
              [ reversed | noreversed ]
              [ renormalise | norenormalise ]
\end{verbatim}

\noindent If the {\tt renor\-malise} option is specified, then the values of
$c_n$ at each point in the data grid are first scaled into the range zero to
one. This is often useful, since the color components passed to the {\tt
rgb()}, {\tt gray()}, {\tt hsb()} and {\tt cmyk()} functions must be in this
range.  Thus, in the example given above, the lowest value of $c_1$ corresponds
to black (i.e.\ brightness 0), and the highest value corresponds to white
(i.e.\ brightness 1). If an explicit range is specified to the {\tt set
c<n>range} command, then the upper limit of this range maps to the value one,
and the lower limit maps to the value zero. An asterisk ({\tt *}) means that
the lowest or highest value found in the map is substituted. The mapping is
inverted if the {\tt reverse} option is specified, such that the upper limit
maps to zero, and the lower limit maps to one.  Intermediate values are scaled
either linearly or logarithmically, and these behaviours can be selected with
the following commands:
\begin{verbatim}
set logscale c1
set linearscale c1
\end{verbatim}

In the example below, a color map of the function $f(z)=3x^2/(x^3+1)$ is made,
using hue to represent the magnitude of $f(z)$ and saturation to represent the
complex argument of $f(z)$:

\vspace{2mm}
\input{examples/tex/ex_branch_cuts_1.tex}
\vspace{2mm}

\centerline{\includegraphics[width=8cm]{examples/eps/ex_branch_cuts}}

In the next example, three circular pools of red, green, and blue illumination are overlapped to show how colors mix together:

\vspace{2mm}
\input{examples/tex/ex_spectrum_colmix1_1.tex}
\vspace{2mm}

\centerline{\includegraphics[width=8cm]{examples/eps/ex_spectrum_colmix1}}

The same is possible with CMYK colors, to demonstrate how substractive color mixing works:

\vspace{2mm}
\input{examples/tex/ex_spectrum_colmix2_1.tex}
\vspace{2mm}

\centerline{\includegraphics[width=8cm]{examples/eps/ex_spectrum_colmix2}}

The function {\tt colors.wavelength(wlen,normalisation)} provides a color representation of any given wavelength of light, useful for producing representations of the electromagnetic spectrum:

\vspace{2mm}
\input{examples/tex/ex_spectrum_1_1.tex}
\vspace{2mm}

\centerline{\includegraphics[width=8cm]{examples/eps/ex_spectrum_1}}

The function {\tt colors.spectrum(spectrum,normalisation)} takes a function as its first input, which should return a spectral energy distribution (in arbitrary units) as a function of wavelength. In this example, we show the colors of blackbodies of different temperatures. We renormalise their brightnesses by $T^{-4}$ to avoid saturating hot blackbodies to white:

\vspace{2mm}
\input{examples/tex/ex_spectrum_bbcol_1.tex}
\vspace{2mm}

\centerline{\includegraphics[width=8cm]{examples/eps/ex_spectrum_bbcol}}

As a final example, we use this function to plot the interference pattern seen when a wedge of stressed plastic, a birefrigent material, is viewed between crossed polars:

\vspace{2mm}
\input{examples/tex/ex_spectrum_biref_1.tex}
\vspace{2mm}

\centerline{\includegraphics[width=8cm]{examples/eps/ex_spectrum_biref}}

\subsection{Color scale bars}

By default, plots with color maps with single-parameter color mappings are
accompanied by color scale bars, which appear by default on the right-hand
side of the plot. Such scale bars may be configured using the \indcmdt{set
colorkey}. Issuing the command

\begin{verbatim}
set colorkey
\end{verbatim}

\noindent by itself causes such a scale to be drawn on graphs in the default
position, usually along the right-hand edge of the graphs. The converse action
is achieved by:

\begin{verbatim}
set nocolorkey
\end{verbatim}

\noindent The command

\begin{verbatim}
unset colorkey
\end{verbatim}

\noindent causes Pyxplot to revert to its default behaviour, as specified in a
configuration file, if present. A position for the key may optionally be
specified after the {\tt set colorkey} command, as in the example:

\begin{verbatim}
set colorkey bottom
\end{verbatim}

Recognised positions are {\tt top}, {\tt bottom}, {\tt left} and {\tt right}.
{\tt above} is an alias for {\tt top}; {\tt below} is an alias for {\tt bottom}
and {\tt outside} is an alias for {\tt right}.

The format of the ticks along such scale bars may be set using the \indcmdt{set
c1format} command, which is similar in syntax to the {\tt set xformat} command
(see Section~\ref{sec:set_xformat}), but which uses {\tt c} as its dummy
variable.

The positions of the ticks along color scale bars may be set using the
\indcmdt{set c1tics} command, which has similar syntax to the {\tt set xtics}
command.

\example{ex:mandelbrot}{An image of the Mandelbrot set}{
The Mandelbrot set is a set of points in the complex plane whose boundary forms
a fractal with a Hausdorff dimension of two.  A point $c$ in the complex plane
is defined to lie within the Mandelbrot set if the complex sequence of numbers
\begin{displaymath}
z_{n+1} = z_n^2 + c,
\end{displaymath}
subject to the starting condition $z_0=0$, remains bounded.
\nlnp
The map of this set of points has become a widely-used image of the power of
chaos theory to produce complicated structure out of simple algorithms. To
produce a more pleasing image, points in the complex plane are often colored
differently, depending on how many iterations $n$ of the above series are
required for $|z_n|$ to exceed~2. This is the point of no return, beyond which
it can be shown that $|z_{n+1}|>|z_n|$ and that divergence is guaranteed. In
numerical implementations of the above iteration, in the absence of any better
way to prove that the iteration remains bounded for a certain value of $c$,
some maximum number of iterations $m$ is chosen, and the series is deemed to
have remained bounded if $|z_m|<2$.  This is implemented in Pyxplot by the
built-in mathematical function {\tt
fractal\_\-mandel\-brot(z,m)}\indfun{fractal\_mandelbrot($z$,$m$)}, which
returns an integer in the range $0\leq i\leq m$.
\nlscf
\input{examples/tex/ex_mandelbrot_1.tex}
\nlscf
\begin{center}
\includegraphics[width=8cm]{examples/eps/ex_mandelbrot}
\end{center}
}

\section{Contour maps}
\label{sec:contourmaps}

Contour maps are similar to color maps, but instead of coloring the whole
$(x,y)$ plane, lines are drawn to indicate paths of constant $c(x,y)$. The
number of contours drawn, and the values $c_1$ that they correspond to, is set
using the \indcmdt{set contour}, which has the following syntax:

\begin{verbatim}
set contours [ ( <number> |
               \( { <value> } \) ) ]
             [ (label | nolabel) ]
\end{verbatim}

If {\tt <number>} is specified, as in the example
\begin{verbatim}
set contours 8
\end{verbatim}
then the specified number of contours are drawn at evenly spaced intervals.
Whether the contours are linearly or logarithmically spaced can be changed
using the commands
\begin{verbatim}
set logscale c1
set linearscale c1
\end{verbatim}
By default, the range of values spanned by the contours is automatically scales
to the range of the data provided. However, it may also be set manually using
the \indcmdt{set c1range} as in the example
\begin{verbatim}
set c1range [0:10]
\end{verbatim}
The default autoscaling behaviour can be restored using the command
\begin{verbatim}
set c1range [*:*]
\end{verbatim}

Alternatively, an explicit list of the values of $c$ for which contours should
be drawn may be specified to the \indcmdt{set contour} as a ()-bracketed
comma-separated list. For example:
\begin{verbatim}
set contours (0,5,10,20,40)
\end{verbatim}

If the option {\tt label} is specified to the \indcmdt{set contour}, then each
contour is labelled with the value of $c$ that it corresponds to. If the option
{\tt nolabel} is specified, then the contours are left unlabelled.

In the following example, a contour map is overlaid on top of a color map of
the function $x^3/20+y^2$:

\vspace{2mm}
\input{examples/tex/ex_contourmap_1.tex}
\vspace{2mm}

\centerline{\includegraphics[width=10cm]{examples/eps/ex_contourmap}}
\vspace{2mm}

The {\tt contourmap} plot style differs from other plot styles in that it is
not permitted to take expressions such as {\tt \$2+1} for style modifiers such
as {\tt linetype} (see Section~\ref{sec:with_modifier}) which use additional
columns of input data to plot different points in different styles. However,
the variable {\tt c1} may be used in such expressions to define different
styles for different contours:

\begin{dontdo}
plot 'datafile' with contourmap linetype \$5
\end{dontdo}

\begin{dodo}
plot 'datafile' with contourmap linetype c1/10
\end{dodo}

\section{Three-dimensional plotting}
\label{sec:threedim}

Three-dimensional graphs may be produced by placing the modifier {\tt 3d}
immediately after the {\tt plot} command, as demonstrated by the following
simple example which draws a helix:

\vspace{2mm}
\input{examples/tex/ex_3d_helix_1.tex}
\vspace{2mm}

\centerline{\includegraphics[width=10cm]{examples/eps/ex_3d_helix}}
\vspace{2mm}

Many plot styles take additional columns of data when used on
three-dimen\-sional plots, reading in three values for the $x$, $y$ and $z$
coordinates of each datapoint, where previously only $x$ and $y$ coordinates
were required. In the above example, the {\tt lines} plot style is used, which
takes three columns of input data when used on three-dimensional plots, as
compared to two on two-dimensional plots.  The descriptions of each plot style
in Section~\ref{sec:list_of_plotstyles} includes information on the number of
columns of data required for two- and three-dimensional plots. 

The example above also demonstrates that the \indcmdt{set size} takes an
additional aspect ratio {\tt zratio} which affects three-dimensional plots;
whereas the aspect ratio {\tt ratio} determines the ratio of the lengths of the
$y$-axes of plots to their $x$-axes, the aspect ratio {\tt zratio} determines
the ratio of the lengths of the $z$-axes of plots to their $x$-axes.

The angle from which three-dimensional plots are viewed can be set using the
\indcmdt{set view}. This should be followed by two angles, which can either be
expressed in degrees, as dimensionless numbers, or as quantities with physical
units of angle:
\begin{verbatim}
set view 60,30

set unit angle nodimensionless
set view unit(0.1*rev),unit(2*rad)
\end{verbatim}
The orientation $(0,0)$ corresponds to having the $x$-axis horizontal, the
$z$-axis vertical, and the $y$-axis directed into the page. The first angle
supplied to the {\tt set view} command rotates the plot in the $(x,y)$ plane,
and the second angle tips the plot up in the plane containing the $z$-axis and
the normal to the user's two-dimensional display.

The \indcmdt{replot} command may be used to add additional datasets to
three-dimensional plots in an entirely analogous fashion to two-dimensional
plots.

\subsection{Surface plotting}
\label{sec:surfaces}

The {\tt surface} plot style is similar to the {\tt colormap} and {\tt
contourmap} plot styles, but produces maps of the values $z(x,y)$ of functions
of two variables using three-dimensional surfaces. The surface is displayed as
a grid of four-sided elements, whose number may be specified using the
\indcmdt{set samples}, as in the example
\begin{verbatim}
set samples grid 40x40
\end{verbatim}
If data is supplied from a \datafile, then it is first re-sampled onto a regular
grid using one of the methods described in Section~\ref{sec:colormaps}.

The example below plots a surface indicating the magnitude of the imaginary
part of $\log(x+iy)$:

\vspace{2mm}
\input{examples/tex/ex_surface_log_1.tex}
\vspace{2mm}

\centerline{\includegraphics[width=10cm]{examples/eps/ex_surface_log}}
\vspace{2mm}

\example{ex:surface-polynomial}{A surface plotted above a contour map}{
In this example, we plot a surface showing the value of the expression
$x^3/20+y^2$, and project below it a series of contours in the $(x,y)$ plane.
\nlscf
\input{examples/tex/ex_surface_polynomial_1.tex}
\nlscf
\begin{center}
\includegraphics[width=10cm]{examples/eps/ex_surface_polynomial}
\end{center}
}

\example{ex:surface-sinc}{The sinc($x$) function represented as a surface}{
In this example, we produce a surface showing the function $\mathrm{sinc}(r)$
where $r=\sqrt{x^2+y^2}$. To produce a prettier result, we vary the color of
the surface such that the hue of the surface varies with azimuthal position,
its saturation varies with radius $r$, and its brightness varies with height
$z$.
\nlscf
\input{examples/tex/ex_surface_sinc_1.tex}
\nlscf
\begin{center}
\includegraphics[width=10cm]{examples/eps/ex_surface_sinc}
\end{center}
}

% \section{Non-Flat Projections}


% TERMINALS.TEX
%
% The documentation in this file is part of PyXPlot
% <http://www.pyxplot.org.uk>
%
% Copyright (C) 2006-2011 Dominic Ford <coders@pyxplot.org.uk>
%               2008-2011 Ross Church
%
% $Id$
%
% PyXPlot is free software; you can redistribute it and/or modify it under the
% terms of the GNU General Public License as published by the Free Software
% Foundation; either version 2 of the License, or (at your option) any later
% version.
%
% You should have received a copy of the GNU General Public License along with
% PyXPlot; if not, write to the Free Software Foundation, Inc., 51 Franklin
% Street, Fifth Floor, Boston, MA  02110-1301, USA

% ----------------------------------------------------------------------------

% LaTeX source for the PyXPlot Users' Guide

\chapter{Producing Image Files}
\label{ch:image_formats}

PyXPlot is able to produce graphical output in a wide range of image formats,
including both vector graphic formats such as PostScript and scalable vector
graphics ({\tt svg}), and rasterised formats such as bitmap ({\tt bmp}) and
jpeg. Additionally, it can produce graphical output for immediate preview on
screen. In this chapter we describe how to select and control which image
format should be used.

\section{The {\tt set terminal} Command}
\label{sec:set_terminal}

The \indcmdt{set terminal} is used to select the image format in which output
should be produced, and also to specify a range of fine controls such as
whether output should be in colour or black-and-white. In its simplest usage,
the command is followed by the name of the output image format which is to be
used, which may be any of the options listed in Table~\ref{tab:output_terminals}.

\begin{table}
\centerline{\includegraphics[width=9cm]{examples/eps/ex_set_terminal}}
\caption{A list of the properties of the graphical output formats supported by PyXPlot.}
\label{tab:output_terminals}
\end{table}

\subsection{Previewing Graphs On Screen}
\index{X11 terminal}

Three output terminal produce immediate previews to the screen: {\tt
X11\_\-Single\-Window}, {\tt X11\_\-Persist}, {\tt X11\_\-Multi\-Window}.  The
default of these options -- i.e.\ the default terminal when PyXPlot is started
up in interactive mode -- is {\tt X11\_\-Single\-Window}.  In this terminal,
each time a new plot is generated, if the previous plot is still open on the
display, the old plot is replaced with the new one. This way, only one plot
window is open at any one time. This behaviour is intended to prevent the
desktop from becoming flooded with plot windows.

The alternative {\tt X11\_\-Multi\-Window} terminal is similar in all respects
except that each new plot is generated in a new window, regardless of whether
any previous plots are still open on the display. This is especially useful
when multiple plots are to be compared side-by-side:\index{multiple windows}

\begin{verbatim}
set terminal X11_SingleWindow
plot 'data1.dat'
plot 'data2.dat'  <-- first plot window disappears
\end{verbatim}

\noindent c.f.:

\begin{verbatim}
set terminal X11_MultiWindow
plot 'data1.dat'
plot 'data2.dat'  <-- first plot window remains
\end{verbatim}

The third of these terminals, {\tt X11\_\-Persist}, is similar to {\tt
X11\_\-Multi\-Window} but keeps plot windows open after PyXPlot terminates in
distinction from the above two terminals, which close all plot windows upon
exit.

\subsection{Producing Images on Disk}

The remaining terminals listed in Table~\ref{tab:output_terminals} direct
graphical output to disk in a selection of rasterised and vector graphics
formats. The filename of the resulting image file may be set using the
\indcmdt{set output}, as in the example:

\begin{verbatim}
set output 'my_plot.eps'
\end{verbatim}

Use of rasterised image formats inevitably results in some loss of image
quality since the plot has to be rasterised into a bitmapped graphic image. By
default, this rasterisation is performed at a resolution of
$300\,\mathrm{dpi}$, though this may be changed using the \indcmdt{set terminal
dpi}, which should be followed by a numerical value. Alternatively, the
resolution may be changed using the {\tt DPI} option in the {\tt settings}
section of a configuration file (see Chapter~\ref{ch:configuration}).\index{set
terminal command!dpi modifier@{\tt dpi} modifier}\index{bitmap
output!resolution}\index{image resolution}

\subsection{The Complete Syntax of the {\tt set terminal} Command}

In addition to being used to select the graphical format in which output should
be produced, the \indcmdt{set terminal} takes many options for fine-tuning the
behaviours of particular terminals. Its complete syntax is:

\begin{verbatim}
set terminal ( X11_SingleWindow | X11_MultiWindow | X11_Persist |
               bmp | eps | gif | jpeg | pdf | png | postscript |
               svg | tiff )
             ( colour | color | monochrome )
             ( dpi <value> )
             ( portrait | landscape )
             ( invert | noinvert )
             ( transparent | solid )
             ( antialias | noantialias )
             ( enlarge | noenlarge )
\end{verbatim}

The following table lists the effects which each of these settings has:

\begin{longtable}{p{3cm}p{9cm}}
{\tt X11\_SingleWindow} & Displays plots on the screen (in X11 windows, using \ghostview or other viewing application selected using the \indcmdt{set viewer}). Each time a new plot is generated, it replaces the old one, to prevent the desktop from becoming flooded with old plots.\footnote{The authors are aware of a bug, that this terminal can occasionally go blank when a new plot is generated. This is a known bug in \ghostview, and can be worked around by selecting File $\to$ Reload within the \ghostview\ window.} {\bf [default when running interactively; see below]}\\
{\tt X11\_MultiWindow} & As above, but each new plot appears in a new window, and the old plots remain visible. As many plots as may be desired can be left on the desktop simultaneously.\\
{\tt X11\_Persist} & As above, but plot windows remain open after PyXPlot closes.\\
{\tt bmp} & Sends output to a Windows bitmap ({\tt .bmp}) file. The filename for this file should be set using {\tt set output}. This is a bitmap graphics terminal. \index{bmp output}\\
{\tt eps} & As above, but produces Encapsulated PostScript.\index{Encapsulated PostScript}\index{PostScript!Encapsulated}\\
{\tt gif} & As above, but produces a gif image. This is a bitmap graphics terminal.\index{gif output}\\
{\tt jpeg} & As above, but produces a jpeg image. This is a bitmap graphics terminal.\index{jpeg output}\\
{\tt pdf} & As above, but produces pdf output.\index{pdf output}\\
{\tt png} & As above, but produces a png image. This is a bitmap graphics terminal.\index{png output}\\
{\tt postscript} & As above, but sends output to a PostScript file. {\bf [default when running non-interactively; see below]}\index{PostScript output}\\
{\tt svg} & As above, but produces an svg image.\footnote{The {\tt svg} output terminal is experimental and may be unstable. It relies upon the use of the {\tt svg} output device in Ghostscript, which may not be present on all systems.}\index{svg output}\\
{\tt tiff} & As above, but produces a tiff image. This is a bitmap graphics terminal.\index{tiff output}\\
{\tt colour} & Allows datasets to be plotted in colour. Automatically they will be displayed in a series of different colours, or alternatively colours may be specified using the {\tt with colour} plot modifier (see below). {\bf [default]}\index{colour output}\\
{\tt color} & Equivalent US spelling of the above. \\
{\tt monochrome} & Opposite to the above; all datasets will be plotted in black by default.\index{monochrome output}\\
{\tt dpi} & Sets the number of dots per inch at which rasterised graphic output should be sampled (i.e.\ the output image resolution)\\
{\tt portrait} & Sets plots to be displayed in upright (normal) orientation. {\bf [default]}\index{portrait orientation}\\
{\tt landscape} & Opposite of the above; produces side-ways plots. Not very useful when displayed on the screen, but you fit more on a sheet of paper that way around.\index{landscape orientation}\\
{\tt invert} & Modifier for the bitmap output terminals identified above -- i.e.\ the {\tt bmp}, {\tt gif}, {\tt jpeg}, {\tt png} and {\tt tiff} terminals -- which produces output with inverted colours.\footnote{This terminal setting is useful for producing plots to embed in talk slideshows, which often contain bright text on a dark background. It only works when producing bitmapped output, though a similar effect can be achieved in PostScript using the {\tt set textcolour} and {\tt set axescolour} commands (see Section~\ref{sec:set_colours}).}\index{colours!inverting}\\
{\tt noinvert} & Modifier for the bitmap output terminals identified above; opposite to the above. {\bf [default]}\\
{\tt transparent} & Modifier for the {\tt gif} and {\tt png} terminals; produces output with a transparent background.\index{transparent terminal}\index{gif output!transparency}\index{png output!transparency}\\
{\tt solid} & Modifier for the {\tt gif} and {\tt png} terminals; opposite to the above. {\bf [default]}\\
{\tt antialias} & Modifier for the bitmap output terminals identified above; produces antialiased output, with colour boundaries smoothed to disguise the effects of pixelisation {\bf [default]}\\
{\tt noantialias} & Modifier for the bitmap output terminals identified above; opposite to the above\\
{\tt enlarge} & Enlarge or shrink contents to fit the current paper size.\index{enlarging output}\\
{\tt noenlarge} & Do not enlarge output; opposite to the above. {\bf [default]}\\
\end{longtable}

\section{The Default Terminal}

The default terminal is normally {\tt X11\_SingleWindow}, except when PyXPlot
is used non-interactively -- i.e.\ one or more command scripts are specified on
the command line, and PyXPlot exits as soon as it finishes executing them. In
this case, the {\tt X11\_SingleWindow} would not be a very sensible terminal to
use as any plot window would close as soon as PyXPlot exited. The default
terminal in this case changes to {\tt eps}.

This rule does not apply when the special `--' filename is specified in a list
of command scripts on the command line, where an interactive terminal will
operate between running a series of scripts. In this case, PyXPlot detects that
the session will be interactive, and defaults to the usual {\tt
X11\_SingleWindow} terminal. Conversely, on machines where the {\tt DISPLAY}
environment variable\index{display environment variable@{\tt DISPLAY}
environment variable} is not set, PyXPlot detects that it has access to no
X-terminal on which to display plots, and defaults to the {\tt eps}
terminal.

\section{PostScript Output}

If the {\tt enlarge} modifier is used with the \indcmdt{set terminal} then the
whole plot is enlarged, or, in the case of large plots, shrunk, to the current
paper size, minus a small margin. The aspect ratio of the plot is preserved.

\subsection{Paper Sizes}
\label{sec:set_papersize}

By default, the {\tt postscript} terminal, and the {\tt enlarge} terminal
option, read the paper size for their output from the user's system locale
settings. It may be changed, however, with \indcmdt{set papersize}, which may
be followed either by the name of a recognised paper size, or by the dimensions
of a user-defined size, specified as a {\tt height}, {\tt width} pair, both
being measured in millimetres. For example:

\begin{verbatim}
set papersize a4
set papersize 100,100
\end{verbatim}

\noindent A complete list of recognised paper size names can be found in
Appendix~\ref{ch:paper_sizes}.\footnote{Marcus Kuhn has written a very complete
treatise on international paper sizes, which can be downloaded from:
\url{http://www.cl.cam.ac.uk/~mgk25/iso-paper.html}. Further details on the
Swedish extensions to this system, and the Japanese B-series, can be found on
Wikipedia: \url{http://en.wikipedia.org/wiki/Paper_size}.}\index{Kuhn,
Marcus}\index{paper sizes}

\section{Backing Up Over-Written Files}
\index{overwriting files}\index{backup files}\label{sec:file_backup}

By default, when graphical output is sent to a file -- i.e.\ a PostScript file
or a bitmap image -- any pre-existing file is overwritten if its filename
matches that of the file which PyXPlot generates. This behaviour may be changed
with the \indcmdt{set backup}, which has the syntax:

\begin{verbatim}
set backup
set nobackup
\end{verbatim}

When this switch is turned on, pre-existing files will be renamed with a tilde
($\sim$) appended to their filenames, rather than being overwritten.


% vector_graphics.tex
%
% The documentation in this file is part of PyXPlot
% <http://www.pyxplot.org.uk>
%
% Copyright (C) 2006-2012 Dominic Ford <coders@pyxplot.org.uk>
%               2008-2012 Ross Church
%
% $Id$
%
% PyXPlot is free software; you can redistribute it and/or modify it under the
% terms of the GNU General Public License as published by the Free Software
% Foundation; either version 2 of the License, or (at your option) any later
% version.
%
% You should have received a copy of the GNU General Public License along with
% PyXPlot; if not, write to the Free Software Foundation, Inc., 51 Franklin
% Street, Fifth Floor, Boston, MA  02110-1301, USA

% ----------------------------------------------------------------------------

% LaTeX source for the PyXPlot Users' Guide

\chapter{Producing vector graphics}
\label{ch:vector_graphics}

In the previous two chapters, we have seen how the \indcmdt{plot} may be used
to produce single graphs of functions and \datafile s, how the \indcmdt{set
terminal} can be used to produce graphical output in a wide range of different
image formats (see Section~\ref{sec:set_terminal}), and how the \indcmdt{set
papersize} can be used to produce PostScript output to fit on different sizes
of paper (see Section~\ref{sec:set_papersize}). Often, however, there is a need
to produce more sophisticated vector graphics.  For example, several plots may
be wanted side-by-side, or some line-art annotations may be wanted on top of a
graph. Additionally, PyXPlot can also produce technical diagrams as well as
graphs. In this chapter, we turn our attention to such cases.

Several of the vector graphics commands described in this chapter take rotation
angles as one of their inputs.  It should be stressed that PyXPlot has two
modes for handling angles: they can either be treated as dimensionless numbers
or as being an additional base unit within the SI system (see
Section~\ref{sec:angles}).  Where the vector graphics commands described in
this chapter take rotation angles, a choice is always provided between
expressing the angles as dimensionless numbers, measured in degrees, or as
quantities with physical dimensions of angle. Thus, the following ways of
specifying angles are all valid:

\begin{dodo}
set unit angle nodimensionless\newline
text "foo" rotate unit(pi*rad)\newline
text "foo" rotate 180\newline
text "foo" rotate unit(0.25*revolution)
\end{dodo}

\noindent as are these ways:

\begin{dodo}
set unit angle dimensionless\newline
text "foo" rotate 180\newline
text "foo" rotate degrees(pi)
\end{dodo}

\noindent The following is {\it not} valid, because the angle of $180^\circ$ is
passed to the {\tt text} command in radians:

\begin{dontdo}
set unit angle dimensionless\newline
text "foo" rotate unit(180*deg)
\end{dontdo}

\section{Multiplot mode}
\label{sec:multiplot}
\index{multiplot}

PyXPlot has two modes in which it can produce graphical output. In {\it
singleplot} mode, the default, each time the {\tt plot} command is issued, the
canvas is wiped clean and the new plot is placed alone on a blank page. In {\it
multiplot} mode, vector graphics objects accumulate on the canvas. Each time
the {\tt plot} command is issued, the new graph is placed on top of any other
objects which were already on the canvas, and many plots can be placed
side-by-side.

The user can switch between these two modes of operation by issuing the
commands \indcmdts{set multiplot} and \indcmdts{set nomultiplot}. The
\indcmdt{set origin} is required for multiplot mode to be useful when placing
plots side-by-side: it sets the position on the page of the lower-left corner
of the next plot. It takes a comma-separated $(x,y)$ coordinate pair, which may
have units of length, or, if dimensionless, are assumed to be measured in
centimetres. The following example plots a graph of $\sin(x)$ to the left of a
plot of $\cos(x)$:
\begin{verbatim}
set multiplot
set width 8
plot sin(x)
set origin 10,0
plot cos(x)
\end{verbatim}

All objects on a multiplot canvas have a unique identification number.  By
default, these count up from one, such that the first item placed on the canvas
is number one, the next is number two, and so forth. Alternatively, the user
may specify a particular number for a particular object by supplying the
modifier {\tt item} to the {\tt plot} command, followed by an integer
identification number, as in the following example:
\begin{verbatim}
plot item 6 'data.dat'
\end{verbatim}
If there were already an object on the canvas with identification number~6,
this object would be deleted and replaced with the new object.

A list of all of the objects on the current multiplot canvas can be obtained
using the \indcmdt{list}, which produces output in the following format:
\begin{verbatim}
# ID   Command
    1  plot item 1 'data1.dat'
    2  plot item 2 'data2.dat'
    3  [deleted] plot item 3 'data3.dat'
\end{verbatim}

A multiplot canvas can be wiped clean by issuing the \indcmdt{clear}, which
removes all items currently on the canvas. Alternatively, individual items may
be removed using the \indcmdt{delete}, which should be followed by a
comma-separated list of the identification numbers of the objects to be
deleted.  Deleted items may be restored using the \indcmdt{undelete}, which
likewise takes a comma-separated list of the identification numbers of the
objects to be restored, e.g.:
\begin{verbatim}
delete 1,2
undelete 2
\end{verbatim}
Once a canvas has been cleared using the \indcmdt{clear}, however, there is no
way to restore it.  Objects may be moved around on the canvas using the
\indcmdt{move}. For example, the following would move item 23 to position
$(8,8)$ measured in inches:
\begin{verbatim}
move 23 to 8*unit(in), 8*unit(in)
\end{verbatim}

\subsection{Settings associated with multiplot items}

Of the settings which can be set with the \indcmdt{set}, some refer to
PyXPlot's global environment and whole multiplot canvases. Others, such as {\tt
set width} and {\tt set origin} refer specifically to individual graphs and
vector graphics items. For this reason, whenever a new multiplot graphics item
is produced, it takes a copy of the settings which are specific to it, allowing
these settings to be changed by the user before producing other multiplot
items, without affecting previous items. The settings associated with a
particular multiplot item can be queried by passing the modifier {\tt item} to
the \indcmdt{show}, followed by the integer identification number of the item,
as in the examples:
\begin{verbatim}
show item 3 width    # Shows the width of item 3
show item 3 settings # Shows all settings associated with item 3
\end{verbatim}

The settings associated with a particular multiplot item can be changed by
passing the same {\tt item} modifier to the \indcmdt{set}, as in the example,
which sets the width of item~3 to be $10\,\mathrm{cm}$:
\begin{verbatim}
set item 3 width 10*unit(cm)
\end{verbatim}
After making such changes, the \indcmdt{refresh} is useful: it produces a new
graphical image of the current multiplot to reflect any settings which have
been changed. The following example would produce a pair of plots, and then
change the color of the text on the first plot:
\begin{verbatim}
set multiplot
plot f(x)
set origin 10,0
plot g(x)
set item 1 textcolor red
refresh
\end{verbatim}

Another common use of the \indcmdt{refresh} is to produce multiple
copies of an image in different graphical formats. For example, having just
developed a multiplot canvas interactively in the {\tt X11\_singlewindow},
copies can be produced as {\tt eps} and {\tt jpeg} images using the following
commands:
\begin{verbatim}
set terminal eps
set output 'figure.eps'
refresh
set terminal jpeg
set output 'figure.jpg'
refresh
\end{verbatim}

\subsection{Reordering multiplot items}

Items on multiplot canvases are drawn in order of increasing identification
number, and thus items with low identification numbers are drawn first, at the
back of the multiplot, and items with higher identification numbers are later,
towards the front of the multiplot. When new items are added, they are given
higher identification numbers than previous items and appear at the front of
the multiplot.

If this is not the desired ordering, then the \indcmdt{swap} may be used to
rearrange items. It takes the identification numbers of two multiplot items and
swaps their identification numbers and hence their positions in the ordered
sequence.  Thus, if, for example, the corner of item~3 disappears behind the
corner of item~5, when the converse effect is actually desired, the following
command should be issued:
\begin{verbatim}
swap 3 5
\end{verbatim}

\subsection{The construction of large multiplots}
\label{sec:set_display}

By default, whenever an item is added to a multiplot, or an existing item moved
or replotted, the whole multiplot is replotted to show the change. This can be
a time consuming process on large and complex multiplots. For this reason, the
\indcmdt{set nodisplay} is provided, which stops PyXPlot from producing any
output. The \indcmdt{set display} can subsequently be issued to return to
normal behaviour.

This can be especially useful in scripts which produce large multiplots. There
is no point in producing output at each step in the construction of a large
multiplot, and a great speed increase can be achieved by wrapping the script
with:

\begin{verbatim}
set nodisplay
[...prepare large multiplot...]
set display
refresh
\end{verbatim}

\section{Linked axes and galleries of plots}

In the previous chapter (Section~\ref{sec:linked_axes}), linked axes were
introduced as a mechanism by which several axes on a single plot could be set
to have the same range, or to be algebraically related to one another. Another
common use for them is to make several plots on a multiplot canvas share common
axes. Just as the following statement links two axes on a single plot to one
another
\begin{verbatim}
set axis x2 linked x
\end{verbatim}
axes on the current plot can be linked to those of previous plots which are
already on the multiplot canvas using syntax of the form:
\begin{verbatim}
set axis x2 linked item 2 x
\end{verbatim}

A common reason for doing this is to produce galleries of side-by-side plots.
The following series of commands would produce a $2\times2$ grid of plots, with
axes only labelled along the bottom and left sides of the grid:

\vspace{3mm}
\input{examples/tex/ex_gallery_1.tex}
\vspace{3mm}

\centerline{\includegraphics[width=\textwidth]{examples/eps/ex_gallery}}

\section{The {\tt replot} command revisited}

In multiplot mode, the \indcmdt{replot} can be used to modify the last plot
added to the page. For example, the following would change the title of the
latest plot to `foo', and add a plot of the function $g(x)$ to it:

\begin{verbatim}
set title 'foo'
replot cos(x)
\end{verbatim}

Additionally, it is possible to modify any plot on the page by adding an {\tt
item} modifier to the {\tt replot} statement to specify which plot should be
replotted.  The following example would produce two plots, and then add an
additional function to the first plot:

\begin{verbatim}
set multiplot
plot f(x)
set origin 10,0
plot g(x)
replot item 1 h(x)
\end{verbatim}

If no {\tt item} number is specified, then the \indcmdt{replot} acts by default
upon the most recent plot to have been added to the multiplot canvas.

\section{Adding other vector graphics objects}

In addition to graphs, a range of other objects can be placed on multiplot
canvases:
\begin{itemize}
\item Arcs of circles (the \indcmdt{arc}).
\item Arrows (the \indcmdt{arrow}).
\item Rectangular boxes (the \indcmdt{box}).
\item Circles (the \indcmdt{circle}).
\item Ellipses (the \indcmdt{ellipse}).
\item Encapsulated PostScript images (the \indcmdt{eps}).
\item Graphical images in {\tt bmp}, {\tt gif}, {\tt jpeg} or {\tt png} formats (the \indcmdt{image}).
\item Lines (the \indcmdt{line}).
\item Piecharts (the \indcmdt{piechart}).
\item Points labelled by crosses and other symbols (the \indcmdt{point}).
\item Text labels (the \indcmdt{text}).
\end{itemize}
Put together, these commands can be used to produce a wide range of vector
graphics. In the remainder of this chapter, we describe these commands in turn,
providing a variety of examples of their use.

\subsection{The {\tt text} command}

Text labels may be added to multiplot canvases using the \indcmdt{text}. This
has the following syntax:

\begin{verbatim}
text 'This is some text' at x,y
\end{verbatim}

In this case, the string `This is some text' would be rendered at position
$(x,y)$ on the multiplot. As with the \indcmdt{set label}, a color may
optionally be specified with the {\tt with color} modifier, as well as a
rotation angle to rotate text labels through any given angle, measured in
degrees counter-clockwise. For example:\indkey{rotate}

\begin{verbatim}
text 'This is some text' at x,y rotate r with color red
\end{verbatim}

The commands \indcmdts{set textcolor}, \indcmdts{set texthalign} and
\indcmdts{set textvalign} can be used to set the color and alignment of the
text produced with the \indcmdt{text}. Alternatively, the \indcmdt{text} takes
three modifiers to control the alignment of the text which override these {\tt
set} commands. The {\tt halign} and {\tt valign} modifiers may be followed by
any of the settings which may follow the {\tt set texthalign} and {\tt set
textvalign} commands respectively, as in the following examples:

\begin{verbatim}
text 'This is some text' at 0,0 halign left valign top
text 'This is some text' at 0,0 halign right valign centre
\end{verbatim}

\noindent The {\tt gap} modifier allows a gap to be inserted in the alignment
of the text. For example, the string {\tt halign left gap 3*unit(mm)} would
cause text to be rendered with its left side $3\,\mathrm{mm}$ to the right of
the position specified for the text. This is useful for labelling points on
diagrams, where the labels should be slightly offset from the points that they
are associated with. If the {\tt gap} modifier is followed by a dimensionless
number, rather than one with dimensions of lengths, then it is assumed to be
measured in centimetres.

It should be noted that the \indcmdt{text} can also be used outside of the
multiplot environment, to render a single piece of short text instead of a
graph. One obvious application is to produce equations rendered as graphical
files which can subsequently be imported into documents, slideshows or
webpages.\index{presentations}

\subsection{The {\tt arrow} and {\tt line} commands}

Arrows may also be added to multiplot canvases using the \indcmdt{arrow}, which
has syntax:

\begin{verbatim}
arrow from x,y to x,y
\end{verbatim}

The \indcmdt{arrow} may be followed by the \indmodt{with} keyword to specify to
style of the arrow. The line type, line width and color of the arrow, may be
specified using the same syntax as used in the plot command, using the {\tt
linetype}, {\tt linewidth} and {\tt color} modifiers after the word {\tt
with}, as in the example:

\begin{verbatim}
arrow from 0,0 to 10,10 \
with linetype 2 linewidth 5 color red
\end{verbatim}

\noindent The style of the arrow may also be specified after the word {\tt
with}, and three options are available: {\tt head} (the default), {\tt nohead},
which produces line segments with no arrowheads on them, and {\tt twoway},
which produces bidirectional  arrows with heads on both ends.

The \indcmdt{arrow} has a twin, the \indcmdt{line}, which has the same syntax
but with a different style setting of {\tt nohead}.

\example{ex:notice}{A simple notice generated with the {\tt text} and {\tt arrow} commands}{
In this example script, we use PyXPlot's {\tt arrow} and {\tt text} commands to
produce a simple notice advertising that a lecture has moved to a different
seminar room:
\nlscf
\input{examples/tex/ex_notice_1.tex}
\nlscf
\centerline{\includegraphics[width=\textwidth]{examples/eps/ex_notice}}
}

\example{ex:euclid}{A diagram from Euclid's {\it Elements}}{
In this more extended example script, we use PyXPlot's {\tt arrow} and {\tt
text} commands to reproduce a diagram illustrating the 47th Proposition from
Euclid's First Book of {\it Elements}, better known as Pythagoras' Theorem. A
full text of the proof which accompanies this diagram can be found at
\url{http://www.gutenberg.org/etext/21076}.
\nlscf
\input{examples/tex/ex_euclid_I_47_1.tex}
\nlscf
\centerline{\includegraphics[width=8cm]{examples/eps/ex_euclid_I_47}}
}

\example{ex:nanotubes}{A diagram of the conductivity of nanotubes}{
In this example we produce a diagram of the {\it irreducible wedge} of possible
carbon nanotube configurations, highlighting those configurations which are
electrically conductive. We use PyXPlot's loop constructs to automate the
production of the hexagonal grid which forms the basis of the diagram.
\nlscf
\input{examples/tex/ex_nanotubes_1.tex}
\nlscf
\centerline{\includegraphics[width=9cm]{examples/eps/ex_nanotubes}}
}

\subsection{The {\tt image} command}

Graphical images in {\tt bmp}, {\tt gif}, {\tt jpeg} or {\tt png} format may be
placed on multiplot canvases using the \indcmdt{image}\footnote{To maintain
compatibility with historic versions of PyXPlot, the {\tt image} command may
also be spelt {\tt jpeg}, with the identical syntax thereafter.}. In its
simplest form, this has the syntax:
\begin{verbatim}
image 'filename' at x,y width w
\end{verbatim}

As an alternative to the \indkeyt{width} keyword the height of the image can be
specified, using the analogous \indkeyt{height} keyword.  An optional angle can
also be specified using the \indkeyt{rotate} keyword; this causes the included
image to be rotated counter-clockwise by a specified angle, measured in
degrees.  The keyword {\tt smooth} may optionally be supplied to cause the
pixels of the image to be interpolated\footnote{Many commonly-used PostScript
display engines, including Ghostscript, do not support this functionality.}.

Images which include transparency are supported. The optional keyword {\tt
notransparent} may be supplied to the \indcmdt{image} to cause transparent
regions to be filled with the image's default background color. Alternatively,
an RGB color may be specified in the form {\tt rgb<r>:<g>:<b>} after the
keyword {\tt transparent} to cause that particular color to become
transparent; the three components of the RGB color should be in the range~0
to~255.

\subsection{The {\tt eps} command}

Vector graphic images in eps format may be placed on multiplot canvases
using the \indcmdt{eps}, which has a syntax analogous to the {\tt image}
command.  However neither height nor width need be specified; in this case the
image will be included at its native size.  For example:

\begin{verbatim}
eps 'filename' at 3,2 rotate 5
\end{verbatim}

\noindent will place the eps file with its bottom-left corner at position
$(3,2)$\,cm from the origin, rotated counter-clockwise through 5 degrees.

\subsection{The {\tt box} and {\tt circle} commands}
\label{sec:rectangle}

Rectangular boxes and circles may be placed on multiplot canvases
using the {\tt box} and {\tt circle} commands\indcmd{box}\indcmd{circle}, as
in:

\begin{verbatim}
box from 0*unit(mm),0*unit(mm) to 25*unit(mm),70*unit(mm)
circle at 0*unit(mm),0*unit(mm) radius 70*unit(mm)
\end{verbatim}

\noindent In the former case, two corners of the rectangle are specified,
meanwhile in the latter case the centre of the circle and its radius are
specified. The \indcmdt{box} may also be invoked by the synonym {\tt
rectangle}\indcmd{rectangle}. Boxes may be rotated using an optional {\tt
rotate} modifier, which may be followed by a counter-clockwise rotational angle
which may either have dimensions of angle, or is assumed to be in degrees if
dimensionless. The rotation is performed about the centre of the rectangle:

\begin{verbatim}
box from 0,0 to 10,3 rotate 45
\end{verbatim}

The positions and dimensions of boxes may also be specified by giving the
position of one of the corners of the box, together with its width and height.
The specified corner is assumed to be the bottom-left corner if both the
specified width and height are positive; other corners may be specified if the
supplied width and/or height are negative. If such boxes are rotated, the
rotation is about the specified corner:

\begin{verbatim}
box at 0,0 width 10 height 3 rotate 45
\end{verbatim}

The line type, line width, and color of line with which the outlines of boxes
and circles are drawn may be specified as in the {\tt arrow} command, for
example:

\begin{verbatim}
circle at 0,0 radius 5 with linetype 1 linewidth 2 color red
\end{verbatim}

\noindent The shapes may be filled by specifying a {\tt fillcolor}:

\begin{verbatim}
circle at 0,0 radius 5 with lw 10 color red fillcolor yellow
\end{verbatim}

\example{ex:noentry}{A simple no-entry sign}{
In this example script, we use PyXPlot's {\tt box} and {\tt circle} commands to
produce a no-entry sign warning passers by that code monkeys can turn nasty
when interrupted from their work.
\nlscf
\input{examples/tex/ex_noentry_1.tex}
\nlscf
\centerline{\includegraphics[width=5cm]{examples/eps/ex_noentry}}
}

\subsection{The {\tt arc} command}
\label{sec:arc}

Partial arcs of circles may be drawn using the \indcmdt{arc}. This has similar
syntax to the \indcmdt{circle}, but takes two additional angles, measured
clockwise from the upward vertical direction, which specify the extent of the
arc to be drawn. The arc is drawn clockwise from start to end, and hence the
following two instructions draw two complementary arcs which together form a
complete circle:

\begin{verbatim}
set multiplot
arc at 0,0 radius 5 from -90 to   0 with lw 3 col red
arc at 0,0 radius 5 from   0 to -90 with lw 3 col green
\end{verbatim}

\noindent If a {\tt fillcolor} is specified, then a pie-wedge is drawn:

\begin{verbatim}
arc at 0,0 radius 5 from 0 to 30 with lw 3 fillcolor red
\end{verbatim}

\example{ex:triangle}{A labelled diagram of a triangle}{
In this example, we make a subroutine to draw labelled diagrams of the interior
angles of triangles, taking as its inputs the lengths of the three sides of the
triangle to be drawn and the position of its lower-left corner. The subroutine
calculates the positions of the three vertices of the triangle and then labels
them. We use PyXPlot's automatic handling of physical units to generate the
\LaTeX\ strings required to label the side lengths in centimetres and the
angles in degrees. We use PyXPlot's {\tt arc} command to draw angle symbols in
the three corners of a triangle.
\nlscf
\input{examples/tex/ex_triangle_1.tex}
\nlscf
\centerline{\includegraphics{examples/eps/ex_triangle}}
}

\example{ex:lens}{A labelled diagram of a converging lens forming a real image}{
In this example, we make a subroutine to draw labelled diagrams of converging
lenses forming real images.
\nlscf
\input{examples/tex/ex_lenses_1.tex}
\nlscf
\centerline{\includegraphics{examples/eps/ex_lenses}}
}

\subsection{The {\tt point} command}
\label{sec:point}

The \indcmdt{point} places a single point on a multiplot canvases, in the same
style which would be used when plotting a dataset on a graph with the {\tt
points} plotting style. It is useful for marking significant points on
technical diagrams with crosses or other motifs.

The \indcmdt{point} that the position of the point to be marked be specified
after the {\tt at} modifier. A text label to be attached next to the point may
optionally be specified using the same {\tt label} modifier as taken by the
{\tt plot} command. A {\tt with} modifier may then be supplied, followed by any
of the style modifiers: {\tt color}, {\tt pointlinewidth}, {\tt pointsize},
{\tt pointtype}, {\tt style}.

The following example labels the origin as such:
\begin{verbatim}
set texthalign left
set textvalign centre
point at 0,0 label "The Origin" with ps 2
\end{verbatim}

\subsection{The {\tt ellipse} command}
\label{sec:ellipse}

Ellipses may be placed on multiplot canvases using the \indcmdt{ellipse}. The
shape of the ellipse may be specified in many different ways, by specifying

\begin{enumerate}[(i)]
\item the position of two corners of the smallest rectangle which can enclose
the ellipse when its major axis is horizontal, together with an optional
counter-clockwise rotation angle, applied about the centre of the ellipse.
For example:

\begin{verbatim}
ellipse from 0,0 to 4,1 rot 70
\end{verbatim}

\item the position of both the centre and one of the foci of the ellipse,
together with any one of the following additional pieces of information: the
ellipse's major axis length, its semi-major axis length, its minor axis length,
its semi-minor axis length, its eccentricity, its latus rectum, or its
semi-latus rectum.  For example:

\begin{verbatim}
ellipse focus 0,0 centre 2,2 majoraxis 4
ellipse focus 0,0 centre 2,2 minoraxis 4
ellipse focus 0,0 centre 2,2 ecc 0.5
ellipse focus 0,0 centre 2,2 LatusRectum 6
ellipse focus 0,0 centre 2,2 slr 3
\end{verbatim}

\item the position of either the centre or one of the foci of the ellipse,
together with any two of the following additional pieces of information: the
ellipse's major axis length, its semi-major axis length, its minor axis length,
its semi-minor axis length, its eccentricity, its latus rectum, or its
semi-latus rectum. An optional counter-clockwise rotation angle may also be
specified, applied about either the centre or one of the foci of the ellipse,
whichever is specified. If no rotation angle is given, then the major axis of
the ellipse is horizontal.  For example:

\begin{verbatim}
ellipse centre 0,0 majoraxis 4 minoraxis 4
\end{verbatim}
\end{enumerate}

The line type, line width, and color of line with which the outlines of
ellipses are drawn may be specified after the keyword {\tt with}, as in the
{\tt box} and {\tt circle} commands above. Likewise, ellipses may be filled in
the same manner.

\example{ex:ellipse}{A labelled diagram of an ellipse}{
In this example script, we illustrate the text of Section~\ref{sec:ellipse} by
using PyXPlot's {\tt ellipse} command, together with arrows and text labels, to
produce a labelled diagram of an ellipse. We label the semi-major axis $a$, the
semi-minor axis $b$, the semi-latus rectum $L$, and the distance between the
centre of the ellipse and one of its foci with the length $ae$, where $e$ is
the eccentricity of the ellipse.
\nlscf
\input{examples/tex/ex_ellipse_1.tex}
\nlscf
\centerline{\includegraphics[width=8cm]{examples/eps/ex_ellipse}}
}

\subsection{The {\tt piechart} command}
\label{sec:piechart}

The \indcmdt{piechart} produces piecharts based upon single columns of data
read from \datafile s, which are taken to indicate the sizes of the pie wedges.
The \indcmdt{piechart} has the following syntax:
\begin{verbatim}
piechart ('<filename>'|<function>)
     [using <using specifier>]
     [select <select specifier>]
     [index <index specifier>]
     [every <every specifier>]
     [label <auto|key|inside|outside> <label>]
     [format <format string>]
     [with <style> [<style modifier> ... ] ]
\end{verbatim}

Immediately after the {\tt piechart} keyword, the file (or indeed, function)
from which the data is to be taken should be specified; any of the modifiers
taken by the {\tt plot} command -- i.e.\ {\tt using}, {\tt index}, etc.\ -- may
be used to specify which data from this \datafile\ should be used. The {\tt
label} modifier should be used to specify how a name for each pie wedge should
be drawn from the \datafile, and has a similar syntax to the equivalent
modifier in the {\tt plot} command, except that the name string may be
prefixed by a keyword to specify how the pie wedge names should be positioned.
Four options are available:

\noindent
\begin{itemize}
\item {\tt auto} -- specifies that the {\tt inside} positioning mode should be used on wide pie wedges, and the {\tt outside} positioning mode should be used on narrow pie wedges. {\bf [default]}
\item {\tt key} -- specifies that all of the labels should be arranged in a vertical list to the right-hand side of the piechart.
\item {\tt inside} -- specifies that the labels should be placed within the pie wedges themselves.
\item {\tt outside} -- specifies that the labels should be arranged around the circumference of the pie chart.
\end{itemize}

Having specified a name for each wedge using the {\tt label} modifier, the {\tt
format} modifier determines the final text which is printed along side each
wedge.  For example, a wedge with name `Europe' might be labelled as `27\%
Europe', applying the default format string:
\begin{verbatim}
"%.1d\%% %s"%(percentage,label)
\end{verbatim}
Three variables may be used in format strings: {\tt label} contains the name of
the wedge as specified by the {\tt label} modifier, {\tt percentage} contains
the numerical percentage size of the wedge, and {\tt wedgesize} contains the
absolute unnormalised size of the wedge, as read from the input \datafile,
before the sizes were renormalised to sum to 100\%.

The {\tt with} modifier may be followed by the keywords {\tt color}, {\tt
linewidth}, {\tt style}, which all apply to the lines drawn around the
circumference of the piechart and between its wedges. The fill color of the
wedges themselves are taken sequentially from the current palette, as set by
the {\tt set palette} command. Note that PyXPlot's default palette is optimised
more for producing plots with datasets in different and distinct colors than
for producing piecharts in aesthetically pleasing shades, where a little more
subtly may be desirable. A suitable call to the {\tt set palette} command is
highly recommended before the \indcmdt{piechart} is used.

As with the {\tt plot} command, the position and size of the piechart are
governed by the {\tt set origin} and {\tt set size} commands. The former
determines where the centre of the piechart is positioned; the latter
determines its diameter.

\example{ex:piechart}{A piechart of the composition of the Universe}{
In this example, we use PyXPlot's {\tt piechart} command to produce a diagram
of the composition of the Universe, showing that of the mass in the Universe,
only 4\% is in the form of the baryonic matter; of the rest, 22\% is in the
form of dark matter and 74\% in the form of dark energy:
\nlscf
\input{examples/tex/ex_piechart_1.tex}
\nlscf
\centerline{\includegraphics{examples/eps/ex_piechart}}
\nlfcf
Below, we show the change produced by replacing the line\vspace{2mm}\newline
\noindent{\tt piechart '--' using \$1 label key "\%s"\%(\$2)}\vspace{2mm}\newline
with\vspace{2mm}\newline
\noindent{\tt piechart '--' using \$1 label auto "\%s"\%(\$2)}\vspace{2mm}\newline
Note that the labels on the piechart are placed either within the pie, in the
cases of large wedges, and around the edge of the pie for those wedges which
are too narrow for this.
\nlscf
\centerline{\includegraphics{examples/eps/ex_piechart2}}
}

\section{LaTeX and PyXPlot}

The \indcmdt{text} can straightforwardly be used to render simple one-line
\LaTeX\index{latex} strings, but sometimes the need arises to place more
substantial blocks of text onto a plot. For this purpose, it can be useful to
use the \LaTeX\ {\tt parbox} or {\tt minipage} environments\footnote{Remember,
any valid \LaTeX\ string can be passed to the \indcmdt{text} and \indcmdt{set
label}.}. For example:

\input{examples/tex/ex_text1_1.tex}

\begin{center}
\fbox{\includegraphics{examples/eps/ex_text1}}
\end{center}

If unusual mathematical symbols are required, for example those in the {\tt
amsmath} package\index{amsmath package@{\tt amsmath} package}, such a package
can be loaded using the \indcmdt{set preamble}. For example:

\input{examples/tex/ex_text2_1.tex}

\begin{center}
\fbox{\includegraphics{examples/eps/ex_text2}}
\end{center}


\part{Reference manual}
% REFERENCE.TEX
%
% The documentation in this file is part of PyXPlot
% <http://www.pyxplot.org.uk>
%
% Copyright (C) 2006-2012 Dominic Ford <coders@pyxplot.org.uk>
%               2008-2012 Ross Church
%
% $Id$
%
% PyXPlot is free software; you can redistribute it and/or modify it under the
% terms of the GNU General Public License as published by the Free Software
% Foundation; either version 2 of the License, or (at your option) any later
% version.
%
% You should have received a copy of the GNU General Public License along with
% PyXPlot; if not, write to the Free Software Foundation, Inc., 51 Franklin
% Street, Fifth Floor, Boston, MA  02110-1301, USA

% ----------------------------------------------------------------------------

% LaTeX source for the PyXPlot Users' Guide

\chapter{Command Reference}
\label{ch:reference}

This chapter contains an alphabetically ordered list of all of PyXPlot's
commands. The syntax of each is specified in a variant of Backus-Naur notation,
in which angle brackets {\tt <>} are used to indicate replaceable tokens,
parentheses {\tt ()} are used to indicate mutually-exclusive options which are
separated by vertical lines {\tt |}, square brackets {\tt []} are used to
indicate optional items, and braces {\tt \{\}} are used to indicate items which
may be repeated. Dots {\tt ...} are used to indicate arbitrary strings of text.
Where any of these punctuation marks appear as a part of PyXPlot's syntax, they
are placed in double quotes, as in {\tt "\{"}. Strings, such as filenames,
which should be placed in quotes are shown in single quotes, as in {\tt
'<filename>'}, though in practice, double and single quotes can always be used
interchangeably. In cases where there is likely to be any ambiguity, worked
examples are usually shown beneath the syntax specification.

\section{?}\indcmd{?}

\begin{verbatim}
? [<topic> {<sub-topic>} ]
\end{verbatim}

The {\tt ?} symbol is a shortcut to the {\tt help} command.


\section{!}\indcmd{!}

\begin{verbatim}
! <shell command>
... `<shell command>` ...
\end{verbatim}

Shell commands can be executed within PyXPlot by prefixing them with
pling (!) characters, as in the example:

\begin{verbatim}
!mkdir foo
\end{verbatim}

\noindent As an alternative, back-quotes (`) can be used to substitute the
output of a shell command into a PyXPlot command, as in the example:

\begin{verbatim}
set xlabel `echo "'" ; ls ; echo "'"`
\end{verbatim}

\noindent Note that back-quotes cannot be used inside quote characters, and so
the following would \textit{not} work:

\begin{verbatim}
set xlabel '`ls`'
\end{verbatim}


\section{arc}\indcmd{arc}

\begin{verbatim}
arc [ item <id> ] [at] <x>, <y> radius <r>
    from <start> to <finish> [ with {<option>} ]
\end{verbatim}

Arcs (curves with constant radius of curvature, that is, segments of circles)
may be drawn on multiplot canvases using the \indcmdt{arc}.  The {\tt at}
modifier specifies the coordinates of the centre of curvature, from which all
points on the arc are at the distance given following the {\tt radius} modifier.
The angles {\tt start} and {\tt finish}, measured clockwise from the vertical,
control where the arc begins and ends.  For example, the command

\begin{verbatim}
arc at 0,0 radius 2 from 90 to 270
\end{verbatim}

\noindent would draw a semi-circle beneath the line $x=0$, centred on the
origin with radius $2\,\mathrm{cm}$.  The usual style modifiers for lines may
be passed after the keyword {\tt with}; if the {\tt fillcolour} modifier is
specified then the arc will be filled to form a pie-chart slice.

All vector graphics objects placed on multiplot canvases receive unique
identification numbers which count sequentially from one, and which may be
listed using the {\tt list} command.  By reference to these numbers, they can
be deleted and subsequently restored with the {\tt delete} and {\tt undelete}
commands respectively.


\section{arrow}\indcmd{arrow}

\begin{verbatim}
arrow [ item <id> ] [from] <x>, <y> [to] <x>, <y>
                  [ with {<option>} ]
\end{verbatim}

Arrows may be drawn on multiplot canvases using the \indcmdt{arrow}. The style
of the arrows produced may be specified by following the \indmodt{with}
modifier by one of the style keywords \indkeyt{nohead}, \indkeyt{head}
(default) or \indkeyt{twohead}. In addition, keywords such as \indkeyt{colour},
\indkeyt{linewidth} and \indkeyt{linetype} have the same syntax and meaning
following the keyword \indmodt{with} as in the {\tt plot} command. The
following example would draw a bidirectional blue arrow:

\begin{verbatim}
arrow from x1,y1 to x2,y2 with twohead linetype 2 colour blue
\end{verbatim}

The {\tt arrow} command has a twin, the {\tt line} command, which has the same
syntax, but uses the default arrow style of \indkeyt{nohead}, producing short
line segments.

All vector graphics objects placed on multiplot canvases receive unique
identification numbers which count sequentially from one, and which may be
listed using the {\tt list} command.  By reference to these numbers, they can
be deleted and subsequently restored with the {\tt delete} and {\tt undelete}
commands respectively.


\section{assert}\indcmd{assert}

\begin{verbatim}
assert ( <expression> | version ( >= | < ) <version> )
       [<error message>]
\end{verbatim}

The \indcmdt{assert} can be used to assert that a logical expression, such as
{\tt x>0}, is true. An error is reported if the expression is false, and
optionally a string can be supplied to provide a more informative error message 
to the user:

\begin{verbatim}
assert x>0
assert y<0 "y must be less than zero."
\end{verbatim}

The \indcmdt{assert} can also be used to test the version number of PyXPlot. It
is possible to test either that the version is newer than or equal to a
specific version, using the {\tt $>$=} operator, or that it is older than a
specific version, using the {\tt $<$} operator, as demonstrated in the
following examples:

\begin{verbatim}
assert version >= 0.8.2
assert version <  0.8  "This script is designed for PyXPlot 0.7"
\end{verbatim}


\section{box}\indcmd{box}

\begin{verbatim}
box [ item <id> ] at <x>, <y> width <w> height <h>
         [ rotate <r> ] [ with {<option>} ]

box [ item <id> ] from <x1>, <y1> to <x2>, <y2>
         [ rotate <r> ] [ with {<option>} ]
\end{verbatim}

The \indcmdt{box} is used to draw and fill rectangular boxes on multiplot
canvases.  The position of each box may be specified in one of two ways.  In the
first, the coordinates of one corner of the box are specified, along with its
width and height. If both the width and the height are positive then the
coordinates are taken to be those of the bottom left-hand corner of the box;
other corners may be specified if the supplied width and/or height are
negative. If a rotation angle is specified then the box is rotated about the
specified corner.  The {\tt with} modifier allows the style of the box to be
specified using similar options to those accepted by the {\tt plot} command.

The second syntax allows two pairs of coordinates to be specified.  PyXPlot
will then draw a rectangular box with opposing corners at the specified
locations.  If an angle is specified the box will be rotated about its centre.
Hence the following two commands both draw a square box centred on the origin:

\begin{verbatim}
box from -1, -1 to 1,1
box at 1, -1 width -2 height 2
\end{verbatim}

All vector graphics objects placed on multiplot canvases receive unique
identification numbers which count sequentially from one, and which may be
listed using the {\tt list} command.  By reference to these numbers, they can
be deleted and subsequently restored with the {\tt delete} and {\tt undelete}
commands respectively.


\section{break}\indcmd{break}

\begin{verbatim}
break [<loopname>]
\end{verbatim}

The \indcmdt{break} terminates execution of {\tt do}, {\tt while}, {\tt for}
and {\tt foreach} loops in an analogous manner to the {\tt break} statement in
the C programming language.  Execution resumes at the statement following the
end of the loop. For example, the following loop would only print the numbers~1
and~2:

\begin{verbatim}
for i = 1 to 10
 {
  print i
  if (i==2)
   {
    break
   }
 }
\end{verbatim}

If several loops are nested, the {\tt break} statement only acts on the
innermost loop. If the {\tt break} statement is encountered outside of any loop
structure, an error results. Optionally, the {\tt for}, {\tt foreach}, {\tt do}
and {\tt while} commands may be supplied with a name for the loop, prefixed by
the word {\tt loopname}, as in the examples:

\begin{verbatim}
for i=0 to 4 loopname iloop

foreach i in "*.dat" loopname DatafileLoop
\end{verbatim}

\noindent When loops are given such names, the {\tt break} statement may be
followed by the name of the loop whose iteration is to be broken, allowing it
to act upon loops other than the innermost one.

See also the {\tt continue} command.


\section{cd}\indcmd{cd}

\begin{verbatim}
cd <directory>
\end{verbatim}

PyXPlot's \indcmdt{cd} is very similar to the shell {\tt cd} command; it
changes the current working directory. The following example would enter the
subdirectory {\tt foo}:

\begin{verbatim}
cd foo
\end{verbatim}


\section{circle}\indcmd{circle}

\begin{verbatim}
circle [ item <id> ] [at] <x>, <y> radius <r>
       [ with {<option>} ]
\end{verbatim}

The \indcmdt{circle} is used to draw circles on multiplot canvases.  The
coordinates of the circle's centre and its radius are specified. The {\tt with}
modifier allows the style of the circle to be specified using similar options
to those accepted by the {\tt plot} command.  The example

\begin{verbatim}
circle at 2,2 radius 1 with colour red fillcolour blue
\end{verbatim}

\noindent would draw a red circle of unit radius filled in blue, centred
$2\,\mathrm{cm}$ above and to the right of the origin.

All vector graphics objects placed on multiplot canvases receive unique
identification numbers which count sequentially from one, and which may be
listed using the {\tt list} command.  By reference to these numbers, they can
be deleted and subsequently restored with the {\tt delete} and {\tt undelete}
commands respectively.


\section{clear}\indcmd{clear}

\begin{verbatim}
clear
\end{verbatim}

In multiplot mode, the \indcmdt{clear} removes all plots, arrows and text
objects from the working multiplot canvas. Outside of multiplot mode, it is not
especially useful; it removes the current plot to leave a blank canvas.  The
{\tt clear} command should not be followed by any parameters.


\section{continue}\indcmd{continue}

\begin{verbatim}
continue [<loopname>]
\end{verbatim}

The \indcmdt{continue} terminates execution of the current iteration of {\tt
for}, {\tt foreach}, {\tt do} and {\tt while} loops in an analogous manner to
the {\tt continue} statement in the C programming language. Execution resumes
at the first statement of the next iteration of the loop, or at the first
statement following the end of the loop in the case of the last iteration of
the loop.  For example, the following script will not print the number~2:

\begin{verbatim}
for i = 0 to 5
 {
  if (i==2)
   {
    continue
   }
  print i
 }
\end{verbatim}

If several loops are nested, the {\tt continue} statement only
acts on the innermost loop. If the {\tt continue} statement is encountered outside of any
loop structure, an error results. Optionally, the {\tt for}, {\tt foreach},
{\tt do} and {\tt while} statements may be supplied with a name for the loop, prefixed by
the word {\tt loopname}, as in the examples:

\begin{verbatim}
for i=0 to 4 loopname iloop

foreach i in "*.dat" loopname DatafileLoop
\end{verbatim}

\noindent When loops are given such names, the {\tt continue} statement may be
followed by the name of the loop whose iteration is to be broken, allowing it
to act upon loops other than the innermost one.

See also the {\tt break} command.


\section{delete}\indcmd{delete}

\begin{verbatim}
delete <item number> {, <item number>}
\end{verbatim}

The \indcmdt{delete} removes vector graphics objects such as plots, arrows or
text items from the current multiplot canvas. All vector graphics objects
placed on multiplot canvases receive unique identification numbers which count
sequentially from one, and which may be listed using the {\tt list} command.
The items to be deleted should be identified using a comma-separated list of
their identification numbers. The example

\begin{verbatim}
delete 1,2,3
\end{verbatim}

\noindent would remove item numbers~1,~2 and~3.

Having been deleted, multiplot items can be restored using the {\tt undelete}
command.


\section{do}\indcmd{do}

\begin{verbatim}
do [loopname <loopname>]
 "{"
  ...
 "}" while <condition>
\end{verbatim}

The \indcmdt{do} executes a block of commands repeatedly, checking the
condition given in the {\tt while} clause at the end of each iteration.  If the
condition is true then the loop executes again. This is similar to a {\tt
while} loop, except that the contents of a {\tt do} loop are {\emph always}
executed at least once.  The following example prints the numbers~1, 2 and~3:

\begin{verbatim}
i=1
do
 {
  print i
  i = i + 1
 } while (i < 4)
\end{verbatim}

\noindent Note that there must always be a newline following the opening brace
after the \indcmdt{do}, and the while clause must always be on the same line as
the closing brace.


\section{ellipse}\indcmd{ellipse}

Ellipses may be drawn on multiplot canvases using the \indcmdt{ellipse}. The shape
of the ellipse may be specified in many different ways, by specifying

\begin{enumerate}[(i)]
\item the position of two corners of the smallest rectangle which can enclose
the ellipse when its major axis is horizontal, together with an optional
counter-clockwise rotation angle, applied about the centre of the ellipse.
For example:

\begin{verbatim}
ellipse from 0,0 to 4,1 rot 70
\end{verbatim}

\item the position of both the centre and one of the foci of the ellipse,
together with any one of the following additional pieces of information: the
ellipse's major axis length, its semi-major axis length, its minor axis length,
its semi-minor axis length, its eccentricity, its latus rectum, or its
semi-latus rectum.  For example:

\begin{verbatim}
ellipse focus 0,0 centre 2,2 majoraxis 4
ellipse focus 0,0 centre 2,2 minoraxis 4
ellipse focus 0,0 centre 2,2 ecc 0.5
ellipse focus 0,0 centre 2,2 LatusRectum 6
ellipse focus 0,0 centre 2,2 slr 3
\end{verbatim}

\item the position of either the centre or one of the foci of the ellipse,
together with any two of the following additional pieces of information: the
ellipse's major axis length, its semi-major axis length, its minor axis length,
its semi-minor axis length, its eccentricity, its latus rectum, or its
semi-latus rectum. An optional counter-clockwise rotation angle may also be
specified, applied about either the centre or one of the foci of the ellipse,
whichever is specified. If no rotation angle is given, then the major axis of
the ellipse is horizontal.  For example:

\begin{verbatim}
ellipse centre 0,0 majoraxis 4 minoraxis 4
\end{verbatim}
\end{enumerate}

The line type, line width, and colour of line with which the outlines of
ellipses are drawn may be specified after the keyword {\tt with}, as in the
{\tt box} and {\tt circle} commands above. Likewise, ellipses may be filled in
the same manner.

All vector graphics objects placed on multiplot canvases receive unique
identification numbers which count sequentially from one, and which may be
listed using the {\tt list} command.  By reference to these numbers, they can
be deleted and subsequently restored with the {\tt delete} and {\tt undelete}
commands respectively.


\section{else}\indcmd{else}

The {\tt else} statement is described in the entry for the {\tt if}
statement, of which it forms part.


\section{eps}\indcmd{eps}

\begin{verbatim}
eps [ item <id> ] '<filename>' [at <x>, <y>] [rotate <angle>]
                            [width <width>] [height <height>]
\end{verbatim}

The \indcmdt{eps} allows Encapsulated PostScript (EPS) images to be inserted
onto multiplot canvases.  The {\tt at} modifier can be used to specify where
the bottom-left corner of the image should be placed; if it is not, then the
image is placed at the origin. The {\tt rotate} modifier can be used to rotate
the image by any angle, measured in degrees counter-clockwise.  The {\tt width}
or {\tt height} modifiers can be used to specify the width or height with which
the image should be rendered; both should be specified in centimetres. If
neither is specified then the image will be rendered with the native dimensions
specified within the PostScript.  The {\tt eps} command is often useful in
multiplot mode, allowing PostScript images to be combined with plots, text
labels, etc.

All vector graphics objects placed on multiplot canvases receive unique
identification numbers which count sequentially from one, and which may be
listed using the {\tt list} command.  By reference to these numbers, they can
be deleted and subsequently restored with the {\tt delete} and {\tt undelete}
commands respectively.


\section{exec}\indcmd{exec}

\begin{verbatim}
exec <command>
\end{verbatim}

The \indcmdt{exec} can be used to execute PyXPlot commands contained within
string variables, as in the following example:

\begin{verbatim}
terminal="eps"
exec "set terminal %s"%(terminal)
\end{verbatim}

\noindent It can also be used to write obfuscated PyXPlot scripts, and its use
should be minimised wherever possible.


\section{exit}\indcmd{exit}

\begin{verbatim}
exit
\end{verbatim}

The \indcmdt{exit} can be used to quit PyXPlot. If multiple command files,
or a mixture of command files and interactive sessions, were specified on
PyXPlot's command line, then PyXPlot moves onto the next command-line item
after receiving the {\tt exit} command.

PyXPlot may also be quit be pressing CTRL-D or using the {\tt quit} command. In
interactive mode, CTRL-C terminates the current command, if one is running.
When running a script, CTRL-C terminates execution of the script.


\section{fft}\indcmd{fft}

\begin{verbatim}
fft {<range>} <function>"()"
    of ( '<filename>' | <function>"()" )
    [using <expression> {:<expression>} ]

ifft {<range>} <function>"()"
    of ( '<filename>' | <function>"()" )
    [using <expression> {:<expression>} ]
\end{verbatim}

The \indcmdt{fft} calculates Fourier transforms of \datafile s or functions.
Transforms can be performed on datasets with arbitrary numbers of dimensions.
To transform an algebraic expression with $n$~degrees of freedom, it must be
wrapped in a function of the form $f(i_2,i_2,\ldots,i_n)$. To transform an
$n$-dimensional dataset stored in a \datafile, the samples must be arranged on
a regular linearly-spaced grid and stored in row-major order.  For each
dimension of the transform, a range specification must be provided to the {\tt
fft} command in the form

\begin{verbatim}
[ <minimum> : <maximum> : <step> ]
\end{verbatim}

When data from a \datafile\ is being transformed, the specified range(s) must
precisely match those of the samples read from the file; the first $n$~columns
of data should contain the values of the $n$~real-space coordinates, and the
$n+1$th column should contain the data to be transformed.  After the range(s),
a function name should be provided for the output transform: a function of
$n$~arguments with this name will be generated to represent the transformed
data.  Note that this function is in general complex -- i.e.\ it has a non-zero
imaginary component. Complex numerics can be enabled using the {\tt set
numerics complex} command and the {\tt fft} command is of little use without
doing so. The {\tt using}, {\tt index}, {\tt every} and {\tt select} modifiers
can be used to specify how data will be sampled from the input function or
\datafile\ in an analogous manner to how they are used in the {\tt plot}
command.

The {\tt ifft} command calculates inverse Fourier transforms; it has the same
syntax as the {\tt fft} command.

\section{fit}\indcmd{fit}

\begin{verbatim}
fit [{<range>}] <function>"()" [withouterrors] '<datafile>'
    [index <value>]
    [using <expression> {:<expression>} ]
    via <variable> {, <variable>}
\end{verbatim}

The \indcmdt{fit} can be used to fit arbitrary functional forms to \datapoint s
read from files. It can be used to produce best-fit lines for datasets or to
determine gradients and other mathematical properties of data by looking at the
parameters associated with the best-fitting functional form.  The following
simple example fits a straight line to data in a file called {\tt data.dat}:

\begin{verbatim}
f(x) = a*x+b
fit f() 'data.dat' index 1 using 2:3 via a,b
\end{verbatim}

\noindent The first line specifies the functional form which is to be used.
The coefficients within this function, {\tt a} and {\tt b}, which are to be
varied during the fitting process are listed after the keyword \indkeyt{via}
in the {\tt fit} command.  The modifiers \indmodt{index}, \indmodt{every},
\indmodt{select} and \indmodt{using} have the same meanings in the {\tt fit}
command as in the {\tt plot} command. When fitting a function of $n$
variables, at least $n+1$ columns (or rows -- see
Section~\ref{sec:horizontal_datafiles}) of data must be specified after the {\tt using}
modifier. By default, the first $n+1$ columns are used. These correspond to the
values of each of the $n$ arguments to the function, plus finally the value which
the output from the function is aiming to match.  If an additional column is
specified, then this is taken to contain the standard error in the value that
the output from the function is aiming to match, and can be used to weight the
\datapoint s which are being used to constrain the fit.

As the {\tt fit} command works, it displays statistics including the best-fit
values of each of the fitting parameters, the uncertainties in each of them,
and the covariance matrix. These can be useful for analysing the security of
the fit achieved, but calculating the uncertainties in the best-fit parameters
and the covariance matrix can be time consuming, especially when many
parameters are being fitted simultaneously. The optional keyword {\tt
withouterrors} can be included immediately before the filename of the
\datafile\ to be fitted to substantially speed up cases where this information
is not required.

By default, the starting values for each of the fitting parameters is
$1.0$. However, if the variables to be used in the fitting process are already
set before the {\tt fit} command is called, these initial values are used
instead. For example, the following would use the initial values
$\{a=100,b=50\}$:
\begin{verbatim}
f(x) = a*x+b
a = 100
b = 50
fit f() 'data.dat' index 1 using 2:3 via a,b
\end{verbatim}

More details can be found in Section~\ref{sec:fit_command}.


\section{for}\indcmd{for}

\begin{verbatim}
for <variable> = <start> to <end> [step <step>]
                          [loopname <loopname>]
 "{"
  ...
 "}"
\end{verbatim}

The \indcmdt{for} executes a set of commands repeatedly, with a specified
variable taking a different value on each iteration. The variable takes the
value {\tt start} on the first iteration, and increases by a fixed value {\tt
step} on each iteration; {\tt step} may be negative if {\tt end} $<$ {\tt
start}. If {\tt step} is not specified then a value of unity is assumed. The
loop terminates when the variable exceeds {\tt end}.  The following example
prints the squares of the first five natural numbers:

\begin{verbatim}
for i = 1 to 5
 {
  print i**2
 }
\end{verbatim}


\section{foreach}\indcmd{foreach}

\begin{verbatim}
foreach <variable> in ( <filename expression> |
                         "("<value> {, <value>}")" )
                      [loopname <loopname>]
 "{"
  ...
 "}"
\end{verbatim}

The \indcmdt{foreach} can be used to run a block of commands repeatedly, once
for each item in a list.  The list of items can be specified in one of two
ways.  In the first case, a set of filenames or filename wildcards is supplied,
and the {\tt foreach} loop iterates once for each supplied filename, with a
string variable set to each filename in succession.  For example, the following
loop would plot the data in the set of files whose names end with {\tt .dat}:

\begin{verbatim}
plot     # Create blank plot
foreach file in *.dat
 {
  replot file with lines
 }
\end{verbatim}

The second form of the command takes a list of string or numerical values
provided explicitly by the user, and the {\tt foreach} loop iterates once for
each value, with a variable set to each value in succession.  For example, the
following script would plot normal distributions of three different widths:

\begin{verbatim}
plot     # Create blank plot
foreach sigma in (1, 2, 3)
 {
  replot 1/sigma*exp(-x**2/(2*sigma**2))
 }
\end{verbatim}


\section{help}\indcmd{help}

\begin{verbatim}
help [<topic> {<sub-topic>} ]
\end{verbatim}

The \indcmdt{help} provides an hierarchical source of information which is
supplementary to this Users' Guide.  To obtain information on any particular
topic, type {\tt help} followed by the name of the topic, as in the following
example

\begin{verbatim}
help commands
\end{verbatim}

\noindent which provides information on PyXPlot's commands. Some topics have
sub-topics; these are listed at the end of each help page. To view them, add
further words to the end of the help request, as in the example:

\begin{verbatim}
help commands help
\end{verbatim}

Information is arranged with general information about PyXPlot under the
heading {\tt about} and information about PyXPlot's commands under {\tt
commands}.  Information about the format that input \datafile s should take can
be found under {\tt datafile}.  Other categories are self-explanatory.

To exit any help page, press the {\tt q} key.


\section{histogram}\indcmd{histogram}

\begin{verbatim}
histogram [<range>] <function name>"()" '<datafile>'
     [every <expression> {:<expression>} ]
     [index <value>]
     [select <expression>]
     [using <expression> {:<expression>} ]
     ( [binwidth <value>] [binorigin <value>] |
       [bins (x1, x2, ...)] )
\end{verbatim}

The \indcmdt{histogram} takes a single column of data from a file and produces
a function that represents the frequency distribution of the supplied data
values. The output function consists of a series of discrete intervals which we
term {\it bins}. Within each interval the output function has a constant value,
determined such that the area under each interval -- i.e.\ the integral of the
function over each interval -- is equal to the number of datapoints found
within that interval.  The following simple example

\begin{verbatim}
histogram f() 'input.dat'
\end{verbatim}

\noindent produces a frequency distribution of the data values found in the
first column of the file {\tt input.dat}, which it stores in the function
$f(x)$. The value of this function at any given point is equal to the number of
items in the bin at that point, divided by the width of the bins used. If the
input datapoints are not dimensionless then the output frequency distribution
adopts appropriate units, thus a histogram of data with units of length has
units of one over length.

The number and arrangement of bins used by the \indcmdt{histogram} can be
controlled by means of various modifiers.  The \indmodt{binwidth} modifier sets
the width of the bins used. The \indmodt{binorigin} modifier controls where
their boundaries lie; the \indcmdt{histogram} selects a system of bins which,
if extended to infinity in both directions, would put a bin boundary at the
value specified in the {\tt binorigin} modifier. Thus, if {\tt binorigin 0.1}
were specified, together with a bin width of~20, bin boundaries might lie
at~$20.1$, $40.1$, $60.1$, and so on. Alternatively global defaults for the bin
width and the bin origin can be specified using the {\tt set binwidth} and {\tt
set binorigin} commands respectively. The example

\begin{verbatim}
histogram h() 'input.dat' binorigin 0.5 binwidth 2
\end{verbatim}

\noindent would bin data into bins between $0.5$ and $2.5$, between $2.5$ and
$4.5$, and so forth.

Alternatively the set of bins to be used can be controlled more precisely using
the \indmodt{bins} modifier, which allows an arbitrary set of bins to be
specified. The example

\begin{verbatim}
histogram g() 'input.dat' bins (1, 2, 4)
\end{verbatim}

\noindent would bin the data into two bins, $x=1\to 2$ and $x=2\to 4$.

A range can be supplied immediately following the {\tt histogram} command,
using the same syntax as in the {\tt plot} and {\tt fit} commands; if such a
range is supplied, only points that fall within that range will be binned.  In
the same way as in the {\tt plot} command, the \indmodt{index},
\indmodt{every}, \indmodt{using} and \indmodt{select} modifiers can be used to
specify which subsets of a \datafile\ should be used.

Two points about the {\tt histogram} command are worthy of note. First,
although histograms are similar to bar charts, they are not the same.  A bar
chart conventionally has the height of each bar equal to the number of points
that it represents, whereas a histogram is a continuous function in which the
area underneath each interval is equal to the number of points within it.
Thus, to produce a bar chart using the {\tt histogram} command, the end result
should be multiplied by the bin width used.

Second, if the function produced by the {\tt histogram} command is plotted
using the {\tt plot} command, samples are automatically taken not at evenly
spaced intervals along the ordinate axis, but at the centres of each bin. If
the \indpst{boxes} plot style is used, the box boundaries are be conveniently
drawn to coincide with the bins into which the data were sorted.


\section{history}\indcmd{history}

\begin{verbatim}
history [<number of items>]
\end{verbatim}

The \indcmdt{history} prints a list of the most recently executed commands on
the terminal.  The optional parameter, {\tt N}, if supplied, causes only the
latest $N$ commands to be displayed.


\section{if}\indcmd{if}\indcmd{else if}\indcmd{else}

\begin{verbatim}
if <criterion>
 "{"
    ...
 "}" { else if <criterion>
 "{"
    ...
 "}" } [ else
 "{"
    ...
 "}" ]
\end{verbatim}

The \indcmdt{if} allows conditional execution of blocks of commands.  The code
enclosed in braces following the {\tt if} statement is executed if, and only
if, the {\tt criterion} is satisfied.  An arbitrary number of subsequent {\tt
else if} statements can optionally follow the initial {\tt if} statement; these
have their own criteria for execution which are only considered if all of the
previous criteria have tested false -- i.e.\ if none of the previous command
blocks have been executed.  A final optional {\tt else} statement can be
provided; the block of commands which follows it are executed only if none of
the preceding criteria have tested true.  The following example illustrates a
chain of {\tt else if} clauses:

\begin{verbatim}
if (x==2)
 {
  print "x is two!"
 } else if (x==3) {
  print "x is three!"
 } else if (x>3) {
  print "x is greater than three!"
 } else {
  x=2
  print "x didn't used to be two, but it is now!"
 }
\end{verbatim}


\section{ifft}\indcmd{ifft}

\begin{verbatim}
ifft {<range>} <function>"()"
    of ( '<filename>' | <function>"()" )
    [using <expression> {:<expression>} ]
\end{verbatim}

See {\tt fft}.


\section{image}\indcmd{image}

\begin{verbatim}
image [ item <id> ] '<filename>' [at <x>, <y>] [rotate <angle>]
                     [width <width>] [height <height>] [smooth]
                   [notransparent] [transparent rgb<r>:<g>:<b>]
\end{verbatim}

The \indcmdt{image} allows graphical images to be inserted onto the current
multiplot canvas from files on disk. Input graphical images may be in bitmap,
gif, jpeg or png formats; the file type of each image is automatically
detected. The {\tt at} modifier can be used to specify where the bottom-left
corner of each image should be placed; if no position is specified then this
corner of the image is placed at the origin. The {\tt rotate} modifier can be
used to rotate images by any angle, measured in degrees counter-clockwise.  The
{\tt width} or {\tt height} modifiers can be used to specify the width or
height with which images should be rendered; both should be specified in
centimetres. If neither is specified then images are rendered with the native
dimensions specified within the metadata present in the image file (if any). If
both are specified, then the aspect ratio of the image is changed.

The keyword {\tt smooth} may optionally be supplied to cause the pixels of
images to be interpolated\footnote{Many commonly-used PostScript display
engines, including Ghostscript, do not support this functionality.}.  Images
which include transparency are supported. The optional keyword {\tt
notransparent} may be supplied to cause transparent regions to be filled with
the image's default background colour. Alternatively, an RGB colour may be
specified in the form {\tt rgb<r>:<g>:<b>} after the keyword {\tt transparent}
to cause that particular colour to become transparent; the three components of
the RGB colour should be in the range~0 to~255.

All vector graphics objects placed on multiplot canvases receive unique
identification numbers which count sequentially from one, and which may be
listed using the {\tt list} command.  By reference to these numbers, they can
be deleted and subsequently restored with the {\tt delete} and {\tt undelete}
commands respectively.


\section{interpolate}\indcmd{interpolate}

\begin{verbatim}
interpolate ( akima | linear | loglinear | polynomial |
              spline | stepwise |
              2d [( bmp_r | bmp_g | bmp_b )] )
            [<range specification>] <function name>"()"
            '<filename>'
            [ every <expression> {:<expression} ]
            [ index <value> ]
            [ select <expression> ]
            [ using <expression> {:<expression} ]
\end{verbatim}

The \indcmdt{interpolate} can be used to generate a special function within
PyXPlot's mathematical environment which interpolates a set of \datapoint s
supplied from a \datafile. Either one- or two-dimensional interpolation is
possible.

In the case of one-dimensional interpolation, various different types of
interpolation are supported: linear interpolation, power law interpolation,
polynomial interpolation, cubic spline interpolation and akima spline
interpolation. Stepwise interpolation returns the value of the datapoint
nearest to the requested point in argument space. The use of polynomial
interpolation with large datasets is strongly discouraged, as polynomial fits
tend to show severe oscillations between \datapoint s.  Except in the case of
stepwise interpolation, extrapolation is not permitted; if an attempt is made
to evaluate an interpolated function beyond the limits of the \datapoint s
which it interpolates, PyXPlot returns an error or value of not-a-number.

In the case of two-dimensional interpolation, the type of interpolation to be
used is set using the {\tt interpolate} modifier to the \indcmdt{set samples},
and may be changed at any time after the interpolation function has been
created.  The options available are nearest neighbour interpolation -- which is
the two-dimensional equivalent of stepwise interpolation, inverse square
interpolation -- which returns a weighted average of the supplied \datapoint s,
using the inverse squares of their distances from the requested point in
argument space as weights, and Monaghan Lattanzio interpolation, which uses the
weighting function (Monaghan \& Lattanzio 1985)
\begin{eqnarray*}
w(x) & = 1 - \nicefrac{3}{2}v^2 + \nicefrac{3}{4}v^3 & \,\mathrm{for~}0\leq v\leq 1 \\
     & = \nicefrac{1}{4}(2-v)^3                      & \,\mathrm{for~}1\leq v\leq 2
\end{eqnarray*}
where $v=r/h$ for $h=\sqrt{A/n}$, $A$ is the product
$(x_\mathrm{max}-x_\mathrm{min})(y_\mathrm{max}-y_\mathrm{min})$ and $n$ is the
number of input datapoints. These are selected as follows:

\begin{verbatim}
set samples interpolate NearestNeighbour
set samples interpolate InverseSquare
set samples interpolate MonaghanLattanzio
\end{verbatim}

Finally, data can be imported from graphical images in bitmap ({\tt .bmp}) 
format to produce a function of two arguments returning a value in the
range~$0\to1$ which represents the data in one of the image's three colour
channels. The two arguments are the horizontal and vertical position within the
bitmap image, as measured in pixels.

A very common application of the \indcmdt{interpolate} is to perform arithmetic
functions such as addition or subtraction on datasets which are not sampled at
the same abscissa values. The following example would plot the difference
between two such datasets:

\begin{verbatim}
interpolate linear f() 'data1.dat'
interpolate linear g() 'data2.dat'
plot [min:max] f(x)-g(x)
\end{verbatim}

\noindent Note that it is advisable to supply a range to the {\tt plot} command
in this example: because the two datasets have been turned into continuous
functions, the {\tt plot} command has to guess a range over which to plot them
unless one is explicitly supplied.

The \indcmdt{spline} is an alias for {\tt interpolate spline}; the following
two statements are equivalent:

\begin{verbatim}
spline f() 'data1.dat'
interpolate spline f() 'data1.dat'
\end{verbatim}


\section{jpeg}\indcmd{jpeg}

\begin{verbatim}
jpeg [ item <id> ] '<filename>' [at <x>, <y>] [rotate <angle>]
                    [width <width>] [height <height>] [smooth]
                  [notransparent] [transparent rgb<r>:<g>:<b>]
\end{verbatim}

See {\tt image}.


\section{let}\indcmd{let}

\begin{verbatim}
let <varname> = <value>
\end{verbatim}

The \indcmdt{let} sets the variable {\tt varname} to equal {\tt value}.


\section{list}\indcmd{list}

\begin{verbatim}
list
\end{verbatim}

The \indcmdt{list} returns a list of all of the items on the current multiplot
canvas, giving their identification numbers and the commands used to produce
them.  The following is an example of the output produced:

{\footnotesize
\begin{verbatim}
pyxplot> list
# ID   Command
    1  plot item 1 f(x) using columns
    2  [deleted] text item 2 "Figure 1: A plot of f(x)" at 0,0 rotate 0 gap 0
    3  text item 3 "Figure 1: A plot of $f(x)$" at 0,0 rotate 0 gap 0
\end{verbatim}
}

In this example, the user has plotted a graph of $f(x)$ and added a caption to
it. The {\tt ID} column lists the reference numbers of each multiplot item.
Item {\tt 1} has been deleted.


\section{load}\indcmd{load}

\begin{verbatim}
load '<filename>'
\end{verbatim}

The \indcmdt{load} executes a PyXPlot command script file, just as if its
contents had been typed into the current terminal. For example:

\begin{verbatim}
load 'foo'
\end{verbatim}

\noindent would have the same effect as typing the contents of the file {\tt
foo} into the present terminal.  Filename wildcard can be supplied, in which
case {\it all} command files matching the given wildcard are executed, as in
the example:

\begin{verbatim}
load '*.script'
\end{verbatim}


\section{maximise}\indcmd{maximise}

\begin{verbatim}
maximise <expression> via <variable> {, variable}
\end{verbatim}

The \indcmdt{maximise} can be used to find the maxima of algebraic expressions.
A single algebraic expression should be supplied for optimisation, together
with a comma-separated list of the variables with respect to which it should be
optimised.  In the following example, a maximum of the sinusoidal function
$\cos(x)$ is sought:

\vspace{3mm}
\noindent{\tt pyxplot> {\bf set numerics real}}\newline
\noindent{\tt pyxplot> {\bf x=0.1}}\newline
\noindent{\tt pyxplot> {\bf maximise cos(x) via x}}\newline
\noindent{\tt pyxplot> {\bf print x/pi}}\newline
\noindent{\tt 0}
\vspace{3mm}

\noindent Note that this particular example doesn't work when complex
arithmetic is enabled, since $\cos(x)$ diverges to $\infty$ at $x=\infty i$.

Various caveats apply the {\tt maximise} command, as well as to the {\tt
minimise} and {\tt solve} commands.  All of these commands operate by searching
numerically for optimal sets of input parameters to meet the criteria set by
the user. As with all numerical algorithms, there is no guarantee that the {\it
locally} optimum solutions returned are the {\it globally} optimum solutions.
It is always advisable to double-check that the answers returned agree with
common sense.

These commands can often find solutions to equations when these solutions are
either very large or very small, but they usually work best when the solution
they are looking for is roughly of order unity.  PyXPlot does have mechanisms
which attempt to correct cases where the supplied initial guess turns out to be
many orders of magnitude different from the true solution, but it cannot be
guaranteed not to wildly overshoot and produce unexpected results in such
cases.  To reiterate, it is always advisable to double-check that the answers
returned agree with common sense.


\section{minimise}\indcmd{minimise}

\begin{verbatim}
minimise <expression> via <variable> {, variable}
\end{verbatim}

The \indcmdt{minimise} can be used to find the minima of algebraic expressions.
A single algebraic expression should be supplied for optimisation, together
with a comma-separated list of the variables with respect to which it should be
optimised. In the following example, a minimum of the sinusoidal function
$\cos(x)$ is sought:

\vspace{3mm}
\noindent{\tt pyxplot> {\bf set numerics real}}\newline
\noindent{\tt pyxplot> {\bf x=0.1}}\newline
\noindent{\tt pyxplot> {\bf minimise cos(x) via x}}\newline
\noindent{\tt pyxplot> {\bf print x/pi}}\newline
\noindent{\tt 1}
\vspace{3mm}

\noindent Note that this particular example doesn't work when complex
arithmetic is enabled, since $\cos(x)$ diverges to $-\infty$ at $x=\pi+\infty
i$.

Various caveats apply the {\tt minimise} command, as well as to the {\tt
maximise} and {\tt solve} commands.  All of these commands operate by searching
numerically for optimal sets of input parameters to meet the criteria set by
the user. As with all numerical algorithms, there is no guarantee that the {\it
locally} optimum solutions returned are the {\it globally} optimum solutions.
It is always advisable to double-check that the answers returned agree with
common sense.

These commands can often find solutions to equations when these solutions are
either very large or very small, but they usually work best when the solution
they are looking for is roughly of order unity.  PyXPlot does have mechanisms
which attempt to correct cases where the supplied initial guess turns out to be
many orders of magnitude different from the true solution, but it cannot be
guaranteed not to wildly overshoot and produce unexpected results in such
cases.  To reiterate, it is always advisable to double-check that the answers
returned agree with common sense.


\section{move}\indcmd{move}

\begin{verbatim}
move <item number> to <x>, <y>
\end{verbatim}

The \indcmdt{move} allows vector graphics objects to be moved around on the
current multiplot canvas. All vector graphics objects placed on multiplot
canvases receive unique identification numbers which count sequentially from
one, and which may be listed using the {\tt list} command. The item to be moved
should be specified using its identification number. The example

\begin{verbatim}
move 23 to 8,8
\end{verbatim}

\noindent would move multiplot item~23 to position $(8,8)$ centimetres. If this
item were a plot, the end result would be the same as if the command {\tt set
origin 8,8} had been executed before it had originally been plotted.


\section{plot}\indcmd{plot}

\begin{verbatim}
plot [3d] [item <id>] [{<range>}] ( '<filename>' | <function> )
     [axes <axes>] [every <expression> {:<expression>}]
     [index <value>] [select <expression>]
     [label <string expression>]
     [title <string>] [using <expression> {:<expression>}]
     [with {<option>}]
\end{verbatim}

The \indcmdt{plot} is used to produce graphs. The following simple example
would plot the sine function:

\begin{verbatim}
plot sin(x)
\end{verbatim}

Ranges for the axes of a graph can be specified by placing them in
square brackets before the name of the function to be plotted. An example of
this syntax would be:

\begin{verbatim}
plot [-pi:pi] sin(x)
\end{verbatim}

\noindent which would plot the function $\sin(x)$ between $-\pi$ and $\pi$.

\Datafile s may also be plotted as well as functions, in which case the
filename of the \datafile\ to be plotted should be enclosed in either single or
double quotation marks. An example of this syntax would be:

\begin{verbatim}
plot 'data.dat' with points
\end{verbatim}

\noindent which would plot data from the file {\tt data.dat}.
Section~\ref{sec:plot_datafiles} provides further details of the format that
input \datafile s should take and how PyXPlot may be directed to plot only
certain portions of \datafile s.

Multiple datasets can be plotted on a single graph by specifying them in a
comma-separated list, as in the example:

\begin{verbatim}
plot sin(x) with colour blue, cos(x) with linetype 2
\end{verbatim}

If the {\tt 3d} modifier is supplied to the {\tt plot} command, then a
three-dimensional plot is produced; many plot styles then take additional
columns of data to signify the positions of datapoints along the $z$-axis. This
is described further in Chapter~\ref{ch:plotting}. The angle from which
three-dimensional plots are viewed can be set using the {\tt set view} command.


\subsection{axes}\indcmd{plot axes}

The {\tt axes} modifier may be used to specify which axes data should be
plotted against when plots have multiple parallel axes -- for example, if a
plot has an {\tt x}-axis along its lower edge and an {\tt x2}-axis along its
upper edge. The following example would plot data using the {\tt x2}-axis as
the ordinate axis and the {\tt y}-axis as the abscissa axis:

\begin{verbatim}
plot sin(x) axes x2y1
\end{verbatim}

\noindent It is also possible to use the {\tt axes} modifier to specify that a
vertical ordinate axis and a horizontal abscissa axis should be used; the
following example would plot data using the {\tt y}-axis as the ordinate axis
and the {\tt x}-axis as the abscissa axis:

\begin{verbatim}
plot sin(x) axes yx
\end{verbatim}


\subsection{label}\indcmd{plot label}

The {\tt label} modifier to the {\tt plot} command may be used to render text 
labels next to datapoints, as in the following examples:
\begin{verbatim}
set samples 8
plot [2:5] x**2 label "$x=%.2f$"%($1) with points

plot 'datafile' using 1:2 label "%s"%($3)
\end{verbatim}

\noindent Note that if a particular column of a \datafile\ contains strings
which are to be used as labels, as in the second example above, syntax such as 
{\tt "\%s"\%(\$3)} must be used to explicitly read the data as strings rather
than as numerical quantities.  As PyXPlot treats any whitespace as separating
columns of data, such labels cannot contain spaces, though \LaTeX's {\tt
$\sim$} character can be used to achieve a space.

Datapoints can be labelled when plotted in any of the following plot styles:
{\tt arrows} (and similar styles), {\tt dots}, {\tt errorbars} (and similar
styles), {\tt errorrange} (and similar styles), {\tt impulses}, {\tt
linespoints}, {\tt lowerlimits}, {\tt points}, {\tt stars} and {\tt 
upperlimits}. It is not possible to label datapoints plotted in other styles.
Labels are rendered in the same colour as the datapoints with which they are
associated.


\subsection{title}\indcmd{plot title}

By default, PyXPlot generates its own entry in the legend of a plot for each
dataset plotted.  This default behaviour can be overridden using the {\tt
title} modifier. The following example labels a dataset as `Dataset~1':

\begin{verbatim}
plot sin(x) title 'Dataset 1'
\end{verbatim}

\noindent If a blank string, i.e.\ {\tt ""}, is supplied, then no entry is made
in the plot's legend for that dataset. The same effect can be achieved using
the {\tt notitle} modifier.


\subsection{with}\indcmd{plot with}

The {\tt with} modifier controls the style in which data should be plotted. For
example, the statement
\begin{verbatim}
plot "data.dat" index 1 using 4:5 with lines
\end{verbatim}
specifies that data should be plotted using lines connecting each \datapoint to
its neighbours. More generally, the {\tt with} modifier can be followed by a
range of settings which fine-tune the manner in which the data are displayed;
for example, the statement
\begin{verbatim}
plot "data.dat" with lines linewidth 2.0
\end{verbatim}
would use twice the default width of line.

The following is a complete list of all of PyXPlot's plot styles -- i.e.\ all
of the words which may be used in place of {\tt lines}: {\tt arrows\_\-head},
{\tt arrows\_\-no\-head}, {\tt arrows\_\-two\-head}, {\tt boxes}, {\tt
Colour\-Map}, {\tt Contour\-Map}, {\tt dots}, {\tt Filled\-Region}, {\tt
fsteps}, {\tt histeps}, {\tt impulses}, {\tt lines}, {\tt Lines\-Points}, {\tt
Lower\-Limits}, {\tt points}, {\tt stars}, {\tt steps}, {\tt surface}, {\tt
Upper\-Limits}, {\tt wbox\-es}, {\tt X\-Error\-Bars}, {\tt X\-Error\-Range},
{\tt XY\-Error\-Bars}, {\tt XY\-Error\-Range}, {\tt XYZ\-Error\-Bars}, {\tt
XYZ\-Error\-Range}, {\tt XZ\-Error\-Bars}, {\tt XZ\-Error\-Range}, {\tt
Y\-Error\-Bars}, {\tt Y\-Error\-Range}, {\tt Y\-Error\-Shaded}, {\tt
YZ\-Error\-Bars}, {\tt YZ\-Error\-Range}, {\tt Z\-Error\-Bars}, {\tt
Z\-Error\-Range}. In addition, {\tt lp} and {\tt pl} are recognised as
abbreviations for {\tt lines\-points}; {\tt error\-bars} is recognised as an
abbreviation for {\tt y\-error\-bars}; {\tt error\-range} is recognised as an
abbreviation for {\tt y\-error\-range}; and {\tt arrows\_\-two\-way} is
recognised as an alternative for {\tt arrows\_\-two\-head}.

As well as the names of these plot styles, the {\tt with} modifier can also be
followed by style modifiers such as {\tt line\-width} which alter the exact
appearance of various plot styles. A complete list of these is as follows:
\begin{itemize}
\item \indmodt{colour} -- used to select the colour in which each dataset is to be plotted. It should be followed either by an integer, to set a colour from the present palette (see Section~\ref{sec:palette}), or by a recognised colour name, a complete list of which can be found in Section~\ref{sec:colour_names}. Alternatively, arbitrary colours may be specified by using one of the forms {\tt rgb0.1:\-0.2:\-0.3}, {\tt hsb0.1:\-0.2:\-0.3} or {\tt cmyk0.4:\-0.3:\-0.2:\-0.1}, where the colon-separated values indicate the RGB, HSB or CMYK components of the desired colour in the range~0 to~1. This modifier may also be spelt {\tt color}.\index{colours!setting for datasets}
\item \indmodt{fillcolour} -- used to select the colour in which each dataset is filled. The colour may be specified using any of the styles listed for {\tt colour}. May also be spelt {\tt fillcolour}.
\item \indmodt{linetype} -- used to numerically select the type of line -- for example, solid, dotted, dashed, etc.\ -- which should be used in line-based plot styles. A complete list of PyXPlot's numbered line types can be found in Chapter~\ref{ch:linetypes_table}. May be abbreviated {\tt lt}.
\item \indmodt{linewidth} -- used to select the width of line which should be used in line-based plot styles, where unity represents the default width. May be abbreviated {\tt lw}.
\item \indmodt{pointlinewidth} -- used to select the width of line which should be used to stroke points in point-based plot styles, where unity represents the default width. May be abbreviated {\tt plw}.
\item \indmodt{pointsize} -- used to select the size of drawn points, where unity represents the default size. May be abbreviated {\tt ps}.
\item \indmodt{pointtype} -- used to numerically select the type of point -- for example, crosses, circles, etc.\ -- used by point-based plot styles. A complete list of PyXPlot's numbered point types can be found in Chapter~\ref{ch:linetypes_table}. May be abbreviated {\tt pt}.
\end{itemize}

Any number of these modifiers may be placed sequentially after the keyword {\tt
with}, as in the following examples:

\begin{verbatim}
plot 'datafile' using 1:2 with points pointsize 2
plot 'datafile' using 1:2 with lines colour red linewidth 2
plot 'datafile' using 1:2 with lp col 1 lw 2 ps 3
\end{verbatim}

\noindent Where modifiers take numerical values, expressions of the form {\tt
\$2+1}, similar to those supplied to the {\tt using} modifier, may be used to
indicate that each datapoint should be displayed in a different style or in a
different colour. The following example would plot a \datafile\ with {\tt
points}, drawing the position of each point from the first two columns of the
supplied \datafile\ and the size of each point from the third column:
\begin{verbatim}
plot 'datafile' using 1:2 with points pointsize $3
\end{verbatim}


\section{print}\indcmd{print}

\begin{verbatim}
print <expression> {, <expression>}
\end{verbatim}

The \indcmdt{print} displays a string or the value of a mathematical expression to the
terminal. It is most often used to find the value of a variable, though it can
also be used to produce formatted textual output from a PyXPlot script. For example,

\begin{verbatim}
print a
\end{verbatim}

\noindent would print the value of the variable {\tt a}, and

\begin{verbatim}
print "a = %s"%(a)
\end{verbatim}

\noindent would produce the same result in the midst of formatted text.


\section{pwd}\indcmd{pwd}

\begin{verbatim}
pwd
\end{verbatim}

The \indcmdt{pwd} prints the location of the current working directory.


\section{quit}\indcmd{quit}

\begin{verbatim}
quit
\end{verbatim}

The \indcmdt{quit} can be used to exit PyXPlot. See {\tt exit} for more
details.


\section{rectangle}\indcmd{rectangle}

\begin{verbatim}
rectangle [ item <id> ] at <x>, <y> width <w> height <h>
               [ rotate <r> ] [ with {<option>} ]

rectangle [ item <id> ] from <x1>, <y1> to <x2>, <y2>
               [ rotate <r> ] [ with {<option>} ]
\end{verbatim}

See {\tt box}.


\section{refresh}\indcmd{refresh}

\begin{verbatim}
refresh
\end{verbatim}

The \indcmdt{refresh} produces an exact copy of the latest display. It can be
useful, for example, after changing the terminal type, to produce a second copy
of a plot in a different graphic format. It differs from the {\tt replot}
command in that it doesn't replot anything; use of the {\tt set} command since
the previous {\tt plot} command has no effect on the output.


\section{point}\indcmd{point}

\begin{verbatim}
point [ item <id> ] [at] <x>, <y> [ label <string> ]
                    [ with {<option>} ]
\end{verbatim}

The \indcmdt{point} allows a single point to be plotted on the current
multiplot canvas independently of any graph.  It is equivalent to plotting a
\datafile\ containing a single datum and with invisible axes.  If an optional
{\tt label} is specified then the text string provided is rendered next to the
point.  The {\tt with} modifier allows the style of the point to be specified
using similar options to those accepted by the {\tt plot} command.

All vector graphics objects placed on multiplot canvases receive unique
identification numbers which count sequentially from one, and which may be
listed using the {\tt list} command.  By reference to these numbers, they can
be deleted and subsequently restored with the {\tt delete} and {\tt undelete}
commands respectively.


\section{replot}\indcmd{replot}

\begin{verbatim}
replot [ item <id> ] ...
\end{verbatim}

The \indcmdt{replot} has the same syntax as the {\tt plot} command and is
used to add more datasets to an existing plot, or to change its axis ranges.
For example, having plotted one \datafile\ using the command

\begin{verbatim}
plot 'datafile1.dat'
\end{verbatim}

\noindent another can be plotted on the same axes using the command

\begin{verbatim}
replot 'datafile2.dat' using 1:3
\end{verbatim}

\noindent or the ranges of the axes on the original plot can be changed using
the command

\begin{verbatim}
replot [0:1][0:1]
\end{verbatim}

\noindent The plot is also updated to reflect any changes to settings made
using the {\tt set} command.  In multiplot mode, the \indcmdt{replot} can
likewise be used to modify the last plot added to the page. For example, the
following would change the title of the latest plot to `foo', and add a plot of
the function $g(x)$ to it:

\begin{verbatim}
set title 'foo'
replot cos(x)
\end{verbatim}

Additionally, in multiplot mode it is possible to modify any plot on the
current multiplot canvas by adding an {\tt item} modifier to the {\tt replot}
statement to specify which plot should be replotted.  The following example
would produce two plots, and then add an additional function to the first plot:

\begin{verbatim}
set multiplot
plot f(x)
set origin 10,0
plot g(x)
replot item 1 h(x)
\end{verbatim}

If no {\tt item} number is specified, then the \indcmdt{replot} acts by default
upon the most recent plot to have been added to the multiplot canvas.


\section{reset}\indcmd{reset}

\begin{verbatim}
reset
\end{verbatim}

The \indcmdt{reset} reverts the values of all settings that have been changed
with the {\tt set} command back to their default values. It also clears the
current multiplot canvas.


\section{save}\indcmd{save}

\begin{verbatim}
save '<filename>'
\end{verbatim}

The \indcmdt{save} saves a list of all of the commands which have been executed
in the current interactive PyXPlot session into a file. The filename to be used
for the output should be placed in quotes, as in the example:

\begin{verbatim}
save 'foo'
\end{verbatim}

\noindent would save a command history into the file named {\tt foo}.


\section{set}\indcmd{set}

\begin{verbatim}
set <option> <value>
\end{verbatim}

The \indcmdt{set} is used to configure the values of a wide range of parameters
within PyXPlot which control its behaviour and the style of the output which it
produces.  For example:

\begin{verbatim}
set pointsize 2
\end{verbatim}

\noindent would set the size of points plotted by PyXPlot to be twice the
default. In the majority of cases, the syntax follows that above: the {\tt set}
command should be followed by a keyword which specifies which parameter
should be set, followed by the value to which that parameter should be
set. Those parameters which work in an on/off fashion take a different syntax
along the lines of:

\vspace{3mm}
\begin{tabular}{ll}
{\tt set key} & {\tt\it \# Set option ON} \\
{\tt set nokey} & {\tt\it \# Set option OFF}
\end{tabular}
\vspace{3mm}

\noindent
More details of the effects of each individual parameter can be found in the
subsections below, which forms a complete list of the recognised setting
keywords.

The reader should also see the {\tt show} command, which can be used to display
the current values of settings, and the {\tt unset} command, which returns
settings to their default values. Chapter~\ref{ch:configuration} describes how
commonly used settings can be saved into a configuration file.


\subsection{arrow}\indcmd{set arrow}

\begin{verbatim}
set arrow <arrow number>
      from [<system>] <x>, [<system>] <y>
      to   [<system>] <x>, [<system>] <y>
      [ with {<option>} ]
\end{verbatim}

\noindent where {\tt <system>} may take any of the values
\newline\noindent
{\tt ( first | second | screen | graph | axis<number> )}
\vspace{5mm}

The \indcmdt{set arrow} is used to add arrows to graphs. The example

\begin{verbatim}
set arrow 1 from 0,0 to 1,1
\end{verbatim}

\noindent would draw an arrow between the points $(0,0)$ and $(1,1)$, as
measured along the {\tt x} and {\tt y}-axes.  The tag {\tt 1} immediately
following the keyword {\tt arrow} is an identification number and allows arrows
to be subsequently removed using the {\tt unset arrow} command.  By default,
the coordinates are specified relative to the first horizontal and vertical
axes, but they can alternatively be specified any one of several of coordinate
systems. The coordinate system to be used is specified as in the example:

\begin{verbatim}
set arrow 1 from first 0, second 0 to axis3 1, axis4 1
\end{verbatim}

\noindent The name of the coordinate system to be used precedes the position
value in that system. The coordinate system {\tt first}, the default, measures
the graph using the {\tt x}- and {\tt y}-axes. {\tt second} uses the {\tt x2}-
and {\tt y2}-axes.  {\tt screen} and {\tt graph} both measure in centimetres
from the origin of the graph.  The syntax {\tt axis<n>} may also be used, to
use the $n$th horizontal or vertical axis; for example, {\tt axis3} above.

The {\tt set arrow} command can be followed by the keyword {\tt with} to
specify the style of the arrow. For example, the specifiers {\tt nohead}, {\tt
head} and {\tt twohead}, when placed after the keyword {\tt with}, can be used
to make arrows with no arrow heads, normal arrow heads, or two arrow heads.
{\tt twoway} is an alias for {\tt twohead}. All of the line type modifiers
accepted by the {\tt plot} command can also be used here, as in the example:

\begin{verbatim}
set arrow 2 from first 0, second 2.5 to axis3 0,
             axis4 2.5 with colour blue nohead
\end{verbatim}


\subsection{autoscale}\indcmd{set autoscale}

\begin{verbatim}
set autoscale {<axis>}
\end{verbatim}

The \indcmdt{set autoscale} causes PyXPlot to choose the scaling for an axis
automatically based on the data and/or functions to be plotted against it. The
example

\begin{verbatim}
set autoscale x1
\end{verbatim}

\noindent would cause the range of the first horizontal axis to be scaled to
fit the data.  Multiple axes can be specified, as in the example

\begin{verbatim}
set autoscale x1y3
\end{verbatim}

\noindent Note that ranges explicitly specified in a \indcmdt{plot} will
override the {\tt set autoscale} command.


\subsection{axescolour}\indcmd{set axescolour}

\begin{verbatim}
set axescolour <colour>
\end{verbatim}

The setting {\tt axescolour} changes the colour used to draw graph axes.  The example

\begin{verbatim}
set axescolour blue
\end{verbatim}

\noindent would specify that graph axes should be drawn in blue. Any of the
recognised colour names listed in Section~\ref{sec:colour_names} can be used.


\subsection{axis}\indcmd{set axis}

\begin{verbatim}
set axis <axis> [ ( visible | invisible ) ]
  [ ( top | bottom | left | right | front | back ) ]
  [ ( atzero | notatzero ) ]
  [ ( automirrored | mirrored | fullmirrored ) ]
  [ ( noarrow | arrow | reversearrow | twowayarrow ) ]
  [ linked [item <number>] <axis> [using <expression>] ]
\end{verbatim}

The \indcmdt{set axis} is used to add additional axes to plots and to
configure their appearance. Where an axis is stated on its own, as in the
example
\begin{verbatim}
set axis x2
\end{verbatim}
additional horizontal or vertical axes are added with default
settings. The converse statements
\begin{verbatim}
set noaxis x2
unset axis x2
\end{verbatim}
are used, respectively, to remove axes from plots and to return them
to their default configurations, which often has the same effect of removing
them from the graph, unless they are configured otherwise in a configuration
file.

The position of any axis can be explicitly set using syntax of the form:
\begin{verbatim}
set axis x top
set axis y right
set axis z back
\end{verbatim}
Horizontal axes can be set to appear either at the {\tt top} or {\tt bottom};
vertical axes can be set to appear either at the {\tt left} or {\tt right}; and
$z$-axes can be set to appear either at the {\tt front} or {\tt back}.  By
default, the {\tt x1}-axis is placed along the bottom of graphs and the {\tt
y1}-axis is placed up the left-hand side of graphs. On three-dimensional plots,
the {\tt z1}-axis is placed at the front of the graph. The second set of axes
are placed opposite to the first: the {\tt x2}-, {\tt y2}- and {\tt z2}-axes
are placed respectively along the top, right and back sides of graphs.
Higher-numbered axes are placed alongside the {\tt x1}-, {\tt y1}- and {\tt
z1}-axes.

The following keywords may also be placed alongside the positional keywords
listed above to specify how the axis should appear:
\begin{itemize}
\item {\tt arrow} -- Specifies that an arrowhead should be drawn on the right/top end of the axis. [{\bf Not default}].
\item {\tt atzero} -- Specifies that rather than being placed along an edge of the plot, the axis should mark the lines where the perpendicular axes {\tt x1}, {\tt y1} and/or {\tt z1} are zero. [{\bf Not default}].
\item {\tt automirrored} -- Specifies that an automatic decision should be made between the behaviour of {\tt mirrored} and {\tt nomirrored}. If there are no axes on the opposite side of the graph, a mirror axis is produced. If there are already axes on the opposite side of the graph, no mirror axis is produced. [{\bf Default}].
\item {\tt fullmirrored} -- Similar to {\tt mirrored}. Specifies that this axis should have a corresponding twin placed on the opposite side of the graph with mirroring ticks and labelling. [{\bf Not default}; see {\tt automirrored}].
\item {\tt invisible} -- Specifies that the axis should not be drawn; data can still be plotted against it, but the axis is unseen. See Example~\ref{ex:australia} for a plot where all of the axes are invisible.
\item {\tt linked} -- Specifies that the axis should be linked to another axis; see Section~\ref{sec:linked_axes}.
\item {\tt mirrored} -- Specifies that this axis should have a corresponding twin placed on the opposite side of the graph with mirroring ticks but with no labels on the ticks. [{\bf Not default}; see {\tt automirrored}].
\item {\tt noarrow} -- Specifies that no arrowheads should be drawn on the ends of the axis. [{\bf Default}].
\item {\tt nomirrored} -- Specifies that this axis should not have any corresponding twins. [{\bf Not default}; see {\tt automirrored}].
\item {\tt notatzero} -- Opposite of {\tt atzero}; the axis should be placed along an edge of the plot. [{\bf Default}].
\item {\tt notlinked} -- Specifies that the axis should no longer be linked to any other; see Section~\ref{sec:linked_axes}. [{\bf Default}].
\item {\tt reversearrow} -- Specifies that an arrowhead should be drawn on the left/bottom end of the axis. [{\bf Not default}].
\item {\tt twowayarrow} -- Specifies that arrowheads should be drawn on both ends of the axis. [{\bf Not default}].
\item {\tt visible} -- Specifies that the axis should be displayed; opposite of {\tt invisible}. [{\bf Default}].
\end{itemize}


\subsection{axisunitstyle}\indcmd{set axisunitstyle}

\begin{verbatim}
set axisunitstyle ( bracketed | squarebracketed | ratio )
\end{verbatim}

The setting {\tt axisunitstyle} controls the style with which the units of
plotted quantities are indicated on the axes of plots. The {\tt bracketed}
option causes the units to be placed in parentheses following the axis labels,
whilst the {\tt square\-bracketed} option using square brackets instead.  The
{\tt ratio} option causes the units to follow the label as a divisor so as to
leave the quantity dimensionless.


\subsection{backup}\indcmd{set backup}

\begin{verbatim}
set backup
\end{verbatim}

The setting {\tt backup} changes PyXPlot's behaviour when it detects that a
file which it is about to write is going to overwrite an existing file. Whereas
by default the existing file would be overwritten by the new one, when the
setting {\tt backup} is turned on, it is renamed, placing a tilde at the end of
its filename. For example, suppose that a plot were to be written with filename
{\tt out.ps}, but such a file already existed.  With the backup setting turned
on the existing file would be renamed {\tt out.ps$\sim$} to save it from being
overwritten.

The setting is turned off using the {\tt set nobackup} command.


\subsection{bar}\indcmd{set bar}

\begin{verbatim}
set bar ( large | small | <value> )
\end{verbatim}

The setting {\tt bar} changes the size of the bar drawn on the end of the error
bars, relative to the current point size.  For example:

\begin{verbatim}
set bar 2
\end{verbatim}

\noindent sets the bars to be twice the size of the points.  The options {\tt large} and
{\tt small} are equivalent to~1 (the default) and~0 (no bar) respectively.


\subsection{binorigin}\indcmd{set binorigin}

\begin{verbatim}
set binorigin <value>
\end{verbatim}

The setting {\tt binorigin} affects the behaviour of the \indcmdt{histogram} by
adjusting where it places the boundaries between the bins into which it places
data. The \indcmdt{histogram} selects a system of bins which, if extended to
infinity in both directions, would put a bin boundary at the value specified in
the \indcmdt{set binorigin}. Thus, if a value of $0.1$ were specified to the
\indcmdt{set binorigin}, and a bin width of~20 were chosen by the
\indcmdt{histogram}, bin boundaries might lie at~$20.1$, $40.1$, $60.1$, and so
on. The specified value may have any physical units, but must be real and
finite.


\subsection{binwidth}\indcmd{set binwidth}

\begin{verbatim}
set binwidth <value>
\end{verbatim}

The setting {\tt binwidth} changes the width of the bins used by the {\tt
histogram} command. The specified width may have any physical units, but must
be real and finite.


\subsection{boxfrom}\indcmd{set boxfrom}

\begin{verbatim}
set boxfrom <value>
\end{verbatim}

The setting {\tt boxfrom} alters the vertical line from which bars are drawn
when PyXPlot draws bar charts.  By default, bars all originate from the line
$y=0$, but the example

\begin{verbatim}
set boxfrom 2
\end{verbatim}

\noindent would make the bars originate from the line $y=2$. The specified
vertical abscissa value may have any physical units, but must be real and
finite.



\subsection{boxwidth}\indcmd{set boxwidth}

\begin{verbatim}
set boxwidth <width>
\end{verbatim}

The setting {\tt boxwidth} alters PyXPlot's behaviour when plotting bar charts.
It sets the default width of the boxes used, in ordinate axis units.  If the
specified width is negative then, as happens by default, the boxes have
automatically selected widths, such that the interfaces between them occur at
the horizontal midpoints between their specified positions.  For example:

\begin{verbatim}
set boxwidth 2
\end{verbatim}

\noindent would set all boxes to be two units wide, and

\begin{verbatim}
set boxwidth -2
\end{verbatim}

\noindent would set all of the bars to have differing widths, centred upon
their specified horizontal positions, such that their interfaces occur at the
horizontal midpoints between them. The specified width may have any physical
units, but must be real and finite.


\subsection{c1format}\indcmd{set x1format}

\begin{verbatim}
set c1format ( auto | '<format>' )
      ( horizontal | vertical | rotate <angle> )
\end{verbatim}

The {\tt c1format} setting is used to manually specify an explicit format for
the axis labels to take along the colour scale bars drawn alongside plots which
make use of the \indpst{colourmap} plot style. It has similar syntax to the
{\tt set xformat} command.

\subsection{c1label}\indcmd{set c1label}

\begin{verbatim}
set c1label '<text>' [ rotate <angle> ]
\end{verbatim}

The setting {\tt c1label} sets the label which should be written alongside the
colour scale bars drawn next to plots when the \indpst{colourmap} plot style
is used. An optional rotation angle may be specified to rotate axis labels
clockwise by arbitrary angles. The angle should be specified either as a
dimensionless number of degrees, or as a quantity with physical dimensions of
angle.


\subsection{calendar}\indcmd{set calendar}

\begin{verbatim}
set calendar [ ( input | output ) ] <calendar>
\end{verbatim}

The \indcmdt{set calendar} sets the calendar that PyXPlot uses to convert dates
between calendar dates and Julian Day numbers. PyXPlot uses two separate
calendars which may be different: an input calendar for processing dates that
the user inputs as calendar dates, and an output calendar that controls how
dates are displayed or written on plots.  The available calendars are {\tt
British}, {\tt French}, {\tt Greek}, {\tt Gregorian}, {\tt Hebrew}, {\tt
Islamic}, {\tt Jewish}, {\tt Julian}, {\tt Muslim}, {\tt Papal} and {\tt
Russian}, where {\tt Jewish} is an alias for {\tt Hebrew} and {\tt Muslim} is
an alias for {\tt Islamic}.


\subsection{clip}\indcmd{set clip}

\begin{verbatim}
set clip
\end{verbatim}

The \indcmdt{set clip} causes PyXPlot to clip points which extend over the edge
of plots. The opposite effect is achieved using the {\tt set noclip}
command.


\subsection{colourkey}\indcmd{set colourkey}

\begin{verbatim}
set colourkey [<position>]
\end{verbatim}

The setting {\tt colourkey} determines whether colour scales are drawn along
the edges of plots drawn using the \indpst{colourmap} plot style, indicating
the mapping between represented values and colours. Note that such scales are
only ever drawn when the \indpst{colourmap} plot style is supplied with only
three columns of data, since the colour mappings are themselves
multi-dimensional when more columns are supplied. Issuing the command

\begin{verbatim}
set colourkey
\end{verbatim}

\noindent by itself causes such a scale to be drawn on graphs in the default
position, usually along the right-hand edge of the graphs. The converse action
is achieved by:

\begin{verbatim}
set nocolourkey
\end{verbatim}

\noindent The command

\begin{verbatim}
unset colourkey
\end{verbatim}

\noindent causes PyXPlot to revert to its default behaviour, as specified in a
configuration file, if present. A position for the key may optionally be
specified after the {\tt set colourkey} command, as in the example:

\begin{verbatim}
set colourkey bottom
\end{verbatim}

Recognised positions are {\tt top}, {\tt bottom}, {\tt left} and {\tt right}.
{\tt above} is an alias for {\tt top}; {\tt below} is an alias for {\tt bottom}
and {\tt outside} is an alias for {\tt right}.


\subsection{colourmap}\indcmd{set colourmap}

\begin{verbatim}
set colourmap ( rgb<r>:<g>:<b> |
                hsb<h>:<s>:<b> |
                cmyk<c>:<m>:<y>:<k> )
              [ mask <expr> ]
\end{verbatim}

The setting {\tt colourmap} is used to specify the mapping between ordinate
values and colours used by the \indpst{colourmap} plot style. As elsewhere in
PyXPlot, the colour components should be numerical expressions which evaluate
to a value in the range zero to one. Within these expressions, the variables
{\tt c1}, {\tt c2}, {\tt c3} and {\tt c4} refer quantities calculated from the
third through sixth columns of data supplied to the \indpst{colourmap} plot
style in a way determined by the {\tt c<n>range} setting.  Thus, the following
colour mapping, which is the default, produces a greyscale colour mapping of
the third column of data supplied to the \indpst{colourmap} plot style; further
columns of data, if supplied, are not used:

\begin{verbatim}
set c1range [*:*] renormalise
set colourmap rgb(c1):(c1):(c1)
\end{verbatim}

If a mask expression is supplied, then any areas in a colour map where this
expression evaluates to zero (i.e.\ false) are made transparent.


\subsection{contours}\indcmd{set contours}

\begin{verbatim}
set contours [ ( <number> |
               \( <value> {, <value>} \) ) ]
             [ (label | nolabel) ]
\end{verbatim}

The setting {\tt contours} is used to define the set of ordinate values for
which contours are drawn when using the \indpst{contourmap} plot style. If {\tt
<number>} is specified, the contours are evenly spaced -- either linearly or
logarithmically, depending upon the state of the {\tt logscale c1} setting --
between the values specified in the {\tt c1range} setting. Otherwise, the list
of ordinate values may be specified as a ()-bracketed comma-separated list.

If the option {\tt label} is specified, then each contour is labelled with the
ordinate value that it follows. If the option {\tt nolabel} is specified, then
the contours are not labelled.


\subsection{c$<$n$>$range}\indcmd{set crange}

\begin{verbatim}
set c<n>range [ <range> ]
              [ reversed | noreversed ]
              [ renormalise | norenormalise ]
\end{verbatim}

The {\tt set c<n>range} command changes the range of ordinate values
represented by different colours in the \indpst{colourmap} plot style, and in
the case of the {\tt set c1range} command, also by contours in the
\indpst{contourmap} plot style. The value {\tt <n>} should be an integer in the
range 1--4.

\subsubsection{Contour Maps}

The effect of the {\tt set c1range} command upon the set of ordinate values for
which contours are drawn in the {\tt contourmap} plot style is dependent upon
whether the {\tt set contours} command has been supplied with a number of
contours to draw, or a list of explicit ordinate values for which they should
be drawn. In the latter case, the {\tt set c1range} command has no effect. In
the former case, the contours are evenly spaced, either linearly or
logarithmically depending upon the state of the {\tt logscale c1} setting,
between the minimum and maximum ordinate values supplied to the {\tt set
c1range} command.  If an asterisk ({\tt *}) is supplied in place of either the
minimum and/or the maximum, then the range of values used is automatically
scaled to fit the range of the data supplied.

\subsubsection{Colour Maps}

The colour of each pixel in a colour map is determined by the {\tt colourmap}
setting, which should contain an expression of the form {\tt
rgb(c1):(c2):(c3)}, specifying the components of a colour in either RGB, HSB or
CMYK space, as a function of the variables {\tt c1}, {\tt c2}, {\tt c3} and
{\tt c4}. The {\tt colourmap} plot style should be supplied with between three
and six columns of data, the first two of which contain the $x$- and
$y$-positions of points, and the remainder of which are used to set the values
of the variables {\tt c1}, {\tt c2}, {\tt c3} and {\tt c4} when calculating the
colour with which that point should be represented. If fewer than six columns
of data are supplied, then not all of these variables will be set.

The {\tt set c<n>range} command is used to determine how the raw data values
are mapped to the values of the variables {\tt c1}--{\tt c4}. If the {\tt
no\-renor\-malise} option is specified, then the raw values are passed directly
to the expression. Otherwise, they are first scaled into the range zero to one.
If an explicit range is specified to the {\tt set c<n>range} command, then the
upper limit of this range maps to the value one, and the lower limit maps to
the value zero. This mapping is inverted if the {\tt reverse} option is
specified, such that the upper limit maps to zero, and the lower limit maps to
one. If an asterisk ({\tt *}) is supplied in place of either the upper and/or
lower limit, then the range automatically scales to fit the data supplied.
Intermediate values are scaled, either linearly or logarithmically, depending
upon the state of the {\tt logscale c<n>} setting. For more details of the
syntax of the range specifier, see the {\tt set xrange} command.


\subsection{data style}\indcmd{set data style}

See {\tt set style data}.


\subsection{display}\indcmd{set display}

\begin{verbatim}
set [no]display
\end{verbatim}

By default, whenever an item is added to a multiplot canvas, or an existing
item is moved or replotted, the whole multiplot is redrawn to reflect the
change.  This can be a time-consuming process when constructing large and
complex multiplot canvases, as fresh output is produced at each step. For this
reason, the {\tt set nodisplay} command is provided, which stops PyXPlot from
producing any graphical output. The {\tt set display} command can subsequently
be issued to return to normal behaviour. Scripts which produces large and
complex multiplot canvases are typically wrapped as follows:

\begin{verbatim}
set nodisplay
...
set display
refresh
\end{verbatim}


\subsection{filter}\indcmd{set filter}

\begin{verbatim}
set filter '<filename wildcard>' '<filter command>'
\end{verbatim}

The \indcmdt{set filter} allows input filter programs to be specified to allow
PyXPlot to deal with file types that are not in the plaintext format which it
ordinarily expects.  Firstly the pattern used to recognise the filenames of the
\datafile s to which the filter should apply to must be specified; the standard
wildcard characters {\tt *} and {\tt ?} may be used.  Then a filter program
should be specified, along with any necessary commandline options which should
be passed to it.  This program should take the name of the file to be filtered
as the final option on its command line, immediately following any commandline
options specified above.  It should output a suitable PyXPlot \datafile on its
standard output stream for PyXPlot to read.  For example, to filter all files
that end in {\tt .foo} through the a program called {\tt defoo} using the {\tt
--text} option one would type:

\begin{verbatim}
set filter "*.foo" "/usr/local/bin/defoo --text"
\end{verbatim}

\subsection{fontsize}\indcmd{set fontsize}

\begin{verbatim}
set fontsize <value>
\end{verbatim}

The setting {\tt fontsize} changes the size of the font used to render all text
labels which appear on graphs and multiplot canvases, including keys, axis
labels, text labels produced using the {\tt text} command, and so forth. The
value specified should be a multiplicative factor greater than zero; a value
of~{\tt 2} would cause text to be rendered at twice its normal size, and a
value of~{\tt 0.5} would cause text to be rendered at half its normal size.
The default value is one.

As an alternative, font sizes can be specified with coarser granulation
directly in the \LaTeX\ text of labels, as in the example:

\begin{verbatim}
set xlabel '\Large This is a BIG label'
\end{verbatim}


\subsection{function style}\indcmd{set function style}

See {\tt set style function}.


\subsection{grid}\indcmd{set grid}

\begin{verbatim}
set [no]grid {<axis>}
\end{verbatim}

The setting {\tt grid} controls whether a grid is placed behind graphs or not.
Issuing the command

\begin{verbatim}
set grid
\end{verbatim}

\noindent would cause a grid to be drawn with its lines connecting to the ticks
of the default axes (usually the first horizontal and vertical axes).
Conversely, issuing the command

\begin{verbatim}
set nogrid
\end{verbatim}

\noindent would remove from the plot all gridlines associated with the ticks of
any axes.  One or more axes can be specified for the {\tt set grid} command to
draw gridlines from; in such cases, gridlines are then drawn only to connect
with the ticks of the specified axes, as in the example

\begin{verbatim}
set grid x1 y3
\end{verbatim}

It is possible, though not always aesthetically pleasing, to draw gridlines
from multiple parallel axes, as in example:

\begin{verbatim}
set grid x1x2x3
\end{verbatim}


\subsection{gridmajcolour}\indcmd{set gridmajcolour}

\begin{verbatim}
set gridmajcolour <colour>
\end{verbatim}

The setting {\tt gridmajcolour} changes the colour that is used to draw the
gridlines (see the {\tt set grid} command) which are associated with the major
ticks of axes (i.e.\ major gridlines). For example:

\begin{verbatim}
set gridmajcolour purple
\end{verbatim}

\noindent would cause the major gridlines to be drawn in purple. Any of the
recognised colour names listed in Section~\ref{sec:colour_names} can be used.

See also the {\tt set gridmincolour} command.


\subsection{gridmincolour}\indcmd{set gridmincolour}

\begin{verbatim}
set gridmincolour <colour>
\end{verbatim}

The setting {\tt gridmincolour} changes the colour that is used to draw the
gridlines (see the {\tt set grid} command) which are associated with the minor
ticks of axes (i.e.\ minor gridlines). For example:

\begin{verbatim}
set gridmincolour purple
\end{verbatim}

\noindent would cause the minor gridlines to be drawn in purple. Any of the
recognised colour names listed in Section~\ref{sec:colour_names} can be used.

See also the {\tt set gridmajcolour} command.


\subsection{key}\indcmd{set key}

\begin{verbatim}
set key <position> [<xoffset>, <yoffset>]
\end{verbatim}

The setting {\tt key} determines whether legends are drawn on graphs, and if
so, where they should be located on the plots. Issuing the command

\begin{verbatim}
set key
\end{verbatim}

\noindent by itself causes legends to be drawn on graphs in the default
position, usually in the upper-right corner of the graphs. The converse action
is achieved by:

\begin{verbatim}
set nokey
\end{verbatim}

\noindent The command

\begin{verbatim}
unset key
\end{verbatim}

\noindent causes PyXPlot to revert to its default behaviour, as specified in a
configuration file, if present. A position for the key may optionally be
specified after the {\tt set key} command, as in the example:

\begin{verbatim}
set key bottom left
\end{verbatim}

Recognised positions are {\tt top}, {\tt bottom}, {\tt left}, {\tt right}, {\tt
below}, {\tt above}, {\tt outside}, {\tt xcentre} and {\tt ycentre}. In
addition, if none of these options quite achieve the desired position, a
horizontal and vertical offset may be specified as a comma-separated pair after
any of the positional keywords above.  The first value is assumed to be the
horizontal offset, and the second the vertical offset, both measured in
centimetres.  The example

\begin{verbatim}
set key bottom left 0.0, -0.5
\end{verbatim}

\noindent would display a key below the bottom left corner of the graph.


\subsection{keycolumns}\indcmd{set keycolumns}

\begin{verbatim}
set keycolumns ( <value> | auto )
\end{verbatim}

The setting {\tt keycolumns} sets how many columns the legend of a plot should
be arranged into. By default, the legends of plots are arranged into an
automatically-selected number of columns, equivalent to the behaviour achieved
by issuing the command {\tt set key\-columns auto}. However, if a different
arrangement is desired, the {\tt set keycolumns} command can be followed by any
positive integer to specify that the legend should be split into that number of
columns, as in the example:

\begin{verbatim}
set keycolumns 3
\end{verbatim}


\subsection{label}\indcmd{set label}

\begin{verbatim}
set label <label number> '<text>'
      [<system>] <x>, [<system>] <y>
      [ rotate <angle> ]
      [ with colour <colour> ]
\end{verbatim}

\noindent where {\tt <system>} may take any of the values
\newline\noindent
{\tt ( first | second | screen | graph | axis<number> )}
\vspace{5mm}

The \indcmdt{set label} is used to place text labels on graphs. The example

\begin{verbatim}
set label 1 'Hello' 0, 0
\end{verbatim}

\noindent would place a label reading `Hello' at the point $(0,0)$ on a graph,
as measured along the {\tt x}- and {\tt y}-axes.  The tag {\tt 1} immediately
following the keyword {\tt label} is an identification number and allows the
label to be subsequently removed using the {\tt unset label} command. By
default, the positional coordinates of the label are specified relative to the
first horizontal and vertical axes, but they can alternatively be specified in
any one of several coordinate systems. The coordinate system to be used is
specified as in the example:

\begin{verbatim}
set label 1 'Hello' first 0, second 0
\end{verbatim}

\noindent The name of the coordinate system to be used precedes the position
value in that system. The coordinate system {\tt first}, the default, measures
the graph using the {\tt x}- and {\tt y}-axes. {\tt second} uses the {\tt x2}-
and {\tt y2}-axes.  {\tt screen} and {\tt graph} both measure in centimetres
from the origin of the graph.  The syntax {\tt axis<n>} may also be used, to
use the $n\,$th horizontal or vertical axis; for example, {\tt axis3}:

\begin{verbatim}
set label 1 'Hello' axis3 1, axis4 1
\end{verbatim}

A rotation angle may optionally be specified after the keyword {\tt rotate}
to produce text rotated to any arbitrary angle, measured in degrees
counter-clockwise. The following example would produce upward-running text:

\begin{verbatim}
set label 1 'Hello' 1.2, 2.5 rotate 90
\end{verbatim}

By default the labels are black; however, an arbitrary colour may be specified
using the {\tt with colour} modifier.  For example,

\begin{verbatim}
set label 3 'A purple label' 0, 0 with colour purple
\end{verbatim}

\noindent would place a purple label at the origin.


\subsection{linewidth}\indcmd{set linewidth}

\begin{verbatim}
set linewidth <value>
\end{verbatim}

The \indcmdt{set linewidth} sets the default line width of the lines used to
plot datasets onto graphs using plot styles such as {\tt lines}, {\tt
errorbars}, etc. The value supplied should be a multiplicative factor relative
to the default line width; a value of~1.0 would result in lines being drawn
with their default thickness. For example, in the following statement, lines of
three times the default thickness are drawn:

\begin{verbatim}
set linewidth 3
plot sin(x) with lines
\end{verbatim}

\noindent The {\tt set linewidth} command only affects plot statements where no
line width is manually specified.


\subsection{logscale}\indcmd{set logscale}

\begin{verbatim}
set logscale {<axis>} [<base>]
\end{verbatim}

The setting {\tt logscale} causes an axis to be laid out with logarithmically,
rather than linearly, spaced intervals.  For example, issuing the command:

\begin{verbatim}
set log
\end{verbatim}

\noindent would cause all of the axes of a plot to be scaled logarithmically.
Alternatively, only one, or a selection of axes, can be set to scale
logarithmically as follows:

\begin{verbatim}
set log x1 y2
\end{verbatim}

\noindent This would cause the first horizontal and second vertical axes to be
scaled logarithmically.  Linear scaling can be restored to all axes using the
command

\begin{verbatim}
set nolog
\end{verbatim}

\noindent meanwhile the command

\begin{verbatim}
unset log
\end{verbatim}

\noindent restores axes to their default scaling, as specified in any
configuration file which may be present. Both of these commands can also be
applied to only one or a selection of axes, as in the examples

\begin{verbatim}
set nolog x1 y2
\end{verbatim}

\noindent and

\begin{verbatim}
unset log x1y2
\end{verbatim}

Optionally, a base may be specified at the end of the {\tt set logscale}
command, to produce axes labelled in logarithms of arbitrary bases.  The
default base is~10.

In addition to acting upon any combination of $x$-, $y$- and $z$-axes, the {\tt
set logscale} command may also be requested to set the {\tt c1}, {\tt c2}, {\tt
c3}, {\tt c4}, {\tt t}, {\tt u} and/or {\tt v} axes to scale logarithmically.
The first four of these options affect whether the colours on colour maps scale
linearly or logarithmically with input ordinate values; see the {\tt set
c<n>range} command for more details. The final three of these options specifies
whether parametric functions are sampled linearly or logarithmically in the
variables {\tt t} (one-dimensional), or {\tt u} and {\tt v} (two-dimensional);
see the {\tt set trange}, {\tt set urange} and {\tt set vrange} commands for
more details.


\subsection{multiplot}\indcmd{set multiplot}

\begin{verbatim}
set multiplot
\end{verbatim}

Issuing the command

\begin{verbatim}
set multiplot
\end{verbatim}

\noindent causes PyXPlot to enter multiplot mode, which allows many graphs to
be plotted together and displayed side-by-side. See Section~\ref{sec:multiplot}
for a full discussion of multiplot mode.


\subsection{mxtics}\indcmd{set mxtics}

See {\tt set xtics}.


\subsection{mytics}\indcmd{set mytics}

See {\tt set xtics}.


\subsection{mztics}\indcmd{set mztics}

See {\tt set ztics}.


\subsection{noarrow}\indcmd{set noarrow}

\begin{verbatim}
set noarrow [<arrow number>]
\end{verbatim}

Issuing the command

\begin{verbatim}
set noarrow
\end{verbatim}

\noindent removes all arrows configured with the {\tt set arrow} command.
Alternatively, individual arrows can be removed using commands of the form

\begin{verbatim}
set noarrow 2
\end{verbatim}

\noindent where the tag {\tt 2} is the identification number given to the arrow
to be removed when it was initially specified with the {\tt set arrow} command.


\subsection{noaxis}\indcmd{set noaxis}

\begin{verbatim}
set noaxis [ <axis> {, <axis> } ]
\end{verbatim}

The {\tt set noaxis} command is used to remove axes from graphs; it achieves
the opposite effect from the {\tt set axis} command. It should be followed by a
comma-separated lists of the axes which are to be removed from the current axis
configuration.


\subsection{nobackup}\indcmd{set nobackup}

See {\tt backup}.


\subsection{noclip}\indcmd{set noclip}

See {\tt clip}.


\subsection{nocolourkey}\indcmd{set nocolourkey}

\begin{verbatim}
set nocolourkey
\end{verbatim}

Issuing the command {\tt set nocolourkey} causes plots to be generated with no
colour scale when the \indpst{colourmap} plot style is used. See the {\tt set
colourkey} command for more details.


\subsection{nodisplay}\indcmd{set nodisplay}

See {\tt display}.


\subsection{nogrid}\indcmd{set nogrid}

\begin{verbatim}
set nogrid {<axis>}
\end{verbatim}

Issuing the command {\tt set nogrid} removes gridlines from the current plot. On
its own, the command removes all gridlines from the plot, but alternatively,
those gridlines connected to the ticks of certain axes can be selectively
removed.  The following example would remove gridlines associated with the
first horizontal axis and the second vertical axis:

\begin{verbatim}
set nogrid x1 y2
\end{verbatim}


\subsection{nokey}\indcmd{set nokey}

\begin{verbatim}
set nokey
\end{verbatim}

Issuing the command {\tt set nokey} causes plots to be generated with no legend.
See the {\tt set key} command for more details.


\subsection{nolabel}\indcmd{set nolabel}

\begin{verbatim}
set nolabel {<label number>}
\end{verbatim}

Issuing the command

\begin{verbatim}
set nolabel
\end{verbatim}

\noindent removes all text labels configured using the {\tt set label} command.
Alternatively, individual labels can be removed using the syntax:

\begin{verbatim}
set nolabel 2
\end{verbatim}

\noindent where the tag {\tt 2} is the identification number given to the label
to be removed when it was initially set using the {\tt set label} command.


\subsection{nologscale}\indcmd{set nologscale}

\begin{verbatim}
set nologscale {<axis>}
\end{verbatim}

The setting {\tt nologscale} causes an axis to be laid out with linearly,
rather than logarithmically, spaced intervals; it is equivalent to the setting
{\tt linearscale}. It is the converse of the setting {\tt logscale}.  For
example, issuing the command

\begin{verbatim}
set nolog
\end{verbatim}

\noindent would cause all of the axes of a plot to be scaled linearly.
Alternatively only one, or a selection of axes, can be set to scale linearly as
follows:

\begin{verbatim}
set nologscale x1 y2
\end{verbatim}

\noindent This would cause the first horizontal and second vertical axes to be
scaled linearly.


\subsection{nomultiplot}\indcmd{set nomultiplot}

\begin{verbatim}
set nomultiplot
\end{verbatim}

The \indcmdt{set nomultiplot} causes PyXPlot to leave multiplot mode; outside
of multiplot mode, only single graphs and vector graphics objects are displayed
at any one time, whereas in multiplot mode, galleries of plots and vector
graphics can be placed alongside one another.  See Section~\ref{sec:multiplot}
for a full discussion of multiplot mode.


\subsection{nostyle}\indcmd{set nostyle}

\begin{verbatim}
set nostyle <style number>
\end{verbatim}

The setting {\tt nostyle} deletes a numbered plot style set using the {\tt set
style} command. For example, the command

\begin{verbatim}
set nostyle 3
\end{verbatim}

\noindent would delete the third numbered plot style, if defined. See the
command {\tt set style} for more details.


\subsection{notitle}\indcmd{set notitle}

\begin{verbatim}
set notitle
\end{verbatim}

Issuing the command {\tt set notitle} will cause graphs to be produced with no
title at the top.


\subsection{noxtics}\indcmd{set noxtics}

\begin{verbatim}
set no<axis>tics
\end{verbatim}

This command causes graphs to be produced with no major tick marks along the
specified axis. For example, the {\tt set noxtics} command removes all major
tick marks from the {\tt x}-axis.


\subsection{noytics}\indcmd{set noytics}

Similar to the {\tt set noxtics} command, but acts on vertical axes.


\subsection{noztics}\indcmd{set noztics}

Similar to the {\tt set noxtics} command, but acts on $z$-axes.


\subsection{numerics}\indcmd{set numerics}

\begin{verbatim}
set numerics [ ( complex | real ) ] [ errors ( explicit | quiet) ]
    [ display ( latex | natural | typeable) ]
    [ sigfig <precision> ]
\end{verbatim}

The \indcmdt{set numerics} is used to adjust the way in which calculations are
carried out and numerical quantities are displayed:

\begin{itemize}

\item The option {\tt complex} causes PyXPlot to switch from performing real
arithmetic (default) to performing complex arithmetic. The option {\tt real}
causes any calculations which return results with finite imaginary components
to generate errors.

\item The option {\tt errors} controls how numerical errors such as divisions
by zero, numerical overflows, and the querying functions outside of the domains
in which they are defined, are communicated to the user.  The option {\tt
explicit} (default) causes an error message to be displayed on the terminal
whenever a calculation causes an error.  The option {\tt quiet} causes such
calculations to silently generate a {\tt nan} (not a number) result. The latter
is especially useful when, for example, plotting an expression with the
ordinate axis range set to extend outside the domain in which that expression
returns a well-defined real result; it suppresses the error messages which
might otherwise result from PyXPlot's attempts to evaluate the expression in
those domains where its result is undefined. The option {\tt nan} is a synonym
for {\tt quiet}.

\item The setting {\tt display} changes the format in which numbers are
displayed on the terminal.  Setting the option to {\tt typeable} causes the
numbers to be printed in a form suitable for pasting back into PyXPlot
commands.  The setting {\tt latex} causes \LaTeX-compatible output to be
generated.  The setting {\tt natural} generates concise, human-readable output
which has neither of the above properties.

\item The setting {\tt sigfig} changes the number of significant figures to
which numbers are displayed on the PyXPlot terminal.  Regardless of the value
set, all calculations are internally carried out and stored at double
precision, accurate to around~16 significant figures.

\end{itemize}


\subsection{origin}\indcmd{set origin}

\begin{verbatim}
set origin <x>, <y>
\end{verbatim}

The \indcmdt{set origin} is used to set the location of the bottom-left corner
of the next graph to be placed on a multiplot canvas.  For example, the
command

\begin{verbatim}
set origin 3,5
\end{verbatim}

\noindent would cause the next graph to be plotted with its bottom-left corner
at position $(3,5)$ centimetres on the multiplot canvas. Alternatively, either
of the coordinates may be specified as quantities with physical units of
length, such as {\tt unit(35*mm)}.  The {\tt set origin} command is of little
use outside of multiplot mode.


\subsection{output}\indcmd{set output}

\begin{verbatim}
set output '<filename>'
\end{verbatim}

The setting {\tt output} controls the name of the file that is produced for
non-interactive terminals ({\tt postscript}, {\tt eps}, {\tt jpeg}, {\tt gif}
and {\tt png}).  For example,

\begin{verbatim}
set output 'myplot.eps'
\end{verbatim}

\noindent causes the output to be written to the file {\tt myplot.eps}.


\subsection{palette}\indcmd{set palette}

\begin{verbatim}
set palette <colour> {, <colour>}
\end{verbatim}

PyXPlot has a palette of colours which it assigns sequentially to datasets when
colours are not manually assigned. This is also the palette to which reference
is made if the user issues a command such as

\begin{verbatim}
plot sin(x) with colour 5
\end{verbatim}

\noindent requesting the fifth colour from the palette. By default, this palette
contains a range of distinctive colours. However, the user can choose to
substitute his own list of colours using the {\tt set palette} command. It
should be followed by a comma-separated list of colour names, for example:

\begin{verbatim}
set palette red,green,blue
\end{verbatim}

\noindent If, after issuing this command, the following plot statement were to
be executed:

\begin{verbatim}
plot sin(x), cos(x), tan(x), exp(x)
\end{verbatim}

\noindent the first function would be plotted in red, the second in green, and
the third in blue. Upon reaching the fourth, the palette would cycle back to
red.

Any of the recognised colour names listed in Section~\ref{sec:colour_names} can
be used.


\subsection{papersize}\indcmd{set papersize}

\begin{verbatim}
set papersize ( <named size> | <height>,<width> )
\end{verbatim}

The setting {\tt papersize} changes the size of output produced by the {\tt
postscript} terminal, and whenever the {\tt enlarge} terminal option is set
(see the {\tt set terminal} command). This can take the form of either a
recognised paper size name -- a list of these is given in
Appendix~\ref{ch:paper_sizes} -- or as a (height, width) pair of values, both
measured in millimetres. The following examples demonstrate these
possibilities:

\begin{verbatim}
set papersize a4
set papersize letter
set papersize 200,100
\end{verbatim}


\subsection{pointlinewidth}\indcmd{set pointlinewidth}

\begin{verbatim}
set pointlinewidth <value>
\end{verbatim}

The setting {\tt pointlinewidth} changes the width of the lines that are used
to plot \datapoint s.  For example,

\begin{verbatim}
set pointlinewidth 20
\end{verbatim}

\noindent would cause points to be plotted with lines~20 times the default
thickness.  The setting {\tt pointlinewidth} can be abbreviated as {\tt plw}.


\subsection{pointsize}\indcmd{set pointsize}

\begin{verbatim}
set pointsize <value>
\end{verbatim}

The setting {\tt pointsize} changes the size at which points are drawn,
relative to their default size. It should be followed by a single value which
can be any positive multiplicative factor. For example,

\begin{verbatim}
set pointsize 1.5
\end{verbatim}

\noindent would cause points to be drawn at~1.5 times their default size.


\subsection{preamble}\indcmd{set preamble}

\begin{verbatim}
set preamble <text>
\end{verbatim}

The setting {\tt preamble} changes the text of the preamble that is passed to
\LaTeX\ prior to the rendering of each text item on the current multiplot
canvas.  This allows, for example, different packages to be loaded by default
and user-defined macros to be set up, as in the examples:

\begin{verbatim}
set preamble \usepackage{marvosym}
set preamble \def\degrees{$^\circ$}
\end{verbatim}


%\subsection{projection}\indcmd{set projection}
%
%\begin{verbatim}
%set projection flat
%\end{verbatim}
%
%The {\tt set projection} command will change between different projections of
%plots in a future version of PyXPlot.  Currently only the {\tt flat} option is
%supported.


\subsection{samples}\indcmd{set samples}

\begin{verbatim}
set samples [<value>]
            [grid <x_samples> [x] <y_samples>]
            [interpolate ( InverseSquare |
                           MonaghanLattanzio |
                           NearestNeighbour ) ]
\end{verbatim}

The setting {\tt samples} determines the number of values along the ordinate
axis at which functions are evaluated when they are plotted. For example, the
command

\begin{verbatim}
set samples 100
\end{verbatim}

\noindent would cause functions to be evaluated at 100~points along the
ordinate axis.  Increasing this value will cause functions to be plotted more
smoothly, but also more slowly, and the PostScript files generated will also be
larger. When functions are plotted with the {\tt points} plot style, this
setting controls the number of points plotted.

After the keyword {\tt grid} may be specified the dimensions of the
two-dimensional grid of samples used in the \indpst{colourmap} and
\indpst{surface} plot styles, and internally when calculating the contours to
be plotted in the \indpst{contourmap} plot style. If a {\tt *} is given in
place of either of the dimensions, then the same number of samples as are
specified in {\tt <value>} are taken.

After the keyword {\tt interpolate}, the method used for interpolating
non-gridded two-dimensional data onto the above-mentioned grid may be
specified. The available options are {\tt Inverse\-Square}, {\tt
Monag\-han\-Lat\-tan\-zio} and {\tt Nearest\-Neigh\-bour}.


\subsection{seed}\indcmd{set seed}

\begin{verbatim}
set seed <value>
\end{verbatim}

The \indcmdt{set seed} sets the seed used by all of those mathematical
functions which generate random samples from probability distributions.  This
allows repeatable sequences of pseudo-random numbers to be generated.  Upon
initialisation, PyXPlot returns the sequence of random numbers obtained after
issuing the command {\tt set seed~0}.


\subsection{size}\indcmd{set size}

\begin{verbatim}
set size [<width>]
         [( ratio <ratio> | noratio  | square)]
         [(zratio <ratio> | nozratio )]
\end{verbatim}

The setting {\tt size} is used to set the width or aspect ratio of the next
graph to be generated. If a width is specified, then it may either take the
form of a dimensionless number implicitly measured in centimetres, or a
quantity with physical dimensions of length such as {\tt unit(50*mm)}.

When the keyword {\tt ratio} is specified, it should be followed by the ratio
of the graph's height to its width, i.e.\ of the length of its $y$-axes to that
of its $x$-axes. The keyword {\tt noratio} returns the aspect ratio to its
default value of the golden ratio, and the keyword {\tt square} sets the aspect
ratio to one.

When the keyword {\tt zratio} is specified, it should be followed by the ratio
of the length of three-dimensional graphs' $z$-axes to that of their $x$-axes.
The keyword {\tt nozratio} returns this aspect ratio to its default value of
the golden ratio.


\subsubsection{noratio}\index{set size command!noratio modifier@{\tt noratio} modifier}

\begin{verbatim}
set size noratio
\end{verbatim}

Executing the command

\begin{verbatim}
set size noratio
\end{verbatim}

\noindent resets PyXPlot to produce plots with its default aspect ratio, which
is the golden ratio. Other aspect ratios can be set with the {\tt set size
ratio} command.


\subsubsection{ratio}\index{set size command!ratio modifier@{\tt ratio} modifier}

\begin{verbatim}
set size ratio <ratio>
\end{verbatim}

This command sets the aspect ratio of plots produced by PyXPlot.  The height of
resulting plots will equal the plot width, as set by the {\tt set width}
command, multiplied by this aspect ratio.  For example,

\begin{verbatim}
set size ratio 2.0
\end{verbatim}

\noindent would cause PyXPlot to produce plots that are twice as high as they
are wide.  The default aspect ratio which PyXPlot uses is a golden ratio of
$2/(1+\sqrt{5})$.


\subsubsection{square}\index{set size command!square modifier@{\tt square} modifier}

\begin{verbatim}
set size square
\end{verbatim}

This command sets PyXPlot to produce square plots, i.e.\ with unit aspect
ratio. Other aspect ratios can be set with the {\tt set size ratio} command.


\subsection{style}\indcmd{set style}

\begin{verbatim}
set style <style number> {<style option>}
\end{verbatim}

At times, the string of style keywords following the {\tt with} modifier in
plot commands can grow rather unwieldy in its length. For clarity, frequently
used plot styles can be stored as numbered plot {\tt styles}. The syntax for
setting a numbered plot style is:

\begin{verbatim}
set style 2 points pointtype 3
\end{verbatim}

\noindent where the {\tt 2} is the identification number of the plot style.
In a subsequent plot statement, this line style can be recalled as follows:

\begin{verbatim}
plot sin(x) with style 2
\end{verbatim}


\subsection{style data | style function}\indcmd{set style data}\indcmd{set style function}

\begin{verbatim}
set style { data | function } {<style option>}
\end{verbatim}

The {\tt set style data} command affects the default style with which data from
files is plotted.  Likewise the {\tt set style function} command changes the
default style with which functions are plotted.  Any valid style modifier which
can follow the keyword {\tt with} in the {\tt plot} command can be used.  For
example, the commands

\begin{verbatim}
set style data points
set style function lines linestyle 1
\end{verbatim}

\noindent would cause \datafile s to be plotted, by default, using points and
functions using lines with the first defined line style.


\subsection{terminal}\indcmd{set terminal}

\begin{verbatim}
set terminal ( X11_SingleWindow | X11_MultiWindow | X11_Persist |
               bmp | eps | gif | jpeg | pdf | png | postscript |
               svg | tiff )
             ( colour | color | monochrome )
             ( dpi <value> )
             ( portrait | landscape )
             ( invert | noinvert )
             ( transparent | solid )
             ( antialias | noantialias )
             ( enlarge | noenlarge )
\end{verbatim}

The \indcmdt{set terminal} controls the graphical format in which PyXPlot
renders plots and multiplot canvases, for example configuring whether it should
output plots to files or display them in a window on the screen. Various
options can also be set within many of the graphical formats which PyXPlot
supports using this command.

The following graphical formats are supported:  {\tt X11\_\-Single\-Window},
{\tt X11\_\-Multi\-Window}, {\tt X11\_\-Persist}, {\tt bmp}, {\tt eps}, {\tt
gif}, {\tt jpeg}, {\tt pdf}, {\tt png}, {\tt postscript}, {\tt
svg}\footnote{The {\tt svg} output terminal is experimental and may be
unstable. It relies upon the use of the {\tt svg} output device in Ghostscript,
which may not be present on all systems.}, {\tt tiff}.  To select one of these
formats, simply type the name of the desired format after the {\tt set
terminal} command. To obtain more details on each, see the subtopics below.
The following settings, which can also be typed following the {\tt set
terminal} command, are used to change the options within some of these graphic
formats: {\tt colour}, {\tt monochrome}, {\tt dpi}, {\tt portrait}, {\tt
landscape}, {\tt invert}, {\tt noinvert}, {\tt transparent}, {\tt solid}, {\tt
enlarge}, {\tt noenlarge}. Details of each of these can be found below.


\subsubsection{antialias}\index{set terminal command!antialias modifier@{\tt antialias} modifier}

The {\tt antialias} terminal option causes plots produced with the bitmap
terminals (i.e.\ {\tt bmp}, {\tt gif}, {\tt jpeg}, {\tt png} and {\tt tiff}) to be
antialiased; this is the default behaviour.


\subsubsection{bmp}\index{set terminal command!bmp modifier@{\tt gif} modifier}

The {\tt bmp} terminal renders output as Windows bitmap images. The filename to
which output is to be sent should be set using the {\tt set output} command;
the default is {\tt pyxplot.bmp}. The number of dots per inch used can be
changed using the {\tt dpi} option. The {\tt invert} option may be used to produce an
image with inverted colours.


\subsubsection{colour}\index{set terminal command!colour modifier@{\tt colour} modifier}

The {\tt colour} terminal option causes plots to be produced in colour; this is
the default behaviour.


\subsubsection{color}\index{set terminal command!color modifier@{\tt color} modifier}

The {\tt color} terminal option is the US-English equivalent of {\tt colour}.


\subsubsection{dpi}\index{set terminal command!dpi modifier@{\tt dpi} modifier}

When PyXPlot is set to produce bitmap graphics output, using the {\tt bmp},
{\tt gif}, {\tt jpg} or {\tt png} terminals, the setting {\tt dpi} changes the
number of dots per inch with which these graphical images are produced. That is
to say, it changes the image resolution of the output images. For example,

\begin{verbatim}
set terminal dpi 100
\end{verbatim}

\noindent sets the output to a resolution of~100 dots per inch. Higher DPI
values yield better quality images, but larger file sizes.


\subsubsection{enlarge}\index{set terminal command!enlarge modifier@{\tt enlarge} modifier}

The {\tt enlarge} terminal option causes plots and multiplot canvases to be
enlarged or shrunk to fit within the margins of the currently selected paper
size. It is especially useful when using the {\tt postscript} terminal, as it
allows for the production of immediately-printable output.


\subsubsection{eps}\index{set terminal command!eps modifier@{\tt eps} modifier}

Sends output to Encapsulated PostScript ({\tt eps}) files.  The filename to
which output should be sent can be set using the {\tt set output} command; the
default is {\tt pyxplot.eps}.  This terminal produces images suitable for
including in, for example, \LaTeX\ documents.


\subsubsection{gif}\index{set terminal command!gif modifier@{\tt gif} modifier}

The {\tt gif} terminal renders output as gif images. The filename to which
output should be sent can be set using the {\tt set output} command; the
default is {\tt pyxplot.gif}. The number of dots per inch used can be changed
using the {\tt dpi} option. Transparent gifs can be produced with the {\tt
transparent} option. The {\tt invert} option may be used to produce an image
with inverted colours.


\subsubsection{invert}\index{set terminal command!invert modifier@{\tt invert} modifier}

The {\tt invert} terminal option causes the bitmap terminals (i.e.\ {\tt bmp},
{\tt gif}, {\tt jpeg}, {\tt png} and {\tt tiff}) to produce output with
inverted colours.


\subsubsection{jpeg}\index{set terminal command!jpeg modifier@{\tt jpeg} modifier}

The {\tt jpeg} terminal renders output as jpeg images. The filename to which
output should be sent can be set using the {\tt set output} command; the
default is {\tt pyxplot.jpg}.  The number of dots per inch used can be changed
using the {\tt dpi} option. The {\tt invert} option may be used to produce an
image with inverted colours.


\subsubsection{landscape}\index{set terminal command!landscape modifier@{\tt landscape} modifier}

The {\tt landscape} terminal option causes PyXPlot's output to be displayed in
rotated orientation.  This can be useful for fitting graphs onto sheets of
paper, but is generally less useful for plotting things on screen.


\subsubsection{monochrome}\index{set terminal command!monochrome modifier@{\tt monochrome} modifier}

The {\tt monochrome} terminal option causes plots to be rendered in black and
white. This changes the default behaviour of the {\tt plot} command to be
optimised for monochrome display, and so, for example, different dash styles
are used to differentiate between lines on plots with several datasets.


\subsubsection{noantialias}\index{set terminal command!noantialias modifier@{\tt noantialias} modifier}

The {\tt noantialias} terminal option causes plots produced with the bitmap
terminals (i.e.\ {\tt bmp}, {\tt gif}, {\tt jpeg}, {\tt png} and {\tt tiff})
not to be antialiased. This can be useful when making plots which will
subsequently have regions cut out and made transparent.


\subsubsection{noenlarge}\index{set terminal command!noenlarge modifier@{\tt noenlarge} modifier}

The {\tt noenlarge} terminal option causes the output not to be scaled to fit
within the margins of the currently-selected papersize. This is the opposite of
{\tt enlarge} option.


\subsubsection{noinvert}\index{set terminal command!noinvert modifier@{\tt noinvert} modifier}

The {\tt noinvert} terminal option causes the bitmap terminals (i.e.\ {\tt
gif}, {\tt jpeg}, {\tt png}) to produce normal output without inverted colours.
This is the opposite of the {\tt inverse} option.


\subsubsection{pdf}\index{set terminal command!pdf modifier@{\tt pdf}
modifier}

The {\tt pdf} terminal renders output in Adobe's Portable Document Format
(PDF).


\subsubsection{png}\index{set terminal command!png modifier@{\tt png} modifier}

The {\tt png} terminal renders output as png images. The filename to which
output should be sent can be set using the {\tt set output} command; the
default is {\tt pyxplot.png}. The number of dots per inch used can be changed
using the {\tt dpi} option. Transparent pngs can be produced with the {\tt
transparent} option. The {\tt invert} option may be used to produce an image
with inverted colours.


\subsubsection{portrait}\index{set terminal command!portrait modifier@{\tt portrait} modifier}

The {\tt portrait} terminal option causes PyXPlot's output to be displayed in
upright (normal) orientation; it is the converse of the {\tt landscape} option.


\subsubsection{postscript}\index{set terminal command!postscript modifier@{\tt postscript} modifier}

The {\tt postscript} terminal renders output as PostScript files. The filename
to which output should be sent can be set using the {\tt set output} command;
the default is {\tt pyxplot.ps}.  This terminal produces non-encapsulated
PostScript suitable for sending directly to a printer; it should not be used
for producing images to be embedded in documents, for which the {\tt eps}
terminal should be used.


\subsubsection{solid}\index{set terminal command!solid modifier@{\tt solid} modifier}

The {\tt solid} option causes the {\tt gif} and {\tt png} terminals to produce
output with a non-transparent background, the converse of {\tt transparent}.


\subsubsection{transparent}\index{set terminal command!transparent modifier@{\tt transparent} modifier}

The {\tt transparent} terminal option causes the {\tt gif} and {\tt png}
terminals to produce output with a transparent background.


\subsubsection{X11\_multiwindow}\index{set terminal command!X11\_multiwindow modifier@{\tt X11\_multiwindow} modifier}

The {\tt X11\_multiwindow} terminal displays plots on the screen in X11
windows. Each time a new plot is generated it appears in a new window, and the
old plots remain visible.  As many plots as may be desired can be left on the
desktop simultaneously. When PyXPlot exits, however, all of the windows are
closed.

\subsubsection{X11\_persist}\index{set terminal command!X11\_persist
modifier@{\tt X11\_persist} modifier}

The {\tt X11\_persist} terminal displays plots on the screen in X11 windows.
Each time a new plot is generated it appears in a new window, and the old plots
remain visible.  When PyXPlot is exited the windows remain in place until they
are closed manually.

\subsubsection{X11\_singlewindow}\index{set terminal command!X11\_singlewindow modifier@{\tt X11\_singlewindow} modifier}

The {\tt X11\_singlewindow} terminal displays plots on the screen in X11
windows. Each time a new plot is generated it replaces the old one, preventing
the desktop from becoming flooded with old plots. This terminal is the default
when running interactively.


\subsection{textcolour}\indcmd{set textcolour}

\begin{verbatim}
set textcolour <colour>
\end{verbatim}

The setting {\tt textcolour} changes the default colour of all text displayed
on plots or multiplot canvases.  For example,

\begin{verbatim}
set textcolour red
\end{verbatim}

\noindent causes all text labels, including the labels on graph axes and
legends, etc.\ to be rendered in red. Any of the recognised colour names listed
in Section~\ref{sec:colour_names} can be used; colours can also be referred to
numerically with reference to the current palette.


\subsection{texthalign}\indcmd{set texthalign}

\begin{verbatim}
set texthalign ( left | centre | right )
\end{verbatim}

The setting {\tt texthalign} controls how text labels are justified
horizontally with respect to their specified positions, acting both upon labels
placed on plots using the {\tt set label} command, and upon text items created
using the {\tt text} command. Three options are available:

\begin{verbatim}
set texthalign left
set texthalign centre
set texthalign right
\end{verbatim}


\subsection{textvalign}\indcmd{set textvalign}

\begin{verbatim}
set textvalign ( bottom | centre | top )
\end{verbatim}

The setting {\tt textvalign} controls how text labels are justified vertically
with respect to their specified positions, acting both upon labels placed on
plots using the {\tt set label} command, and upon text items created using the
{\tt text} command. Three options are available:

\begin{verbatim}
set textvalign bottom
set textvalign centre
set textvalign top
\end{verbatim}


\subsection{title}\indcmd{set title}

\begin{verbatim}
set title '<title>'
\end{verbatim}

The setting {\tt title} can be used to set a title for a plot, to be displayed
above it.  For example, the command:

\begin{verbatim}
set title 'foo'
\end{verbatim}

\noindent would cause a title `foo' to be displayed above a graph. The easiest
way to remove a title, having set one, is using the command:

\begin{verbatim}
set notitle
\end{verbatim}


\subsection{trange}\indcmd{set trange}

\begin{verbatim}
set trange [<range>] [reverse]
\end{verbatim}

The {\tt set trange} command changes the range of the free parameter {\tt t}
used when generating parametric plots.  For more details of the syntax of the
range specifier, see the {\tt set xrange} command. Note that {\tt t} is not
allowed to autoscale, and so the {\tt *} character is not permitted in the
specified range.

\subsection{unit}\indcmd{set unit}

\begin{verbatim}
set unit [ angle ( dimensionless | nodimensionless ) ]
         [ of <dimension> <unit> ]
         [ scheme <unit scheme> ]
         [ preferred <unit> ]
         [ nopreferred <unit> ]
         [ display ( full | abbreviated | prefix | noprefix ) ]
\end{verbatim}

The \indcmdt{set unit} controls how quantities with physical units are
displayed by PyXPlot. The \indcmdt{set unit scheme} provides the most general
configuration option, allowing one of several {\it units
schemes}\index{units!unit schemes} to be selected, each of which comprises a
list of units which are deemed to be members of that particular scheme. For
example, in the CGS unit scheme\index{CGS units}\index{units!CGS}, all lengths
are displayed in centimetres, all masses are displayed in grammes, all energies
are displayed in ergs, and so forth.  In the imperial unit
scheme\index{imperial units}\index{units!imperial}, quantities are displayed in
British imperial units -- inches, pounds, pints, and so forth -- and in the US
unit scheme, US customary units are used. The available schemes are: {\tt
ancient}, {\tt cgs}, {\tt imperial}, {\tt planck}, {\tt si}, and {\tt us}.

To fine-tune the unit used to display quantities with a particular set of
physical dimensions, the {\tt set unit of} form of the command should be used.
For example, the following command would cause all lengths to be displayed in
inches:

\begin{verbatim}
set unit of length inch
\end{verbatim}

The \indcmdt{set unit preferred} offers a slightly more flexible way of
achieving the same result. Whereas the \indcmdt{set unit of} can only operate
on named quantities such as lengths and powers, and cannot act upon compound
units such as {\tt W/Hz}, the \indcmdt{set unit preferred} can act upon any
unit or combination of units, as in the examples:
\begin{verbatim}
set unit preferred parsec
set unit preferred W/Hz
set unit preferred N*m
\end{verbatim}
The latter two examples are particularly useful when working with spectral
densities (powers per unit frequency) or torques (forces multiplied by
distances). Unfortunately, both of these units are dimensionally equal to
energies, and so are displayed by PyXPlot in Joules by default. The above
statement overrides such behaviour. Having set a particular unit to be
preferred, this can be unset as in the following example:
\begin{verbatim}
set unit nopreferred parsec
\end{verbatim}

By default, units are displayed in their abbreviated forms, for example {\tt A}
instead of {\tt amperes} and {\tt W} instead of {\tt watts}. Furthermore, SI
prefixes such as milli- and kilo- are applied to SI units where they are
appropriate. Both of these behaviours can be turned on or off, in the former
case with the commands

\begin{verbatim}
set unit display abbreviated
set unit display full
\end{verbatim}

\noindent and in the latter case using the following pair of commands:

\begin{verbatim}
set unit display prefix
set unit display noprefix
\end{verbatim}


\subsection{urange}\indcmd{set urange}

\begin{verbatim}
set urange [<range>] [reverse]
\end{verbatim}

The {\tt set urange} command changes the range of the free parameter {\tt u}
used when generating parametric plots sampled over grids of ({\tt u},{\tt v})
values.  For more details of the syntax of the range specifier, see the {\tt
set xrange} command. Note that {\tt u} is not allowed to autoscale, and so the
{\tt *} character is not permitted in the specified range.

Specifying the {\tt set urange} command by itself specified that parametric
plots should be sampled over two-dimensional grids of ({\tt u},{\tt v}) values,
rather than one-dimensional ranges of {\tt t} values.


\subsection{view}\indcmd{set view}

\begin{verbatim}
set view <theta>, <phi>
\end{verbatim}

The \indcmdt{set view} is used to specify the angle from which
three-dimensional plots are viewed. It should be followed by two angles, which
can either be expressed in degrees, as dimensionless numbers, or as quantities
with physical units of angle:
\begin{verbatim}
set view 60,30

set unit angle nodimensionless
set view unit(0.1*rev),unit(2*rad)
\end{verbatim}
The orientation $(0,0)$ corresponds to having the $x$-axis horizontal, the
$z$-axis vertical, and the $y$-axis directed into the page. The first angle
supplied to the {\tt set view} command rotates the plot in the $(x,y)$ plane,
and the second angle tips the plot up in the plane containing the $z$-axis and
the normal to the user's two-dimensional display.


\subsection{viewer}\indcmd{set viewer}

\begin{verbatim}
set viewer ( auto | <command> )
\end{verbatim}

The \indcmdt{set viewer} is used to select which external PostScript viewing
application is used to display PyXPlot output on screen in the {\tt X11}
terminals. If the option {\tt auto} is selected, then either \ghostview\ or
{\tt ggv} is used, if installed. Alternatively, any other application such as
{\tt evince} or {\tt okular} can be selected by name, providing it is installed
in within your shell's search path or an absolute path is provided, as in the
examples:

\begin{verbatim}
set viewer evince
set viewer /usr/bin/okular
\end{verbatim}

\noindent Additional commandline switches may also be provided after the name
of the application to be used, as in the example

\begin{verbatim}
set viewer gv --grayscale
\end{verbatim}


\subsection{vrange}\indcmd{set vrange}

\begin{verbatim}
set vrange [<range>] [reverse]
\end{verbatim}

See the {\tt set urange} command.


\subsection{width}\indcmd{set width}

\begin{verbatim}
set width <value>
\end{verbatim}

The setting {\tt width} is used to set the width of the next graph to be
generated. The width is specified either as a dimensionless number
implicitly measured in centimetres, or as a quantity with physical dimensions
of length such as {\tt unit(50*mm)}.


\subsection{xformat}\indcmd{set xformat}

\begin{verbatim}
set <axis>format ( auto | '<format>' )
      ( horizontal | vertical | rotate <angle> )
\end{verbatim}

By default, the major tick marks along axes are labelled with representations
of the ordinate values at each point, each accurate to the number of
significant figures specified using the \indcmdt{set numerics sigfig}. These
labels may appear as decimals, such as $3.142$, in scientific notion, as in
$3\times10^8$, or, on logarithmic axes where a base has been specified for the
logarithms, using syntax such as\footnote{Note that the {\tt x} axis must be
referred to as {\tt x1} here to distinguish this statement from {\tt set log
x2}.}
\begin{verbatim}
set log x1 2
\end{verbatim}
in a format such as $1.5\times2^8$.

The \indcmdt{set xformat} -- together with its companions such as {\tt set
yformat} -- is used to manually specify an explicit format for the axis labels
to take, as demonstrated by the following pair of examples:
\begin{verbatim}
set xformat "%.2f"%(x)
set yformat "%s$^\prime$"%(y/unit(feet))
\end{verbatim}
The first example specifies that values should be displayed to two
decimal places along the {\tt x}-axis; the second specifies that distances should
be displayed in feet along the {\tt y}-axis. Note that the dummy variable used to
represent the represented value is {\tt x}, {\tt y} or {\tt z} depending upon the
direction of the axis, but that the dummy variable used in the {\tt set
x2format} command is still {\tt x}. The following pair of examples both have
the equivalent effect of returning the {\tt x2}-axis to its default system of
tick labels:
\begin{verbatim}
set x2format auto
set x2format "%s"%(x)
\end{verbatim}

The following example specifies that ordinate values should be displayed as
multiples of $\pi$:
\begin{verbatim}
set xformat "%s$\pi$"%(x/pi)
plot [-pi:2*pi] sin(x)
\end{verbatim}

Note that where possible, PyXPlot intelligently changes the positions along
axes where it places the ticks to reflect significant points in the chosen
labelling system.  The extent to which this is possible depends upon the format
string supplied. It is generally easier when continuous-varying numerical
values are substituted into strings, rather than discretely-varying values or
strings.

\subsection{xlabel}\indcmd{set xlabel}

\begin{verbatim}
set <axis>label '<text>' [ rotate <angle> ]
\end{verbatim}

The setting {\tt xlabel} sets the label which should be written along the {\tt
x}-axis.  For example,

\begin{verbatim}
set xlabel '$x$'
\end{verbatim}

\noindent sets the label on the {\tt x}-axis to read `$x$'.  Labels can be
placed on higher numbered axes by inserting their number after the `{\tt x}';
for example,

\begin{verbatim}
set x10label 'foo'
\end{verbatim}

\noindent would label the tenth horizontal axis. Similarly, labels can be
placed on vertical axes as follows:

\begin{verbatim}
set ylabel '$y$'
set y2label 'foo'
\end{verbatim}

An optional rotation angle may be specified to rotate axis labels clockwise by
arbitrary angles. The angle should be specified either as a dimensionless
number of degrees, or as a quantity with physical dimensions of angle.

\subsection{xrange}\indcmd{set xrange}

\begin{verbatim}
set <axis>range <range> [reverse]
\end{verbatim}

The setting {\tt xrange} controls the range of values spanned by the {\tt
x}-axes of plots.  For function plots, this is also the domain across which the
function will be evaluated.  For example,

\begin{verbatim}
set xrange [0:10]
\end{verbatim}

\noindent sets the first horizontal axis to run from~0 to~10.  Higher numbered
axes may be referred to be inserting their number after the {\tt x}; the ranges
of vertical axes may similarly be set by replacing the {\tt x} with a {\tt y}.
Hence,

\begin{verbatim}
set y23range [-5:5]
\end{verbatim}

\noindent sets the range of the 23rd vertical axis to run from~$-5$ to~5.  To
request a range to be automatically scaled an asterisk can be used.  The
following command would set the {\tt x}-axis to have an upper limit of 10, but
does not affect the lower limit; its range remains at its previous setting:

\begin{verbatim}
set xrange [:10][*:*]
\end{verbatim}

The keyword {\tt reverse} is used to indicate that the two limits of an axis
should be swapped. This is useful for setting auto-scaling axes to be displayed
running from right to left, or from top to bottom.


\subsection{xtics}\indcmd{set xtics}

\begin{verbatim}
set [m]<axis>tics
    [ ( axis | border | inward | outward | both ) ]
    [ ( autofreq
          | [<minimum>,] <increment> [, <maximum>]
          | \( { '<label>' <position> } \)
         ] )
\end{verbatim}

By default, PyXPlot places a series of tick marks at significant points along
each axis, with the most significant points being labelled.  Labelled tick
marks are termed {\it major} ticks, and unlabelled tick marks are termed {\it
minor} ticks.  The position and appearance of the major ticks along the {\tt
x}-axis can be configured using the \indcmdt{set xtics}; the corresponding
{\tt set mxtics} command configures the appearance of the minor ticks along the
{\tt x}-axis. Analogous commands such as {\tt set ytics} and {\tt set mx2tics}
configure the major and minor ticks along other axes.

The keywords \indkeyt{inward}, \indkeyt{outward} and \indkeyt{both} are used to
configure how the ticks appear -- whether they point inward, towards the plot,
as is default, or outwards towards the axis labels, or in both directions.  The
keyword \indkeyt{axis} is an alias for \indkeyt{inward}, and \indkeyt{border}
is an alias for \indkeyt{outward}.

The remaining options are used to configure where along the axis ticks are
placed. If a series of comma-separated values {\tt <minimum>, <increment>,
<maximum>} are specified, then ticks are placed at evenly spaced intervals
between the specified limits. The {\tt <minimum>} and {\tt <maximum>} values
are optional; if only one value is specified then it is taken to be the step
size between ticks. If two values are specified, then the first is taken to be
{\tt <minimum>}. In the case of logarithmic axes, {\tt <increment>} is applied
as a multiplicative step size.

Alternatively, if a bracketed list of quoted tick labels and tick positions are
provided, then ticks can be placed on an axis manually and each given its own
textual label. The quoted tick labels may be omitted, in which case they are
automatically generated:
\begin{verbatim}
set xtics ("a" 1, "b" 2, "c" 3)
set xtics (1,2,3)
\end{verbatim}
The keyword \indkeyt{autofreq} overrides any manual selection of ticks which
may have been placed on an axis and resumes the automatic placement of ticks
along it. The \indcmdt{show xtics}, together with its companions such as {\tt
show x2tics} and {\tt show ytics}, is used to query the current ticking
options. The \indcmdt{set noxtics} is used to stipulate that no ticks should
appear along a particular axis:

\begin{verbatim}
set noxtics
show xtics
\end{verbatim}


\subsection{yformat}\indcmd{set yformat}

See {\tt xformat}.


\subsection{ylabel}\indcmd{set ylabel}

See {\tt xlabel}.


\subsection{yrange}\indcmd{set yrange}

See {\tt xrange}.


\subsection{ytics}\indcmd{set ytics}

See {\tt xtics}.


\subsection{zformat}\indcmd{set zformat}

See {\tt xformat}.


\subsection{zlabel}\indcmd{set zlabel}

See {\tt xlabel}.


\subsection{zrange}\indcmd{set zrange}

See {\tt xrange}.


\subsection{ztics}\indcmd{set ztics}

See {\tt xtics}.


\section{show}\indcmd{show}

\begin{verbatim}
show { all | axes | functions | settings | units
       | userfunctions | variables | <parameter> }
\end{verbatim}

The \indcmdt{show} displays the present state of parameters which can be set
with the {\tt set} command. For example,

\begin{verbatim}
show pointsize
\end{verbatim}

\noindent displays the currently set point size.

Details of the various parameters which can be queried can be found under the
{\tt set} command; any keyword which can follow the {\tt set} command can also
follow the {\tt show} command.

In addition, {\tt show all} shows a complete list of the present values of all
of PyXPlot's configurable parameters.  The command {\tt show settings} shows
all of these parameters, but does not list the currently-configured variables,
functions and axes. {\tt show axes} shows the configuration states of all graph
axes. {\tt show variables} lists all of the currently defined variables. And
finally, {\tt show functions} lists all of the current user-defined functions.


\section{solve}\indcmd{solve}

\begin{verbatim}
solve <equation> {, <equation>}
    via <variable> {, <variable>}
\end{verbatim}

The \indcmdt{solve} can be used to solve simple systems of one or more
equations numerically. It takes as its arguments a comma-separated list of the
equations which are to be solved, and a comma-separated list of the variables
which are to be found. The latter should be prefixed by the word {\tt
via}, to separate it from the list of equations.

Note that the time taken by the solver dramatically increases with the number
of variables which are simultaneously found, whereas the accuracy achieved
simultaneously decreases. The following example solves a simple pair of
simultaneous equations of two variables:

\begin{verbatim}
pyxplot> solve x+y=10, x-y=3 via x,y
pyxplot> print x
6.5
pyxplot> print y
3.5
\end{verbatim}

\noindent No output is returned to the terminal if the numerical solver
succeeds, otherwise an error message is displayed. If any of the fitting
variables are already defined prior to the {\tt solve} command's being called,
their values are used as initial guesses, otherwise an initial guess of unity
for each fitting variable is assumed. Thus, the same \indcmdt{solve} returns
two different values in the following two cases:

\begin{verbatim}
pyxplot> x= # Undefine x
pyxplot> solve cos(x)=0 via x
pyxplot> print x/pi
0.5
pyxplot> x=10
pyxplot> solve cos(x)=0 via x
pyxplot> print x/pi
3.5
\end{verbatim}

\noindent In cases where any of the variables being solved for are not
dimensionless, it is essential that an initial guess with appropriate units be
supplied, otherwise the solver will try and fail to solve the system of
equations using dimensionless values:

\begin{verbatim}
x = unit(m)
y = 5*unit(km)
solve x=y via x
\end{verbatim}

The {\tt solve} command works by minimising the squares of the residuals of all
of the equations supplied, and so even when no exact solution can be found, the
best compromise is returned. The following example has no solution -- a system
of three equations with two variables is over-constrained -- but PyXPlot
nonetheless finds a compromise solution:

\begin{verbatim}
pyxplot> solve x+y=10, x-y=3, 2*x+y=16 via x,y
pyxplot> print x
6.3571429
pyxplot> print y
3.4285714
\end{verbatim}

When complex arithmetic is enabled, the {\tt solve} command allows each of the
variables being fitted to take any value in the complex plane, and thus the
number of dimensions of the fitting problem is effectively doubled -- the real
and imaginary components of each variable are solved for separately -- as in
the following example:

\begin{verbatim}
pyxplot> set numerics complex
pyxplot> solve exp(x)=e*i via x
pyxplot> print Re(x)
1
pyxplot> print Im(x)/pi
0.5
\end{verbatim}


\section{spline}\indcmd{spline}

\begin{verbatim}
spline [ <range> ] <function name>"()" '<filename>'
       [ every <expression> {:<expression>} ]
       [ index <value> ]
       [ select <expression> ]
       [ using <expression>  {:<expression>} ]
\end{verbatim}

The \indcmdt{spline} is an alias for the {\tt interpolate spline} command.
See the entry for the {\tt interpolate} command for more details.


\section{swap}\indcmd{swap}

\begin{verbatim}
swap <item1> <item2>
\end{verbatim}

Items on multiplot canvases are drawn in order of increasing identification
number, and thus items with low identification numbers are drawn first, at the
back of the multiplot, and items with higher identification numbers are later,
towards the front of the multiplot. When new items are added, they are given
higher identification numbers than previous items and appear at the front of
the multiplot.

If this is not the desired ordering, then the \indcmdt{swap} may be used to
rearrange items. It takes the identification numbers of two multiplot items and
swaps their identification numbers and hence their positions in the ordered
sequence.  Thus, if, for example, the corner of item~3 disappears behind the
corner of item~5, when the converse effect is actually desired, the following
command should be issued:
\begin{verbatim}
swap 3 5
\end{verbatim}


\section{tabulate}\indcmd{tabulate}

\begin{verbatim}
tabulate [ <range> ] ( <expression> | <filename> )
       [ every <expression> {:<expression>} ]
       [ index <value> ]
       [ select <expression> ]
       [ sortby <expression> ]
       [ using <expression>  {:<expression>} ]
       [ with <output format> ]
\end{verbatim}

PyXPlot's \indcmdt{tabulate} is similar to its {\tt plot} command, but instead
of plotting a series of \datapoint s onto a graph, it outputs them to \datafile
s. This can be used to produce text files containing samples of functions, to
rearrange/filter the columns in \datafile s, to change the units in which data
is expressed in \datafile s, and so forth.  The following example would produce
a \datafile\ called {\tt gamma.dat} containing a list of values of the gamma
function:

\begin{verbatim}
set output 'gamma.dat'
tabulate [1:5] gamma(x)
\end{verbatim}

\noindent Multiple functions may be tabulated into the same file, either by
using the \indmodt{using} modifier:

\begin{verbatim}
tabulate [0:2*pi] sin(x):cos(x):tan(x) u 1:2:3:4
\end{verbatim}

\noindent or by placing them in a comma-separated list, as in the {\tt plot}
command:

\begin{verbatim}
tabulate [0:2*pi] sin(x), cos(x), tan(x)
\end{verbatim}

In the former case, the functions are tabulated horizontally alongside one
another in a series of columns. In the latter case, the functions are tabulated
one after another in a series of index blocks separated by double linefeeds
(see Section~\ref{sec:plot_datafiles}).

The setting {\tt samples} can be used to control the number of points that are
produced when tabulating functions, in the same way that it controls the {\tt
plot} command:\indcmd{set samples}

\begin{verbatim}
set samples 200
\end{verbatim}

\noindent If the ordinate axis is set to be logarithmic then the points at which
functions are evaluated are spaced logarithmically, otherwise they are spaced
linearly.

The \indmodt{select}, \indmodt{using} and \indmodt{every} modifiers operate in
the same manner in the {\tt tabulate} command as in the {\tt plot} command.
Thus, the following example would write out the third, sixth and ninth columns
of the \datafile\ {\tt input.dat}, but only when the arcsine of the value in the
fourth column is positive:

\begin{verbatim}
set output 'filtered.dat'
tabulate 'input.dat' u 3:6:9 select (asin($4)>0)
\end{verbatim}

The numerical display format used in each column of the output file is chosen
automatically to preserve accuracy whilst simultaneously being as easily human
readable as possible.  Thus, columns which contain only integers are displayed
as such, and scientific notation is only used in columns which contain very
large or very small values.  If desired, however, a format statement may be
specified using the {\tt with format} specifier. The syntax for this is similar
to that expected by the string substitution operator ({\tt \%}; see
Section~\ref{sec:stringsubop}). As an example, to tabulate the values of $x^2$
to very many significant figures with some additional text, one could use:

\begin{verbatim}
tabulate x**2 with format "x = %f ; x**2 = %27.20e"
\end{verbatim}

\noindent This might produce the following output:

\begin{verbatim}
x = 0.000000 ; x**2 =  0.00000000000000000000e+00
x = 0.833333 ; x**2 =  6.94444444444442421371e-01
x = 1.666667 ; x**2 =  2.77777777777778167589e+00
\end{verbatim}

The data produced by the {\tt tabulate} command can be sorted in order of any
arbitrary metric by supplying an expression after the {\tt sortby} modifier;
where such expressions are supplied, the data is sorted in order from the
smallest value of the expression to the largest.


\section{text}\indcmd{text}

\begin{verbatim}
text [ item <id> ] '<text string>' [ at <x>, <y> ]
     [ rotate <angle> ] [ gap <gap> ]
     [ halign <alignment> ] [ valign <alignment> ]
     [ with colour <colour> ]
\end{verbatim}

The \indcmdt{text} allows strings of text to a added as labels on multiplot
canvases. The example

\begin{verbatim}
text 'Hello World!' at 0,2
\end{verbatim}

\noindent would render the text `Hello World!' at position $(0,2)$, measured in
centimetres. The alignment of the text item with respect to this position can
be set using the {\tt set texthalign} and {\tt set textvalign} commands, or
using the {\tt halign} and {\tt valign} modifiers supplied to the {\tt text}
command itself.

A gap may be specified, which should either have dimensions of length, or be
dimensionless, in which case it is interpreted as being measured in
centimetres. If a gap is specified, then the text string is slightly displaced
from the specified position, in the direction in which it is being aligned.

A rotation angle may optionally be specified after the keyword {\tt rotate}
to produce text rotated to any arbitrary angle, measured in degrees
counter-clockwise. The following example would produce upward-running text:

\begin{verbatim}
text 'Hello' at 1.5, 3.6 rotate 90
\end{verbatim}

By default the text is black; however, an arbitrary colour may be specified
using the {\tt with colour} modifier.  For example:

\begin{verbatim}
text 'A purple label' at 0, 0 with colour purple
\end{verbatim}

\noindent would add a purple label at the origin of the multiplot.

Outside of multiplot mode, the text command can be used to produce images
consisting simply of one single text item. This can be useful for importing
\LaTeX ed equations as gif images into slideshow programs such as Microsoft
Powerpoint which are incapable of producing such neat mathematical notation
by themselves.

All vector graphics objects placed on multiplot canvases receive unique
identification numbers which count sequentially from one, and which may be
listed using the {\tt list} command.  By reference to these numbers, they can
be deleted and subsequently restored with the {\tt delete} and {\tt undelete}
commands respectively.


\section{undelete}\indcmd{undelete}

\begin{verbatim}
undelete <item number> { , <item number> }
\end{verbatim}

The \indcmdt{undelete} allows vector graphics objects which have previously
been deleted from the current multiplot canvas to be restored.  The item(s)
which are to be restored should be identified using the reference number(s)
which were used to delete them, and can be queried using the {\tt list}
command. The example

\begin{verbatim}
undelete 1
\end{verbatim}

\noindent would cause the previously deleted item numbered {\tt 1} to reappear.


\section{unset}\indcmd{unset}

\begin{verbatim}
unset <setting>
\end{verbatim}

The \indcmdt{unset} causes a configuration option which has been changed using
the {\tt set} command to be returned to its default value.  For example:

\begin{verbatim}
unset linewidth
\end{verbatim}

\noindent returns the linewidth to its default value.

Any keyword which can follow the {\tt set} command to identify a configuration
parameter can also follow the {\tt unset} command; a complete list of these can
be found under the {\tt set} command.


\section{while}\indcmd{while}

\begin{verbatim}
while <condition> [loopname <loopname>]
 "{"
    ...
 "}"
\end{verbatim}

The \indcmdt{while} executes a block of commands repeatedly, checking the
provided condition at the start of each iteration. If the condition is true,
the loop executes again. This is similar to a {\tt do} loop, except that the
contents of a {\tt while} loop may not be executed at all if the iteration
criterion tests false upon the first iteration. For example, the following code
prints out the low-valued Fibonacci numbers:

\begin{verbatim}
i = 1
j = 1
while (j < 50)
 {
  print j
  i = i + j
  print i
  j = j + i
 }
\end{verbatim}


% FUNCTIONS.TEX
%
% The documentation in this file is part of PyXPlot
% <http://www.pyxplot.org.uk>
%
% Copyright (C) 2006-2012 Dominic Ford <coders@pyxplot.org.uk>
%               2008-2012 Ross Church
%
% $Id$
%
% PyXPlot is free software; you can redistribute it and/or modify it under the
% terms of the GNU General Public License as published by the Free Software
% Foundation; either version 2 of the License, or (at your option) any later
% version.
%
% You should have received a copy of the GNU General Public License along with
% PyXPlot; if not, write to the Free Software Foundation, Inc., 51 Franklin
% Street, Fifth Floor, Boston, MA  02110-1301, USA

% ----------------------------------------------------------------------------

% LaTeX source for the PyXPlot Users' Guide

\chapter{List of Mathematical Functions}
\label{ch:function_list}

The following is a complete list of the mathematical functions which are defined by default within PyXPlot.

\newcommand{\funcdef}[2]{
\vspace{5mm}
\begin{samepage}
\noindent
{\large \bf #1}
\newline
\indfun{#1}
The #1 function #2
\end{samepage}

}

\funcdef{abs($z$)}{returns the absolute magnitude of $z$, where $z$ may be any general complex number. The output shares the physical dimensions of $z$, if any.}
\funcdef{acos($z$)}{returns the arccosine of $z$, where $z$ may be any general dimensionless complex number. The output has physical dimensions of angle.}
\funcdef{acosh($z$)}{returns the hyperbolic arccosine of $z$, where $z$ may be any general dimensionless complex number. The output has physical dimensions of angle.}
\funcdef{acot($z$)}{returns the arccotangent of $z$, where $z$ may be any general dimensionless complex number. The output has physical dimensions of angle.}
\funcdef{acoth($z$)}{returns the hyperbolic arccotangent of $z$, where $z$ may be any general dimensionless complex number. The output has physical dimensions of angle.}
\funcdef{acsc($z$)}{returns the arccosecant of $z$, where $z$ may be any general dimensionless complex number. The output has physical dimensions of angle.}
\funcdef{acsch($z$)}{returns the hyperbolic arccosecant of $z$, where $z$ may be any general dimensionless complex number. The output has physical dimensions of angle.}
\funcdef{airy\_\-ai($z$)}{returns the Airy function Ai evaluated at $z$, where $z$ may be any dimensionless complex number.}
\funcdef{airy\_\-ai\_\-diff($z$)}{returns the first derivative of the Airy function Ai evaluated at $z$, where $z$ may be any dimensionless complex number.}
\funcdef{airy\_\-bi($z$)}{returns the Airy function Bi evaluated at $z$, where $z$ may be any dimensionless complex number.}
\funcdef{airy\_\-bi\_\-diff($z$)}{returns the first derivative of the Airy function Bi evaluated at $z$, where $z$ may be any dimensionless complex number.}
\funcdef{arg($z$)}{returns the argument of the complex number $z$, which may have any physical dimensions. The output has physical dimensions of angle.}
\funcdef{asec($z$)}{returns the arcsecant of $z$, where $z$ may be any general dimensionless complex number. The output has physical dimensions of angle.}
\funcdef{asech($z$)}{returns the hyperbolic arcsecant of $z$, where $z$ may be any general dimensionless complex number. The output has physical dimensions of angle.}
\funcdef{asin($z$)}{returns the arcsine of $z$, where $z$ may be any general dimensionless complex number. The output has physical dimensions of angle.}
\funcdef{asinh($z$)}{returns the hyperbolic arcsine of $z$, where $z$ may be any general dimensionless complex number. The output has physical dimensions of angle.}
\funcdef{ast\_\-Lcdm\_\-age($H_0$,\-$\Omega_\mathrm{M}$,\-$\Omega_\Uplambda$)}{is a specialist cosmology function. It returns the current age of the Universe in a standard $\Uplambda_\mathrm{CDM}$ cosmology with specified values for Hubble's constant, $\Omega_\mathrm{M}$ and $\Omega_\Uplambda$. Hubble's constant should be specified either with physical units of recession velocity per unit distance, or as a dimensionless number, assumed to have implicit units of km/s/Mpc. Suitable input values for a standard cosmology are: $H_0=70$, $\Omega_\mathrm{M}=0.27$ and $\Omega_\Uplambda=0.73$. For more details, see David W.\ Hogg's short article {\it Distance measures in cosmology}, available online at:\newline\url{http://arxiv.org/abs/astro-ph/9905116}.}
\funcdef{ast\_\-Lcdm\_\-angscale($z$,\-$H_0$,\-$\Omega_\mathrm{M}$,\-$\Omega_\Uplambda$)}{is a specialist cosmology function. It returns the angular scale of the sky at a redshift of $z$ in a standard $\Uplambda_\mathrm{CDM}$ cosmology. For details, see the {\tt ast\_Lcdm\_age()} function above. The returned value has dimensions of distance per unit angle.}
\funcdef{ast\_\-Lcdm\_\-DA($z$,\-$H_0$,\-$\Omega_\mathrm{M}$,\-$\Omega_\Uplambda$)}{is a specialist cosmology function. It returns the angular size distance of objects at a redshift of $z$ in a standard $\Uplambda_\mathrm{CDM}$ cosmology. For details, see the {\tt ast\_Lcdm\_age()} function above. The returned value has dimensions of distance.}
\funcdef{ast\_\-Lcdm\_\-DL($z$,\-$H_0$,\-$\Omega_\mathrm{M}$,\-$\Omega_\Uplambda$)}{is a specialist cosmology function. It returns the luminosity distance of objects at a redshift of $z$ in a standard $\Uplambda_\mathrm{CDM}$ cosmology. For details, see the {\tt ast\_Lcdm\_age()} function above. The returned value has dimensions of distance.}
\funcdef{ast\_\-Lcdm\_\-DM($z$,\-$H_0$,\-$\Omega_\mathrm{M}$,\-$\Omega_\Uplambda$)}{is a specialist cosmology function. It returns the proper motion distance of objects at a redshift of $z$ in a standard $\Uplambda_\mathrm{CDM}$ cosmology. For details, see the {\tt ast\_Lcdm\_age()} function above. The returned value has dimensions of distance.}
\funcdef{ast\_\-Lcdm\_\-t($z$,\-$H_0$,\-$\Omega_\mathrm{M}$,\-$\Omega_\Uplambda$)}{is a specialist cosmology function. It returns the lookback time to objects at a redshift of $z$ in a standard $\Uplambda_\mathrm{CDM}$ cosmology. For details, see the {\tt ast\_Lcdm\_age()} function above. The returned value has dimensions of time. To find the age of the Universe at a redshift of $z$, this value should be subtracted from the output of the {\tt ast\_Lcdm\_age()} function.}
\funcdef{ast\_\-Lcdm\_\-z($t$,\-$H_0$,\-$\Omega_\mathrm{M}$,\-$\Omega_\Uplambda$)}{is a specialist cosmology function. It returns the redshift corresponding to a lookback time of $t$ in a standard $\Uplambda_\mathrm{CDM}$ cosmology. For details, see the {\tt ast\_Lcdm\_age()} function above. The returned value is dimensionless.}
\funcdef{ast\_\-moonphase($JD$)}{returns the phase of the Moon, with dimensions of angle, at the time corresponding to the supplied Julian Day number.}
\funcdef{ast\_\-sidereal\_\-time($JD$)}{returns the sidereal time at Greenwich, with dimensions of angle, at the time corresponding to the supplied Julian Day number. This is equal to the right ascension of the stars which are transiting the Greenwich meridian at that time. This function uses the expression for sidereal time adopted in 1982 by the International Astronomical Union (IAU), and which is reproduced in Chapter~12 of Jean Meeus' book {\it Astronomical Algorithms} (1998).}
\funcdef{atan($z$)}{returns the arctangent of $z$, where $z$ may be any general dimensionless complex number. The output has physical dimensions of angle.}
\funcdef{atanh($z$)}{returns the hyperbolic arctangent of $z$, where $z$ may be any general dimensionless complex number. The output has physical dimensions of angle.}
\funcdef{atan2($x,y$)}{returns the arctangent of $x/y$. Unlike atan($y/x$), atan2($x,y$) takes account of the signs of both $x$ and $y$ to remove the degeneracy between $(1,1)$ and $(-1,-1)$. $x$ and $y$ must be real numbers, and must have matching physical dimensions.}
\funcdef{besseli($l,x$)}{evaluates the $l$th regular modified spherical Bessel function at $x$. $l$ must be a positive dimensionless real integer. $x$ must be a real dimensionless number.}
\funcdef{besselI($l,x$)}{evaluates the $l$th regular modified cylindrical Bessel function at $x$. $l$ must be a positive dimensionless real integer. $x$ must be a real dimensionless number.}
\funcdef{besselj($l,x$)}{evaluates the $l$th regular spherical Bessel function at $x$. $l$ must be a positive dimensionless real integer. $x$ must be a real dimensionless number.}
\funcdef{besselJ($l,x$)}{evaluates the $l$th regular cylindrical Bessel function at $x$. $l$ must be a positive dimensionless real integer. $x$ must be a real dimensionless number.}
\funcdef{besselk($l,x$)}{evaluates the $l$th irregular modified spherical Bessel function at $x$. $l$ must be a positive dimensionless real integer. $x$ must be a real dimensionless number.}
\funcdef{besselK($l,x$)}{evaluates the $l$th irregular modified cylindrical Bessel function at $x$. $l$ must be a positive dimensionless real integer. $x$ must be a real dimensionless number.}
\funcdef{bessely($l,x$)}{evaluates the $l$th irregular spherical Bessel function at $x$. $l$ must be a positive dimensionless real integer. $x$ must be a real dimensionless number.}
\funcdef{besselY($l,x$)}{evaluates the $l$th irregular cylindrical Bessel function at $x$. $l$ must be a positive dimensionless real integer. $x$ must be a real dimensionless number.}
\funcdef{beta($a,b$)}{evaluates the beta function $B(a,b)$, where $a$ and $b$ must be dimensionless real numbers.}
\funcdef{binomialCDF($k,p,n$)}{evaluates the probability of getting fewer than or exactly $k$ successes out of $n$ trials in a binomial distribution with success probability $p$. $k$ and $n$ must be positive real integers. $p$ must be a real number in the range $0\leq p \leq 1$.}
\funcdef{binomialPDF($k,p,n$)}{evaluates the probability of getting $k$ successes out of $n$ trials in a binomial distribution with success probability $p$. $k$ and $n$ must be positive real integers. $p$ must be a real number in the range $0\leq p \leq 1$.}
\funcdef{Bv($\nu,T$)}{returns the power emitted by a blackbody of temperature $T$ at frequency $\nu$ per unit area, per unit solid angle, per unit frequency. $T$ should have physical dimensions of temperature, or be a dimensionless number, in which case it is understood to be a temperature in Kelvin. $\nu$ should have physical dimensions of frequency, or be a dimensionless number, in which case it is understood to be a frequency measured in Hertz. The output has physical dimensions of power per unit area per unit solid angle per unit frequency.}
\funcdef{Bvmax($T$)}{returns the frequency at which the function Bv($\nu,T$) reaches its maximum, as calculated by the Wien Displacement Law. The inputs are subject to the same constraints as above.}
\funcdef{ceil($x$)}{returns the smallest integer value greater than or equal to $x$, where $x$ must be a dimensionless real number.}
\funcdef{chisqCDF($x,nu$)}{returns the cumulative probability density at $x$ in a $\chi$-squared distribution with $\nu$ degrees of freedom. $\nu$ must be a positive real dimensionless integer. $x$ must be a positive real dimensionless number.}
\funcdef{chisqCDFi($P,nu$)}{returns the point $x$ at which the cumulative probability density in a $\chi$-squared distribution with $\nu$ degrees of freedom is $P$. $\nu$ must be a positive real dimensionless integer. $P$ must be a real number in the range $0\leq p \leq 1$.}
\funcdef{chisqPDF($x,nu$)}{returns the probability density at $x$ in a $\chi$-squared distribution with $\nu$ degrees of freedom. $\nu$ must be a positive real dimensionless integer. $x$ must be a positive real dimensionless number.}
\funcdef{conditionalN($a,b,c$)}{returns either $b$, if $a$ is true, or $c$ otherwise. Arguments $b$ and $c$ may be complex and may have any physical dimensions, but may not be strings. Argument $a$ must be a dimensionless real number.}
\funcdef{conditionalS($a,b,c$)}{returns either $b$, if $a$ is true, or $c$ otherwise. Arguments $b$ and $c$ should be string variables or expressions. Argument $a$ must be a dimensionless real number.}
\funcdef{conjugate($z$)}{returns the complex conjugate of the complex number $z$, which may have any physical dimensions.}
\funcdef{cos($z$)}{returns the cosine of $z$, where $z$ may be any complex number and must either have physical dimensions of angle or be a dimensionless number, in which case it is understood to be measured in radians.}
\funcdef{cosh($z$)}{returns the hyperbolic cosine of $z$, where $z$ may be any complex number and must either have physical dimensions of angle or be a dimensionless number, in which case it is understood to be measured in radians.}
\funcdef{cot($z$)}{returns the cotangent of $z$, where $z$ may be any complex number and must either have physical dimensions of angle or be a dimensionless number, in which case it is understood to be measured in radians.}
\funcdef{coth($z$)}{returns the hyperbolic cotangent of $z$, where $z$ may be any complex number and must either have physical dimensions of angle or be a dimensionless number, in which case it is understood to be measured in radians.}
\funcdef{csc($z$)}{returns the cosecant of $z$, where $z$ may be any complex number and must either have physical dimensions of angle or be a dimensionless number, in which case it is understood to be measured in radians.}
\funcdef{csch($z$)}{returns the hyperbolic cosecant of $z$, where $z$ may be any complex number and must either have physical dimensions of angle or be a dimensionless number, in which case it is understood to be measured in radians.}
\funcdef{degrees($x$)}{takes a real input which should either have physical units of angle, or be dimensionless, in which case it is assumed to be measured in radians. The output is the dimensionless number of degrees in $x$.}
\funcdef{diff\_dx($e,x,step$)}{numerically differentiates an expression $e$ with respect to $a$ at $x$, using a step size of $step$. `{\tt x}' can be replaced by any variable name of fewer than 16 characters, and so, for example, the {\tt diff\_dfoobar()} function differentiates an expression with respect to the variable {\tt foobar}. The expression $e$ may optionally be enclosed in quotes. Both $x$, and the output differential, may be complex numbers with any physical unit. The step size may optionally be omitted, in which case a value of $10^{-6}$ is used. The following example would differentiate the expression $x^2$ with respect to $x$:\newline{\tt print diff\_dx("x**2", 1, 1e-6)}.}
\funcdef{ellipticintE($k$)}{evaluates the following complete elliptic integral: \begin{displaymath} E(k) = \int_0^1 \sqrt{\frac{1-k^2 t^2}{1-t^2}}\,\mathrm{d}t. \end{displaymath} }
\funcdef{ellipticintK($k$)}{evaluates the following complete elliptic integral: \begin{displaymath} K(k) = \int_0^1 \frac{\mathrm{d}t}{\sqrt{(1-t^2)(1-k^2 t^2)}}. \end{displaymath} }
\funcdef{ellipticintP($k,n$)}{evaluates the following complete elliptic integral: \begin{displaymath} P(k,n) = \int_0^{\nicefrac{\pi}{2}} \frac{\mathrm{d}\theta}{(1+n\sin^2\theta)(1-k^2\sin^2\theta)}.\end{displaymath} }
\funcdef{erf($x$)}{evaluates the error function at $x$, where $x$ must be a dimensionless real number.}
\funcdef{erfc($x$)}{evaluates the complementary error function at $x$, where $x$ must be a dimensionless real number.}
\funcdef{exp($z$)}{returns $e^z$, where $z$ can be a complex number but must either be dimensionless or be an angle.}
\funcdef{expint($n,x$)}{evaluates the following integral: \begin{displaymath} \int_{t=1}^{t=\infty} \exp(-xt)/t^n \, \mathrm{d}t. \end{displaymath} $n$ must be a positive real dimensionless integer and $x$ must be a real dimensionless number.}
\funcdef{expm1($x$)}{accurately evaluates $\exp(x)-1$, where $x$ must be a dimensionless real number.}
\funcdef{finite($x$)}{returns one if $x$ is a finite number, and zero otherwise.}
\funcdef{floor($x$)}{returns the largest integer value smaller than or equal to $x$, where $x$ must be a dimensionless real number.}
\funcdef{fractal\_julia($z$,$z_c$,$m$)}{tests whether the point $z$ in the complex plane lies within the Julia set associated with the point $z_c$ in the complex plane. The expression $z_{n+1} = z_n^2 + z_c$ is iterated until either $|z_n|>2$, in which case the iteration is deemed to have diverged, or until $m$ iterations have been exceeded, in which case it is deemed to have remained bounded. The number of iterations required for divergence is returned, or $m$ is returned if the iteration remained bounded -- i.e.\ the point lies within the numerical approximation to the Julia set.}
\funcdef{fractal\_mandelbrot($z$,$m$)}{tests whether the point $z$ in the complex plane lies within the Mandelbrot set. The expression $z_{n+1} = z_n^2 + z_0$ is iterated until either $|z_n|>2$, in which case the iteration is deemed to have diverged, or until $m$ iterations have been exceeded, in which case it is deemed to have remained bounded. The number of iterations required for divergence is returned, or $m$ is returned if the iteration remained bounded -- i.e.\ the point lies within the numerical approximation to the Mandelbrot set.}
\funcdef{gamma($x$)}{evaluates the gamma function $\Gamma(x)$, where $x$ must be a dimensionless real number.}
\funcdef{gaussianCDF($x,\sigma$)}{evaluates the Gaussian cumulative distribution function of standard deviation $\sigma$ at $x$. The distribution is centred upon $x=0$. $x$ and $\sigma$ must both be real, but may have any physical dimensions so long as they match.}
\funcdef{gaussianCDFi($x,\sigma$)}{evaluates the inverse Gaussian cumulative distribution function of standard deviation $\sigma$ at $x$. The distribution is centred upon $x=0$. $x$ and $\sigma$ must both be real, but may have any physical dimensions so long as they match.}
\funcdef{gaussianPDF($x,\sigma$)}{evaluates the Gaussian probability density function of standard deviation $\sigma$ at $x$. The distribution is centred upon $x=0$. $x$ and $\sigma$ must both be real, but may have any physical dimensions so long as they match.}
\funcdef{heaviside($x$)}{returns the Heaviside function, defined to be one for $x\geq0$ and zero otherwise. $x$ must be a dimensionless real number.}
\funcdef{hyperg\_0F1($c,x$)}{evaluates the hypergeometric function $_0F_1(c,x)$. All inputs must be dimensionless real numbers. For reference, the implementation used is GSL's {\tt gsl\_sf\_hyperg\_0F1} function.}
\funcdef{hyperg\_1F1($a,b,x$)}{evaluates the hypergeometric function $_1F_1(a,b,x)$. All inputs must be dimensionless real numbers. For reference, the implementation used is GSL's {\tt gsl\_sf\_hyperg\_1F1} function.}
\funcdef{hyperg\_2F0($a,b,x$)}{evaluates the hypergeometric function $_2F_0(a,b,x)$. All inputs must be dimensionless real numbers.For reference, the implementation used is GSL's {\tt gsl\_sf\_hyperg\_2F0} function.}
\funcdef{hyperg\_2F1($a,b,c,x$)}{evaluates the hypergeometric function $_2F_1(a,b,c,x)$. All inputs must be dimensionless real numbers. For reference, the implementation used is GSL's {\tt gsl\_sf\_hyperg\_2F1} function. This implementation cannot evaluate the region $|x|<1$.}
\funcdef{hyperg\_U($a,b,x$)}{evaluates the hypergeometric function $U(a,b,x)$. All inputs must be dimensionless real numbers. For reference, the implementation used is GSL's {\tt gsl\_sf\_hyperg\_U} function.}
\funcdef{hypot($x,y$)}{returns the quadrature sum of $x$ and $y$, $\sqrt{x^2+y^2}$, where $x$ and $y$ may have any physical dimensions so long as they match, and can be complex numbers.}
\funcdef{Im($z$)}{returns the imaginary part of the complex number $z$, which may have any physical units. The number returned shares the same physical units as $z$.}
\funcdef{int\_dx($e,min,max$)}{numerically integrates an expression $e$ with respect to $x$ between $min$ and $max$. `{\tt x}' can be replaced by any variable name of fewer than 16 characters, and so, for example, the {\tt int\_dfoobar()} function integrates an expression with respect to the variable {\tt foobar}. The expression $e$ may optionally be enclosed in quotes. $min$ and $max$ may have any physical units, so long as they match, but must be real numbers. The output integral may be a complex number, and may have any physical dimensions. The following example would integrate the expression $x^2$ with respect to $x$ between $1$\,m and $2$\,m:\newline{\tt print int\_dx("x**2", 1*unit(m), 2*unit(m))}.}
\funcdef{jacobi\_cn($u,m$)}{evaluates a Jacobi elliptic function; it returns the value $\cos\phi$ where $\phi$ is defined by the integral \begin{displaymath} \int_{0}^{\phi}\frac{\mathrm{d}\theta}{\sqrt{1-m\sin^2\theta}}.\end{displaymath}}
\funcdef{jacobi\_dn($u,m$)}{evaluates a Jacobi elliptic function; it returns the value $\sqrt{1-m\sin^2\theta}$ where $\phi$ is defined by the integral \begin{displaymath} \int_{0}^{\phi}\frac{\mathrm{d}\theta}{\sqrt{1-m\sin^2\theta}}.\end{displaymath}}
\funcdef{jacobi\_sn($u,m$)}{evaluates a Jacobi elliptic function; it returns the value $\sin\phi$ where $\phi$ is defined by the integral \begin{displaymath} \int_{0}^{\phi}\frac{\mathrm{d}\theta}{\sqrt{1-m\sin^2\theta}}.\end{displaymath}}
\funcdef{lambert\_W0($x$)}{evaluates the principal real branch of the Lambert~W function, for which $W>-1$ when $x<0$.}
\funcdef{lambert\_W1($x$)}{evaluates the secondary real branch of the Lambert~W function, for which $W<-1$ when $x<0$.}
\funcdef{ldexp($x,y$)}{returns $x$ times $2^y$ for integer y, where both $x$ and $y$ must be real.}
\funcdef{legendreP($l,x$)}{evaluates the $l$th Legendre polynomial at $x$, where $l$ must be a positive dimensionless real integer and $x$ must be a real dimensionless number.}
\funcdef{legendreQ($l,x$)}{evaluates the $l$th Legendre function at $x$, where $l$ must be a positive dimensionless real integer and $x$ must be a real dimensionless number.}
\funcdef{log($z$)}{returns the natural logarithm of $z$, where $z$ may be any complex dimensionless number.}
\funcdef{log10($z$)}{returns the logarithm to base~10 of $z$, where $z$ may be any complex dimensionless number.}
\funcdef{lognormalCDF($x,\zeta,\sigma$)}{evaluates the log normal cumulative distribution function of standard deviation $\sigma$, centred upon $\zeta$, at $x$. $\sigma$ must be real, positive and dimensionless. $x$ and $\zeta$ must both be real, but may have any physical dimensions so long as they match.}
\funcdef{lognormalCDFi($x,\zeta,\sigma$)}{evaluates the inverse log normal cumulative distribution function of standard deviation $\sigma$, centred upon $\zeta$, at $x$. $\sigma$ must be real, positive and dimensionless. $x$ and $\zeta$ must both be real, but may have any physical dimensions so long as they match.}
\funcdef{lognormalPDF($x,\zeta,\sigma$)}{evaluates the log normal probability density function of standard deviation $\sigma$, centred upon $\zeta$, at $x$. $\sigma$ must be real, positive and dimensionless. $x$ and $\zeta$ must both be real, but may have any physical dimensions so long as they match.}
\funcdef{max($x,y$)}{returns the greater of the two values $x$ and $y$, where $x$ and $y$ may have any physical dimensions so long as they match. If either input is complex, the input with the larger magnitude is returned.}
\funcdef{min($x,y$)}{returns the lesser of the two values $x$ and $y$, where $x$ and $y$ may have any physical dimensions so long as they match. If either input is complex, the input with the smaller magnitude is returned.}
\funcdef{mod($x,y$)}{returns the remainder of $x/y$, where $x$ and $y$ may have any physical dimensions so long as they match but must both be real.}
\funcdef{ordinal($x$)}{returns an ordinal string, for example, ``1st'', ``2nd'' or ``3rd'', for any positive dimensionless real number $x$.}
\funcdef{poissonCDF($x,\mu$)}{returns the probability of getting $\leq x$ from a Poisson distribution with mean $\mu$, where $\mu$ must be real, positive and dimensionless and $x$ must be real and dimensionless.}
\funcdef{poissonPDF($x,\mu$)}{returns the probability of getting $x$ from a Poisson distribution with mean $\mu$, where $\mu$ must be real, positive and dimensionless and $x$ must be a real dimensionless integer.}
\funcdef{pow($x,y$)}{returns $x$ to the power of $y$, where $x$ and $y$ may both be complex numbers and $x$ may have any physical dimensions but $y$ must be dimensionless. It not not permitted for $y$ to be complex if $x$ is not dimensionless, since this would lead to an output with complex physical dimensions.}
\funcdef{prime($x$)}{returns one if floor($x$) is a prime number and zero otherwise.}
\funcdef{radians($x$)}{takes a real input which should either have physical units of angle, or be dimensionless, in which case it is assumed to be measured in degrees. The output is the dimensionless number of radians in $x$.}
\funcdef{random()}{returns a random real number between 0 and~1.}
\funcdef{random\_\-binomial($p,n$)}{returns a random sample from a binomial distribution with $n$ independent trials and a success probability $p$. $n$ must be a real positive dimensionless integer. $p$ must be a dimensionless number in the range $0\leq p\leq 1$.}
\funcdef{random\_\-chisq($\nu$)}{returns a random sample from a $\chi$-squared distribution with $\nu$ degrees of freedom, where $\nu$ must be a real positive dimensionless integer.}
\funcdef{random\_\-gaussian($\sigma$)}{returns a random sample from a Gaussian (normal) distribution of standard deviation $\sigma$ and centred upon zero. $\sigma$ must be real, but may have any physical units. The returned random sample shares the physical units of $\sigma$.}
\funcdef{random\_\-lognormal($\zeta,\sigma$)}{returns a random sample from the log normal distribution centred on $\zeta$, and of width $\sigma$. $\sigma$ must be a real positive dimensionless number. $\zeta$ must be real, but may have any physical units. The returned random sample shares the physical units of $\zeta$.}
\funcdef{random\_\-poisson($n$)}{returns a random integer from a Poisson distribution with mean $n$, where $n$ must be a real positive dimensionless number.}
\funcdef{random\_\-tdist($\nu$)}{returns a random sample from a $t$-distribution with $\nu$ degrees of freedom, where $\nu$ must be a real positive dimensionless integer.}
\funcdef{Re($z$)}{returns the real part of the complex number $z$, which may have any physical units. The number returned shares the same physical units as $z$.}
\funcdef{root($z,n$)}{returns the $n$th root of $z$. $z$ may be any complex number, and may have any physical dimensions. $n$ must be a dimensionless integer. When complex arithmetic is enabled, and whenever $z$ is positive, this function is entirely equivalent to {\tt pow(z,1/n)}. However, when $z$ is negative and complex arithmetic is disabled, the expression {\tt pow(z,1/n)} may not be evaluated, since it will in general have a small imaginary part for any finite-precision floating-point representation of $1/n$. The expression {\tt root(z,n)}, on the other hand, may be evaluated under such conditions, providing that $n$ is an odd integer.}
\funcdef{sec($z$)}{returns the secant of $z$, where $z$ may be any complex number and must either have physical dimensions of angle or be a dimensionless number, in which case it is understood to be measured in radians.}
\funcdef{sech($z$)}{returns the hyperbolic secant of $z$, where $z$ may be any complex number and must either have physical dimensions of angle or be a dimensionless number, in which case it is understood to be measured in radians.}
\funcdef{sin($z$)}{returns the sine of $z$, where $z$ may be any complex number and must either have physical dimensions of angle or be a dimensionless number, in which case it is understood to be measured in radians.}
\funcdef{sinc($z$)}{returns the sinc function $\sin(z)/z$ for any complex number $z$, which may either be dimensionless, in which case it is understood to be measured in radians, or have physical dimensions of angle. The result is dimensionless.}
\funcdef{sinh($z$)}{returns the hyperbolic sine of $z$, where $z$ may be any complex number and must either have physical dimensions of angle or be a dimensionless number, in which case it is understood to be measured in radians.}
\funcdef{sqrt($z$)}{returns the square root of $z$, which may be any complex number and may have any physical dimensions.}
\funcdef{strcmp($s1,s2$)}{returns zero if the strings $s1$ and $s2$ are the same, one if $s1$ should be placed after $s2$ in alphabetical sequence, minus one if $s1$ should be placed before $s2$ in alphabetical sequence.}
\funcdef{strlen($s$)}{returns the length of the string $s$.}
\funcdef{strlower($s$)}{returns a version of the string $s$ in which all alphabetic characters are converted to lowercase.}
\funcdef{strrange($s,i,j$)}{returns a slice of the string $s$ containing only the $i$th through until the $j$th characters of the string. If either $i$ or $j$ are negative, they are counted from the end of the string; for example, $-1$ refers to the last character of the string.}
\funcdef{strupper($s$)}{returns a version of the string $s$ in which all alphabetic characters are converted to uppercase.}
\funcdef{tan($z$)}{returns the tangent of $z$, where $z$ may be any complex number and must either have physical dimensions of angle or be a dimensionless number, in which case it is understood to be measured in radians.}
\funcdef{tanh($z$)}{returns the hyperbolic tangent of $z$, where $z$ may be any complex number and must either have physical dimensions of angle or be a dimensionless number, in which case it is understood to be measured in radians.}
\funcdef{tdistCDF($x,nu$)}{returns the cumulative probability density at $x$ in a $t$-distribution with $\nu$ degrees of freedom. $\nu$ must be a positive real dimensionless integer. $x$ must be a positive real dimensionless number.}
\funcdef{tdistCDFi($P,nu$)}{returns the point $x$ at which the cumulative probability density in a $t$-distribution with $\nu$ degrees of freedom is $P$. $\nu$ must be a positive real dimensionless integer. $P$ must be a real number in the range $0\leq p \leq 1$.}
\funcdef{tdistPDF($x,nu$)}{returns the probability density at $x$ in a $t$-distribution with $\nu$ degrees of freedom. $\nu$ must be a positive real dimensionless integer. $x$ must be a positive real dimensionless number.}
\funcdef{texify(\ldots)}{returns a string of \LaTeX\ text corresponding to the algebraic expression or string supplied between the brackets.}
\funcdef{time\_\-daymonth({\it JD})}{returns the number of the day of the month on which the supplied Julian Day number falls. For more details see the {\tt time\_\-julianday()} function.}
\funcdef{time\_\-dayweekname({\it JD})}{returns as a string the name of the day of the week on which the supplied Julian Day number falls. For more details see the {\tt time\_\-julianday()} function.}
\funcdef{time\_\-dayweeknum({\it JD})}{returns the number, in the range 1~(Sunday) to 7~(Saturday), of the day of the week on which the supplied Julian Day number falls. For more details see the {\tt time\_\-julianday()} function.}
\funcdef{time\_\-diff({\it JD}$_1$,\-{\it JD}$_2$)}{returns the time interval elapsed between the first and second supplied Julian Day numbers. For more details see the {\tt time\_\-julianday()} function.}
\funcdef{time\_\-diff\_\-string({\it JD}$_1$,\-{\it JD}$_2$,\-{\it format})}{returns a string representation of the time interval elapsed between the first and second supplied Julian Day numbers. For more details about Julian Day numbers, see the {\tt time\_\-julianday()} function. The third input is used to control the format of the output, with the following tokens being substituted for:
\begin{longtable}{|>{\columncolor{LightGrey}}l|>{\columncolor{LightGrey}}l|}
\hline \endfoot
\hline
Token & Value \\
\hline \endhead
{\tt \%\%} & A literal \% sign.\\
{\tt \%d} & The number of days elapsed, modulo 365.\\
{\tt \%D} & The number of days elapsed. \\
{\tt \%h} & The number of hours elapsed, modulo 24.\\
{\tt \%H} & The number of hours elapsed.\\
{\tt \%m} & The number of minutes elapsed, modulo 60.\\
{\tt \%M} & The number of minutes elapsed.\\
{\tt \%s} & The number of seconds elapsed, modulo 60.\\
{\tt \%S} & The number of seconds elapsed.\\
{\tt \%Y} & The number of years elapsed.\\
\end{longtable}}
\funcdef{time\_\-fromunix({\it u})}{returns the Julian Day number corresponding to the specified Unix time $u$.}
\funcdef{time\_\-hour({\it JD})}{returns the integer hour of the day, in the range 0-23, in which the supplied Julian Day number falls. For more details see the {\tt time\_\-julianday()} function.}
\funcdef{time\_\-julianday({\it year},\-{\it month},\-{\it day},\-{\it hour},\-{\it min},\-{\it sec})}{returns the Julian Day number corresponding to the supplied time and date. Each field should be supplied numerically as an integer: for example, the {\it month} argument should be an integer in the range 1-12. In the default British calendar, the {\it year} argument should be the number of years elapsed since the Christian epoch. To enter dates before {\footnotesize AD\,1}, $0$ should be passed to indicate the year 1\,{\footnotesize BC}, $-1$ should be passed to indicate the year 2\,{\footnotesize BC}, and so forth. Dates may be entered in other calendars using the {\tt set calendar} command.

Julian Day numbers are defined to be the number of days elapsed since noon on 1st January, 4713\,{\footnotesize BC} in the Julian Calendar. Consequently, Julian Day numbers are rather large numbers: for example, midnight on 1st January 2000 corresponds to the Julian Day number 2451544.5. The hour of the day is indicated by the fractional part of the Julian Day number.

Julian Day numbers provide a useful means for analysing time-series data because they are not subject to the complicated non-decimal units in which time is conventionally measured. The task of calculating the time interval elapsed between two calendar dates is in general rather difficult, especially when leap years need be considered. However, the number of 24-hour periods elapsed between two Julian Day numbers is simply the numerical difference between the two day numbers.

PyXPlot's implementation of the {\tt time\_\-julianday()} function includes a consideration of the transition which was made from the Julian calendar to the Gregorian calendar on various dates in various countries. By default the transition is made at midnight on 14th~September 1752 (Gregorian calendar), when Britain and the British Empire adopted the Gregorian calendar. However, this may be changed using the {\tt set calendar} command.}
\funcdef{time\_\-min({\it JD})}{returns the integer number of minutes elapsed within the hour when the supplied Julian Day number falls. For more details see the {\tt time\_\-julianday()} function.}
\funcdef{time\_\-monthname({\it JD})}{returns as a string the English name of the calendar month in which the supplied Julian Day number falls. For more details see the {\tt time\_\-julianday()} function.}
\funcdef{time\_\-monthnum({\it JD})}{returns the number, in the range 1-12, of the calendar month in which the supplied Julian Day number falls. For more details see the {\tt time\_\-julianday()} function.}
\funcdef{time\_\-now()}{returns the Julian Day number corresponding to the present epoch. For more details see the {\tt time\_\-julianday()} function.}
\funcdef{time\_\-sec({\it JD})}{returns the number of seconds elapsed within the minute when the supplied Julian Day number falls. The number of seconds includes the fractional part. For more details see the {\tt time\_\-julianday()} function.}
\funcdef{time\_\-string({\it JD},\-{\it format})}{returns a string representation of the date and time corresponding to the supplied Julian Day number. For more details about Julian Day numbers, see the {\tt time\_\-julianday()} function. The second input is optional, and may be used to control the format of the output. If no format string is provided, then the format \newline\noindent{\tt "\%a \%Y \%b \%d \%H:\%M:\%S"}\newline\noindent is used. In such format strings, the following tokens are substituted for various parts of the date:
\begin{longtable}{|>{\columncolor{LightGrey}}l|>{\columncolor{LightGrey}}l|}
\hline \endfoot
\hline
Token & Value \\
\hline \endhead
{\tt \%\%} & A literal \% sign.\\
{\tt \%a} & Three-letter abbreviated weekday name.\\
{\tt \%A} & Full weekday name.\\
{\tt \%b} & Three-letter abbreviated month name.\\
{\tt \%B} & Full month name.\\
{\tt \%C} & Century number, e.g. 21 for the years 2000-2100.\\
{\tt \%d} & Day of month.\\
{\tt \%H} & Hour of day, in range~00-23.\\
{\tt \%I} & Hour of day, in range~01-12.\\
{\tt \%k} & Hour of day, in range~0-23.\\
{\tt \%l} & Hour of day, in range~1-12.\\
{\tt \%m} & Month number, in range~01-12.\\
{\tt \%M} & Minute, in range~00-59.\\
{\tt \%p} & Either {\tt am} or {\tt pm}.\\
{\tt \%S} & Second, in range~00-59.\\
{\tt \%y} & Last two digits of year number.\\
{\tt \%Y} & Year number.\\
\end{longtable}}
\funcdef{time\_\-unix({\it JD})}{returns the Unix time corresponding to the specified Julian Day number.}
\funcdef{time\_\-year({\it JD})}{returns the year in which the supplied Julian Day number falls. For more details see the {\tt time\_\-julianday()} function. The returned value is a dimensionless integer. A value of $0$ corresponds to the year 1\,{\footnotesize BC}; a value of $-1$ corresponds to the year 2\,{\footnotesize BC}, and so forth.}
\funcdef{tophat($x,\sigma$)}{returns one if $|x| \leq |\sigma|$, and zero otherwise. Both inputs must be real, but may have any physical dimensions so long as they match.}
\funcdef{unit(\ldots)}{multiplies a number by a physical unit. The string inside the brackets should consist of a string of the names of physical units, multiplied together with the {\tt *} operator, divided using the {\tt /} operator, or raised by numeric powers using the {\tt \^{}} operator. The list may be commenced with a numeric constant, for example: {\tt unit(2*m\^{}2/s)}.}
\funcdef{zernike($n,m,r,\phi$)}{evaluates the Zernike polynomial $Z^m_n(r,\phi)$, where $m$ and $n$ are non-negative integers with $n\geq m$, $r$ is the radial coordinate in the range $0<r<1$ and $\phi$ is the azimuthal coordinate.}
\funcdef{zernikeR($n,m,r$)}{evaluates the radial Zernike polynomial $R^m_n(r)$, where $m$ and $n$ are non-negative integers with $n\geq m$ and $r$ is the radial coordinate in the range $0<r<1$.}
\funcdef{zeta($x$)}{evaluates the Riemann zeta function for any dimensionless number $x$.}


% types.tex
%
% The documentation in this file is part of Pyxplot
% <http://www.pyxplot.org.uk>
%
% Copyright (C) 2006-2013 Dominic Ford <coders@pyxplot.org.uk>
%               2008-2013 Ross Church
%
% $Id$
%
% Pyxplot is free software; you can redistribute it and/or modify it under the
% terms of the GNU General Public License as published by the Free Software
% Foundation; either version 2 of the License, or (at your option) any later
% version.
%
% You should have received a copy of the GNU General Public License along with
% Pyxplot; if not, write to the Free Software Foundation, Inc., 51 Franklin
% Street, Fifth Floor, Boston, MA  02110-1301, USA

% ----------------------------------------------------------------------------

% LaTeX source for the Pyxplot Users' Guide

\chapter{List of data types}
\label{ch:types_list}

The following is a list of Pyxplot's data types:

\begin{itemize}
\item boolean
\item color
\item date
\item dictionary
\item exception
\item fileHandle
\item function
\item instance
\item list
\item matrix
\item module
\item null
\item number
\item string
\item type
\item vector
\end{itemize}

Each of these data types has a prototype object in the module {\tt types},
which can be called like a function to create a new object of the type. See
Section~\ref{sec:functions_types} for details of the arguments accepted by each
prototype.

All objects in Pyxplot have methods that can be called on them, using the
generic syntax:

\begin{verbatim}
object.methodName(arguments)
\end{verbatim}

\noindent Some methods are common to all objects. For example, all objects have
a method {\tt str()} which produces a string representation of the object, as
used by the {\tt print} command. They also all have a {\tt methods()} method,
which returns a list of the names of all of the methods available for the
object. For example:

\vspace{3mm}
\input{fragments/tex/types_methods.tex}
\vspace{3mm}

\noindent As the above examples show, printing a method object returns brief
documentation about it. The sections below list the methods of each data type.

\section{Methods common to all data types}
\label{sec:common_methods}

\methdef{class()}{returns the class prototype of an object.}
\methdef{contents()}{returns a list of all the methods and internal variables of an object.}
\methdef{data()}{returns a list of all the internal variables (not methods) of an object.}
\methdef{methods()}{returns a list of the methods of an object.}
\methdef{str()}{returns a string representation of an object.}
\methdef{type()}{returns the type of an object.}

\section{The \texttt{boolean} type}
\label{sec:boolean_methods}

The \texttt{boolean} type has no methods other than those common to all types.
Objects of type \texttt{boolean} have no methods other than those common to all types.

\section{The \texttt{color} type}
\label{sec:color_methods}

\methdef{componentsCMYK()}{returns a vector CMYK representation of a color.}
\methdef{componentsHSB()}{returns a vector HSB representation of a color.}
\methdef{componentsRGB()}{returns a vector RGB representation of a color.}
\methdef{toCMYK()}{returns color object containing a CMYK representation of a color.}
\methdef{toHSB()}{returns color object containing an HSB representation of a color.}
\methdef{toRGB()}{returns color object containing an RGB representation of a color.}

\section{The \texttt{date} type}
\label{sec:date_methods}

For more information about manipulating dates in Pyxplot, see
Section~\ref{sec:time_series}.  For more information about manipulating times
and dates in Pyxplot, see Section~\ref{sec:time_series}. Many of the methods
listed below take an optional timezone string as their final argument. This
should be specified in the form {\tt Europe/London}, {\tt America/New\_York} or
{\tt Australia/Perth}, as used by the {\tt tz database}. A complete list of
available timezones can be found here:
\url{http://en.wikipedia.org/wiki/List_of_tz_database_time_zones}.  If
universal time is used, the timezone may be specified as {\tt UTC}.

\methdef{str($<format>,<timezone>$)}{converts a date object to a string with an optional format string supplied as an argument (see the {\tt time.string()} function).}
\methdef{toDayOfMonth($<timezone>$)}{returns the day of the month of a date object in the current calendar.}
\methdef{toDayWeekName($<timezone>$)}{returns the name of the day of the week of a date object.}
\methdef{toDayWeekNum($<timezone>$)}{returns the day of the week (1--7) of a date object.}
\methdef{toHour($<timezone>$)}{returns the integer hour component (0--23) of a date object.}
\methdef{toJD()}{converts a date object to a numerical Julian date.}
\methdef{toMinute($<timezone>$)}{returns the integer minute component (0--59) of a date object.}
\methdef{toMJD()}{converts a date object to a modified Julian date.}
\methdef{toMonthName($<timezone>$)}{returns the name of the month in which a date object falls.}
\methdef{toMonthNum($<timezone>$)}{returns the number (1--12) of the month in which a date object falls.}
\methdef{toSecond($<timezone>$)}{returns the seconds component (0--60) of a date object, including the non-integer component.}
\methdef{toUnix()}{converts a date object to a Unix time.}
\methdef{toYear($<timezone>$)}{returns the year in which a date object falls in the current calendar.}

\section{The \texttt{dictionary} type}
\label{sec:dictionary_methods}

\methdef{delete($s$)}{deletes any element with string key $s$ from the dictionary.}
\methdef{hasKey($x$)}{returns a boolean indicating whether the key $x$ exists in the dictionary.}
\methdef{items()}{returns a list of the [key,value] pairs in a dictionary.}
\methdef{keys()}{returns a list of the keys defined in a dictionary.}
\methdef{len()}{returns the number of entries in a dictionary.}
\methdef{values()}{returns a list of the values in a dictionary.}

\section{The \texttt{exception} type}
\label{sec:exception_methods}

\methdef{raise($x$)}{raises an exception with error message string $x$.}

\section{The \texttt{fileHandle} type}
\label{sec:fileHandle_methods}

\methdef{close()}{closes a file handle.}
\methdef{dump($x$)}{stores a typeable ASCII representation of the object $x$ to a file. Note that this method has no checking for recursive hierarchical data structures.}
\methdef{eof()}{returns a boolean flag to indicate whether the end of a file has been reached.}
\methdef{flush()}{flushes any buffered data which has not yet physically been written to a file.}
\methdef{getPos()}{returns a file handle's current position in a file.}
\methdef{isOpen()}{returns a boolean flag indicating whether a file is open.}
\methdef{read()}{returns the contents of a file as a string.}
\methdef{readline()}{returns a single line of a file as a string.}
\methdef{readlines()}{returns the lines of a file as a list of strings.}
\methdef{setPos($x$)}{sets a file handle's current position in a file.}
\methdef{write($x$)}{writes the string $x$ to a file.}

\section{The \texttt{function} type}

Objects of type \texttt{function} have no methods other than those common to all types.

\section{The \texttt{instance} type}

\methdef{delete($s$)}{deletes any element with string key $s$ from the instance.}
\methdef{hasKey($x$)}{returns a boolean indicating whether the key $x$ exists in the instance.}
\methdef{items()}{returns a list of the [key,value] pairs in a instance.}
\methdef{keys()}{returns a list of the keys defined in a instance.}
\methdef{len()}{returns the number of entries in a instance.}
\methdef{values()}{returns a list of the values in a instance.}

\section{The \texttt{list} type}
\label{sec:list_methods}

\methdef{append($x$)}{appends the object $x$ to a list and returns the new list.}
\methdef{count($x$)}{returns the number of items in a list that equal $x$.}
\methdef{extend($x$)}{appends the members of a list or vector $x$ to the operand and returns the new list.}
\methdef{filter($f$)}{takes a pointer to a function of one argument, $f(a)$. It calls the function for every element of the list, and returns a new list of those elements for which $f(a)$ tests true.}
\methdef{index($x$)}{returns the index of the first element of a list that equals $x$, or $-1$ if no elements match.}
\methdef{insert$n,x$)}{inserts the number $x$ into a list at position $n$, and returns the new list.}
\methdef{len()}{returns the number of elements in a list.}
\methdef{map($f$)}{takes a pointer to a function of one argument, $f(a)$. It calls the function for every element of the list, and returns a list of the results.}
\methdef{max()}{returns the highest-valued item in a list.}
\methdef{min()}{returns the lowest-valued item in a list.}
\methdef{pop()}{returns the last item in a list, and removes it from the list.}
\methdef{reduce($f$)}{takes a pointer to a function of two arguments. It first calls $f(a,b)$ on the first two elements of the list, and then continues through the list calling $f(a,b)$ on the result and the next item in the list. The final result is returned.}
\methdef{reverse()}{reverses the order of the members of a list, and returns the new list.}
\methdef{sort()}{sorts the members of a list into ascending order, and returns the new list.}
\methdef{sortOn($f$)}{sorts the members of a list using the user-supplied function $f(a,b)$ to determine the sort order. $f$ should return $1$, $0$ or $-1$ depending whether $a>b$, $a=b$ or $a<b$.}
\methdef{sortOnElement($n$)}{sorts a list of lists on the $n$th element of each sublist. If $n$ is negative, it counts from the final item of each list, $n=-1$ being the last item.}
\methdef{vector()}{returns the elements in a list as a vector.}

\section{The \texttt{matrix} type}
\label{sec:matrix_methods}

\methdef{det()}{returns the determinant of a square matrix.}
\methdef{diagonal()}{returns a boolean indicating whether a matrix is diagonal.}
\methdef{eigenvalues()}{returns a vector containing the eigenvalues of a square symmetric matrix.}
\methdef{eigenvectors()}{returns a list of the eigenvectors of a square symmetric matrix.}
\methdef{inv()}{returns the inverse of a square matrix.}
\methdef{size()}{returns the dimensions of a matrix.}
\methdef{symmetric()}{returns a boolean indicating whether a matrix is symmetric.}
\methdef{transpose()}{returns the transpose of a matrix.}

\section{The \texttt{module} type}

\methdef{delete($s$)}{deletes any element with string key $s$ from the module.}
\methdef{hasKey($x$)}{returns a boolean indicating whether the key $x$ exists in the module.}
\methdef{items()}{returns a list of the [key,value] pairs in a module.}
\methdef{keys()}{returns a list of the keys defined in a module.}
\methdef{len()}{returns the number of entries in a module.}
\methdef{values()}{returns a list of the values in a module.}

\section{The \texttt{null} type}

Objects of type \texttt{null} have no methods other than those common to all types.

\section{The \texttt{number} type}

Objects of type \texttt{number} have no methods other than those common to all types.

\section{The \texttt{string} type}
\label{sec:string_methods}

\methdef{append($x$)}{appends the string $x$ to the end of a string.}
\methdef{beginsWith($x$)}{returns a boolean indicating whether a string begins with the substring $x$.}
\methdef{endsWith($x$)}{returns a boolean indicating whether a string ends with the substring $x$.}
\methdef{find($x$)}{returns the position of the first occurrence of $x$ in a string, or $-1$ if it is not found.}
\methdef{findAll($x$)}{returns a list of the positions where the substring $x$ occurs in a string.}
\methdef{isalnum()}{returns a boolean indicating whether all of the characters of a string are alphanumeric.}
\methdef{isalpha()}{returns a boolean indicating whether all of the characters of a string are alphabetic.}
\methdef{isdigit()}{returns a boolean indicating whether all of the characters of a string are numeric.}
\methdef{len()}{returns the length of a string.}
\methdef{lower()}{converts a string to lowercase.}
\methdef{lstrip()}{strips whitespace off the beginning of a string.}
\methdef{split()}{returns a list of all the whitespace-separated words in a string.}
\methdef{splitOn(...)}{splits a string whenever it encounters any of the substrings supplied as arguments, and returns a list of the split string segments.}
\methdef{strip()}{strips whitespace off the beginning and end of a string.}
\methdef{rstrip()}{strips whitespace off the end of a string.}
\methdef{upper()}{converts a string to uppercase.}

\section{The \texttt{type} type}

Objects of type \texttt{type} have no methods other than those common to all types.

\section{The \texttt{vector} type}
\label{sec:vector_methods}

\methdef{append($x$)}{appends the number $x$ to a vector and returns the new vector.}
\methdef{extend($x$)}{appends the members of a list or vector $x$ to the operand and returns the new vector.}
\methdef{filter($f$)}{takes a pointer to a function of one argument, $f(a)$. It calls the function for every element of the vector, and returns a new vector of those elements for which $f(a)$ tests true.}
\methdef{insert($n,x$)}{inserts the number $x$ into a vector at position $n$, and returns the new vector.}
\methdef{len()}{returns the number of elements in a vector.}
\methdef{list()}{returns the elements in a vector as a list.}
\methdef{map($f$)}{takes a pointer to a function of one argument, $f(a)$. It calls the function for every element of the vector, and returns a vector of the results.}
\methdef{norm()}{returns the quadrature sum of the elements in a vector.}
\methdef{reduce($f$)}{takes a pointer to a function of two arguments. It first calls $f(a,b)$ on the first two elements of the vector, and then continues through the vector calling $f(a,b)$ on the result and the next item in the vector. The final result is returned.}
\methdef{reverse()}{reverses the order of the elements of a vector, and returns the new vector.}
\methdef{sort()}{sorts the elements of a vector into ascending order, and returns the new vector.}


% constants.tex
%
% The documentation in this file is part of PyXPlot
% <http://www.pyxplot.org.uk>
%
% Copyright (C) 2006-2012 Dominic Ford <coders@pyxplot.org.uk>
%               2008-2012 Ross Church
%
% $Id$
%
% PyXPlot is free software; you can redistribute it and/or modify it under the
% terms of the GNU General Public License as published by the Free Software
% Foundation; either version 2 of the License, or (at your option) any later
% version.
%
% You should have received a copy of the GNU General Public License along with
% PyXPlot; if not, write to the Free Software Foundation, Inc., 51 Franklin
% Street, Fifth Floor, Boston, MA  02110-1301, USA

% ----------------------------------------------------------------------------

% LaTeX source for the PyXPlot Users' Guide

\chapter{List of physical constants}
\label{ch:constants}

The following table lists all of the physical constants which are defined by default in PyXPlot:

\begin{landscape}
\begin{center}
\begin{longtable}{|lll|}
\hline \endfoot
\hline
{\bf Name} & {\bf Description} & {\bf Approximate Value} \\ \hline \endhead
{\tt e} & $e$ & $2.7182818$ \\
{\tt euler} & The Euler constant & $0.57721566$ \\
{\tt False} & Boolean truth value & 0 \\
{\tt GoldenRatio} & The golden ratio & $1.618034$ \\
{\tt i} & The square-root of $-1$ & $i$ \\
{\tt phy\_alpha} & The fine-structure constant & $0.0072973525$ \\
{\tt phy\_c} & The speed of light & $299792458\,\mathrm{m}/\mathrm{s}$ \\
{\tt phy\_epsilon\_0} & The permittivity of free space & $8.85418782\times10^{-12}\,\mathrm{F}/\mathrm{m}$ \\
{\tt phy\_G} & The gravitational constant & $6.673\times10^{-11}\,\mathrm{m}^3/\mathrm{kg}/\mathrm{s}^{2}$ \\
{\tt phy\_g} & The mean terrestrial acceleration due to gravity & $9.80665\,\mathrm{m}/\mathrm{s}^{2}$ \\
{\tt phy\_h} & The Planck constant & $6.62606896\times10^{-34}\,\mathrm{J}\,\mathrm{s}$ \\
{\tt phy\_hbar} & The Planck constant$\,/\,2\pi$ & $1.05457163\times10^{-34}\,\mathrm{J}\,\mathrm{s}$ \\
{\tt phy\_kB} & The Boltzmann constant & $1.3806504\times10^{-23}\,\mathrm{J}/\mathrm{K}$ \\
{\tt phy\_Lsun} & The luminosity of the Sun & $3.839\times10^{26}\,\mathrm{W}$ \\
{\tt phy\_Msun} & The mass of the Sun & $1.98892\times10^{30}\,\mathrm{kg}$ \\
{\tt phy\_mu\_0} & The permeability of free space & $1.25663706\times10^{-6}\,\mathrm{N}/\mathrm{A}^{2}$ \\
{\tt phy\_mu\_b} & The Bohr magneton & $9.27400899\times10^{-24}\,\mathrm{J}/\mathrm{T}$ \\
{\tt phy\_m\_e} & The mass of the electron & $9.10938188\times10^{-31}\,\mathrm{kg}$ \\
{\tt phy\_m\_muon} & The mass of the muon & $1.88353109\times10^{-28}\,\mathrm{kg}$ \\
{\tt phy\_m\_n} & The mass of the neutron & $1.67492716\times10^{-27}\,\mathrm{kg}$ \\
{\tt phy\_m\_p} & The mass of the proton & $1.67262158\times10^{-27}\,\mathrm{kg}$ \\
{\tt phy\_m\_u} & The unified mass constant & $1.66053878\times10^{-27}\,\mathrm{kg}$ \\
{\tt phy\_NA} & Avogadro's number & $6.02214199\times10^{23}\,\mathrm{mol}^{-1}$ \\
{\tt phy\_q} & The fundamental charge & $1.60217649\times10^{-19}\,\mathrm{C}$ \\
{\tt phy\_R} & The gas constant & $8.314472\,\mathrm{J}/\mathrm{K}/\mathrm{mol}$ \\
{\tt phy\_Rsun} & The radius of the Sun & $695500000\,\mathrm{m}$ \\
{\tt phy\_Ry} & The Rydberg constant & $10973732\,\mathrm{m}^{-1}$ \\
{\tt phy\_sigma} & The Stefan-Boltzmann constant & $5.67040047\times10^{-8}\,\mathrm{kg}/\mathrm{s}^{3}/\mathrm{K}^{4}$ \\
{\tt pi} & $\pi$ & $3.1415927$ \\
{\tt True} & Boolean truth value & 1 \\
{\tt version} & PyXPlot version string & ``\version'' \\
\hline
\end{longtable}
\end{center}
\end{landscape}


% units.tex
%
% The documentation in this file is part of PyXPlot
% <http://www.pyxplot.org.uk>
%
% Copyright (C) 2006-2012 Dominic Ford <coders@pyxplot.org.uk>
%               2008-2012 Ross Church
%
% $Id$
%
% PyXPlot is free software; you can redistribute it and/or modify it under the
% terms of the GNU General Public License as published by the Free Software
% Foundation; either version 2 of the License, or (at your option) any later
% version.
%
% You should have received a copy of the GNU General Public License along with
% PyXPlot; if not, write to the Free Software Foundation, Inc., 51 Franklin
% Street, Fifth Floor, Boston, MA  02110-1301, USA

% ----------------------------------------------------------------------------

% LaTeX source for the PyXPlot Users' Guide

\chapter{List of physical units}
\label{ch:unit_list}
\index{units!list}

The following table lists all of the physical units which PyXPlot recognises by
default. Each unit may be referred to by either a long or an abbreviated name,
and both of these have singular and plural forms. Some units also have further
alternative names.

\begin{landscape}
\begin{center}
\begin{longtable}{|lllll|}
\hline \endfoot
\hline
\multicolumn{4}{|l}{\bf Name} & \\
\multicolumn{2}{|l}{\bf Full} & \multicolumn{2}{l}{\bf Abbrev} & {\bf Unit of} \\
{\bf sing.} & {\bf pl.} & {\bf sing.} & {\bf pl.} & \\ \hline \endhead
{\tt\footnotesize acre} & {\tt\footnotesize acres} & {\tt\footnotesize acre} & {\tt\footnotesize acres} & area \\
{\tt\footnotesize ampere} & {\tt\footnotesize amperes} & {\tt\footnotesize A} & {\tt\footnotesize A} & current \\
\multicolumn{5}{|r|}{\footnotesize Also known as the {\tt amp} and the {\tt amps}.} \\
{\tt\footnotesize angstrom} & {\tt\footnotesize angstroms} & {\tt\footnotesize ang} & {\tt\footnotesize ang} & length \\
{\tt\footnotesize arcminute} & {\tt\footnotesize arcminutes} & {\tt\footnotesize arcmin} & {\tt\footnotesize arcmins} & angle \\
{\tt\footnotesize arcsecond} & {\tt\footnotesize arcseconds} & {\tt\footnotesize arcsec} & {\tt\footnotesize arcsecs} & angle \\
{\tt\footnotesize are} & {\tt\footnotesize ares} & {\tt\footnotesize are} & {\tt\footnotesize ares} & area \\
{\tt\footnotesize astronomical\_unit} & {\tt\footnotesize astronomical\_units} & {\tt\footnotesize AU} & {\tt\footnotesize AU} & length \\
{\tt\footnotesize atmosphere} & {\tt\footnotesize atmospheres} & {\tt\footnotesize atm} & {\tt\footnotesize atms} & pressure \\
{\tt\footnotesize bar} & {\tt\footnotesize bars} & {\tt\footnotesize bar} & {\tt\footnotesize bars} & pressure \\
{\tt\footnotesize barn} & {\tt\footnotesize barns} & {\tt\footnotesize barn} & {\tt\footnotesize barns} & area \\
{\tt\footnotesize barye} & {\tt\footnotesize baryes} & {\tt\footnotesize Ba} & {\tt\footnotesize Ba} & pressure \\
{\tt\footnotesize bath} & {\tt\footnotesize baths} & {\tt\footnotesize bath} & {\tt\footnotesize baths} & volume \\
{\tt\footnotesize becquerel} & {\tt\footnotesize becquerel} & {\tt\footnotesize Bq} & {\tt\footnotesize Bq} & frequency \\
{\tt\footnotesize billion\_electronvolts} & {\tt\footnotesize billion\_electronvolts} & {\tt\footnotesize BeV} & {\tt\footnotesize BeV} & energy \\
{\tt\footnotesize bit} & {\tt\footnotesize bits} & {\tt\footnotesize bit} & {\tt\footnotesize bits} & information\_content \\
{\tt\footnotesize British Thermal Unit} & {\tt\footnotesize British Thermal Units} & {\tt\footnotesize BTU} & {\tt\footnotesize BTU} & energy \\
{\tt\footnotesize bushel\_UK} & {\tt\footnotesize bushels\_UK} & {\tt\footnotesize bushel\_UK} & {\tt\footnotesize bushels\_UK} & volume (UK imperial) \\
{\tt\footnotesize bushel\_US} & {\tt\footnotesize bushels\_US} & {\tt\footnotesize bushel\_US} & {\tt\footnotesize bushels\_US} & volume (US customary) \\
{\tt\footnotesize byte} & {\tt\footnotesize bytes} & {\tt\footnotesize B} & {\tt\footnotesize B} & information\_content \\
{\tt\footnotesize cable} & {\tt\footnotesize cables} & {\tt\footnotesize cable} & {\tt\footnotesize cables} & length \\
{\tt\footnotesize calorie} & {\tt\footnotesize calories} & {\tt\footnotesize cal} & {\tt\footnotesize cal} & energy \\
{\tt\footnotesize candela} & {\tt\footnotesize candelas} & {\tt\footnotesize cd} & {\tt\footnotesize cd} & light\_intensity \\
{\tt\footnotesize candlepower} & {\tt\footnotesize candlepower} & {\tt\footnotesize candlepower} & {\tt\footnotesize candlepower} & light\_intensity \\
{\tt\footnotesize carat} & {\tt\footnotesize carats} & {\tt\footnotesize CD} & {\tt\footnotesize CDs} & mass \\
{\tt\footnotesize centimetre} & {\tt\footnotesize centimetres} & {\tt\footnotesize cm} & {\tt\footnotesize cm} & length \\
{\tt\footnotesize chain} & {\tt\footnotesize chains} & {\tt\footnotesize chain} & {\tt\footnotesize chains} & length \\
{\tt\footnotesize clo} & {\tt\footnotesize clos} & {\tt\footnotesize clo} & {\tt\footnotesize clos} & thermal\_insulation \\
{\tt\footnotesize coulomb} & {\tt\footnotesize coulombs} & {\tt\footnotesize C} & {\tt\footnotesize C} & charge \\
{\tt\footnotesize cubic\_centimetre} & {\tt\footnotesize cubic\_centimetres} & {\tt\footnotesize cubic\_cm} & {\tt\footnotesize cubic\_cm} & volume \\
{\tt\footnotesize cubic\_foot} & {\tt\footnotesize cubic\_feet} & {\tt\footnotesize cubic\_ft} & {\tt\footnotesize cubic\_ft} & volume \\
{\tt\footnotesize cubic\_inch} & {\tt\footnotesize cubic\_inches} & {\tt\footnotesize cubic\_in} & {\tt\footnotesize cubic\_in} & volume \\
{\tt\footnotesize cubic\_metre} & {\tt\footnotesize cubic\_metres} & {\tt\footnotesize cubic\_m} & {\tt\footnotesize cubic\_m} & volume \\
{\tt\footnotesize cubit} & {\tt\footnotesize cubits} & {\tt\footnotesize cubit} & {\tt\footnotesize cubits} & length \\
{\tt\footnotesize cup\_US} & {\tt\footnotesize cups\_US} & {\tt\footnotesize cup\_US} & {\tt\footnotesize cups\_US} & volume (US customary) \\
{\tt\footnotesize day} & {\tt\footnotesize days} & {\tt\footnotesize day} & {\tt\footnotesize days} & time \\
{\tt\footnotesize decimetre} & {\tt\footnotesize decimetres} & {\tt\footnotesize dm} & {\tt\footnotesize dm} & length \\
{\tt\footnotesize degree} & {\tt\footnotesize degrees} & {\tt\footnotesize deg} & {\tt\footnotesize deg} & angle \\
{\tt\footnotesize degree\_celsius} & {\tt\footnotesize degrees\_celsius} & {\tt\footnotesize oC} & {\tt\footnotesize oC} & temperature \\
\multicolumn{5}{|r|}{\footnotesize Also known as the {\tt degree\_centigrade}, the {\tt degrees\_centigrade}, the {\tt centigrade} and the {\tt celsius}.} \\
{\tt\footnotesize degree\_fahrenheit} & {\tt\footnotesize degrees\_fahrenheit} & {\tt\footnotesize oF} & {\tt\footnotesize oF} & temperature \\
\multicolumn{5}{|r|}{\footnotesize Also known as the {\tt fahrenheit}.} \\
{\tt\footnotesize dioptre} & {\tt\footnotesize dioptres} & {\tt\footnotesize dioptre} & {\tt\footnotesize dioptres} & lens\_power \\
{\tt\footnotesize drachm} & {\tt\footnotesize drachms} & {\tt\footnotesize drachm} & {\tt\footnotesize drachms} & mass \\
{\tt\footnotesize dyne} & {\tt\footnotesize dynes} & {\tt\footnotesize dyn} & {\tt\footnotesize dyn} & force \\
{\tt\footnotesize earth\_mass} & {\tt\footnotesize earth\_masses} & {\tt\footnotesize Mearth} & {\tt\footnotesize Mearth} & mass \\
{\tt\footnotesize earth\_radius} & {\tt\footnotesize earth\_radii} & {\tt\footnotesize Rearth} & {\tt\footnotesize Rearth} & length \\
{\tt\footnotesize electronvolt} & {\tt\footnotesize electronvolts} & {\tt\footnotesize eV} & {\tt\footnotesize eV} & energy \\
{\tt\footnotesize erg} & {\tt\footnotesize ergs} & {\tt\footnotesize erg} & {\tt\footnotesize erg} & energy \\
{\tt\footnotesize euro} & {\tt\footnotesize euros} & {\tt\footnotesize euro} & {\tt\footnotesize euros} & cost \\
{\tt\footnotesize farad} & {\tt\footnotesize farad} & {\tt\footnotesize F} & {\tt\footnotesize F} & capacitance \\
{\tt\footnotesize fathom} & {\tt\footnotesize fathoms} & {\tt\footnotesize fathom} & {\tt\footnotesize fathoms} & length \\
{\tt\footnotesize firkin\_UK\_ale} & {\tt\footnotesize firkins\_UK\_ale} & {\tt\footnotesize firkin\_UK\_ale} & {\tt\footnotesize firkins\_UK\_ale} & volume \\
{\tt\footnotesize firkin\_wine} & {\tt\footnotesize firkins\_wine} & {\tt\footnotesize firkin\_UK\_wine} & {\tt\footnotesize firkins\_UK\_wine} & volume \\
{\tt\footnotesize fluid\_ounce\_UK} & {\tt\footnotesize fluid\_ounce\_UK} & {\tt\footnotesize fl\_oz\_UK} & {\tt\footnotesize fl\_oz\_UK} & volume (UK imperial) \\
{\tt\footnotesize fluid\_ounce\_US} & {\tt\footnotesize fluid\_ounce\_US} & {\tt\footnotesize fl\_oz\_US} & {\tt\footnotesize fl\_oz\_US} & volume (US customary) \\
{\tt\footnotesize foot} & {\tt\footnotesize feet} & {\tt\footnotesize ft} & {\tt\footnotesize ft} & length \\
{\tt\footnotesize furlong} & {\tt\footnotesize furlongs} & {\tt\footnotesize furlong} & {\tt\footnotesize furlongs} & length \\
{\tt\footnotesize gallon\_UK} & {\tt\footnotesize gallons\_UK} & {\tt\footnotesize gallon\_UK} & {\tt\footnotesize gallons\_UK} & volume (UK imperial) \\
{\tt\footnotesize gallon\_US} & {\tt\footnotesize gallons\_US} & {\tt\footnotesize gallon\_US} & {\tt\footnotesize gallons\_US} & volume (US customary) \\
{\tt\footnotesize gauss} & {\tt\footnotesize gauss} & {\tt\footnotesize G} & {\tt\footnotesize G} & magnetic\_field \\
{\tt\footnotesize gibibit} & {\tt\footnotesize gibibits} & {\tt\footnotesize Gib} & {\tt\footnotesize Gib} & information\_content \\
{\tt\footnotesize gibibyte} & {\tt\footnotesize gibibytes} & {\tt\footnotesize GiB} & {\tt\footnotesize GiB} & information\_content \\
{\tt\footnotesize grain} & {\tt\footnotesize grains} & {\tt\footnotesize grain} & {\tt\footnotesize grains} & mass \\
{\tt\footnotesize gram} & {\tt\footnotesize grams} & {\tt\footnotesize g} & {\tt\footnotesize g} & mass \\
{\tt\footnotesize gramme} & {\tt\footnotesize grammes} & {\tt\footnotesize g} & {\tt\footnotesize g} & mass \\
{\tt\footnotesize gray} & {\tt\footnotesize gray} & {\tt\footnotesize Gy} & {\tt\footnotesize Gy} & radiation\_dose \\
{\tt\footnotesize hectare} & {\tt\footnotesize hectares} & {\tt\footnotesize hectare} & {\tt\footnotesize hectares} & area \\
{\tt\footnotesize henry} & {\tt\footnotesize henry} & {\tt\footnotesize H} & {\tt\footnotesize H} & inductance \\
{\tt\footnotesize hertz} & {\tt\footnotesize hertz} & {\tt\footnotesize Hz} & {\tt\footnotesize Hz} & frequency \\
{\tt\footnotesize homer} & {\tt\footnotesize homers} & {\tt\footnotesize homer} & {\tt\footnotesize homers} & volume \\
{\tt\footnotesize horsepower} & {\tt\footnotesize horsepower} & {\tt\footnotesize horsepower} & {\tt\footnotesize horsepower} & power \\
{\tt\footnotesize hour} & {\tt\footnotesize hours} & {\tt\footnotesize hr} & {\tt\footnotesize hr} & time \\
{\tt\footnotesize hundredweight\_UK} & {\tt\footnotesize hundredweight\_UK} & {\tt\footnotesize cwt\_UK} & {\tt\footnotesize cwt\_UK} & mass (UK imperial) \\
{\tt\footnotesize hundredweight\_US} & {\tt\footnotesize hundredweight\_US} & {\tt\footnotesize cwt\_US} & {\tt\footnotesize cwt\_US} & mass (US customary) \\
{\tt\footnotesize inch} & {\tt\footnotesize inches} & {\tt\footnotesize in} & {\tt\footnotesize in} & length \\
{\tt\footnotesize inch\_of\_mercury} & {\tt\footnotesize inches\_of\_mercury} & {\tt\footnotesize inHg} & {\tt\footnotesize inHg} & pressure \\
{\tt\footnotesize inch\_of\_water} & {\tt\footnotesize inches\_of\_water} & {\tt\footnotesize inAq} & {\tt\footnotesize inAq} & pressure \\
{\tt\footnotesize jansky} & {\tt\footnotesize janskys} & {\tt\footnotesize Jy} & {\tt\footnotesize Jy} & flux\_density \\
{\tt\footnotesize joule} & {\tt\footnotesize joules} & {\tt\footnotesize J} & {\tt\footnotesize J} & energy \\
{\tt\footnotesize jupiter\_mass} & {\tt\footnotesize jupiter\_masses} & {\tt\footnotesize Mjupiter} & {\tt\footnotesize Mjupiter} & mass \\
\multicolumn{5}{|r|}{\footnotesize Also known as the {\tt Mjove} and the {\tt Mjovian}.} \\
{\tt\footnotesize jupiter\_radius} & {\tt\footnotesize jupiter\_radii} & {\tt\footnotesize Rjupiter} & {\tt\footnotesize Rjupiter} & length \\
\multicolumn{5}{|r|}{\footnotesize Also known as the {\tt Rjove} and the {\tt Rjovian}.} \\
{\tt\footnotesize katal} & {\tt\footnotesize katals} & {\tt\footnotesize kat} & {\tt\footnotesize kat} & catalytic\_activity \\
{\tt\footnotesize kayser} & {\tt\footnotesize kaysers} & {\tt\footnotesize kayser} & {\tt\footnotesize kaysers} & wavenumber \\
{\tt\footnotesize kelvin} & {\tt\footnotesize kelvin} & {\tt\footnotesize K} & {\tt\footnotesize K} & temperature \\
{\tt\footnotesize kibibit} & {\tt\footnotesize kibibits} & {\tt\footnotesize Kib} & {\tt\footnotesize Kib} & information\_content \\
{\tt\footnotesize kibibyte} & {\tt\footnotesize kibibytes} & {\tt\footnotesize KiB} & {\tt\footnotesize KiB} & information\_content \\
{\tt\footnotesize kilderkin\_UK\_ale} & {\tt\footnotesize kilderkins\_UK\_ale} & {\tt\footnotesize kilderkin\_UK\_ale} & {\tt\footnotesize kilderkins\_UK\_ale} & volume \\
{\tt\footnotesize kilogram} & {\tt\footnotesize kilograms} & {\tt\footnotesize kg} & {\tt\footnotesize kg} & mass \\
{\tt\footnotesize kilowatt\_hour} & {\tt\footnotesize kilowatt\_hours} & {\tt\footnotesize kWh} & {\tt\footnotesize kWh} & energy \\
{\tt\footnotesize knot} & {\tt\footnotesize knots} & {\tt\footnotesize kn} & {\tt\footnotesize kn} & velocity \\
{\tt\footnotesize light\_year} & {\tt\footnotesize light\_years} & {\tt\footnotesize lyr} & {\tt\footnotesize lyr} & length \\
{\tt\footnotesize link} & {\tt\footnotesize links} & {\tt\footnotesize link} & {\tt\footnotesize links} & length \\
{\tt\footnotesize litre} & {\tt\footnotesize litres} & {\tt\footnotesize l} & {\tt\footnotesize l} & volume \\
{\tt\footnotesize long\_ton} & {\tt\footnotesize long\_tons} & {\tt\footnotesize ton} & {\tt\footnotesize tons} & mass (UK imperial) \\
{\tt\footnotesize lumen} & {\tt\footnotesize lumens} & {\tt\footnotesize lm} & {\tt\footnotesize lm} & power \\
{\tt\footnotesize lunar\_distance} & {\tt\footnotesize lunar\_distances} & {\tt\footnotesize lunar\_distance} & {\tt\footnotesize lunar\_distances} & length \\
{\tt\footnotesize lux} & {\tt\footnotesize luxs} & {\tt\footnotesize lx} & {\tt\footnotesize lx} & power \\
{\tt\footnotesize maxwell} & {\tt\footnotesize maxwell} & {\tt\footnotesize Mx} & {\tt\footnotesize Mx} & magnetic\_flux \\
{\tt\footnotesize mebibit} & {\tt\footnotesize mebibits} & {\tt\footnotesize Mib} & {\tt\footnotesize Mib} & information\_content \\
{\tt\footnotesize mebibyte} & {\tt\footnotesize mebibytes} & {\tt\footnotesize MiB} & {\tt\footnotesize MiB} & information\_content \\
{\tt\footnotesize metre} & {\tt\footnotesize metres} & {\tt\footnotesize m} & {\tt\footnotesize m} & length \\
{\tt\footnotesize mho} & {\tt\footnotesize mhos} & {\tt\footnotesize mho} & {\tt\footnotesize mhos} & conductance \\
{\tt\footnotesize mile} & {\tt\footnotesize miles} & {\tt\footnotesize mi} & {\tt\footnotesize mi} & length \\
{\tt\footnotesize mile\_per\_hour} & {\tt\footnotesize miles\_per\_hour} & {\tt\footnotesize mph} & {\tt\footnotesize mph} & velocity \\
{\tt\footnotesize mina} & {\tt\footnotesize minas} & {\tt\footnotesize mina} & {\tt\footnotesize minas} & mass \\
{\tt\footnotesize minute} & {\tt\footnotesize minutes} & {\tt\footnotesize min} & {\tt\footnotesize min} & time \\
{\tt\footnotesize mole} & {\tt\footnotesize moles} & {\tt\footnotesize mol} & {\tt\footnotesize mol} & moles \\
{\tt\footnotesize nautical\_mile} & {\tt\footnotesize nautical\_miles} & {\tt\footnotesize nautical\_mile} & {\tt\footnotesize nautical\_miles} & length \\
{\tt\footnotesize newton} & {\tt\footnotesize newtons} & {\tt\footnotesize N} & {\tt\footnotesize N} & force \\
{\tt\footnotesize ohm} & {\tt\footnotesize ohms} & {\tt\footnotesize ohm} & {\tt\footnotesize ohms} & resistance \\
{\tt\footnotesize ounce} & {\tt\footnotesize ounces} & {\tt\footnotesize oz} & {\tt\footnotesize oz} & mass \\
{\tt\footnotesize parsec} & {\tt\footnotesize parsecs} & {\tt\footnotesize pc} & {\tt\footnotesize pc} & length \\
{\tt\footnotesize parts\_per\_billion} & {\tt\footnotesize parts\_per\_billion} & {\tt\footnotesize ppb} & {\tt\footnotesize ppb} & dimensionlessness \\
{\tt\footnotesize parts\_per\_million} & {\tt\footnotesize parts\_per\_million} & {\tt\footnotesize ppm} & {\tt\footnotesize ppm} & dimensionlessness \\
{\tt\footnotesize pascal} & {\tt\footnotesize pascals} & {\tt\footnotesize Pa} & {\tt\footnotesize Pa} & pressure \\
{\tt\footnotesize percent} & {\tt\footnotesize percent} & {\tt\footnotesize percent} & {\tt\footnotesize percent} & dimensionlessness \\
{\tt\footnotesize perch} & {\tt\footnotesize perches} & {\tt\footnotesize perch} & {\tt\footnotesize perches} & length \\
{\tt\footnotesize pica} & {\tt\footnotesize picas} & {\tt\footnotesize pica} & {\tt\footnotesize picas} & length \\
{\tt\footnotesize pint\_UK} & {\tt\footnotesize pints\_UK} & {\tt\footnotesize pint\_UK} & {\tt\footnotesize pints\_UK} & volume (UK imperial) \\
{\tt\footnotesize pint\_US} & {\tt\footnotesize pints\_US} & {\tt\footnotesize pint\_US} & {\tt\footnotesize pints\_US} & volume (US customary) \\
{\tt\footnotesize planck\_charge} & {\tt\footnotesize planck\_charges} & {\tt\footnotesize Q\_planck} & {\tt\footnotesize Q\_planck} & charge \\
{\tt\footnotesize planck\_current} & {\tt\footnotesize planck\_current} & {\tt\footnotesize I\_planck} & {\tt\footnotesize I\_planck} & current \\
{\tt\footnotesize planck\_energy} & {\tt\footnotesize planck\_energy} & {\tt\footnotesize E\_planck} & {\tt\footnotesize E\_planck} & energy \\
{\tt\footnotesize planck\_force} & {\tt\footnotesize planck\_force} & {\tt\footnotesize F\_planck} & {\tt\footnotesize F\_planck} & force \\
{\tt\footnotesize planck\_impedence} & {\tt\footnotesize planck\_impedence} & {\tt\footnotesize Z\_planck} & {\tt\footnotesize Z\_planck} & resistance \\
{\tt\footnotesize planck\_length} & {\tt\footnotesize planck\_lengths} & {\tt\footnotesize L\_planck} & {\tt\footnotesize L\_planck} & length \\
{\tt\footnotesize planck\_mass} & {\tt\footnotesize planck\_masses} & {\tt\footnotesize M\_planck} & {\tt\footnotesize M\_planck} & mass \\
{\tt\footnotesize planck\_momentum} & {\tt\footnotesize planck\_momentum} & {\tt\footnotesize p\_planck} & {\tt\footnotesize p\_planck} & momentum \\
{\tt\footnotesize planck\_power} & {\tt\footnotesize planck\_power} & {\tt\footnotesize P\_planck} & {\tt\footnotesize P\_planck} & power \\
{\tt\footnotesize planck\_temperature} & {\tt\footnotesize planck\_temperature} & {\tt\footnotesize Theta\_planck} & {\tt\footnotesize Theta\_planck} & temperature \\
{\tt\footnotesize planck\_time} & {\tt\footnotesize planck\_times} & {\tt\footnotesize T\_planck} & {\tt\footnotesize T\_planck} & time \\
{\tt\footnotesize planck\_voltage} & {\tt\footnotesize planck\_voltage} & {\tt\footnotesize V\_planck} & {\tt\footnotesize V\_planck} & potential \\
{\tt\footnotesize point} & {\tt\footnotesize points} & {\tt\footnotesize pt} & {\tt\footnotesize pt} & length \\
{\tt\footnotesize poise} & {\tt\footnotesize poises} & {\tt\footnotesize P} & {\tt\footnotesize P} & viscosity \\
{\tt\footnotesize pole} & {\tt\footnotesize poles} & {\tt\footnotesize pole} & {\tt\footnotesize poles} & length \\
{\tt\footnotesize pound} & {\tt\footnotesize pounds} & {\tt\footnotesize lb} & {\tt\footnotesize lbs} & mass \\
{\tt\footnotesize pound\_force} & {\tt\footnotesize pounds\_force} & {\tt\footnotesize lbf} & {\tt\footnotesize lbf} & force \\
{\tt\footnotesize pound\_per\_square\_inch} & {\tt\footnotesize pounds\_per\_square\_inch} & {\tt\footnotesize psi} & {\tt\footnotesize psi} & pressure \\
{\tt\footnotesize quart\_UK} & {\tt\footnotesize quarts\_UK} & {\tt\footnotesize quart\_UK} & {\tt\footnotesize quarts\_UK} & volume (UK imperial) \\
{\tt\footnotesize quart\_US} & {\tt\footnotesize quarts\_US} & {\tt\footnotesize quart\_US} & {\tt\footnotesize quarts\_US} & volume (US customary) \\
{\tt\footnotesize radian} & {\tt\footnotesize radians} & {\tt\footnotesize rad} & {\tt\footnotesize rad} & angle \\
{\tt\footnotesize rankin} & {\tt\footnotesize rankin} & {\tt\footnotesize R} & {\tt\footnotesize R} & temperature \\
{\tt\footnotesize revolution} & {\tt\footnotesize revolutions} & {\tt\footnotesize rev} & {\tt\footnotesize rev} & angle \\
{\tt\footnotesize rod} & {\tt\footnotesize rods} & {\tt\footnotesize rod} & {\tt\footnotesize rods} & length \\
{\tt\footnotesize roman\_league} & {\tt\footnotesize roman\_leagues} & {\tt\footnotesize roman\_league} & {\tt\footnotesize roman\_leagues} & length \\
{\tt\footnotesize roman\_mile} & {\tt\footnotesize roman\_miles} & {\tt\footnotesize roman\_mile} & {\tt\footnotesize roman\_miles} & length \\
{\tt\footnotesize second} & {\tt\footnotesize seconds} & {\tt\footnotesize s} & {\tt\footnotesize s} & time \\
\multicolumn{5}{|r|}{\footnotesize Also known as the {\tt sec} and the {\tt secs}.} \\
{\tt\footnotesize shekel} & {\tt\footnotesize shekels} & {\tt\footnotesize shekel} & {\tt\footnotesize shekels} & mass \\
{\tt\footnotesize short\_ton} & {\tt\footnotesize short\_tons} & {\tt\footnotesize short\_ton} & {\tt\footnotesize short\_tons} & mass (US customary) \\
{\tt\footnotesize siemens} & {\tt\footnotesize siemens} & {\tt\footnotesize S} & {\tt\footnotesize S} & conductance \\
{\tt\footnotesize sievert} & {\tt\footnotesize sieverts} & {\tt\footnotesize Sv} & {\tt\footnotesize Sv} & radiation\_dose \\
{\tt\footnotesize slug} & {\tt\footnotesize slugs} & {\tt\footnotesize slug} & {\tt\footnotesize slugs} & mass \\
{\tt\footnotesize sol} & {\tt\footnotesize sols} & {\tt\footnotesize sol} & {\tt\footnotesize sols} & time \\
{\tt\footnotesize solar\_luminosity} & {\tt\footnotesize solar\_luminosities} & {\tt\footnotesize Lsun} & {\tt\footnotesize Lsun} & power \\
\multicolumn{5}{|r|}{\footnotesize Also known as the {\tt Lsolar}.} \\
{\tt\footnotesize solar\_mass} & {\tt\footnotesize solar\_masses} & {\tt\footnotesize Msun} & {\tt\footnotesize Msun} & mass \\
\multicolumn{5}{|r|}{\footnotesize Also known as the {\tt Msolar}.} \\
{\tt\footnotesize solar\_radius} & {\tt\footnotesize solar\_radii} & {\tt\footnotesize Rsun} & {\tt\footnotesize Rsun} & length \\
\multicolumn{5}{|r|}{\footnotesize Also known as the {\tt Rsolar}.} \\
{\tt\footnotesize square\_centimetre} & {\tt\footnotesize square\_centimetres} & {\tt\footnotesize sq\_cm} & {\tt\footnotesize sq\_cm} & area \\
{\tt\footnotesize square\_degree} & {\tt\footnotesize square\_degrees} & {\tt\footnotesize sqdeg} & {\tt\footnotesize sqdeg} & solidangle \\
{\tt\footnotesize square\_foot} & {\tt\footnotesize square\_feet} & {\tt\footnotesize sq\_ft} & {\tt\footnotesize sq\_ft} & area \\
{\tt\footnotesize square\_inch} & {\tt\footnotesize square\_inches} & {\tt\footnotesize sq\_in} & {\tt\footnotesize sq\_in} & area \\
{\tt\footnotesize square\_kilometre} & {\tt\footnotesize square\_kilometres} & {\tt\footnotesize sq\_km} & {\tt\footnotesize sq\_km} & area \\
{\tt\footnotesize square\_metre} & {\tt\footnotesize square\_metres} & {\tt\footnotesize sq\_m} & {\tt\footnotesize sq\_m} & area \\
{\tt\footnotesize square\_mile} & {\tt\footnotesize square\_miles} & {\tt\footnotesize sq\_mi} & {\tt\footnotesize sq\_mi} & area \\
{\tt\footnotesize steradian} & {\tt\footnotesize steradians} & {\tt\footnotesize sterad} & {\tt\footnotesize sterad} & solidangle \\
{\tt\footnotesize stone} & {\tt\footnotesize stone} & {\tt\footnotesize stone} & {\tt\footnotesize stone} & mass \\
{\tt\footnotesize tablespoon} & {\tt\footnotesize tablespoons} & {\tt\footnotesize tablespoon} & {\tt\footnotesize tablespoons} & volume \\
{\tt\footnotesize talent} & {\tt\footnotesize talents} & {\tt\footnotesize talent} & {\tt\footnotesize talents} & mass \\
{\tt\footnotesize teaspoon} & {\tt\footnotesize teaspoons} & {\tt\footnotesize teaspoon} & {\tt\footnotesize teaspoons} & volume \\
{\tt\footnotesize tesla} & {\tt\footnotesize tesla} & {\tt\footnotesize T} & {\tt\footnotesize T} & magnetic\_field \\
{\tt\footnotesize therm} & {\tt\footnotesize therms} & {\tt\footnotesize therm} & {\tt\footnotesize therms} & energy \\
{\tt\footnotesize tog} & {\tt\footnotesize togs} & {\tt\footnotesize tog} & {\tt\footnotesize togs} & thermal\_insulation \\
{\tt\footnotesize tonne} & {\tt\footnotesize tonnes} & {\tt\footnotesize t} & {\tt\footnotesize t} & mass \\
\multicolumn{5}{|r|}{\footnotesize Also known as the {\tt metric\_tonne} and the {\tt metric\_tonnes}.} \\
{\tt\footnotesize troy\_ounce} & {\tt\footnotesize troy\_ounces} & {\tt\footnotesize oz\_troy} & {\tt\footnotesize oz\_troy} & mass \\
{\tt\footnotesize volt} & {\tt\footnotesize volts} & {\tt\footnotesize V} & {\tt\footnotesize V} & potential \\
{\tt\footnotesize watt} & {\tt\footnotesize watts} & {\tt\footnotesize W} & {\tt\footnotesize W} & power \\
{\tt\footnotesize weber} & {\tt\footnotesize weber} & {\tt\footnotesize Wb} & {\tt\footnotesize Wb} & magnetic\_flux \\
{\tt\footnotesize week} & {\tt\footnotesize weeks} & {\tt\footnotesize week} & {\tt\footnotesize weeks} & time \\
{\tt\footnotesize yard} & {\tt\footnotesize yards} & {\tt\footnotesize yd} & {\tt\footnotesize yd} & length \\
{\tt\footnotesize year} & {\tt\footnotesize years} & {\tt\footnotesize yr} & {\tt\footnotesize yr} & time \\
\end{longtable}
\end{center}
\end{landscape}

\noindent The following units of {\bf angle} are recognised:\newline
\noindent the {\tt arcminute}, the {\tt arcsecond}, the {\tt degree}, the {\tt radian} and the {\tt revolution}.\vspace{5mm}

\noindent The following units of {\bf area} are recognised:\newline
\noindent the {\tt acre}, the {\tt are}, the {\tt barn}, the {\tt hectare}, the {\tt square\_\-centimetre}, the {\tt square\_\-foot}, the {\tt square\_\-inch}, the {\tt square\_\-kilometre}, the {\tt square\_\-metre} and the {\tt square\_\-mile}.\vspace{5mm}

\noindent The following units of {\bf capacitance} are recognised:\newline
\noindent the {\tt farad}.\vspace{5mm}

\noindent The following units of {\bf catalytic activity} are recognised:\newline
\noindent the {\tt katal}.\vspace{5mm}

\noindent The following units of {\bf charge} are recognised:\newline
\noindent the {\tt coulomb} and the {\tt planck\_\-charge}.\vspace{5mm}

\noindent The following units of {\bf conductance} are recognised:\newline
\noindent the {\tt mho} and the {\tt siemens}.\vspace{5mm}

\noindent The following units of {\bf cost} are recognised:\newline
\noindent the {\tt euro}.\vspace{5mm}

\noindent The following units of {\bf current} are recognised:\newline
\noindent the {\tt ampere} and the {\tt planck\_\-current}.\vspace{5mm}

\noindent The following units of {\bf dimensionlessness} are recognised:\newline
\noindent the {\tt parts\_\-per\_\-billion}, the {\tt parts\_\-per\_\-million} and the {\tt percent}.\vspace{5mm}

\noindent The following units of {\bf energy} are recognised:\newline
\noindent the {\tt British Thermal Unit}, the {\tt billion\_\-electronvolts}, the {\tt calorie}, the {\tt electronvolt}, the {\tt erg}, the {\tt joule}, the {\tt kilowatt\_\-hour}, the {\tt planck\_\-energy} and the {\tt therm}.\vspace{5mm}

\noindent The following units of {\bf flux density} are recognised:\newline
\noindent the {\tt jansky}.\vspace{5mm}

\noindent The following units of {\bf force} are recognised:\newline
\noindent the {\tt dyne}, the {\tt newton}, the {\tt planck\_\-force} and the {\tt pound\_\-force}.\vspace{5mm}

\noindent The following units of {\bf frequency} are recognised:\newline
\noindent the {\tt becquerel} and the {\tt hertz}.\vspace{5mm}

\noindent The following units of {\bf inductance} are recognised:\newline
\noindent the {\tt henry}.\vspace{5mm}

\noindent The following units of {\bf information content} are recognised:\newline
\noindent the {\tt bit}, the {\tt byte}, the {\tt gibibit}, the {\tt gibibyte}, the {\tt kibibit}, the {\tt kibibyte}, the {\tt mebibit} and the {\tt mebibyte}.\vspace{5mm}

\noindent The following units of {\bf length} are recognised:\newline
\noindent the {\tt angstrom}, the {\tt astronomical\_\-unit}, the {\tt cable}, the {\tt centimetre}, the {\tt chain}, the {\tt cubit}, the {\tt decimetre}, the {\tt earth\_\-radius}, the {\tt fathom}, the {\tt foot}, the {\tt furlong}, the {\tt inch}, the {\tt jupiter\_\-radius}, the {\tt light\_\-year}, the {\tt link}, the {\tt lunar\_\-distance}, the {\tt metre}, the {\tt mile}, the {\tt nautical\_\-mile}, the {\tt parsec}, the {\tt perch}, the {\tt pica}, the {\tt planck\_\-length}, the {\tt point}, the {\tt pole}, the {\tt rod}, the {\tt roman\_\-league}, the {\tt roman\_\-mile}, the {\tt solar\_\-radius} and the {\tt yard}.\vspace{5mm}

\noindent The following units of {\bf lens power} are recognised:\newline
\noindent the {\tt dioptre}.\vspace{5mm}

\noindent The following units of {\bf light intensity} are recognised:\newline
\noindent the {\tt candela} and the {\tt candlepower}.\vspace{5mm}

\noindent The following units of {\bf magnetic field} are recognised:\newline
\noindent the {\tt gauss} and the {\tt tesla}.\vspace{5mm}

\noindent The following units of {\bf magnetic flux} are recognised:\newline
\noindent the {\tt maxwell} and the {\tt weber}.\vspace{5mm}

\noindent The following units of {\bf mass} are recognised:\newline
\noindent the {\tt carat}, the {\tt drachm}, the {\tt earth\_\-mass}, the {\tt grain}, the {\tt gram}, the {\tt gramme}, the {\tt hundredweight\_\-UK}, the {\tt hundredweight\_\-US}, the {\tt jupiter\_\-mass}, the {\tt kilogram}, the {\tt long\_\-ton}, the {\tt mina}, the {\tt ounce}, the {\tt planck\_\-mass}, the {\tt pound}, the {\tt shekel}, the {\tt short\_\-ton}, the {\tt slug}, the {\tt solar\_\-mass}, the {\tt stone}, the {\tt talent}, the {\tt tonne} and the {\tt troy\_\-ounce}.\vspace{5mm}

\noindent The following units of {\bf moles} are recognised:\newline
\noindent the {\tt mole}.\vspace{5mm}

\noindent The following units of {\bf momentum} are recognised:\newline
\noindent the {\tt planck\_\-momentum}.\vspace{5mm}

\noindent The following units of {\bf potential} are recognised:\newline
\noindent the {\tt planck\_\-voltage} and the {\tt volt}.\vspace{5mm}

\noindent The following units of {\bf power} are recognised:\newline
\noindent the {\tt horsepower}, the {\tt lumen}, the {\tt lux}, the {\tt planck\_\-power}, the {\tt solar\_\-luminosity} and the {\tt watt}.\vspace{5mm}

\noindent The following units of {\bf pressure} are recognised:\newline
\noindent the {\tt atmosphere}, the {\tt bar}, the {\tt barye}, the {\tt inch\_\-of\_\-mercury}, the {\tt inch\_\-of\_\-water}, the {\tt pascal} and the {\tt pound\_\-per\_\-square\_\-inch}.\vspace{5mm}

\noindent The following units of {\bf radiation dose} are recognised:\newline
\noindent the {\tt gray} and the {\tt sievert}.\vspace{5mm}

\noindent The following units of {\bf resistance} are recognised:\newline
\noindent the {\tt ohm} and the {\tt planck\_\-impedence}.\vspace{5mm}

\noindent The following units of {\bf solidangle} are recognised:\newline
\noindent the {\tt square\_\-degree} and the {\tt steradian}.\vspace{5mm}

\noindent The following units of {\bf temperature} are recognised:\newline
\noindent the {\tt degree\_\-celsius}, the {\tt degree\_\-fahrenheit}, the {\tt kelvin}, the {\tt planck\_\-temperature} and the {\tt rankin}.\vspace{5mm}

\noindent The following units of {\bf thermal insulation} are recognised:\newline
\noindent the {\tt clo} and the {\tt tog}.\vspace{5mm}

\noindent The following units of {\bf time} are recognised:\newline
\noindent the {\tt day}, the {\tt hour}, the {\tt minute}, the {\tt planck\_\-time}, the {\tt second}, the {\tt sol}, the {\tt week} and the {\tt year}.\vspace{5mm}

\noindent The following units of {\bf velocity} are recognised:\newline
\noindent the {\tt knot} and the {\tt mile\_\-per\_\-hour}.\vspace{5mm}

\noindent The following units of {\bf viscosity} are recognised:\newline
\noindent the {\tt poise}.\vspace{5mm}

\noindent The following units of {\bf volume} are recognised:\newline
\noindent the {\tt bath}, the {\tt bushel\_\-UK}, the {\tt bushel\_\-US}, the {\tt cubic\_\-centimetre}, the {\tt cubic\_\-foot}, the {\tt cubic\_\-inch}, the {\tt cubic\_\-metre}, the {\tt cup\_\-US}, the {\tt firkin\_\-UK\_\-ale}, the {\tt firkin\_\-wine}, the {\tt fluid\_\-ounce\_\-UK}, the {\tt fluid\_\-ounce\_\-US}, the {\tt gallon\_\-UK}, the {\tt gallon\_\-US}, the {\tt homer}, the {\tt kilderkin\_\-UK\_\-ale}, the {\tt litre}, the {\tt pint\_\-UK}, the {\tt pint\_\-US}, the {\tt quart\_\-UK}, the {\tt quart\_\-US}, the {\tt tablespoon} and the {\tt teaspoon}.\vspace{5mm}

\noindent The following units of {\bf wavenumber} are recognised:\newline
\noindent the {\tt kayser}.\vspace{5mm}

% PAPERSIZES.TEX
%
% The documentation in this file is part of PyXPlot
% <http://www.pyxplot.org.uk>
%
% Copyright (C) 2006-2012 Dominic Ford <coders@pyxplot.org.uk>
%               2008-2012 Ross Church
%
% $Id$
%
% PyXPlot is free software; you can redistribute it and/or modify it under the
% terms of the GNU General Public License as published by the Free Software
% Foundation; either version 2 of the License, or (at your option) any later
% version.
%
% You should have received a copy of the GNU General Public License along with
% PyXPlot; if not, write to the Free Software Foundation, Inc., 51 Franklin
% Street, Fifth Floor, Boston, MA  02110-1301, USA

% ----------------------------------------------------------------------------

% LaTeX source for the PyXPlot Users' Guide

\chapter{List of Paper Sizes}
\label{ch:paper_sizes}

The following table lists all of the named paper sizes which PyXPlot recognises:

{\twocolumn
\begin{supertabular}{|lll|}
\shrinkheight{5mm}
\hline
\tabletail{\hline}
\tablehead{\hline
{\bf Name} & {\bf $h$/mm} & {\bf $w$/mm} \\ \hline}
{\bf Name} & {\bf $h$/mm} & {\bf $w$/mm} \\ \hline
                        2a0 &   1681 &   1189 \\
                        4a0 &   2378 &   1681 \\
                         a0 &   1189 &    840 \\
                         a1 &    840 &    594 \\
                        a10 &     37 &     26 \\
                         a2 &    594 &    420 \\
                         a3 &    420 &    297 \\
                         a4 &    297 &    210 \\
                         a5 &    210 &    148 \\
                         a6 &    148 &    105 \\
                         a7 &    105 &     74 \\
                         a8 &     74 &     52 \\
                         a9 &     52 &     37 \\
                         b0 &   1414 &    999 \\
                         b1 &    999 &    707 \\
                        b10 &     44 &     31 \\
                         b2 &    707 &    499 \\
                         b3 &    499 &    353 \\
                         b4 &    353 &    249 \\
                         b5 &    249 &    176 \\
                         b6 &    176 &    124 \\
                         b7 &    124 &     88 \\
                         b8 &     88 &     62 \\
                         b9 &     62 &     44 \\
                         c0 &   1296 &    917 \\
                         c1 &    917 &    648 \\
                        c10 &     40 &     28 \\
                         c2 &    648 &    458 \\
                         c3 &    458 &    324 \\
                         c4 &    324 &    229 \\
                         c5 &    229 &    162 \\
                         c6 &    162 &    114 \\
                         c7 &    114 &     81 \\
                         c8 &     81 &     57 \\
                         c9 &     57 &     40 \\
                      crown &    508 &    381 \\
                       demy &    572 &    445 \\
               double\_demy &    889 &    597\\
                   elephant &    711 &    584 \\
               envelope\_dl &    110 &    220\\
                  executive &    267 &    184 \\
                   foolscap &    330 &    203\\
         government\_       &    267 &    203\\
\hfill               letter &        &       \\
international\_             &     85 &     53\\
\hfill         businesscard &        &       \\
               japanese\_b0 &   1435 &   1015\\
               japanese\_b1 &   1015 &    717\\
              japanese\_b10 &     44 &     31\\
               japanese\_b2 &    717 &    507\\
               japanese\_b3 &    507 &    358\\
               japanese\_b4 &    358 &    253\\
               japanese\_b5 &    253 &    179\\
               japanese\_b6 &    179 &    126\\
               japanese\_b7 &    126 &     89\\
               japanese\_b8 &     89 &     63\\
               japanese\_b9 &     63 &     44\\ \shrinkheight{5mm}
            japanese\_kiku4 &    306 &    227\\
            japanese\_kiku5 &    227 &    151\\
         japanese\_shiroku4 &    379 &    264\\
         japanese\_shiroku5 &    262 &    189\\
         japanese\_shiroku6 &    188 &    127\\
                large\_post &    533 &    419\\
                     ledger &    432 &    279 \\
                      legal &    356 &    216 \\
                     letter &    279 &    216 \\
                     medium &    584 &    457 \\
                    monarch &    267 &    184 \\
                       post &    489 &    394 \\
                 quad\_demy &   1143 &   889 \\
                     quarto &    254 &    203 \\
                      royal &    635 &    508 \\
                  statement &    216 &    140 \\
                swedish\_d0 &   1542 &   1090 \\
                swedish\_d1 &   1090 &    771 \\
               swedish\_d10 &     48 &     34 \\
                swedish\_d2 &    771 &    545 \\
                swedish\_d3 &    545 &    385 \\
                swedish\_d4 &    385 &    272 \\
                swedish\_d5 &    272 &    192 \\
                swedish\_d6 &    192 &    136 \\
                swedish\_d7 &    136 &     96 \\
                swedish\_d8 &     96 &     68 \\
                swedish\_d9 &     68 &     48 \\
                swedish\_e0 &   1241 &    878 \\
                swedish\_e1 &    878 &    620 \\
               swedish\_e10 &     38 &     27 \\
                swedish\_e2 &    620 &    439 \\
                swedish\_e3 &    439 &    310 \\
                swedish\_e4 &    310 &    219 \\
                swedish\_e5 &    219 &    155 \\
                swedish\_e6 &    155 &    109 \\
                swedish\_e7 &    109 &     77 \\
                swedish\_e8 &     77 &     54 \\
                swedish\_e9 &     54 &     38 \\
                swedish\_f0 &   1476 &   1044 \\
                swedish\_f1 &   1044 &    738 \\
               swedish\_f10 &     46 &     32 \\
                swedish\_f2 &    738 &    522 \\
                swedish\_f3 &    522 &    369 \\ \shrinkheight{5mm}
                swedish\_f4 &    369 &    261 \\
                swedish\_f5 &    261 &    184 \\
                swedish\_f6 &    184 &    130 \\
                swedish\_f7 &    130 &     92 \\
                swedish\_f8 &     92 &     65 \\
                swedish\_f9 &     65 &     46 \\
                swedish\_g0 &   1354 &    957 \\
                swedish\_g1 &    957 &    677 \\
               swedish\_g10 &     42 &     29 \\
                swedish\_g2 &    677 &    478 \\
                swedish\_g3 &    478 &    338 \\
                swedish\_g4 &    338 &    239 \\
                swedish\_g5 &    239 &    169 \\
                swedish\_g6 &    169 &    119 \\
                swedish\_g7 &    119 &     84 \\
                swedish\_g8 &     84 &     59 \\
                swedish\_g9 &     59 &     42 \\
                swedish\_h0 &   1610 &   1138 \\
                swedish\_h1 &   1138 &    805 \\
               swedish\_h10 &     50 &     35 \\
                swedish\_h2 &    805 &    569 \\
                swedish\_h3 &    569 &    402 \\
                swedish\_h4 &    402 &    284 \\
                swedish\_h5 &    284 &    201 \\
                swedish\_h6 &    201 &    142 \\
                swedish\_h7 &    142 &    100 \\
                swedish\_h8 &    100 &     71 \\
                swedish\_h9 &     71 &     50 \\
                    tabloid &    432 &    279 \\
           us\_businesscard &     89 &     51 \\
\end{supertabular}
}\onecolumn


% colors.tex
%
% The documentation in this file is part of Pyxplot
% <http://www.pyxplot.org.uk>
%
% Copyright (C) 2006-2012 Dominic Ford <coders@pyxplot.org.uk>
%               2008-2012 Ross Church
%
% $Id$
%
% Pyxplot is free software; you can redistribute it and/or modify it under the
% terms of the GNU General Public License as published by the Free Software
% Foundation; either version 2 of the License, or (at your option) any later
% version.
%
% You should have received a copy of the GNU General Public License along with
% Pyxplot; if not, write to the Free Software Foundation, Inc., 51 Franklin
% Street, Fifth Floor, Boston, MA  02110-1301, USA

% ----------------------------------------------------------------------------

% LaTeX source for the Pyxplot Users' Guide

\chapter{Color tables}
\label{ch:color_charts}

\index{colors!charts} Figures~\ref{fig:color_table1}, \ref{fig:color_table2}
and \ref{fig:color_table3} show the default named colors which Pyxplot recognises.
In addition to using these colors in statements such as

\begin{verbatim}
plot 'data' with color red
\end{verbatim}

\noindent it is also possible to make custom colors using the {\tt rgb(r,g,b)},
{\tt cmyk(c,m,y,k)}, {\tt gray(g)} and {\tt hsb(h,s,b)} functions, whose inputs
should be in the range 0--1. For example:

\begin{verbatim}
plot 'data' with color rgb(0.8,0.8,0.2)

myColor = cmyk(0.2,0.8,0.8,0.1)
plot 'data' with color myColor
\end{verbatim}

These figures also exclude the 100~shades of gray which Pyxplot recognises,
which are named from {\tt gray00} (black) to {\tt gray99} (almost white).
These shades of gray may also be spelt in the UK English form {\tt grey??}.
\index{colors!shades of gray}

\begin{figure}
\begin{center}
\includegraphics[width=\textwidth]{figures/pyx_colors2}
\end{center}
\caption[A list of the named colors which Pyxplot recognises, sorted alphabetically]
{A list of the named colors which Pyxplot recognises, sorted alphabetically. The numerous shades of gray which it recognises are not shown.}
\label{fig:color_table1}
\end{figure}

\begin{figure}
\begin{center}
\includegraphics[width=\textwidth]{figures/pyx_colors3}
\end{center}
\caption[A list of the named colors which Pyxplot recognises, sorted by hue]
{A list of the named colors which Pyxplot recognises, sorted by hue. The numerous shades of gray which it recognises are not shown.}
\label{fig:color_table2}
\end{figure}

\begin{figure}
\begin{center}
\includegraphics[width=\textwidth]{figures/pyx_colors}
\end{center}
\caption[The named colors which Pyxplot recognises, arranged in HSB color space]
{The named colors which Pyxplot recognises, arranged in HSB color space, with the brightness axis orientated into the page. Some colors are not shown as they lie too close to others.}
\label{fig:color_table3}
\end{figure}


% LINESTYLES.TEX
%
% The documentation in this file is part of PyXPlot
% <http://www.pyxplot.org.uk>
%
% Copyright (C) 2006-2011 Dominic Ford <coders@pyxplot.org.uk>
%               2008-2011 Ross Church
%
% $Id$
%
% PyXPlot is free software; you can redistribute it and/or modify it under the
% terms of the GNU General Public License as published by the Free Software
% Foundation; either version 2 of the License, or (at your option) any later
% version.
%
% You should have received a copy of the GNU General Public License along with
% PyXPlot; if not, write to the Free Software Foundation, Inc., 51 Franklin
% Street, Fifth Floor, Boston, MA  02110-1301, USA

% ----------------------------------------------------------------------------

% LaTeX source for the PyXPlot Users' Guide

\chapter{Line and Point Types}
\label{ch:linetypes_table}

The tables in this chapter show the appearances of each of the numbered line,
point and star types available in the {\tt lines}, {\tt points} and {\tt stars}
plot styles respectively.

\begin{table}
\begin{center}
\includegraphics{examples/eps/ex_linestyles2}
\end{center}
\caption{The numbered line types available in the {\tt lines} plot style.}
\end{table}

\begin{table}
\begin{center}
\includegraphics{examples/eps/ex_linestyles}
\end{center}
\caption{The numbered point types available in the {\tt points} plot style.}
\end{table}

\begin{table}
\begin{center}
\includegraphics[width=\textwidth]{examples/eps/ex_linestyles3}
\end{center}
\caption{The numbered star types available in the {\tt stars} plot style.}
\end{table}


% configuration.newst
%
% The documentation in this file is part of PyXPlot
% <http://www.pyxplot.org.uk>
%
% Copyright (C) 2006-2012 Dominic Ford <coders@pyxplot.org.uk>
%               2008-2012 Ross Church
%
% $Id$
%
% PyXPlot is free software; you can redistribute it and/or modify it under the
% terms of the GNU General Public License as published by the Free Software
% Foundation; either version 2 of the License, or (at your option) any later
% version.
%
% You should have received a copy of the GNU General Public License along with
% PyXPlot; if not, write to the Free Software Foundation, Inc., 51 Franklin
% Street, Fifth Floor, Boston, MA  02110-1301, USA

% ----------------------------------------------------------------------------

% LaTeX source for the PyXPlot Users' Guide

\chapter{Configuring PyXPlot}
\label{ch:configuration}

\renewcommand{\arraystretch}{1.80}

In Parts~I and~II, we encountered numerous configuration options within PyXPlot
which can controlled using the \indcmdt{set}. There are times, however, when
many plots are wanted in a homogeneous style, or when a single plot is
repeatedly generated, when it is desirable to change the default set of
configuration options with which PyXPlot starts up, in order to avoid having to
repeated enter a large number of {\tt set} commands. In this chapter, we
describe the use of configuration files to program PyXPlot's default state.

\section{Configuration files}

Configuration files for PyXPlot have the filename {\tt .pyxplotrc}, and may be
placed either in a user's home directory, in which case they globally affect
all of that user's PyXPlot sessions, or in particular directories, in which
case they only affect PyXPlot sessions which are instantiated with that
particular directory as the current working directory.  When configuration
files are present in both locations, both are read; settings found in the {\tt
.pyxplotrc} file in the current working directory take precedence over those
found in the user's home directory. Configuration files are read only once,
upon startup, and subsequent changes to the configuration files do not affect
copies of PyXPlot which are already running.

Changes to settings made in configuration files affect not only the values that
these settings have upon startup; they also changes the values to which the
\indcmdt{unset} returns settings. Thus, whilst the command
\begin{verbatim}
unset multiplot
\end{verbatim}
ordinarily turns off multiplot mode, it may turn it on if a configuration file
contains the line
\begin{verbatim}
multiPlot=on
\end{verbatim}
When colored terminal output is enabled, the color-coding of the
\indcmdt{show} also reflects the current default configuration: settings which
match their default values are shown in green\footnote{This color can be
changed using the {\tt color\_rep} setting in a configuration file.} whilst
those settings which have been changed from their default values are shown in
amber\footnote{This color can be changed using the {\tt color\_wrn} setting
in a configuration file.}.

Configuration files should take the form of a series of sections, each headed
by a section heading enclosed in square brackets. Each section heading should
be followed by a series of settings, which often take the form of
\begin{verbatim}
setting_name = value
\end{verbatim}
In {\it most} cases, neither the setting name nor the value are case sensitive.

The following sections are used, although they do not all need to be present in
any given file, and they may appear in any order:

\begin{itemize}
\item {\tt colors} -- contains a single setting {\tt palette}, which should be
set to a comma-separated list of colors which should make up the palette used
to plot datasets. The first will be called color~1 in PyXPlot, the second
color~2, etc. A list of recognised color names is given in
Section~\ref{sec:color_names}.
\item {\tt filters} -- can be used to define input filters which should be used
for certain file types (see Section~\ref{sec:filters}).
\item {\tt functions} -- contains user-defined function definitions which
become predefined in PyXPlot's mathematical environment, for example
\begin{verbatim}
sinc(x) = sin(x)/(x)
\end{verbatim}
\item {\tt latex} -- contains a single setting {\tt preamble}, which is
prefixed to the beginning of all \LaTeX\ text items, before the {\tt
\textbackslash begin\{document\}} statement. It can be used to define custom
\LaTeX\ macros or to include packages using the {\tt \textbackslash
includepackage\{\}} command.  The preamble can be also changed using the
\indcmdt{set preamble}.
\item {\tt script} -- can contain a list of \indcmdt{set}s, using the same
syntax which would be used to enter them at a PyXPlot command prompt. This
section provides an alternative and more general way of controlling the
settings which can be changed in the {\tt settings} section. Note that this
section may only contain instances of the \indcmdt{set}; other PyXPlot
commands may not be used. The \indcmdt{set}'s {\tt item} modifier may not be
used.
\item {\tt settings} -- contains settings similar to those which can be set
with the \indcmdt{set}. A complete list is given in
Section~\ref{sec:configfile_settings} below.
\item {\tt styling} -- contains settings which control various detailed aspects
of the graphical output which PyXPlot produces. These settings cannot be
accessed by any other means.
\item {\tt terminal} -- contains settings for altering the behaviour and
appearance of PyXPlot's interactive terminal. These cannot be changed with the
\indcmdt{set}, and can only be controlled via configuration files. A complete
list of the available settings is given in
Section~\ref{sec:configfile_terminal}.
\item {\tt units} -- can be used to define new physical units for use in
PyXPlot's mathematical environment.
\item {\tt variables} -- contains variable definitions, in the format
\begin{verbatim}
variable = value
\end{verbatim}
Any variables defined in this section will be pre-defined in PyXPlot's
mathematical environment upon startup.

\end{itemize}

\section{An example configuration file}
\index{configuration files}

\noindent The following configuration file represents PyXPlot's default
configuration, and provides a useful index to all of the settings which are
available. In subsequent sections, we describe the effect of each setting in
detail.

%Projection = Flat
\begin{verbatim}
[settings]
aspect = auto
axesColor = black
axisUnitStyle = ratio
backup = off
bar = 1.0
binOrigin = auto
binWidth = auto
boxFrom = auto
boxWidth = auto
calendarIn = British
calendarOut = British
clip = off
colKey = on
colKeyPos = Right
color = on
contours = 12
c1Range_log = false
c1Range_max = 0
c1Range_max_Auto = true
c1Range_min = 0
c1Range_min_Auto = true
c1Range_renorm = true
c1Range_reverse = false
c2Range_log = false
c2Range_max = 0
c2Range_max_auto = true
c2Range_min = 0
c2Range_min_auto = true
c2Range_renorm = true
c2Range_reverse = false
c3Range_log = false
c3Range_max = 0
c3Range_max_auto = true
c3Range_min = 0
c3Range_min_auto = true
c3Range_renorm = true
c3Range_reverse = false
c4Range_log = false
c4Range_max = 0
c4Range_max_auto = true
c4Range_min = 0
c4Range_min_auto = true
c4Range_renorm = true
c4Range_reverse = false
dataStyle = Points
display = on
dpi = 300
fontSize = 1
funcStyle = Lines
grid = off
gridAxisX = 1
gridAxisY = 1
gridAxisZ = 1
gridMajColor = grey70
gridMinColor = grey85
key = on
keyColumns = 0
keyPos = top right
key_Xoff = 0.0
key_Yoff = 0.0
landscape = off
lineWidth = 1.0
multiPlot = off
numComplex = off
numDisplay = natural
numErr = on
numSF = 8
originX = 0.0
originY = 0.0
output =
paperHeight = 297
paperName = a4
paperWidth = 210
pointLineWidth = 1.0
pointSize = 1.0
samples = 250
samples_method = nearestNeighbor
samples_x_auto = false
samples_x = 40
samples_y_auto = false
samples_y = 40
termAntiAlias = on
termEnlarge = off
termInvert = off
termTransparent = off
termType = X11_singleWindow
textColor = black
textHAlign = left
textVAlign = bottom
title =
title_Xoff = 0.0
title_Yoff = 0.0
uRange_log = false
uRange_max = 1.0
uRange_min = 0.0
vRange_log = false
vRange_max = 1.0
vRange_min = 0.0
tRange_log = false
tRange_max = 1.0
tRange_min = 0.0
unitAbbrev = on
unitAngleDimless = on
unitPrefix = on
unitScheme = si
width = 8.0
view_xy = 60
view_yz = 30
zAspect = auto

[terminal]
color = on
color_err = red
color_rep = green
color_wrn = amber
splash = on

[styling]
arrow_headAngle = 45
arrow_headSize = 1.0
arrow_headBackIndent = 0.2
axes_lineWidth = 1.0
axes_majTickLen = 1.0
axes_minTickLen = 1.0
axes_separation = 1.0
axes_textGap = 1.0
colorScale_margin = 1.0
colorScale_width = 1.0
grid_majLineWidth = 1.0
grid_minLineWidth = 0.5
baseline_lineWidth = 1.0
baseline_pointSize = 1.0

[variables]
pi = 3.14159265358979

[colors]
palette = black, red, blue, magenta, cyan, brown, salmon, gray,
green, navyBlue, periwinkle, pineGreen, seaGreen, greenYellow,
orange, carnationPink, plum

[latex]
preamble =
\end{verbatim}

\section{Setting definitions}

We now provide a more detailed description of the effect of each of the
settings which can be found in configuration files, including where appropriate
a list of possible values for each. Settings are arranged by section.

\subsection{The {\tt filters} section}

The {\tt filters} section allows input filters to be specified for \datafile s
whose filenames match particular wildcards. Each line should be in the format
\begin{verbatim}
wildcard = filter_binary
\end{verbatim}
For example, the line
\begin{verbatim}
*.gz = /usr/bin/gzip
\end{verbatim}
would set the application {\tt /usr/bin/gzip} to be used as an input filter for
all \datafile s with a {\tt .gz} suffix. For more information about input
filters, see Section~\ref{sec:filters}.

\subsection{The {\tt settings} section}
\label{sec:configfile_settings}

The {\tt settings} section can contain any of the following settings in any order:

\begin{longtable}{p{3.4cm}p{9cm}}
{\tt aspect} & {\bf Possible values:} {\tt auto}, or any floating-point number.

               {\bf Analogous set command:} \indcmdts{set size ratio}

               Sets the $y/x$ aspect ratio of plots.
               \\
{\tt autoAspect} & {\bf Possible values:} {\tt on}, {\tt off}.

               {\bf Analogous set command:} {\tt set size ratio}

               Sets whether plots have the automatic $y/x$ aspect ratio, which is the golden ratio. If {\tt on}, then the {\tt aspect} setting is ignored. Deprecated: new scripts should use {\tt aspect=auto} instead.
               \\
{\tt autoZAspect} & {\bf Possible values:} {\tt on}, {\tt off}.

               {\bf Analogous set command:} {\tt set size zratio}

               Sets whether 3d plots have the automatic $z/x$ aspect ratio, which is the golden ratio. If {\tt on}, then the {\tt zAspect} setting is ignored. Deprecated: new scripts should use {\tt zAspect=auto} instead.
               \\
{\tt axesColor} & {\bf Possible values:} Any recognised color.

               {\bf Analogous set command:} \indcmdts{set axescolor}

               Sets the color of axis lines and ticks.
               \\
{\tt axisUnitStyle} & {\bf Possible values:} {\tt Bracketed}, {\tt Ratio}, {\tt SquareBracketed}

               {\bf Analogous set command:} \indcmdts{set axisunitstyle}

               Sets the style in which the physical units of quantities plotted against axes are appended to axis labels.
               \\
{\tt backup} & {\bf Possible values:} {\tt on}, {\tt off}.

               {\bf Analogous set command:} \indcmdts{set backup}

               When this switch is set to {\tt on}, and plot output is being directed to file, attempts to write output over existing files cause a copy of the existing file to be preserved, with a tilde after its old filename (see Section~\ref{sec:file_backup}).
               \\
{\tt bar}     & {\bf Possible values:}  Any floating-point number.

               {\bf Analogous set command:} \indcmdts{set bar}

               Sets the horizontal length of the lines drawn at the end of errorbars, in units of their default length.
               \\
{\tt binOrigin} & {\bf Possible values:} {\tt auto}, or any floating-point number.

               {\bf Analogous set command:} \indcmdts{set binorigin}

               Sets the point along the abscissa axis from which the bins used by the \indcmdt{histogram} originate.
               \\
{\tt binWidth} & {\bf Possible values:} {\tt auto}, or any floating-point number.

               {\bf Analogous set command:} \indcmdts{set binwidth}

               Sets the widths of the bins used by the \indcmdt{histogram}.
               \\
{\tt boxFrom} & {\bf Possible values:} {\tt auto}, or any floating-point number.

               {\bf Analogous set command:} \indcmdts{set boxfrom}

               Sets the horizontal point from which bars on bar charts appear to emanate.
               \\
{\tt boxWidth} & {\bf Possible values:} {\tt auto}, or any floating-point number.

               {\bf Analogous set command:} \indcmdts{set boxwidth}

               Sets the default width of boxes on barcharts. If negative, then the boxes have automatically selected widths, so that the interfaces between bars occur at the horizontal midpoints between the specified datapoints.
               \\
{\tt calendarIn} & {\bf Possible values:} {\tt British}, {\tt French}, {\tt Greek}, {\tt Gregorian}, {\tt Hebrew}, {\tt Islamic}, {\tt Julian}, {\tt Papal}, {\tt Russian}.

               {\bf Analogous set command:} \indcmdts{set calendar}

               Sets the default calendar for the input of dates from day, month and year representation into Julian Date representation. See Section~\ref{sec:time_series} for more details.
               \\
{\tt calendarOut} & {\bf Possible values:} {\tt British}, {\tt French}, {\tt Greek}, {\tt Gregorian}, {\tt Hebrew}, {\tt Islamic}, {\tt Julian}, {\tt Papal}, {\tt Russian}.

               {\bf Analogous set command:} \indcmdts{set calendar}

               Sets the default calendar for the output of dates from Julian Date representation to day, month and year representation. See Section~\ref{sec:time_series} for more details.
               \\
{\tt clip} & {\bf Possible values:} {\tt on}, {\tt off}.

               {\bf Analogous set command:} \indcmdts{set clip}

               Sets whether datapoints close to the edges of graphs should be clipped at the edges ({\tt on}) or allowed to overrun the axes ({\tt off}).
               \\
{\tt colKey} & {\bf Possible values:} {\tt on}, {\tt off}.

               {\bf Analogous set command:} \indcmdts{set colkey}

               Sets whether \indpst{colormap} plots have a scale along one side relating color to ordinate value.
               \\
{\tt colKeyPos} & {\bf Possible values:} {\tt top}, {\tt bottom}, {\tt left}, {\tt right}.

               {\bf Analogous set command:} \indcmdts{set colkey}

               Sets the side of the plot along which the color legend should appear on \indpst{colormap} plots.
               \\
{\tt color} & {\bf Possible values:} {\tt on}, {\tt off}.

               {\bf Analogous set command:} \indcmdts{set terminal}

               Sets whether output should be color ({\tt on}) or monochrome ({\tt off}).
               \\
{\tt contour} & {\bf Possible values:} Any integer.

               {\bf Analogous set command:} \indcmdts{set contour}

               Sets the number of contours which are drawn in the \indpst{contourmap} plot style.
               \\
{\tt c?Range\_log} & {\bf Possible values:} {\tt true}, {\tt false}.

               {\bf Analogous set command:} \indcmdts{set logscale c}

               When the variables {\tt c1}--{\tt c4} are set to renormalise in the {\tt c?Range\_renorm} setting, this setting determines whether color maps are drawn with logarithmic or linear color scales. The {\tt ?} wildcard should be replaced with an integer in the range 1--4 to alter the renormalisation of the variables {\tt c1} through {\tt c4} respectively in the expressions supplied to the {\tt colmap} setting. In the case of {\tt c1}, this setting also determines whether contours demark linear or logarithmic intervals on contour maps.\indps{contourmap}\indps{colormap}
               \\
{\tt c?Range\_max} & {\bf Possible values:} Any floating-point number.

               {\bf Analogous set command:} \indcmdts{set crange}

               When the variables {\tt c1}--{\tt c4} are set to renormalise in the {\tt c?Range\_renorm} setting, this setting determines the upper limit of the range of values demarked by differing colors on color maps. The {\tt ?} wildcard should be replaced with an integer in the range 1--4 to alter the renormalisation of the variables {\tt c1} through {\tt c4} respectively in the expressions supplied to the {\tt colmap} setting. In the case of {\tt c1}, this setting also determines the range of ordinate values for which contours are drawn on contour maps.
               \\
{\tt c?Range\_max\_auto} & {\bf Possible values:} {\tt true}, {\tt false}.

               {\bf Analogous set command:} \indcmdts{set crange}

               When the variables {\tt c1}--{\tt c4} are set to renormalise in the {\tt c?Range\_renorm} setting, this setting determines whether the upper limit of the range of values demarked by differing colors on color maps should autoscale to fit the data, or be a fixed value as specified in the {\tt C?Range\_max} setting. The {\tt ?} wildcard should be replaced with an integer in the range 1--4 to alter the renormalisation of the variables {\tt c1} through {\tt c4} respectively. In the case of {\tt c1}, this setting also affects the range of ordinate values for which contours are drawn on contour maps.
               \\
{\tt c?Range\_min} & {\bf Possible values:} Any floating-point number.

               {\bf Analogous set command:} \indcmdts{set crange}

               When the variables {\tt c1}--{\tt c4} are set to renormalise in the {\tt c?Range\_renorm} setting, this setting determines the lower limit of the range of values demarked by differing colors on color maps. The {\tt ?} wildcard should be replaced with an integer in the range 1--4 to alter the renormalisation of the variables {\tt c1} through {\tt c4} respectively in the expressions supplied to the {\tt colmap} setting. In the case of {\tt c1}, this setting also determines the range of ordinate values for which contours are drawn on contour maps.
               \\
{\tt c?Range\_min\_auto} & {\bf Possible values:} {\tt true}, {\tt false}.

               {\bf Analogous set command:} \indcmdts{set crange}

               When the variables {\tt c1}--{\tt c4} are set to renormalise in the {\tt c?Range\_renorm} setting, this setting determines whether the lower limit of the range of values demarked by differing colors on color maps should autoscale to fit the data, or be a fixed value as specified in the {\tt C?Range\_min} setting. The {\tt ?} wildcard should be replaced with an integer in the range 1--4 to alter the renormalisation of the variables {\tt c1} through {\tt c4} respectively. In the case of {\tt c1}, this setting also affects the range of ordinate values for which contours are drawn on contour maps.
               \\
{\tt c?Range\_renorm} & {\bf Possible values:} {\tt true}, {\tt false}.

               {\bf Analogous set command:} \indcmdts{set crange}

               Sets whether the variables {\tt c1}--{\tt c4}, used in the construction of color maps, should be renormalised into the range 0--1 before being passed to the expressions supplied to the {\tt set colmap} command, or whether they should contain the exact data values supplied in the 3rd--6th columns of data to the {\tt colormap} plot style. The {\tt ?} wildcard should be replaced with an integer in the range 1--4 to alter the renormalisation of the variables {\tt c1} through {\tt c4} respectively.
               \\
{\tt c?Range\_reverse} & {\bf Possible values:} {\tt true}, {\tt false}.

               {\bf Analogous set command:} \indcmdts{set crange}

               When the variables {\tt c1}--{\tt c4} are set to renormalise in the {\tt c?Range\_renorm} setting, this setting determines whether the renormalisation into the range 0--1 is inverted such that the maximum value maps to zero and the minimum value maps to one. The {\tt ?} wildcard should be replaced with an integer in the range 1--4 to alter the renormalisation of the variables {\tt c1} through {\tt c4} respectively.
               \\
{\tt dataStyle} & {\bf Possible values:} Any plot style.

               {\bf Analogous set command:} \indcmdts{set data style}

               Sets the plot style used by default when plotting \datafile s.
               \\
{\tt display} & {\bf Possible values:} {\tt on}, {\tt off}.

               {\bf Analogous set command:} \indcmdts{set display}

               When set to {\tt on}, no output is produced until the \indcmdt{set display} is issued. This is useful for speeding up scripts which produce large multiplots; see Section~\ref{sec:set_display} for more details.
               \\
{\tt dpi} & {\bf Possible values:} Any floating-point number.

               {\bf Analogous set command:} \indcmdts{set terminal dpi}

               Sets the sampling quality used, in dots per inch, when output is sent to a bitmapped terminal (the bmp, jpeg, gif, png and tif terminals).
               \\
{\tt fontSize} & {\bf Possible values:} Any floating-point number.

               {\bf Analogous set command:} \indcmdts{set fontsize}

               Sets the fontsize of text, where $1.0$ represents 10-point text, and other values differ multiplicatively.
               \\
{\tt funcStyle} & {\bf Possible values:} Any plot style.

               {\bf Analogous set command:} \indcmdts{set function style}

               Sets the plot style used by default when plotting functions.
               \\
{\tt grid} & {\bf Possible values:} {\tt on}, {\tt off}.

               {\bf Analogous set command:} \indcmdts{set grid}

               Sets whether a grid should be displayed on plots.
               \\
{\tt gridAxisX} & {\bf Possible values:} Any integer.

               {\bf Analogous set command:} None

               Sets the default horizontal axis to which gridlines should attach, if the {\tt set grid} command is called without specifying which axes to use.
               \\
{\tt gridAxisY} & {\bf Possible values:} Any integer.

               {\bf Analogous set command:} None

               Sets the default vertical axis to which gridlines should attach, if the {\tt set grid} command is called without specifying which axes to use.
               \\
{\tt gridAxisZ} & {\bf Possible values:} Any integer.

               {\bf Analogous set command:} None

               Sets the default $z$-axis to which gridlines should attach, if the {\tt set grid} command is called without specifying which axes to use.
               \\
{\tt gridMajColor} & {\bf Possible values:} Any recognised color.

               {\bf Analogous set command:} \indcmdts{set gridmajcolor}

               Sets the color of major grid lines.
               \\
{\tt gridMinColor} & {\bf Possible values:} Any recognised color.

               {\bf Analogous set command:} \indcmdts{set gridmincolor}

               Sets the color of minor grid lines.
               \\
{\tt key} & {\bf Possible values:} {\tt on}, {\tt off}.

               {\bf Analogous set command:} \indcmdts{set key}

               Sets whether a legend is displayed on plots.
               \\
{\tt keyColumns} & {\bf Possible values:} Any integer $\geq 0$.

               {\bf Analogous set command:} \indcmdts{set keycolumns}

               Sets the number of columns into which the legends of plots should be divided. If a value of zero is given, then the number of columns is decided automatically for each plot.
               \\
{\tt keyPos} & {\bf Possible values:} {\tt top right}, {\tt top xcenter}, {\tt top left}, {\tt ycenter right}, {\tt ycenter xcenter}, {\tt ycenter left}, {\tt bottom right}, {\tt bottom xcenter}, {\tt bottom left}, {\tt above}, {\tt below}, {\tt outside}.

               {\bf Analogous set command:} \indcmdts{set key}

               Sets where the legend should appear on plots.
               \\
{\tt key\_xOff} & {\bf Possible values:} Any floating-point number.

               {\bf Analogous set command:} \indcmdts{set key}

               Sets the horizontal offset, in approximate graph-widths, that should be applied to the legend, relative to its default position, as set by {\tt KEYPOS}.
               \\
{\tt key\_yOff} & {\bf Possible values:} Any floating-point number.

               {\bf Analogous set command:} \indcmdts{set key}

               Sets the vertical offset, in approximate graph-heights, that should be applied to the legend, relative to its default position, as set by {\tt KEYPOS}.
               \\
{\tt landscape} & {\bf Possible values:} {\tt on}, {\tt off}.

               {\bf Analogous set command:} \indcmdts{set terminal}

               Sets whether output is in portrait orientation ({\tt off}), or landscape orientation ({\tt on}).
               \\
{\tt lineWidth} & {\bf Possible values:} Any floating-point number.

               {\bf Analogous set command:} \indcmdts{set linewidth}

               Sets the width of lines on plots, as a multiple of the default.
               \\
{\tt multiPlot} & {\bf Possible values:} {\tt on}, {\tt off}.

               {\bf Analogous set command:} \indcmdts{set multiplot}

               Sets whether multiplot mode is on or off.
               \\
{\tt numComplex} & {\bf Possible values:} {\tt on}, {\tt off}.

               {\bf Analogous set command:} \indcmdts{set numerics}

               Sets whether complex arithmetic is enabled, or whether all non-real results to calculations should raise numerical exceptions.
               \\
{\tt numDisplay} & {\bf Possible values:} {\tt latex}, {\tt natural}, {\tt typeable}.

               {\bf Analogous set command:} \indcmdts{set numerics}

               Sets whether numerical results are displayed in a natural human-readable way, e.g.\ $2\,\mathrm{m}$, in LaTeX, e.g.\ {\tt \$$2\backslash$,$\backslash$mathrm\{m\}\$}, or in a way which may be pasted back into PyXPlot, e.g.\ {\tt 2*unit(m)}.
               \\
{\tt numErr} & {\bf Possible values:} {\tt on}, {\tt off}.

               {\bf Analogous set command:} \indcmdts{set numerics}

               Sets whether explicit error messages are thrown when calculations yield undefined results, as in the cases of division by zero or the evaluation of functions in regions where they are undefined or infinite. If explicit error messages are disabled, such calculations quietly return {\tt nan}.
               \\
{\tt numSF} & {\bf Possible values:} Any integer between 0 and 30.

               {\bf Analogous set command:} \indcmdts{set numerics}

               Sets the number of significant figures to which numerical quantities are displayed by default.
               \\
{\tt originX} & {\bf Possible values:} Any floating point number.

               {\bf Analogous set command:} \indcmdts{set origin}

               Sets the horizontal position, in centimetres, of the default origin of plots on the page. Most useful when multiplotting many plots.
               \\
{\tt originY} & {\bf Possible values:} Any floating point number.

               {\bf Analogous set command:} \indcmdts{set origin}

               Sets the vertical position, in centimetres, of the default origin of plots on the page. Most useful when multiplotting many plots.
               \\
{\tt output} & {\bf Possible values:} Any string (case sensitive).

               {\bf Analogous set command:} \indcmdts{set output}

               Sets the output filename for plots. If blank, the default filename of {\tt pyxplot.foo} is used, where {\tt foo} is an extension appropriate for the file format.
               \\
{\tt paperHeight} & {\bf Possible values:} Any floating-point number.

               {\bf Analogous set command:} \indcmdts{set papersize}

               Sets the height of the papersize for PostScript output in millimetres.
               \\
{\tt paperName} & {\bf Possible values:} A string matching any of the papersizes listed in Chapter~\ref{ch:paper_sizes}.

               {\bf Analogous set command:} \indcmdts{set papersize}

               Sets the papersize for PostScript output to one of the pre-defined papersizes listed in Chapter~\ref{ch:paper_sizes}.
               \\
{\tt paperWidth} & {\bf Possible values:} Any floating-point number.

               {\bf Analogous set command:} \indcmdts{set papersize}

               Sets the width of the papersize for PostScript output in millimetres.
               \\
{\tt pointLineWidth} & {\bf Possible values:} Any floating-point number.

               {\bf Analogous set command:} \indcmdts{set pointlinewidth}

               Sets the linewidth used to stroke points onto plots, as a multiple of the default.
               \\
{\tt pointSize} & {\bf Possible values:} Any floating-point number.

               {\bf Analogous set command:} \indcmdts{set pointsize}

               Sets the sizes of points on plots, as a multiple of their normal sizes.
               \\
%{\tt projection} & {\bf Possible values:} {\tt gnomonic}, {\tt flat}.
%
%               {\bf Analogous set command:} \indcmdts{set projection}
%
%               Sets the projection used on graphs. Flat projection is useful for most purposes, but gnomonic projection is useful for representing curved spaces on flat pieces of paper, when drawing world maps, for example.
%               \\
{\tt samples} & {\bf Possible values:} Any integer.

               {\bf Analogous set command:} \indcmdts{set samples}

               Sets the number of samples (datapoints) to be evaluated along the abscissa axis when plotting a function.
               \\
{\tt samples\_method} & {\bf Possible values:} {\tt inverse\-Square}, {\tt monag\-han\-Lattan\-zio}, {\tt nearest\-Neigh\-bor}.

               {\bf Analogous set command:} \indcmdts{set samples}

               Sets the method used to interpolate two-dimensional non-gridded arrays of datapoints from datafiles within the \indcmdt{interpolate} and when plotting using the \indpst{colormap}, \indpst{contourmap} and \indpst{surface} plot styles.
               \\
{\tt samples\_x} & {\bf Possible values:} Any integer.

               {\bf Analogous set command:} \indcmdts{set samples}

               Sets the number of samples (datapoints) to be evaluated along the first abscissa axis when drawing color maps and surfaces, and when calculating contour maps.
               \\
{\tt samples\_x\_auto} & {\bf Possible values:} {\tt true}, {\tt false}.

               {\bf Analogous set command:} \indcmdts{set samples}

               Sets whether the number of samples (datapoints) to be evaluated along the first abscissa axis when drawing color maps and surfaces, and when calculating contour maps should follow the number of samples set with the \indcmdts{set samples} command.
               \\
{\tt samples\_y} & {\bf Possible values:} Any integer.

               {\bf Analogous set command:} \indcmdts{set samples}

               Sets the number of samples (datapoints) to be evaluated along the second abscissa axis when drawing color maps and surfaces, and when calculating contour maps.
               \\
{\tt samples\_y\_auto} & {\bf Possible values:} {\tt true}, {\tt false}.

               {\bf Analogous set command:} \indcmdts{set samples}

               Sets whether the number of samples (datapoints) to be evaluated along the second abscissa axis when drawing color maps and surfaces, and when calculating contour maps should follow the number of samples set with the \indcmdts{set samples} command.
               \\
{\tt termAntiAlias} & {\bf Possible values:} {\tt on}, {\tt off}.

               {\bf Analogous set command:} \indcmdts{set terminal}

               Sets whether output sent to the bitmapped graphics output terminals -- i.e.\ the bmp, jpeg, gif, png and tif terminals -- is antialiased. Antialiasing smooths the color boundaries to disguise the effects of pixelisation and is almost invariably desirable.
               \\
{\tt termEnlarge} & {\bf Possible values:} {\tt on}, {\tt off}.

               {\bf Analogous set command:} \indcmdts{set terminal}

               When set to {\tt on} output is enlarged or shrunk to fit the current paper size.
               \\
{\tt termInvert} & {\bf Possible values:} {\tt on}, {\tt off}.

               {\bf Analogous set command:} \indcmdts{set terminal}

               Sets whether jpeg/gif/png output has normal colors ({\tt off}), or inverted colors ({\tt on}).
               \\
{\tt termTransparent} & {\bf Possible values:} {\tt on}, {\tt off}.

               {\bf Analogous set command:} \indcmdts{set terminal}

               Sets whether jpeg/gif/png output has transparent background ({\tt on}), or solid background ({\tt off}).
               \\
{\tt termType} & {\bf Possible values:} {\tt bmp}, {\tt eps}, {\tt gif}, {\tt jpg}, {\tt pdf}, {\tt png}, {\tt ps}, {\tt svg}, {\tt tif}, {\tt X11\_multiWindow}, {\tt X11\_persist}, {\tt X11\_singleWindow}.

               {\bf Analogous set command:} \indcmdts{set terminal}

               Sets whether output is sent to the screen, using one of the {\tt X11\_}... terminals, or to disk. In the latter case, output may be produced in a wide variety of graphical formats.
               \\
{\tt textColor} & {\bf Possible values:} Any recognised color.

               {\bf Analogous set command:} \indcmdts{set textcolor}

               Sets the color of all text output.
               \\
{\tt textHAlign} & {\bf Possible values:} {\tt left}, {\tt center}, {\tt right}.

               {\bf Analogous set command:} \indcmdts{set texthalign}

               Sets the horizontal alignment of text labels to their given reference positions.
               \\
{\tt textVAlign} & {\bf Possible values:} {\tt top}, {\tt center}, {\tt bottom}.

               {\bf Analogous set command:} \indcmdts{set textvalign}

               Sets the vertical alignment of text labels to their given reference positions.
               \\
{\tt title} & {\bf Possible values:} Any string (case sensitive).

               {\bf Analogous set command:} \indcmdts{set title}

               Sets the title to appear at the top of the plot.
               \\
{\tt title\_xOff} & {\bf Possible values:} Any floating point number.

               {\bf Analogous set command:} \indcmdts{set title}

               Sets the horizontal offset of the title of the plot from its default central location.
               \\
{\tt title\_yOff} & {\bf Possible values:} Any floating point number.

               {\bf Analogous set command:} \indcmdts{set title}

               Sets the vertical offset of the title of the plot from its default location at the top of the plot.
               \\
{\tt tRange\_log} & {\bf Possible values:} {\tt true}, {\tt false}.

               {\bf Analogous set command:} \indcmdts{set logscale t}

               Sets whether the $t$-axis -- used for parametric plotting -- is linear or logarithmic.
               \\
{\tt tRange\_max} & {\bf Possible values:} Any floating-point number.

               {\bf Analogous set command:} \indcmdts{set trange}

               Sets upper limit of the $t$-axis, used for parametric plotting.
               \\
{\tt tRange\_min} & {\bf Possible values:} Any floating-point number.

               {\bf Analogous set command:} \indcmdts{set trange}

               Sets lower limit of the $t$-axis, used for parametric plotting.
               \\
{\tt unitAbbrev} & {\bf Possible values:} {\tt on}, {\tt off}.

               {\bf Analogous set command:} \indcmdts{set unit}

               Sets whether physical units are displayed in abbreviated form, e.g.\ {\tt mm}, or in full, e.g.\ {\tt millimetres}.
               \\
{\tt unitAngleDimless} & {\bf Possible values:} {\tt on}, {\tt off}.

               {\bf Analogous set command:} \indcmdts{set unit}

               Sets whether angles are treated as dimensionless units, or whether the radian is treated as a base unit.
               \\
{\tt unitPrefix} & {\bf Possible values:} {\tt on}, {\tt off}.

               {\bf Analogous set command:} \indcmdts{set unit}

               Sets whether SI prefixes, such as {\tt milli-} and {\tt mega-} are prepended to SI units where appropriate.
               \\
{\tt unitScheme} & {\bf Possible values:} {\tt ancient}, {\tt cgs}, {\tt imperial}, {\tt planck}, {\tt si}, {\tt USCustomary}.

               {\bf Analogous set command:} \indcmdts{set unit}

               Sets the scheme of physical units in which quantities are displayed.
               \\
{\tt uRange\_log} & {\bf Possible values:} {\tt true}, {\tt false}.

               {\bf Analogous set command:} \indcmdts{set logscale u}

               Sets whether the $u$-axis -- used for parametric plotting -- is linear or logarithmic.
               \\
{\tt uRange\_max} & {\bf Possible values:} Any floating-point number.

               {\bf Analogous set command:} \indcmdts{set urange}

               Sets upper limit of the $u$-axis, used for parametric plotting.
               \\
{\tt uRange\_min} & {\bf Possible values:} Any floating-point number.

               {\bf Analogous set command:} \indcmdts{set urange}

               Sets lower limit of the $t$-axis, used for parametric plotting.
               \\
{\tt vRange\_log} & {\bf Possible values:} {\tt true}, {\tt false}.

               {\bf Analogous set command:} \indcmdts{set logscale v}

               Sets whether the $v$-axis -- used for parametric plotting -- is linear or logarithmic.
               \\
{\tt vRange\_max} & {\bf Possible values:} Any floating-point number.

               {\bf Analogous set command:} \indcmdts{set vrange}

               Sets upper limit of the $v$-axis, used for parametric plotting.
               \\
{\tt vRange\_min} & {\bf Possible values:} Any floating-point number.

               {\bf Analogous set command:} \indcmdts{set vrange}

               Sets lower limit of the $v$-axis, used for parametric plotting.
               \\
{\tt width} & {\bf Possible values:} Any floating-point number.

               {\bf Analogous set commands:} \indcmdts{set width}, \indcmdts{set size}

               Sets the width of plots in centimetres.
               \\
{\tt view\_xy} & {\bf Possible values:} Any floating-point number.

               {\bf Analogous set commands:} \indcmdts{set view}

               Sets the viewing angle of three-dimensional plots in the $x$-$y$ plane in degrees.
               \\
{\tt view\_yz} & {\bf Possible values:} Any floating-point number.

               {\bf Analogous set commands:} \indcmdts{set view}

               Sets the viewing angle of three-dimensional plots in the $y$-$z$ plane in degrees.
               \\
{\tt zAspect} & {\bf Possible values:} {\tt auto}, or any floating-point number.

               {\bf Analogous set command:} \indcmdts{set size ratio}

               Sets the $z/x$ aspect ratio of 3d plots.
               \\
\end{longtable}

\subsection{The {\tt styling} section}

The {\tt styling} section can contain any of the following settings in any order:

\begin{longtable}{p{3.4cm}p{9cm}}
{\tt arrow\_headAngle} & {\bf Possible values:} Any floating-point number.

               Sets the angle, in degrees, at which the two sides of arrow heads meet at its point.
               \\
{\tt arrow\_headSize} & {\bf Possible values:} Any floating-point number.

               Sets the size of all arrow heads. A value of 1.0 corresponds to PyXPlot's default size.
               \\
{\tt arrow\_headBackIndent} & {\bf Possible values:} Any floating-point number.

               Sets the size of the indentation in the back of arrow heads. The default size is 0.2. Sensible values lie in the range 0 (no indentation) to 1 (the indentation extends the whole length of the arrow head). Less sensible values may be used by the aesthetically adventurous.
               \\
{\tt axes\_lineWidth}  & {\bf Possible values:} Any floating-point number.

               Sets the line width used to draw graph axes.
               \\
{\tt axes\_majTickLen} & {\bf Possible values:} Any floating-point number.

               Sets the length of major axis ticks. A value of 1.0 corresponds to PyXPlot's default length of $1.2\,\mathrm{mm}$; other values differ from this multiplicatively.
               \\
{\tt axes\_minTickLen} & {\bf Possible values:} Any floating-point number.

               Sets the length of minor axis ticks. A value of 1.0 corresponds to PyXPlot's default length of $0.85\,\mathrm{mm}$; other values differ from this multiplicatively.
               \\
{\tt axes\_separation} & {\bf Possible values:} Any floating-point number.

               Sets the separation between parallel axes on graphs, less the width of any text labels associated with the axes. A value of 1.0 corresponds to PyXPlot's default spacing of $8\,\mathrm{mm}$; other values differ from this multiplicatively.
               \\
{\tt axes\_textGap} & {\bf Possible values:} Any floating-point number.

               Sets the separation between axes and the text labels which are associated with them. A value of 1.0 corresponds to PyXPlot's default spacing of $3\,\mathrm{mm}$; other values differ from this multiplicatively.
               \\
{\tt colorScale\_margin} & {\bf Possible values:} Any floating-point number.

               Sets the separation left between the axes of plots drawn using the {\tt colormap} plot style, and of the color scales drawn alongside them. A value of 1.0 corresponds to PyXPlot's default spacing; other values differ from this multiplicatively.
               \\
{\tt colorScale\_width} & {\bf Possible values:} Any floating-point number.

               Sets the width of the color scale bars drawn alonside plots drawn using the {\tt colormap} plot style. A value of 1.0 corresponds to PyXPlot's width; other values differ from this multiplicatively.
               \\
{\tt grid\_majLineWidth} & {\bf Possible values:} Any floating-point number.

               Sets the line width used to draw major gridlines (default $1.0$).
               \\
{\tt grid\_minLineWidth} & {\bf Possible values:} Any floating-point number.

               Sets the line width used to draw minor gridlines (default $0.5$).
               \\
{\tt baseline\_lineWidth} & {\bf Possible values:} Any floating-point number.

               Sets the PostScript line width which corresponds to a {\tt linewidth} of 1.0. A value of 1.0 corresponds to PyXPlot's default line width of $0.2\,\mathrm{mm}$; other values differ from this multiplicatively.
               \\
{\tt baseline\_pointSize} & {\bf Possible values:} Any floating-point number.

               Sets the baseline point size which corresponds to a {\tt pointsize} of 1.0. A value of 1.0 corresponds to PyXPlot's default; other values differ from this multiplicatively.
               \\
\end{longtable}

\subsection{The {\tt terminal} section}
\label{sec:configfile_terminal}

The {\tt terminal} section can contain any of the following settings in any order:

\begin{longtable}{p{3.4cm}p{9cm}}
{\tt color} & {\bf Possible values:} {\tt on}, {\tt off}.

               {\bf Analogous command-line switches:} {\tt -c}, {\tt --color}, {\tt -m}, {\tt --monochrome}.

               Sets whether color highlighting should be used in the interactive terminal. If turned on, output is displayed in green, warning messages in amber, and error messages in red; these colors are configurable, as described below. Note that not all UNIX terminals support the use of color.
               \\
{\tt color\_err} & {\bf Possible values:} Any recognised terminal color (see below).

               {\bf Analogous command-line switches:} None.

               Sets the color in which error messages are displayed when color highlighting is used. Note that the list of recognised color names differs from that used in PyXPlot; a list is given at the end of this section.
               \\
{\tt color\_rep} & {\bf Possible values:} Any recognised terminal color (see below).

               {\bf Analogous command-line switches:} None.

               As above, but sets the color in which PyXPlot displays its non-error-related output.
               \\
{\tt color\_wrn} & {\bf Possible values:} Any recognised terminal color (see below).

               {\bf Analogous command-line switches:} None.

               As above, but sets the color in which PyXPlot displays its warning messages.
               \\
{\tt splash} & {\bf Possible values:} {\tt on}, {\tt off}.

               {\bf Analogous command-line switches:} {\tt -q}, {\tt --quiet}, {\tt -V}, {\tt --verbose}

               Sets whether the standard welcome message is displayed upon startup.
               \\
\end{longtable}

The colors recognised by the {\tt COLOR\_XXX} configuration options above
are: {\tt Red}, {\tt Green}, {\tt Amber}, {\tt Blue}, {\tt Purple}, {\tt
Magenta}, {\tt Cyan}, {\tt White}, {\tt Normal}. The final option produces the
default foreground color of the terminal.

\renewcommand{\arraystretch}{1.00}

\subsection{The {\tt units} section}
\label{sec:configfile_units}

The {\tt units} section can be used to define new physical units for use within
PyXPlot's mathematical environment. Each line should take the format of
\begin{verbatim}
<l_sing> \[ / <s_sing> \] \[ / <lt_sing> \]
   \[ / <l_plur> \] \[ / <s_plur> \] \[ / <lt_plur> \]
   : <quantity_name> = \[ <definition> \]
\end{verbatim}
where
\begin{longtable}{p{3.4cm}p{9cm}}
{\tt l\_sing} & is the long singular name of the unit, e.g.\ {\tt metre}.\\
{\tt s\_sing} & is the short singular name of the unit, e.g.\ {\tt m}.\\
{\tt lt\_sing} & is the singular name of the unit to be used in \LaTeX.\\
{\tt l\_plur} & is the long plural name of the unit, e.g.\ {\tt metres}.\\
{\tt s\_plur} & is the short plural name of the unit, e.g.\ {\tt m}.\\
{\tt lt\_plur} & is the plural name of the unit to be used in \LaTeX.\\
{\tt quantity\_name} & is the physical quantity which the unit measures, e.g.\ {\tt length}.\\
{\tt definition} & is a definition of the unit in terms of other units which
PyXPlot already recognises, e.g.\ {\tt 0.001*km}. The syntax used is identical
to that used in the {\tt unit()} function.\\
\end{longtable}
For example, a definition of the metre would look like
\begin{verbatim}
metre/m/m/metres/m/m:length=0.001*km
\end{verbatim}

Not all of the various names which a unit may have need to be specified. If
plural names are not specified then they are assumed to be the same as the
singular names. If short and/or \LaTeX names are not specified they are assumed
to be the same as the long name. If the definition is left blank then the unit
is assumed to be a new base unit which is not related to any pre-existing
units.

\section{Recognised color names}
\label{sec:color_names}

The following is a complete list of the color names which PyXPlot recognises
in the {\tt set textcolor}, {\tt set axescolor} commands, and in the {\tt
colors} section of its configuration file.  A color chart of these can be
found in Appendix~\ref{ch:color_charts}.  All color names are case
insensitive.

\vspace{5mm}\noindent
\index{configuration file!colors}\index{colors!configuration file}
{\tt
greenYellow, yellow, goldenrod, dandelion, apricot, peach, melon,\newline\noindent
yellowOrange, orange, burntOrange, bittersweet, redOrange,\newline\noindent
mahogany, maroon, brickRed, red, orangeRed, rubineRed,\newline\noindent
wildStrawberry, salmon, carnationPink, magenta, violetRed,\newline\noindent
rhodamine, mulberry, redViolet, fuchsia, lavender, thistle,\newline\noindent
orchid, darkOrchid, purple, plum, violet, royalPurple,\newline\noindent
blueViolet, periwinkle, cadetBlue, cornflowerBlue, midnightBlue,\newline\noindent
navyBlue, royalBlue, blue, cerulean, cyan, processBlue, skyBlue,\newline\noindent
turquoise, tealBlue, aquamarine, blueGreen, emerald, jungleGreen,\newline\noindent
seaGreen, green, forestGreen, pineGreen, limeGreen, yellowGreen,\newline\noindent
springGreen, oliveGreen, rawSienna, sepia, brown, tan, gray,\newline\noindent
grey, black, white.
}

\vspace{5mm}
In addition, a scale of 100~shades of grey is available, ranging from {\tt
grey00}, which is black, to {\tt grey99}, which is very nearly white.  The US
spelling, {\tt gray??}, is also accepted.

Arbitrary colors may be specified in the forms {\tt rgb0:0:0}, {\tt hsb0:0:0}
or {\tt cmyk0:0:0:0}, where the colon-separated zeros should be replaced by
values in the range of~0 to~1 to represent the components of the desired color
in RGB, HSB or CMYK space
respectively.\index{colors!RGB}\index{colors!HSB}\index{colors!CMYK}


\part{Appendices}
\appendix
% OTHER_APPS.TEX
%
% The documentation in this file is part of PyXPlot
% <http://www.pyxplot.org.uk>
%
% Copyright (C) 2006-2011 Dominic Ford <coders@pyxplot.org.uk>
%               2008-2011 Ross Church
%
% $Id$
%
% PyXPlot is free software; you can redistribute it and/or modify it under the
% terms of the GNU General Public License as published by the Free Software
% Foundation; either version 2 of the License, or (at your option) any later
% version.
%
% You should have received a copy of the GNU General Public License along with
% PyXPlot; if not, write to the Free Software Foundation, Inc., 51 Franklin
% Street, Fifth Floor, Boston, MA  02110-1301, USA

% ----------------------------------------------------------------------------

% LaTeX source for the PyXPlot Users' Guide

\chapter{Other Applications of PyXPlot}

In this chapter we present a short cookbook describing a few common yet
miscellaneous tasks for which PyXPlot may prove useful.

\section{Conversion of JPEG Images to PostScript}
\index{jpeg images}

The need to convert bitmap images -- for example, those in jpeg format -- into
PostScript representations is commonly encountered by users of the \LaTeX\
typesetting system, since \LaTeX's {\tt includegraphics} command can only
incorporate Encapsulated PostScript (EPS) images into documents.  A small
number of graphics packages provide facilities for making such conversions,
including ImageMagick\index{ImageMagick}'s {\tt convert} command, but these
almost invariable produce excessively large PostScript files on account of
their failure to use PostScript's native facilities for image compression.
PyXPlot's \indcmdt{image} can in many cases perform much more efficient
conversion:

\begin{verbatim}
set output image.eps
image 'image.jpg' width 10
\end{verbatim}

\section{Inserting Equations in Powerpoint Presentations}
\index{Microsoft Powerpoint}\index{presentations}

The two tools most commonly used for presenting talks\index{presentations} --
Microsoft {\it Powerpoint}\index{Microsoft Powerpoint} and
OpenOffice\index{OpenOffice} {\it Impress} -- have limited facilities for importing
text rendered in \LaTeX\ into slides. {\it Powerpoint} does
include its own {\it Equation Editor}, but its output is considerably less
professional than that produced by \LaTeX.  This can prove a frustration for
anyone who works in a field with notation which makes use of non-standard
characters, but especially for those who work in mathematical and
equation-centric disciplines.

It is possible to import graphic images into {\it Powerpoint}, but it cannot
read images in PostScript format, the format in which \LaTeX\ usually produces
its output.  PyXPlot's {\tt gif} and {\tt png} terminals provide a fix for this
problem, as the following example demonstrates:

\begin{verbatim}
set term transparent noantialias gif
set term dpi 300
set output 'equation.gif'
set multiplot

# Render the Planck blackbody formula in LaTeX
set textcolour yellow
text '$B_\nu = \frac{8\pi h}{c^3} \
\frac{\nu^3}{\exp \left( h\nu / kT \right) -1 }$' at 0,0
text 'The Planck Blackbody Formula:' at 0 , 0.75
\end{verbatim}

The result is a {\tt gif} image of the desired equation, with yellow text on a
transparent background. This can readily be imported into {\it Powerpoint} and
re-scaled to the desired size.

\section{Delivering Talks in PyXPlot}

Going one step further, PyXPlot can be used as a stand-alone tool for designing
slides for talks; it has several advantages over other presentation tools.  All
of the text which is placed on slides is rendered neatly in \LaTeX.  Images can
be placed on slides using the \indcmdts{jpeg} and \indcmdts{eps} commands, and
placed at any arbitrary coordinate position on the slide.  In comparison with
programs such as Microsoft {\it Powerpoint}\index{Microsoft Powerpoint} and
OpenOffice\index{OpenOffice} {\it Impress}, the text looks much neater,
especially if equations or unusual characters are required. In comparison with
\TeX-based programs such as Foil\TeX, it is much easier to incorporate images
around text to create colourful slides which will keep an audience attentive.

As an additional advantage, graphs can be plotted within the scripts describing
each slide, directly from \datafile s in your local filesystem. If you receive
new data shortly before giving a talk, it is a simple matter to re-run the
PyXPlot scripts and your slides will automatically pick up the new \datafile s.

Below, we outline our recipe for designing slides in PyXPlot. There are many
steps, but they do not take much time; many simply involve pasting text into
various files. Readers of the printed version of the manual may find it easier
to copy these files from the HTML version of this manual on the PyXPlot
website.

\subsection{Setting up Infrastructure}

First, a bit of infrastructure needs to be set up. Note that once this has been
done for one talk, the infrastructure can be copied directly from a previous
talk.

\begin{enumerate}
\item Make a new directory in which to put your talk:
\begin{verbatim}
mkdir my_talk
cd my_talk
\end{verbatim}
\item Make a directory into which you will put the PyXPlot scripts for your
individual slides:
\begin{verbatim}
mkdir scripts
\end{verbatim}
\item Make a directory into which you will put any graphic images which you
want to put into your talk to make it look pretty:
\begin{verbatim}
mkdir images
\end{verbatim}
\item Make a directory into which PyXPlot will put graphic images of your
slides:
\begin{verbatim}
mkdir slides
\end{verbatim}
\item Design a background for your slides. Open a paint program such as the
{\tt gimp}, create a new image which measures $1024\times768$\,pixels, and fill
it with colour. My preference tends to be for a blue colour gradient, running
from bright blue at the top to dark blue at the bottom, but you may be more
inventive than me. You may wish to add institutional and/or project logos in
the corners. Alternatively, you can download a ready-made background image from
the PyXPlot website: \url{http://foo}. You should store this image as {\tt
images/background.jpg}.
\item We need a simple PyXPlot script to set up a slide template. Paste the
following text into the file {\tt scripts/slide\_init}; there's a bit of black
magic in the {\tt arrow} commands in this script which it isn't necessary to
understand at this stage:\label{stp:presentation_magic}
\begin{verbatim}
scale  = 1.25        ; inch = 2.54 # cm
width  = 10.24*scale ; height =  7.68*scale
x = width/100.0      ; y = height/100.0
set term gif ; set dpi (1024.0/width) * inch
set multiplot ; set nodisplay
set texthalign centre ; set textvalign centre
set textcolour yellow
jpeg "images/background.jpg" width width
arrow -x* 25,-y* 25 to -x* 25, y*125 with nohead
arrow -x* 25, y*125 to  x*125, y*125 with nohead
arrow  x*125, y*125 to  x*125,-y* 25 with nohead
arrow  x*125,-y* 25 to -x* 25,-y* 25 with nohead
\end{verbatim}
\item We also need a simple PyXPlot script to round off each slide. Paste the
following text into the file {\tt scripts/slide\_finish}:
\begin{verbatim}
set display ; refresh
\end{verbatim}
\item Paste the following text into the file {\tt compile}. This is a simple
shell script which instructs {\tt pyxplot\_watch} to compile your slides using
PyXPlot every time you edit any of the them:
\begin{verbatim}
#!/bin/bash
pyxplot_watch --verbose scripts/0\*
\end{verbatim}
\item Paste the following text into the file {\tt make\_slides}. This is a
simple shell script which crops your slides to measure exactly
$1024\times768$\,pixels, cropping any text boxes which may go off the side of
them. It links up with the black magic of Step~\ref{stp:presentation_magic}:
\begin{verbatim}
#!/bin/bash
mkdir -p slides_cropped
for all in slides/*.gif ; do
convert $all -crop 1024x768+261+198 `echo $all | \
sed 's@slides@slides_cropped@' | sed 's@gif@jpg@'`
done
\end{verbatim}
\item Make the scripts {\tt compile} and {\tt make\_slides} executable:
\begin{verbatim}
chmod 755 compile make_slides
\end{verbatim}
\end{enumerate}

\subsection{Writing A Short Example Talk}

The infrastructure is now completely set up, and you are ready to start
designing slides. We will now design an example talk with three slides.

\begin{enumerate}
\item Run the script {\tt compile} and leave it running in the background.
PyXPlot will then re-run the scripts describing your slides whenever you edit
them.
\item As an example, we will now make a title slide. Paste the following script
into the file {\tt scripts/0001}:
\begin{verbatim}
set output 'slides/0001.gif'
load 'scripts/slide_init'

text '\parbox[t]{10cm}{\center \LARGE \bf \
  A Tutorial in the use of PyXPlot \\ \
  to present Talks \
} ' at x*50, y*75
text '\Large \bf Prof A.N.\ Other' at x*50, y*45
text '\parbox[t]{9cm}{\center \
  Director, \\ \
  Atlantis Island University \
} ' at x*50, y*38
text 'Annual Lecture, 1st January 2010' at x*50, y*22

load 'scripts/slide_finish'
\end{verbatim}
Note that the variables {\tt x} and {\tt y} are defined to be 1~per cent of the
width and height of your slides respectively, such that the bottom-left of each
slide is at $(0,0)$ and the top-right of each slide is at $({\tt 100*x},{\tt
100*y})$.
\item Next we will make a second slide with a series of bullet points. Paste
the following script into the file {\tt scripts/0002}:
\begin{verbatim}
set output 'slides/0002.gif'
load 'scripts/slide_init'

text '\Large \textbf{Talk Overview}' at x*50, y*92
text "\parbox[t]{9cm}{\begin{itemize} \
 \item Setting up the Infrastructure. \
 \item Writing a Short Example Talk. \
 \item Delivering your Talk. \
 \item Conclusion. \
 \end{itemize} \
} " at x*50 , y*60

set textcol cyan
text '{\bf With thanks to my collaborator, \
           Prof Y.E.\ Tanother.}' at x*50,y*15

load 'scripts/slide_finish'
\end{verbatim}
\item Finally, we will make a third slide with a graph on it. Paste the
following script into the file {\tt scripts/0003}:
\begin{verbatim}
set output 'slides/0003.gif'
load 'scripts/slide_init'

text '\Large \bf The Results of Our Model' at x*50, y*92
set axescolour yellow ; set nogrid
set origin x*17.5, y*20 ; set width x*70
set xrange [0.01:0.7]
set xlabel '$x$'
set yrange [0.01:0.7]
set ylabel '$f(x)$'
set palette Red, Green, Orange, Purple

set key top left
plot x t 'Model 1', exp(x)-1 t 'Model 2', \
     log(x+1) t 'Model 3', sin(x) t 'Model 4'

load 'scripts/slide_finish'
\end{verbatim}
\item To view your slides, run the script {\tt make\_slides}. Afterwards, you
will find your slides as a series of $1024\times768$\,pixel jpeg images in the
directory {\tt slides\_cropped}.  If you have the {\it Quick Image
Viewer}\index{Quick Image Viewer} ({\tt qiv}) installed, then you can view them
as follows:
\begin{verbatim}
qiv slides_cropped/*
\end{verbatim}
If you're in a hurry, you can skip the step of running the script {\tt
make\_slides} and view your slides as images in the {\tt slides} directory, but
note that the slides in here may not be properly cropped. This approach is
generally preferable when viewing your slides in a semi-live fashion as you are
editing them.
\item If you'd like to make the text on your slides larger or smaller, you can
do so by varying the {\tt scale} parameter in the file {\tt
scripts/slide\_init}.
\end{enumerate}

%The three slides which we have designed can been seen in
%Figures~\ref{fig:presentation_slide1}, \ref{fig:presentation_slide2} and
%\ref{fig:presentation_slide3}.

\subsection{Delivering your Talk}

There are two straightforward ways in which you can give your talk. The
quickest way is simply to use the {\it Quick Image Viewer}\index{Quick Image
Viewer} ({\tt qiv}):
\begin{verbatim}
qiv slides_cropped/*
\end{verbatim}
Press the left mouse button to move forward through your talk, and the right
mouse button to go back a slide.

This method does lack some of the niceties of Microsoft {\it Powerpoint} -- for
example, the ability to jump to any arbitrary slide number, compatibility with
wireless remote controls to advance your slides, and the ability to use
animated slide transitions. It may be preferably, therefore, to paste the jpeg
images of your slides into a {\it Powerpoint} or OpenOffice {\it Impress}
presentation before you give your talk.


% GNUPLOT_DIFFS.TEX
%
% The documentation in this file is part of PyXPlot
% <http://www.pyxplot.org.uk>
%
% Copyright (C) 2006-2011 Dominic Ford <coders@pyxplot.org.uk>
%               2008-2011 Ross Church
%
% $Id$
%
% PyXPlot is free software; you can redistribute it and/or modify it under the
% terms of the GNU General Public License as published by the Free Software
% Foundation; either version 2 of the License, or (at your option) any later
% version.
%
% You should have received a copy of the GNU General Public License along with
% PyXPlot; if not, write to the Free Software Foundation, Inc., 51 Franklin
% Street, Fifth Floor, Boston, MA  02110-1301, USA

% ----------------------------------------------------------------------------

% LaTeX source for the PyXPlot Users' Guide

\chapter{Summary of Differences Between PyXPlot and \gnuplot}
\chaptermark{Differences Between PyXPlot \& \gnuplot}
\label{ch:gnuplot_diffs}

PyXPlot's commandline interface is based loosely upon that of \gnuplot, but
does not completely re-implement the entirety of \gnuplot's command language.
Moreover, PyXPlot's command language includes many extensions of \gnuplot's
interface. In this Appendix, we outline some of the most significant areas in
which \gnuplot\ and PyXPlot differ. This is far from an exhaustive list, but
may provide a useful reference for \gnuplot\ users.

\section{The Typesetting of Text}

PyXPlot renders all text labels automatically in the \LaTeX\ typesetting
environment. This brings many advantages: it produces neater labels than the
default typesetting engine used by \gnuplot, makes it straightforward to label
graphs with mathematical expressions, and moreover makes it straightforward
when importing graphs into \LaTeX\ documents to match the fonts used in figures
with those used in the main text of the document.  It does, however, also
necessarily introduce some incompatibility with \gnuplot.  Some strings which
are valid in \gnuplot\ are not valid in PyXPlot (see
Section~\ref{sec:latex_incompatibility} for more details). For
example,\index{latex}

\begin{dontdo}
set xlabel 'x\^{}2'
\end{dontdo}

\noindent is a valid label in \gnuplot, but is not valid input for \LaTeX\ and
therefore fails in PyXPlot.  In PyXPlot, it needs to be written in \LaTeX\
mathmode as:

\begin{dodo}
set xlabel '\$x\^{}2\$'
\end{dodo}

\noindent A useful introduction to \LaTeX's syntax can be found in Tobias
Oetiker's\index{Tobias Oetiker} excellent free tutorial, {\it The Not So Short
Guide to \LaTeX\ $2\epsilon$}\index{Not So Short Guide to \LaTeX\ $2\epsilon$,
The}, which is available for free download from:

\noindent \url{http://www.ctan.org/tex-archive/info/lshort/english/lshort.pdf}

PyXPlot's built-in {\tt texify()} function can also assist by automatically
converting mathematical expressions and  strings of text into \LaTeX, as in the
following examples:

\vspace{3mm}
\noindent{\tt pyxplot> {\bf a=50}}\newline
\noindent{\tt pyxplot> {\bf print texify("A \%d\% increase"\%(a))}}\newline
\noindent{\tt A 50$\backslash$\% increase}\newline
\noindent{\tt pyxplot> {\bf print texify(sqrt(x**2+1))}}\newline
\noindent{\tt \$\\displaystyle $\backslash$sqrt\{x\^{}\{2\}+1\}\$}
\vspace{3mm}

\section{Complex Numbers}

The syntax used for representing complex numbers in PyXPlot differs from that
used in \gnuplot. Whereas \gnuplot\ expects the real and imaginary components
of complex numbers to be represented {\tt \{a,b\}}, PyXPlot uses the syntax
{\tt a+b*i}, assuming that the variable {\tt i} has been defined to equal {\tt
sqrt(-1)}.  In addition, in PyXPlot complex arithmetic must first be enabled
using the {\tt set numerics complex} command before complex numbers may be
entered.  This is illustrated by the following example:

\vspace{3mm}
\noindent{\tt gnuplot> {\bf print \{1,2\} + \{3,4\}}}\newline
\noindent{\tt \{4.0, 6.0\}}
\vspace{3mm}\newline
\noindent{\tt pyxplot> {\bf set numerics complex}}\newline
\noindent{\tt pyxplot> {\bf print (1+2*i) + (3+4*i)}}\newline
\noindent{\tt (4+6i)}
\vspace{3mm}

\section{The Multiplot Environment}

\gnuplot's multiplot environment, used for placing many graphs alongside one
another, is massively extended in PyXPlot.  As well as making it much easier to
produce galleries of plots and inset graphs, a wide range of vector graphs
objects can also be added to the multiplot canvas. This is described in detail
in Chapter~\ref{ch:vector_graphics}.

\section{Plots with Multiple Axes}

In \gnuplot, a maximum of two horizontal and two vertical axes may be
associated with each graph, placed in each case with one on either side of the
plot. These are referred to as the {\tt x} (bottom) and {\tt x2} (top), or {\tt
y} (left) and {\tt y2} (right) axes.  This behaviour is reproduced in PyXPlot,
and so the syntax

\begin{verbatim}
set x2label 'Axis label'
\end{verbatim}

\noindent works similarly in both programs. However, in PyXPlot the position of
each axis may be set individually using syntax such as

\begin{verbatim}
set axis x2 top
\end{verbatim}

\noindent and furthermore up to~128 axes may be placed parallel to one another:

\begin{verbatim}
set axis x127 top
set x127label "This is axis number 127"
\end{verbatim}

\noindent More details of how to configure axes can be found in
Section~\ref{sec:multiple_axes}.

\section{Plotting Parametric Functions}

The syntax used for plotting parametric functions differs between \gnuplot\ and
PyXPlot. Whereas parametric plotting is enabled in \gnuplot\ using the {\tt set
parametric} command, in PyXPlot it is enabled on a per-dataset basis by placing
the keyword {\tt parametric} before the algebraic expression to be plotted:

\vspace{3mm}
\noindent{\tt gnuplot> {\bf set parametric}}\newline
\noindent{\tt gnuplot> {\bf set trange [0:2*pi]}}\newline
\noindent{\tt gnuplot> {\bf plot sin(t),cos(t)}}
\vspace{3mm}\newline
\noindent{\tt pyxplot> {\bf set trange [0:2*pi]}}\newline
\noindent{\tt pyxplot> {\bf plot parametric sin(t):cos(t)}}
\vspace{3mm}

\noindent This makes it straightforward to plot parametric functions alongside
non-parametric functions. For more information, see
Section~\ref{sec:parametric_plotting}.

%\section{Displaying Times and Dates on Axes}


% fit_maths.tex
%
% The documentation in this file is part of Pyxplot
% <http://www.pyxplot.org.uk>
%
% Copyright (C) 2006-2012 Dominic Ford <coders@pyxplot.org.uk>
%               2008-2012 Ross Church
%
% $Id$
%
% Pyxplot is free software; you can redistribute it and/or modify it under the
% terms of the GNU General Public License as published by the Free Software
% Foundation; either version 2 of the License, or (at your option) any later
% version.
%
% You should have received a copy of the GNU General Public License along with
% Pyxplot; if not, write to the Free Software Foundation, Inc., 51 Franklin
% Street, Fifth Floor, Boston, MA  02110-1301, USA

% ----------------------------------------------------------------------------

% LaTeX source for the Pyxplot Users' Guide

\chapter{The {\tt fit} command: mathematical details}
\chaptermark{Details of the {\tt fit} command}
\label{ch:fit_maths}

In this section, the mathematical details of the workings of the \indcmdt{fit}
are described. This may be of interest in diagnosing its limitations, and also
in understanding the various quantities that it outputs after a fit is found.
This discussion must necessarily be a rather brief treatment of a large
subject; for a fuller account, the reader is referred to D.S.\ Sivia's {\it
Data Analysis: A Bayesian Tutorial}.

\section{Notation}
\label{sec:bayes_notation}

I shall assume that we have some function $f()$, which takes $n_\mathrm{x}$
parameters, $x_0$...$x_{n_\mathrm{x}-1}$, the set of which may collectively be
written as the vector $\mathbf{x}$. We are supplied a datafile, containing a
number $n_\mathrm{d}$ of datapoints, each consisting of a set of values for
each of the $n_\mathrm{x}$ parameters, and one for the value which we are
seeking to make $f(\mathbf{x})$ match. I shall call of parameter values for the
$i$th datapoint $\mathbf{x}_i$, and the corresponding value which we are trying
to match $f_i$. The \datafile\ may contain error estimates for the values $f_i$,
which I shall denote $\sigma_i$. If these are not supplied, then I shall
consider these quantities to be unknown, and equal to some constant
$\sigma_\mathrm{data}$.

Finally, I assume that there are $n_\mathrm{u}$ coefficients within the
function $f()$ that we are able to vary, corresponding to those variable names
listed after the {\tt via} statement in the {\tt fit} command. I shall
call these coefficients $u_0$...$u_{n_\mathrm{u}-1}$, and refer to them
collectively as $\mathbf{u}$.

I model the values $f_i$ in the supplied \datafile\ as being noisy
Gaussian-distributed observations of the true function $f()$, and within this
framework, seek to find that vector of values $\mathbf{u}$ which is most
probable, given these observations. The probability of any given $\mathbf{u}$
is written
$\mathrm{P}\left( \mathbf{u} | \left\{ \mathbf{x}_i, f_i, \sigma_i \right\} \right)$.

\section{The probability density function}
\label{sec:bayes_pdf}

Bayes' Theorem states that:

\begin{equation}
\mathrm{P}\left( \mathbf{u} | \left\{ \mathbf{x}_i, f_i, \sigma_i \right\} \right) =
\frac{
\mathrm{P}\left( \left\{f_i \right\} | \mathbf{u}, \left\{ \mathbf{x}_i, \sigma_i \right\} \right)
\mathrm{P}\left( \mathbf{u} | \left\{ \mathbf{x}_i, \sigma_i \right\} \right)
}{
\mathrm{P}\left( \left\{f_i \right\} | \left\{ \mathbf{x}_i, \sigma_i \right\} \right)
}
\end{equation}

Since we are only seeking to maximise the quantity on the left, and the
denominator, termed the Bayesian \textit{evidence}, is independent of
$\mathbf{u}$, we can neglect it and replace the equality sign with a
proportionality sign.  Furthermore, if we assume a uniform prior, that is, we
assume that we have no prior knowledge to bias us towards certain more favoured
values of $\mathbf{u}$, then $\mathrm{P}\left( \mathbf{u} \right)$ is also a
constant which can be neglected. We conclude that maximising $\mathrm{P}\left(
\mathbf{u} | \left\{ \mathbf{x}_i, f_i, \sigma_i \right\} \right)$ is
equivalent to maximising $\mathrm{P}\left( \left\{f_i \right\} | \mathbf{u},
\left\{ \mathbf{x}_i, \sigma_i \right\} \right)$.

Since we are assuming $f_i$ to be Gaussian-distributed observations of the true
function $f()$, this latter probability can be written as a product of
$n_\mathrm{d}$ Gaussian distributions:

\begin{equation}
\mathrm{P}\left( \left\{f_i \right\} | \mathbf{u}, \left\{ \mathbf{x}_i, \sigma_i \right\} \right)
=
\prod_{i=0}^{n_\mathrm{d}-1} \frac{1}{\sigma_i\sqrt{2\pi}} \exp \left(
\frac{
-\left[f_i - f_\mathbf{u}(\mathbf{x}_i)\right]^2
}{
2 \sigma_i^2
} \right)
\end{equation}

The product in this equation can be converted into a more computationally
workable sum by taking the logarithm of both sides. Since logarithms are
monotonically increasing functions, maximising a probability is equivalent to
maximising its logarithm. We may write the logarithm $L$ of $\mathrm{P}\left(
\mathbf{u} | \left\{ \mathbf{x}_i, f_i, \sigma_i \right\} \right)$ as:

\begin{equation}
L = \sum_{i=0}^{n_\mathrm{d}-1}
\left( \frac{
-\left[f_i - f_\mathbf{u}(\mathbf{x}_i)\right]^2
}{
2 \sigma_i^2
} \right) + k
\end{equation}

\noindent where $k$ is some constant which does not affect the maximisation
process. It is this quantity, the familiar sum-of-square-residuals, that we
numerically maximise to find our best-fitting set of parameters, which I shall
refer to from here on as $\mathbf{u}^0$.

\section{Estimating the error in $\mathbf{u}^0$}

To estimate the error in the best-fitting parameter values that we find, we
assume $\mathrm{P}\left( \mathbf{u} | \left\{ \mathbf{x}_i, f_i, \sigma_i
\right\} \right)$ to be approximated by an $n_\mathrm{u}$-dimensional Gaussian
distribution around $\mathbf{u}^0$. Taking a Taylor expansion of
$L(\mathbf{u})$ about $\mathbf{u}^0$, we can write:

\begin{eqnarray}
L(\mathbf{u}) & = & L(\mathbf{u}^0) +
    \underbrace{
      \sum_{i=0}^{n_\mathrm{u}-1} \left( u_i - u^0_i \right)
      \left.\frac{\partial L}{\partial u_i}\right|_{\mathbf{u}^0}
    }_{\textrm{Zero at $\mathbf{u}^0$ by definition}} + \label{eqa:L_taylor_expand}\\
& & \sum_{i=0}^{n_\mathrm{u}-1} \sum_{j=0}^{n_\mathrm{u}-1} \frac{\left( u_i - u^0_i \right) \left( u_j - u^0_j \right)}{2}
    \left.\frac{\partial^2 L}{\partial u_i \partial u_j}\right|_{\mathbf{u}^0} +
    \mathcal{O}\left( \mathbf{u} - \mathbf{u}^0\right)^3 \nonumber
\end{eqnarray}

Since the logarithm of a Gaussian distribution is a parabola, the quadratic
terms in the above expansion encode the Gaussian component of the probability
distribution $\mathrm{P}\left( \mathbf{u} | \left\{ \mathbf{x}_i, f_i, \sigma_i
\right\} \right)$ about $\mathbf{u}^0$.\footnote{The use of this is called
\textit{Gauss' Method}. Higher order terms in the expansion represent any
non-Gaussianity in the probability distribution, which we neglect. See MacKay,
D.J.C., \textit{Information Theory, Inference and Learning Algorithms}, CUP
(2003).} We may write the sum of these terms, which we denote $Q$, in matrix
form:

\begin{equation}
Q = \frac{1}{2} \left(\mathbf{u} - \mathbf{u}^0\right)^\mathbf{T} \mathbf{A} \left(\mathbf{u} - \mathbf{u}^0\right)
\label{eqn:Q_vector}
\end{equation}

\noindent where the superscript $^\mathbf{T}$ represents the transpose of the
vector displacement from $\mathbf{u}^0$, and $\mathbf{A}$ is the Hessian matrix
of $L$, given by:

\begin{equation}
A_{ij} = \nabla\nabla L = \left.\frac{\partial^2 L}{\partial u_i \partial u_j}\right|_{\mathbf{u}^0}
\end{equation}
\index{Hessian matrix}

This is the Hessian matrix which is output by the {\tt fit} command. In
general, an $n_\mathrm{u}$-dimensional Gaussian distribution such as that given
by Equation~(\ref{eqa:L_taylor_expand}) yields elliptical contours of
equi-probability in parameter space, whose principal axes need not be aligned
with our chosen coordinate axes -- the variables $u_0 ... u_{n_u-1}$. The
eigenvectors $\mathbf{e}_i$ of $\mathbf{A}$ are the principal axes of these
ellipses, and the corresponding eigenvalues $\lambda_i$ equal $1/\sigma_i^2$,
where $\sigma_i$ is the standard deviation of the probability density function
along the direction of these axes.

This can be visualised by imagining that we diagonalise $\mathbf{A}$, and
expand Equation~(\ref{eqn:Q_vector}) in our diagonal basis. The resulting
expression for $L$ is a sum of square terms; the cross terms vanish in this
basis by definition. The equations of the equi-probability contours become the
equations of ellipses:

\begin{equation}
Q = \frac{1}{2} \sum_{i=0}^{n_\mathrm{u}-1} A_{ii} \left(u_i - u^0_i\right)^2 = k
\end{equation}

\noindent where $k$ is some constant. By comparison with the equation for the
logarithm of a Gaussian distribution, we can associate $A_{ii}$ with
$-1/\sigma_i^2$ in our eigenvector basis.

The problem of evaluating the standard deviations of our variables $u_i$ is
more complicated, however, as we are attempting to evaluate the width of these
elliptical equi-probability contours in directions which are, in general, not
aligned with their principal axes. To achieve this, we first convert our
Hessian matrix into a covariance matrix.

\section{The covariance matrix}
\index{covariance matrix}

The terms of the covariance matrix $V_{ij}$ are defined by:

\begin{equation}
V_{ij} = \left< \left(u_i - u^0_i\right) \left(u_j - u^0_j\right) \right>
\label{eqn:def_covar}
\end{equation}

\noindent Its leading diagonal terms may be recognised as equalling the
variances of each of our $n_\mathrm{u}$ variables; its cross terms measure the
correlation between the variables. If a component $V_{ij} > 0$, it implies that
higher estimates of the coefficient $u_i$ make higher estimates of $u_j$ more
favourable also; if $V_{ij} < 0$, the converse is true.

It is a standard statistical result that $\mathbf{V} = (-\mathbf{A})^{-1}$. In
the remainder of this section we prove this; readers who are willing to accept
this may skip onto Section~\ref{sec:correlation_matrix}.

Using $\Delta u_i$ to denote $\left(u_i - u^0_i\right)$, we may proceed by
rewriting Equation~(\ref{eqn:def_covar}) as:

\begin{eqnarray}
V_{ij} & = & \idotsint_{u_i=-\infty}^{\infty}
\Delta u_i \Delta u_j
\mathrm{P}\left(
\mathbf{u} | \left\{ \mathbf{x}_i, f_i, \sigma_i \right\} \right)
\,\mathrm{d}^{n_\mathrm{u}}\mathbf{u} \\
 & = & \frac{
\idotsint_{u_i=-\infty}^{\infty} \Delta u_i \Delta u_j \exp(-Q) \,\mathrm{d}^{n_\mathrm{u}}\mathbf{u}
}{
\idotsint_{u_i=-\infty}^{\infty} \exp(-Q) \,\mathrm{d}^{n_\mathrm{u}}\mathbf{u}
}
\nonumber
\end{eqnarray}

The normalisation factor in the denominator of this expression, which we denote
as $Z$, the \textit{partition function}, may be evaluated by
$n_\mathrm{u}$-dimensional Gaussian integration, and is a standard result:

\begin{eqnarray}
Z & = & \idotsint_{u_i=-\infty}^{\infty} \exp\left(\frac{1}{2} \Delta \mathbf{u}^\mathbf{T} \mathbf{A} \Delta \mathbf{u} \right) \,\mathrm{d}^{n_\mathrm{u}}\mathbf{u} \\
& = & \frac{(2\pi)^{n_\mathrm{u}/2}}{\mathrm{Det}(\mathbf{-A})} \nonumber
\end{eqnarray}

Differentiating $\log_e(Z)$ with respect of any given component of the Hessian
matrix $A_{ij}$ yields:

\begin{equation}
-2 \frac{\partial}{\partial A_{ij}} \left[ \log_e(Z) \right] = \frac{1}{Z}
\idotsint_{u_i=-\infty}^{\infty} \Delta u_i \Delta u_j \exp(-Q) \,\mathrm{d}^{n_\mathrm{u}}\mathbf{u}
\end{equation}

\noindent which we may identify as equalling $V_{ij}$:

\begin{eqnarray}
\label{eqa:v_zrelate}
V_{ij} & = & -2 \frac{\partial}{\partial A_{ij}} \left[ \log_e(Z) \right] \\
& = & -2 \frac{\partial}{\partial A_{ij}} \left[ \log_e((2\pi)^{n_\mathrm{u}/2}) - \log_e(\mathrm{Det}(\mathbf{-A})) \right] \nonumber \\
& = & 2 \frac{\partial}{\partial A_{ij}} \left[ \log_e(\mathrm{Det}(\mathbf{-A})) \right] \nonumber
\end{eqnarray}

\noindent This expression may be simplified by recalling that the determinant
of a matrix is equal to the scalar product of any of its rows with its
cofactors, yielding the result:

\begin{equation}
\frac{\partial}{\partial A_{ij}} \left[\mathrm{Det}(\mathbf{-A})\right] = -a_{ij}
\end{equation}

\noindent where $a_{ij}$ is the cofactor of $A_{ij}$. Substituting this into
Equation~(\ref{eqa:v_zrelate}) yields:

\begin{equation}
V_{ij} = \frac{-a_{ij}}{\mathrm{Det}(\mathbf{-A})}
\end{equation}

Recalling that the adjoint $\mathbf{A}^\dagger$ of the Hessian matrix is the
matrix of cofactors of its transpose, and that $\mathbf{A}$ is symmetric, we
may write:

\begin{equation}
V_{ij} = \frac{-\mathbf{A}^\dagger}{\mathrm{Det}(\mathbf{-A})} \equiv (-\mathbf{A})^{-1}
\end{equation}

\noindent which proves the result stated earlier.

\section{The correlation matrix}
\label{sec:correlation_matrix}
\index{correlation matrix}

Having evaluated the covariance matrix, we may straightforwardly find the
standard deviations in each of our variables, by taking the square roots of the
terms along its leading diagonal. For datafiles where the user does not specify
the standard deviations $\sigma_i$ in each value $f_i$, the task is not quite
complete, as the Hessian matrix depends critically upon these uncertainties,
even if they are assumed the same for all of our $f_i$. This point is returned
to in Section~\ref{sec:finding_sigmai}.

The correlation matrix $\mathbf{C}$, whose terms are given by:

\begin{equation}
C_{ij} = \frac{V_{ij}}{\sigma_i\sigma_j}
\end{equation}

\noindent may be considered a more user-friendly version of the covariance
matrix for inspecting the correlation between parameters. The leading diagonal
terms are all clearly equal unity by construction. The cross terms lie in the
range $-1 \leq C_{ij} \leq 1$, the upper limit of this range representing
perfect correlation between parameters, and the lower limit perfect
anti-correlation.

\section{Finding $\sigma_i$}
\label{sec:finding_sigmai}

Throughout the preceding sections, the uncertainties in the supplied target
values $f_i$ have been denoted $\sigma_i$ (see
Section~\ref{sec:bayes_notation}).  The user has the option of supplying these
in the source datafile, in which case the provisions of the previous sections
are now complete; both best-estimate parameter values and their uncertainties
can be calculated. The user may also, however, leave the uncertainties in $f_i$
unstated, in which case, as described in Section~\ref{sec:bayes_notation}, we
assume all of the data values to have a common uncertainty
$\sigma_\mathrm{data}$, which is an unknown.

In this case, where $\sigma_i = \sigma_\mathrm{data} \,\forall\, i$, the best
fitting parameter values are independent of $\sigma_\mathrm{data}$, but the
same is not true of the uncertainties in these values, as the terms of the
Hessian matrix do depend upon $\sigma_\mathrm{data}$. We must therefore
undertake a further calculation to find the most probable value of
$\sigma_\mathrm{data}$, given the data. This is achieved by maximising
$\mathrm{P}\left( \sigma_\mathrm{data} | \left\{ \mathbf{x}_i, f_i \right\}
\right)$. Returning once again to Bayes' Theorem, we can write:

\begin{equation}
\mathrm{P}\left( \sigma_\mathrm{data} | \left\{ \mathbf{x}_i, f_i \right\} \right)
= \frac{
\mathrm{P}\left( \left\{ f_i \right\} | \sigma_\mathrm{data}, \left\{ \mathbf{x}_i \right\} \right)
\mathrm{P}\left( \sigma_\mathrm{data} | \left\{ \mathbf{x}_i \right\} \right)
}{
\mathrm{P}\left( \left\{ f_i \right\} | \left\{ \mathbf{x}_i \right\} \right)
}
\end{equation}

As before, we neglect the denominator, which has no effect upon the
maximisation problem, and assume a uniform prior $\mathrm{P}\left(
\sigma_\mathrm{data} | \left\{ \mathbf{x}_i \right\} \right)$. This reduces the
problem to the maximisation of $\mathrm{P}\left( \left\{ f_i \right\} |
\sigma_\mathrm{data}, \left\{ \mathbf{x}_i \right\} \right)$, which we may
write as a marginalised probability distribution over $\mathbf{u}$:

\begin{eqnarray}
\label{eqa:p_f_given_sigma}
\mathrm{P}\left( \left\{ f_i \right\} | \sigma_\mathrm{data}, \left\{ \mathbf{x}_i \right\} \right) =
\idotsint_{-\infty}^{\infty}
&
\mathrm{P}\left( \left\{ f_i \right\} | \sigma_\mathrm{data}, \left\{ \mathbf{x}_i \right\}, \mathbf{u} \right)
\times & \\ &
\mathrm{P}\left( \mathbf{u} | \sigma_\mathrm{data}, \left\{ \mathbf{x}_i \right\} \right)
\,\mathrm{d}^{n_\mathrm{u}}\mathbf{u}
& \nonumber
\end{eqnarray}

Assuming a uniform prior for $\mathbf{u}$, we may neglect the latter term in
the integral, but even with this assumption, the integral is not generally
tractable, as $\mathrm{P}\left( \left\{ f_i \right\} | \sigma_\mathrm{data},
\left\{ \mathbf{x}_i \right\}, \left\{ \mathbf{u}_i \right\} \right)$ may well
be multimodal in form. However, if we neglect such possibilities, and assume
this probability distribution to be approximate a Gaussian \textit{globally},
we can make use of the standard result for an $n_\mathrm{u}$-dimensional Gaussian integral:

\begin{equation}
\idotsint_{-\infty}^{\infty}
\exp \left(
\frac{1}{2}\mathbf{u}^\mathbf{T} \mathbf{A} \mathbf{u}
\right) \,\mathrm{d}^{n_\mathrm{u}}\mathbf{u}
=
\frac{
(2\pi)^{n_\mathrm{u}/2}
}{
\sqrt{\mathrm{Det}\left(-\mathbf{A}\right)}
}
\end{equation}

\noindent We may thus approximate Equation~(\ref{eqa:p_f_given_sigma}) as:

\begin{eqnarray}
\mathrm{P}\left( \left\{ f_i \right\} | \sigma_\mathrm{data}, \left\{ \mathbf{x}_i \right\} \right)
& \approx &
\mathrm{P}\left( \left\{ f_i \right\} | \sigma_\mathrm{data}, \left\{ \mathbf{x}_i \right\}, \mathbf{u}^0 \right)
\times \\
& &
\mathrm{P}\left( \mathbf{u}^0 | \sigma_\mathrm{data}, \left\{ \mathbf{x}_i, f_i \right\} \right)
\frac{
(2\pi)^{n_\mathrm{u}/2}
}{
\sqrt{\mathrm{Det}\left(-\mathbf{A}\right)}
}
\nonumber
\end{eqnarray}

As in Section~\ref{sec:bayes_pdf}, it is numerically easier to maximise this
quantity via its logarithm, which we denote $L_2$, and can write as:

\begin{eqnarray}
L_2 & = &
\sum_{i=0}^{n_\mathrm{d}-1}
\left(
\frac{
-\left[f_i - f_{\mathbf{u}^0}(\mathbf{x}_i)\right]^2
}{
2\sigma_\mathrm{data}^2
}
- \log_e \left(2\pi\sqrt{\sigma_\mathrm{data}} \right)
\right) +
\\ & & \nonumber
\log_e \left(
\frac{
(2\pi)^{n_\mathrm{u}/2}
}{
\sqrt{\mathrm{Det}\left(-\mathbf{A}\right)}
}
\right)
\end{eqnarray}

This quantity is maximised numerically, a process simplified by the fact that
$\mathbf{u}^0$ is independent of $\sigma_\mathrm{data}$.

% changelog.tex
%
% The documentation in this file is part of Pyxplot
% <http://www.pyxplot.org.uk>
%
% Copyright (C) 2006-2012 Dominic Ford <coders@pyxplot.org.uk>
%               2008-2012 Ross Church
%
% $Id$
%
% Pyxplot is free software; you can redistribute it and/or modify it under the
% terms of the GNU General Public License as published by the Free Software
% Foundation; either version 2 of the License, or (at your option) any later
% version.
%
% You should have received a copy of the GNU General Public License along with
% Pyxplot; if not, write to the Free Software Foundation, Inc., 51 Franklin
% Street, Fifth Floor, Boston, MA  02110-1301, USA

% ----------------------------------------------------------------------------

% LaTeX source for the Pyxplot Users' Guide

\chapter{ChangeLog}
\index{ChangeLog}

\subsection*{2012 Aug 29: Pyxplot 0.9.1}

Version 0.9.1 is a minor update with new support for running Pyxplot on Raspberry Pi. It fixes SIGBUS errors in Pyxplot's math engine when run on armhf architectures.

\subsection*{2012 Aug 1: Pyxplot 0.9.0}

Version 0.9 is a major update. Many new data types have been introduced, each
of which has methods which can be called in an object-orientated fashion. These
include:

\begin{itemize}
\item {\bf Colors}, which can be stored in variables for subsequent use in vector graphics commands. The addition and subtraction operators act on colors to allow color mixing.
\item {\bf Dates}, which can be imported from calendar dates, unix times or Julian dates. Dates can be subtracted to give time intervals.
\item {\bf Lists} and {\bf dictionaries}, which can be iterated over, or used to feed calculated data into the {\tt plot} and {\tt tabulate} commands.
\item {\bf Vectors} and {\bf matrices}, which allow matrix algebra. These types interface cleanly with Pyxplot's vector-graphics commands, allowing positions to be specified as vector expressions.
\item {\bf File handles}, which allow Pyxplot to read data from files, or write data or logs to files.
\item {\bf Modules} and {\bf classes}, which allow object-orientated programming.
\end{itemize}

In addition, Pyxplot's range of operators has been extended to include most of those in the C programming language, allowing expressions such as

\vspace{3mm}
\input{fragments/tex/fs_operators.tex}
\vspace{3mm}

\noindent to be written.

\subsubsection*{Incompatibilities with Pyxplot 0.8}

The extensions to Pyxplot in version 0.9 mean that some minor changes to syntax have been necessary. These include:

\begin{itemize}
\item Some functions and variables have been renamed. Variables whose names used to begin {\tt phy\_} now live in a module called {\tt phy}. They may be accessed as, for example, {\tt phy.c}. Similarly, random number generating functions now live in a module called {\tt random}; statistics functions in a module called {\tt stats}; time-handling functions in {\tt time}; operating system functions in {\tt os}; and astronomy functions in {\tt ast}. The contents of these modules can be listed by typing, for example, {\tt print phy}.
\item Custom colors, which used to be specified using syntax such as {\tt rgb0.2:0.3:0.4}, should now be specified using the {\tt rgb(r,g,b)} functions, as, for example, {\tt rgb(0.2,0.3,0.4)}. Custom colors can now be stored in variables for later use (see Section~\ref{sec:colorObjects}).
\item The range of escape characters which can be used in strings has been increased, so that, for example, {\tt $\backslash$n} is a newline and {\tt $\backslash$t} a tab. As in python, prepending the string with the character {\tt r} disables all escape character expansion. As backslashes are common characters in latex command strings, the easiest approach is to always prepend latex strings with an {\tt r}. As in python, triple quotes, e.g.\ {\tt r"""2 $\backslash$times 3"""} can be used where required (see Section~\ref{sec:latex_incompatibility}).
\item In the {\tt foreach} command, square brackets should be used to delimit lists of items to iterate over. The Pyxplot 0.8 syntax {\tt foreach i in (1,2,3)} should now be written {\tt foreach i in [1,2,3]} (see Section~\ref{sec:foreach}).
\end{itemize}

\subsection*{2011 Jan 7: Pyxplot 0.8.4}

\subsubsection*{Summary:}

This is a minor bugfix release.

\subsubsection*{Details:}

\begin{itemize}
\item Two-dimensional parametric grid plotting implemented.
\item Bugfix to the dots plot style; filled triangles replaces with filled circles.
\item Bugfix to linewidths used when drawing line icons on graph legends.
\item Bugfix to Makefile to ensure libraries link correctly under Red Hat and SUSE.
\item Code cleanup to ensure correct compilation with {\tt -O2} optimisation.
\end{itemize}

\subsection*{2010 Sep 15: Pyxplot 0.8.3}

\subsubsection*{Summary:}

This is a minor bugfix release.

\subsubsection*{Details:}

\begin{itemize}
\item @ macro expansion operator implemented.
\item assert command implemented.
\item for command behaviour changed such that {\tt for i=1 to 10} includes a final iteration with {\tt i}=10.
\item Point types rearranged into a more logical order.
\item Improved support for newer Windows bitmap images.
\item Bugfix to the {\tt set unit preferred} command.
\item Binary not operator bugfixed.
\item Bugfix to handling of comma-separated horizontal datafiles.
\item Mathematical function {\tt finite()} added.
\end{itemize}

\subsection*{2010 Aug 4: Pyxplot 0.8.2}

\subsubsection*{Summary:}

This release introduces three-dimensional plotting, as well as the ability to plot two-dimensional maps of functions as either color maps, contour plots, or as three-dimensional surfaces. A large number of bugs have also been fixed.

\subsubsection*{Details:}

\begin{itemize}
\item 3D plotting implemented.
\item New plot styles colormap, contourmap and surface added.
\item Interpolation of 2D datagrids and bitmap images implemented.
\item Stepwise interpolation mode added.
\item Dependency on libkpathsea relaxed to make installation under MacOS easier; linking to the library is still strongly recommended on systems where it is readily available.
\item Mathematical functions {\tt frac-tal\_\-julia()}, {\tt frac\-tal\_\-man\-del\-brot()} and {\tt prime()} added.
\item Many bug fixes, especially to the ticking of axes.
\end{itemize}

\subsection*{2010 Jun 1: Pyxplot 0.8.1}

\subsubsection*{Summary:}

This release has no major new features, but fixes several significant bugs in version 0.8.0.

\subsubsection*{Details:}

\begin{itemize}
\item Mathematical functions {\tt time\_\-from\-unix()}, {\tt time\_\-unix()}, {\tt zernike()} and {\tt zernikeR()} added.
\item Bug fix to the ticking of linked axes.
\item Bug fix to the ticking of axes with blank axis tick labels.
\item Makefile and configure script improved for portability.
\end{itemize}

\subsection*{2010 May 19: Pyxplot 0.8.0}

\subsubsection*{Summary:}

This release is a major update, for which Pyxplot's original python code has
been completely rewritten in C with the addition of many new features. Because
of the scale of this update, there is some minor syntax incompatibility with
previous versions where features have undergone particularly heavy change. The
most apparent change is the increase in speed and efficiency resultant from the
use of a compiled language: especially when handling large \datafile s, Pyxplot
0.8.0 can run more than an order-of-magnitude faster than previous versions.

\subsubsection*{Details:}

\begin{itemize}
\item The handling of large \datafile s has been streamlined to require around an order-of-magnitude less time and memory.
\item Pyxplot's mathematical environment has been extended to handle complex numbers and quantities with physical units.
\item The range of mathematical functions built into Pyxplot has been massively extended.
\item The {\tt solve} command has been added to allow the solution of systems of equations.
\item The {\tt maximize} and {\tt minimize} commands have been added to allow searches for local extrema of functions.
\item An {\tt fft} command has been added for performing Fourier transforms on data.
\item New plot styles -- {\tt filledregion} and {\tt yerrorshaded} -- have been added for plotting filled error regions.
\item The configuration of linked axes has been entirely redesigned.
\item Parametric function plotting has been implemented.
\item Colours can now be specified by RGB, HSB or CMYK components, as well as by name.
\item Several commands, e.g. {\tt box}, {\tt circle}, {\tt ellipse}, etc., have been added to allow vector graphics to be produced in Pyxplot's multiplot environment.
\item The {\tt jpeg} command has been generalised to allow the incorporation of not only {\tt jpeg} images, but also {\tt bmp}, {\tt gif} and {\tt png} images, onto multiplot canvases. The command has been renamed {\tt image} in recognition of its wider applicability. Image transparency is now supported in {\tt gif} and {\tt png} images.
\item The {\tt spline} command, now renamed the {\tt interpolate} command, has been extended up provide many types of interpolation between datapoints.
\item A wide range of conditional and flow control structures have been added to Pyxplot's command language -- these are the {\tt do}, {\tt for}, {\tt foreach}, {\tt if} and {\tt while} commands and the {\tt cond\-ition\-alS} and {\tt con\-dition\-alN} mathematical functions.
\item Input filters have been introduced as a mechanism by which datafiles in arbitrary formats can be read.
\item Pyxplot's command-line interface now supports tab completion.
\item The {\tt show} command has been reworked to produce pastable output.
\item Many minor bugs have been fixed.
\end{itemize}

\subsection*{2009 Nov 17: Pyxplot 0.7.1}

\subsubsection*{Summary:}

This release has no major new features, but fixes several serious bugs in version 0.7.0.

\subsubsection*{Details:}

\begin{itemize}
\item The {\tt exec} command did not work in Pyxplot 0.7.0; this issue has been resolved.
\item The {\tt xyerrorrange} plot style did not work in Pyxplot 0.7.0; this issue has been resolved.
\item Pyxplot 0.7.0 produces large numbers of python deprecation error messages when run under python 2.6; the code has been updated to remove references to deprecated python functions.
\end{itemize}

\subsubsection*{Details -- Change of System Requirements:}

\begin{itemize}
\item In order to fix some of the bugs listed above, it has been necessary to
fix bugs in the PyX graphics library as well as those in Pyxplot. As a result,
and to ensure that these bugfixes reach users as quickly as possible, we have
opted to ship our own modified version of PyX 0.10, called dcfPyX with Pyxplot.
\end{itemize}

\subsection*{2008 Oct 14: Pyxplot 0.7.0}

\subsubsection*{Summary:}

Third Pyxplot beta-release. The code has undergone significant streamlining,
and now runs approximately twice as fast as version 0.6.3 when handling large
datafiles. Memory usage has also been radically reduced. Two new data
processing commands have been introduced. The {\tt tabulate} command can be
used to produce textual datafiles, allowing the user to read data in from
files, apply some analysis, and then write the processed data back to file. The
{\tt histogram} command can be used to estimate the frequency densities of sets
of data points, either by binning them into a bar chart, or by fitting a
functional form to their frequency density.

\subsubsection*{Details -- New and Extended Commands:}

\begin{itemize}
\item {\tt tabulate}
\item {\tt histogram}
\item {\tt set label} and {\tt text} commands extended to allow a color to be
specified.
\end{itemize}

\subsubsection*{Details -- API changes}

\begin{itemize}
\item {\tt diff\_dx()} and {\tt int\_dx()} functions -- the function to be
differentiated or integrated must now be placed in quotation marks.
\end{itemize}

\subsubsection*{Details -- Change of System Requirements:}

\begin{itemize}
\item Requirement of PyX version 0.9 has been updated to PyX version 0.10. Note that new versions of the PyX graphics library are not generally backwardly compatible.
\end{itemize}

\subsection*{2007 Feb 26: Pyxplot 0.6.3}

\subsubsection*{Summary:}

Second Pyxplot beta-release. The most significant change is the introduction of
a new command-line parser, with greatly improved handling of complex
expressions and much more meaningful syntax error messages. Multi-platform
compatibility has also been massively improved, and dependencies loosened.  A
small number of new commands have been added; most notable among them are the
{\tt jpeg} and {\tt eps} commands, which embed images in multiplots.

\subsubsection*{Details -- New and Extended Commands:}

\begin{itemize}
\item {\tt jpeg}
\item {\tt eps}
\item {\tt set xtics} and {\tt set mxtics}
\item {\tt text} and {\tt set label} commands extended to allow text rotation.
\item {\tt set log} command extended to allow the use of logarithms with bases other than 10.
\item {\tt set preamble}
\item {\tt set term enlarge | noenlarge}
\item {\tt set term pdf}
\item {\tt set term x11\_persist}
\end{itemize}

\subsubsection*{Details -- Eased System Requirements:}

\begin{itemize}
\item Requirement on Python 2.4 minimum eased to version 2.3 minimum.
\item Requirements on scipy and readline eased; Pyxplot will now work in reduced form when they are absent, though they are still strongly recommended.
\item dvips and Ghostscript are no longer required.
\end{itemize}

\subsubsection*{Details -- Removed Commands:}

Due to a general refinement of Pyxplot's API, some of the less sensible pieces
of syntax from Version~0.5 are no longer supported. The author apologises for
any inconvenience caused.

\begin{itemize}
\item The {\tt delete\_arrow}, {\tt delete\_text}, {\tt move\_text}, {\tt undelete\_arrow} and {\tt undelete\_text} commands have been removed from the Pyxplot API. The {\tt move}, {\tt delete} and {\tt undelete} commands should now be used to act upon all types of multiplot objects.
\item The {\tt set terminal} command no longer accepts the {\tt enhanced} and {\tt noenhanced} modifiers. The {\tt postscript} and {\tt eps} terminals should be used instead.
\item The {\tt select} modifier, used after the {\tt plot}, {\tt replot}, {\tt fit} and {\tt spline} command can now only be used once; to specify multiple {\tt select} criteria, use the {\tt and} logical operator.
\end{itemize}

\subsection*{2006 Sep 09: Pyxplot 0.5.8}

First beta-release.


\printindex
\end{document}

